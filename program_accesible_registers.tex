
\section{PROGRAM ACCESSIBLE REGISTERS}

%% ===== \end{flushleft}
%% ===== 
%% ===== 
%% ===== \begin{flushleft}
%% ===== A processor register is a hardware assembly that holds information for use in some
%% ===== \end{flushleft}
%% ===== 
%% ===== 
%% ===== \begin{flushleft}
%% ===== specified way. An accessible register is a register whose contents are available to the user for his
%% ===== \end{flushleft}
%% ===== 
%% ===== 
%% ===== \begin{flushleft}
%% ===== purposes. Some accessible registers are explicitly addressed by particular instructions, some are
%% ===== \end{flushleft}
%% ===== 
%% ===== 
%% ===== \begin{flushleft}
%% ===== implicitly referenced during the course of execution of instructions, and some are used in both
%% ===== \end{flushleft}
%% ===== 
%% ===== 
%% ===== \begin{flushleft}
%% ===== ways. The accessible registers are listed in Table 3-1. See Section 4 for a discussion of each
%% ===== \end{flushleft}
%% ===== 
%% ===== 
%% ===== \begin{flushleft}
%% ===== instruction to determine the way in which the registers are used.
%% ===== \end{flushleft}
%% ===== 
%% ===== 
%% ===== 
%% ===== 
%% ===== 
%% ===== \begin{flushleft}
%% ===== Table 3-1. Processor Registers
%% ===== \end{flushleft}
%% ===== 
%% ===== 
%% ===== \begin{flushleft}
%% ===== Register name
%% ===== \end{flushleft}
%% ===== 
%% ===== 
%% ===== 
%% ===== 
%% ===== 
%% ===== \begin{flushleft}
%% ===== Mnemonic Length (bits) Quantity
%% ===== \end{flushleft}
%% ===== 
%% ===== 
%% ===== 
%% ===== 
%% ===== 
%% ===== \begin{flushleft}
%% ===== Accumulator Register
%% ===== \end{flushleft}
%% ===== 
%% ===== 
%% ===== 
%% ===== 
%% ===== 
%% ===== \begin{flushleft}
%% ===== A
%% ===== \end{flushleft}
%% ===== 
%% ===== 
%% ===== 
%% ===== 
%% ===== 
%% ===== 36
%% ===== 
%% ===== 
%% ===== 
%% ===== 
%% ===== 
%% ===== 1
%% ===== 
%% ===== 
%% ===== 
%% ===== 
%% ===== 
%% ===== \begin{flushleft}
%% ===== Quotient Register
%% ===== \end{flushleft}
%% ===== 
%% ===== 
%% ===== 
%% ===== 
%% ===== 
%% ===== \begin{flushleft}
%% ===== Q
%% ===== \end{flushleft}
%% ===== 
%% ===== 
%% ===== 
%% ===== 
%% ===== 
%% ===== 36
%% ===== 
%% ===== 
%% ===== 
%% ===== 
%% ===== 
%% ===== 1
%% ===== 
%% ===== 
%% ===== 
%% ===== 
%% ===== 
%% ===== \begin{flushleft}
%% ===== AQ
%% ===== \end{flushleft}
%% ===== 
%% ===== 
%% ===== 
%% ===== 
%% ===== 
%% ===== 72
%% ===== 
%% ===== 
%% ===== 
%% ===== 
%% ===== 
%% ===== 1
%% ===== 
%% ===== 
%% ===== 
%% ===== 
%% ===== 
%% ===== \begin{flushleft}
%% ===== E
%% ===== \end{flushleft}
%% ===== 
%% ===== 
%% ===== 
%% ===== 
%% ===== 
%% ===== 8
%% ===== 
%% ===== 
%% ===== 
%% ===== 
%% ===== 
%% ===== 1
%% ===== 
%% ===== 
%% ===== 
%% ===== 
%% ===== 
%% ===== \begin{flushleft}
%% ===== EAQ
%% ===== \end{flushleft}
%% ===== 
%% ===== 
%% ===== 
%% ===== 
%% ===== 
%% ===== 80
%% ===== 
%% ===== 
%% ===== 
%% ===== 
%% ===== 
%% ===== 1
%% ===== 
%% ===== 
%% ===== 
%% ===== 
%% ===== 
%% ===== \begin{flushleft}
%% ===== Index Registers
%% ===== \end{flushleft}
%% ===== 
%% ===== 
%% ===== 
%% ===== 
%% ===== 
%% ===== \begin{flushleft}
%% ===== Xn
%% ===== \end{flushleft}
%% ===== 
%% ===== 
%% ===== 
%% ===== 
%% ===== 
%% ===== 18
%% ===== 
%% ===== 
%% ===== 
%% ===== 
%% ===== 
%% ===== 8
%% ===== 
%% ===== 
%% ===== 
%% ===== 
%% ===== 
%% ===== \begin{flushleft}
%% ===== Indicator Register
%% ===== \end{flushleft}
%% ===== 
%% ===== 
%% ===== 
%% ===== 
%% ===== 
%% ===== \begin{flushleft}
%% ===== IR
%% ===== \end{flushleft}
%% ===== 
%% ===== 
%% ===== 
%% ===== 
%% ===== 
%% ===== 14
%% ===== 
%% ===== 
%% ===== 
%% ===== 
%% ===== 
%% ===== 1
%% ===== 
%% ===== 
%% ===== 
%% ===== 
%% ===== 
%% ===== \begin{flushleft}
%% ===== BAR
%% ===== \end{flushleft}
%% ===== 
%% ===== 
%% ===== 
%% ===== 
%% ===== 
%% ===== 18
%% ===== 
%% ===== 
%% ===== 
%% ===== 
%% ===== 
%% ===== 1
%% ===== 
%% ===== 
%% ===== 
%% ===== 
%% ===== 
%% ===== \begin{flushleft}
%% ===== TR
%% ===== \end{flushleft}
%% ===== 
%% ===== 
%% ===== 
%% ===== 
%% ===== 
%% ===== 27
%% ===== 
%% ===== 
%% ===== 
%% ===== 
%% ===== 
%% ===== 1
%% ===== 
%% ===== 
%% ===== 
%% ===== 
%% ===== 
%% ===== \begin{flushleft}
%% ===== RALR
%% ===== \end{flushleft}
%% ===== 
%% ===== 
%% ===== 
%% ===== 
%% ===== 
%% ===== 3
%% ===== 
%% ===== 
%% ===== 
%% ===== 
%% ===== 
%% ===== 1
%% ===== 
%% ===== 
%% ===== 
%% ===== 
%% ===== 
%% ===== \begin{flushleft}
%% ===== Pointer Registers
%% ===== \end{flushleft}
%% ===== 
%% ===== 
%% ===== 
%% ===== 
%% ===== 
%% ===== \begin{flushleft}
%% ===== PRn
%% ===== \end{flushleft}
%% ===== 
%% ===== 
%% ===== 
%% ===== 
%% ===== 
%% ===== 42
%% ===== 
%% ===== 
%% ===== 
%% ===== 
%% ===== 
%% ===== 8
%% ===== 
%% ===== 
%% ===== 
%% ===== 
%% ===== 
%% ===== \begin{flushleft}
%% ===== Address Registers
%% ===== \end{flushleft}
%% ===== 
%% ===== 
%% ===== 
%% ===== 
%% ===== 
%% ===== \begin{flushleft}
%% ===== ARn
%% ===== \end{flushleft}
%% ===== 
%% ===== 
%% ===== 
%% ===== 
%% ===== 
%% ===== 24
%% ===== 
%% ===== 
%% ===== 
%% ===== 
%% ===== 
%% ===== 8
%% ===== 
%% ===== 
%% ===== 
%% ===== 
%% ===== 
%% ===== \begin{flushleft}
%% ===== PPR
%% ===== \end{flushleft}
%% ===== 
%% ===== 
%% ===== 
%% ===== 
%% ===== 
%% ===== 37
%% ===== 
%% ===== 
%% ===== 
%% ===== 
%% ===== 
%% ===== 1
%% ===== 
%% ===== 
%% ===== 
%% ===== 
%% ===== 
%% ===== \begin{flushleft}
%% ===== TPR
%% ===== \end{flushleft}
%% ===== 
%% ===== 
%% ===== 
%% ===== 
%% ===== 
%% ===== 42
%% ===== 
%% ===== 
%% ===== 
%% ===== 
%% ===== 
%% ===== 1
%% ===== 
%% ===== 
%% ===== 
%% ===== 
%% ===== 
%% ===== \begin{flushleft}
%% ===== DSBR
%% ===== \end{flushleft}
%% ===== 
%% ===== 
%% ===== 
%% ===== 
%% ===== 
%% ===== 51
%% ===== 
%% ===== 
%% ===== 
%% ===== 
%% ===== 
%% ===== 1
%% ===== 
%% ===== 
%% ===== 
%% ===== 
%% ===== 
%% ===== \begin{flushleft}
%% ===== Segment Descriptor Word Associative Memory
%% ===== \end{flushleft}
%% ===== 
%% ===== 
%% ===== 
%% ===== 
%% ===== 
%% ===== \begin{flushleft}
%% ===== SDWAM
%% ===== \end{flushleft}
%% ===== 
%% ===== 
%% ===== 
%% ===== 
%% ===== 
%% ===== 88
%% ===== 
%% ===== 
%% ===== 
%% ===== 
%% ===== 
%% ===== 16
%% ===== 
%% ===== 
%% ===== 
%% ===== 
%% ===== 
%% ===== \begin{flushleft}
%% ===== Page Table Word Associative Memory
%% ===== \end{flushleft}
%% ===== 
%% ===== 
%% ===== 
%% ===== 
%% ===== 
%% ===== \begin{flushleft}
%% ===== PTWAM
%% ===== \end{flushleft}
%% ===== 
%% ===== 
%% ===== 
%% ===== 
%% ===== 
%% ===== 51
%% ===== 
%% ===== 
%% ===== 
%% ===== 
%% ===== 
%% ===== 16
%% ===== 
%% ===== 
%% ===== 
%% ===== 
%% ===== 
%% ===== \begin{flushleft}
%% ===== Fault Register
%% ===== \end{flushleft}
%% ===== 
%% ===== 
%% ===== 
%% ===== 
%% ===== 
%% ===== \begin{flushleft}
%% ===== FR
%% ===== \end{flushleft}
%% ===== 
%% ===== 
%% ===== 
%% ===== 
%% ===== 
%% ===== 35
%% ===== 
%% ===== 
%% ===== 
%% ===== 
%% ===== 
%% ===== 1
%% ===== 
%% ===== 
%% ===== 
%% ===== 
%% ===== 
%% ===== \begin{flushleft}
%% ===== Mode Register
%% ===== \end{flushleft}
%% ===== 
%% ===== 
%% ===== 
%% ===== 
%% ===== 
%% ===== \begin{flushleft}
%% ===== MR
%% ===== \end{flushleft}
%% ===== 
%% ===== 
%% ===== 
%% ===== 
%% ===== 
%% ===== 33
%% ===== 
%% ===== 
%% ===== 
%% ===== 
%% ===== 
%% ===== 1
%% ===== 
%% ===== 
%% ===== 
%% ===== 
%% ===== 
%% ===== \begin{flushleft}
%% ===== CMR
%% ===== \end{flushleft}
%% ===== 
%% ===== 
%% ===== 
%% ===== 
%% ===== 
%% ===== 28
%% ===== 
%% ===== 
%% ===== 
%% ===== 
%% ===== 
%% ===== 1
%% ===== 
%% ===== 
%% ===== 
%% ===== 
%% ===== 
%% ===== \begin{flushleft}
%% ===== Control Unit (CU) History Register
%% ===== \end{flushleft}
%% ===== 
%% ===== 
%% ===== 
%% ===== 
%% ===== 
%% ===== 72
%% ===== 
%% ===== 
%% ===== 
%% ===== 
%% ===== 
%% ===== 16
%% ===== 
%% ===== 
%% ===== 
%% ===== 
%% ===== 
%% ===== \begin{flushleft}
%% ===== Operations Unit (OU) History Register
%% ===== \end{flushleft}
%% ===== 
%% ===== 
%% ===== 
%% ===== 
%% ===== 
%% ===== 72
%% ===== 
%% ===== 
%% ===== 
%% ===== 
%% ===== 
%% ===== 16
%% ===== 
%% ===== 
%% ===== 
%% ===== 
%% ===== 
%% ===== \begin{flushleft}
%% ===== Decimal Unit (DU) History Register
%% ===== \end{flushleft}
%% ===== 
%% ===== 
%% ===== 
%% ===== 
%% ===== 
%% ===== 72
%% ===== 
%% ===== 
%% ===== 
%% ===== 
%% ===== 
%% ===== 16
%% ===== 
%% ===== 
%% ===== 
%% ===== 
%% ===== 
%% ===== \begin{flushleft}
%% ===== Appending Unit (APU) History Register
%% ===== \end{flushleft}
%% ===== 
%% ===== 
%% ===== 
%% ===== 
%% ===== 
%% ===== 72
%% ===== 
%% ===== 
%% ===== 
%% ===== 
%% ===== 
%% ===== 16
%% ===== 
%% ===== 
%% ===== 
%% ===== 
%% ===== 
%% ===== \begin{flushleft}
%% ===== Configuration Switch Data
%% ===== \end{flushleft}
%% ===== 
%% ===== 
%% ===== 
%% ===== 
%% ===== 
%% ===== 36
%% ===== 
%% ===== 
%% ===== 
%% ===== 
%% ===== 
%% ===== 5
%% ===== 
%% ===== 
%% ===== 
%% ===== 
%% ===== 
%% ===== \begin{flushleft}
%% ===== Control Unit Data
%% ===== \end{flushleft}
%% ===== 
%% ===== 
%% ===== 
%% ===== 
%% ===== 
%% ===== 288
%% ===== 
%% ===== 
%% ===== 
%% ===== 
%% ===== 
%% ===== 1
%% ===== 
%% ===== 
%% ===== 
%% ===== 
%% ===== 
%% ===== \begin{flushleft}
%% ===== Decimal Unit Data
%% ===== \end{flushleft}
%% ===== 
%% ===== 
%% ===== 
%% ===== 
%% ===== 
%% ===== 288
%% ===== 
%% ===== 
%% ===== 
%% ===== 
%% ===== 
%% ===== 1
%% ===== 
%% ===== 
%% ===== 
%% ===== 
%% ===== 
%% ===== \begin{flushleft}
%% ===== Accumulator-Quotient Register
%% ===== \end{flushleft}
%% ===== 
%% ===== 
%% ===== 
%% ===== 
%% ===== 
%% ===== \begin{flushleft}
%% ===== (a)
%% ===== \end{flushleft}
%% ===== 
%% ===== 
%% ===== 
%% ===== 
%% ===== 
%% ===== \begin{flushleft}
%% ===== Exponent Register
%% ===== \end{flushleft}
%% ===== 
%% ===== 
%% ===== \begin{flushleft}
%% ===== Exponent-Accumulator-Quotient Register
%% ===== \end{flushleft}
%% ===== 
%% ===== 
%% ===== 
%% ===== 
%% ===== 
%% ===== \begin{flushleft}
%% ===== (a)
%% ===== \end{flushleft}
%% ===== 
%% ===== 
%% ===== 
%% ===== 
%% ===== 
%% ===== \begin{flushleft}
%% ===== Base Address Register
%% ===== \end{flushleft}
%% ===== 
%% ===== 
%% ===== \begin{flushleft}
%% ===== Timer Register
%% ===== \end{flushleft}
%% ===== 
%% ===== 
%% ===== \begin{flushleft}
%% ===== Ring Alarm Register
%% ===== \end{flushleft}
%% ===== 
%% ===== 
%% ===== 
%% ===== 
%% ===== 
%% ===== \begin{flushleft}
%% ===== Procedure Pointer Register
%% ===== \end{flushleft}
%% ===== 
%% ===== 
%% ===== 
%% ===== 
%% ===== 
%% ===== \begin{flushleft}
%% ===== (b)
%% ===== \end{flushleft}
%% ===== 
%% ===== 
%% ===== 
%% ===== 
%% ===== 
%% ===== \begin{flushleft}
%% ===== Temporary Pointer Register
%% ===== \end{flushleft}
%% ===== 
%% ===== 
%% ===== 
%% ===== 
%% ===== 
%% ===== \begin{flushleft}
%% ===== (b)
%% ===== \end{flushleft}
%% ===== 
%% ===== 
%% ===== 
%% ===== 
%% ===== 
%% ===== \begin{flushleft}
%% ===== Descriptor Segment Base Register
%% ===== \end{flushleft}
%% ===== 
%% ===== 
%% ===== 
%% ===== 
%% ===== 
%% ===== \begin{flushleft}
%% ===== Cache Mode Register
%% ===== \end{flushleft}
%% ===== 
%% ===== 
%% ===== 
%% ===== 
%% ===== 
%% ===== \begin{flushleft}
%% ===== (a)This register is not a separate physical assembly but is a combination of its constituent
%% ===== \end{flushleft}
%% ===== 
%% ===== 
%% ===== \begin{flushleft}
%% ===== registers.
%% ===== \end{flushleft}
%% ===== 
%% ===== 
%% ===== 
%% ===== 
%% ===== 
%% ===== \begin{flushleft}
%% ===== \newpage
%% ===== (b)This register is not explicitly addressable, but is included because of its vital role in instruction
%% ===== \end{flushleft}
%% ===== 
%% ===== 
%% ===== \begin{flushleft}
%% ===== and operand address preparation.
%% ===== \end{flushleft}
%% ===== 
%% ===== 
%% ===== \begin{flushleft}
%% ===== In the descriptions that follow, the diagrams given for register formats do not imply that a
%% ===== \end{flushleft}
%% ===== 
%% ===== 
%% ===== \begin{flushleft}
%% ===== physical assembly possessing the pictured bit pattern exists.
%% ===== \end{flushleft}
%% ===== 
%% ===== 
%% ===== \begin{flushleft}
%% ===== The diagram is a graphic
%% ===== \end{flushleft}
%% ===== 
%% ===== 
%% ===== \begin{flushleft}
%% ===== representation of the form of the register data as it appears in main memory when the register
%% ===== \end{flushleft}
%% ===== 
%% ===== 
%% ===== \begin{flushleft}
%% ===== contents are stored or how data bits must be assembled for loading into the register.
%% ===== \end{flushleft}
%% ===== 
%% ===== 
%% ===== \begin{flushleft}
%% ===== If the diagrams contain the characters {``}x'' or {``}0'', the values of the bits in the positions
%% ===== \end{flushleft}
%% ===== 
%% ===== 
%% ===== \begin{flushleft}
%% ===== shown are irrelevant to the register. Bits pictured as {``}x'' are not changed when the register is
%% ===== \end{flushleft}
%% ===== 
%% ===== 
%% ===== \begin{flushleft}
%% ===== stored. Bits pictured as {``}0'' are set to 0 when the register is stored. Neither {``}x'' bits or {``}0'' bits are
%% ===== \end{flushleft}
%% ===== 
%% ===== 
%% ===== \begin{flushleft}
%% ===== loaded into the register.
%% ===== \end{flushleft}
%% ===== 
%% ===== 
%% ===== 
%% ===== 
%% ===== 
%% ===== \begin{flushleft}

\subsection{ACCUMULATOR REGISTER (A)}

%% ===== \end{flushleft}
%% ===== 
%% ===== 
%% ===== \begin{flushleft}
%% ===== Format: - 36 bits
%% ===== \end{flushleft}
%% ===== 
%% ===== 
%% ===== 0
%% ===== 
%% ===== 
%% ===== 0
%% ===== 
%% ===== 
%% ===== 
%% ===== 
%% ===== 
%% ===== 1 1
%% ===== 
%% ===== 
%% ===== 7 8
%% ===== 
%% ===== 
%% ===== \begin{flushleft}
%% ===== A-Upper
%% ===== \end{flushleft}
%% ===== 
%% ===== 
%% ===== 
%% ===== 
%% ===== 
%% ===== 3
%% ===== 
%% ===== 
%% ===== 5
%% ===== 
%% ===== 
%% ===== \begin{flushleft}
%% ===== A-Lower
%% ===== \end{flushleft}
%% ===== 
%% ===== 
%% ===== 
%% ===== 
%% ===== 
%% ===== 18
%% ===== 
%% ===== 
%% ===== 
%% ===== 
%% ===== 
%% ===== 18
%% ===== 
%% ===== 
%% ===== 
%% ===== 
%% ===== 
%% ===== \begin{flushleft}
%% ===== Figure 3-1. Accumulator Register (A) Format
%% ===== \end{flushleft}
%% ===== 
%% ===== 
%% ===== \begin{flushleft}
%% ===== Description:
%% ===== \end{flushleft}
%% ===== 
%% ===== 
%% ===== \begin{flushleft}
%% ===== A 36-bit physical register located in the operations unit.
%% ===== \end{flushleft}
%% ===== 
%% ===== 
%% ===== \begin{flushleft}
%% ===== Function:
%% ===== \end{flushleft}
%% ===== 
%% ===== 
%% ===== \begin{flushleft}
%% ===== In fixed-point binary instructions, holds operands and results.
%% ===== \end{flushleft}
%% ===== 
%% ===== 
%% ===== \begin{flushleft}
%% ===== In floating-point binary instructions, holds the most significant part of the mantissa.
%% ===== \end{flushleft}
%% ===== 
%% ===== 
%% ===== \begin{flushleft}
%% ===== In shifting instructions, holds original data and shifted results.
%% ===== \end{flushleft}
%% ===== 
%% ===== 
%% ===== \begin{flushleft}
%% ===== In address preparation, may hold two logically independent word offsets, A-upper and Alower, or an extended range bit- or character-string length.
%% ===== \end{flushleft}
%% ===== 
%% ===== 
%% ===== 
%% ===== 
%% ===== 
%% ===== \begin{flushleft}

\subsection{QUOTIENT REGISTER (Q)}

%% ===== \end{flushleft}
%% ===== 
%% ===== 
%% ===== \begin{flushleft}
%% ===== Format: - 36 bits
%% ===== \end{flushleft}
%% ===== 
%% ===== 
%% ===== 0
%% ===== 
%% ===== 
%% ===== 0
%% ===== 
%% ===== 
%% ===== 
%% ===== 
%% ===== 
%% ===== 1 1
%% ===== 
%% ===== 
%% ===== 7 8
%% ===== 
%% ===== 
%% ===== \begin{flushleft}
%% ===== Q-Upper
%% ===== \end{flushleft}
%% ===== 
%% ===== 
%% ===== 
%% ===== 
%% ===== 
%% ===== 3
%% ===== 
%% ===== 
%% ===== 5
%% ===== 
%% ===== 
%% ===== \begin{flushleft}
%% ===== Q-Lower
%% ===== \end{flushleft}
%% ===== 
%% ===== 
%% ===== 
%% ===== 
%% ===== 
%% ===== 18
%% ===== 
%% ===== 
%% ===== 
%% ===== 
%% ===== 
%% ===== \begin{flushleft}
%% ===== Figure 3-2. Quotient Register (Q) Format
%% ===== \end{flushleft}
%% ===== 
%% ===== 
%% ===== \begin{flushleft}
%% ===== Description:
%% ===== \end{flushleft}
%% ===== 
%% ===== 
%% ===== \begin{flushleft}
%% ===== A 36-bit physical register located in the operations unit.
%% ===== \end{flushleft}
%% ===== 
%% ===== 
%% ===== 
%% ===== 
%% ===== 
%% ===== 18
%% ===== 
%% ===== 
%% ===== 
%% ===== 
%% ===== 
%% ===== \begin{flushleft}
%% ===== \newpage
%% ===== Function:
%% ===== \end{flushleft}
%% ===== 
%% ===== 
%% ===== \begin{flushleft}
%% ===== In fixed-point binary instructions, holds operands and results.
%% ===== \end{flushleft}
%% ===== 
%% ===== 
%% ===== \begin{flushleft}
%% ===== In floating-point binary instructions, holds the least significant part of the mantissa.
%% ===== \end{flushleft}
%% ===== 
%% ===== 
%% ===== \begin{flushleft}
%% ===== In shifting instructions, holds original data and shifted results.
%% ===== \end{flushleft}
%% ===== 
%% ===== 
%% ===== \begin{flushleft}
%% ===== In address preparation, may hold two logically independent word offsets, Q-upper and Qlower, or an extended range bit- or character-string length.
%% ===== \end{flushleft}
%% ===== 
%% ===== 
%% ===== 
%% ===== 
%% ===== 
%% ===== \begin{flushleft}

\subsection{ACCUMULATOR-QUOTIENT REGISTER (AQ)}

%% ===== \end{flushleft}
%% ===== 
%% ===== 
%% ===== \begin{flushleft}
%% ===== Format: - 72 bits
%% ===== \end{flushleft}
%% ===== 
%% ===== 
%% ===== 0
%% ===== 
%% ===== 
%% ===== 0
%% ===== 
%% ===== 
%% ===== 
%% ===== 
%% ===== 
%% ===== 3 3
%% ===== 
%% ===== 
%% ===== 5 6
%% ===== 
%% ===== 
%% ===== 
%% ===== 
%% ===== 
%% ===== 7
%% ===== 
%% ===== 
%% ===== 1
%% ===== 
%% ===== 
%% ===== 
%% ===== 
%% ===== 
%% ===== \begin{flushleft}
%% ===== A
%% ===== \end{flushleft}
%% ===== 
%% ===== 
%% ===== 
%% ===== 
%% ===== 
%% ===== \begin{flushleft}
%% ===== Q
%% ===== \end{flushleft}
%% ===== 
%% ===== 
%% ===== 36
%% ===== 
%% ===== 
%% ===== 
%% ===== 
%% ===== 
%% ===== \begin{flushleft}
%% ===== Even word
%% ===== \end{flushleft}
%% ===== 
%% ===== 
%% ===== 
%% ===== 
%% ===== 
%% ===== 36
%% ===== 
%% ===== 
%% ===== \begin{flushleft}
%% ===== Odd word
%% ===== \end{flushleft}
%% ===== 
%% ===== 
%% ===== 
%% ===== 
%% ===== 
%% ===== \begin{flushleft}
%% ===== Figure 3-3. Accumulator-Quotient Register (AQ) Format
%% ===== \end{flushleft}
%% ===== 
%% ===== 
%% ===== \begin{flushleft}
%% ===== Description:
%% ===== \end{flushleft}
%% ===== 
%% ===== 
%% ===== \begin{flushleft}
%% ===== A combination of the accumulator (A) and quotient (Q) registers.
%% ===== \end{flushleft}
%% ===== 
%% ===== 
%% ===== \begin{flushleft}
%% ===== Function:
%% ===== \end{flushleft}
%% ===== 
%% ===== 
%% ===== \begin{flushleft}
%% ===== In fixed-point binary instructions, holds double-precision operands and results.
%% ===== \end{flushleft}
%% ===== 
%% ===== 
%% ===== \begin{flushleft}
%% ===== In floating-point binary instructions, holds the mantissa.
%% ===== \end{flushleft}
%% ===== 
%% ===== 
%% ===== \begin{flushleft}
%% ===== In shifting instructions, holds original data and shifted results.
%% ===== \end{flushleft}
%% ===== 
%% ===== 
%% ===== 
%% ===== 
%% ===== 
%% ===== \begin{flushleft}

\subsection{EXPONENT REGISTER (E)}

%% ===== \end{flushleft}
%% ===== 
%% ===== 
%% ===== \begin{flushleft}
%% ===== Format: - 8 bits
%% ===== \end{flushleft}
%% ===== 
%% ===== 
%% ===== 0
%% ===== 
%% ===== 
%% ===== 0
%% ===== 
%% ===== 
%% ===== 
%% ===== 
%% ===== 
%% ===== 0 0
%% ===== 
%% ===== 
%% ===== 7 8
%% ===== 
%% ===== 
%% ===== \begin{flushleft}
%% ===== exponent
%% ===== \end{flushleft}
%% ===== 
%% ===== 
%% ===== 
%% ===== 
%% ===== 
%% ===== 3
%% ===== 
%% ===== 
%% ===== 5
%% ===== 
%% ===== 
%% ===== 
%% ===== 
%% ===== 
%% ===== 0 0 0 0 0 0 0 0 0 0 0 0 0 0 0 0 0 0 0 0 0 0 0 0 0 0 0 0
%% ===== 
%% ===== 
%% ===== 8
%% ===== 
%% ===== 
%% ===== 
%% ===== 
%% ===== 
%% ===== \begin{flushleft}
%% ===== Figure 3-4. Exponent Register (E) Format
%% ===== \end{flushleft}
%% ===== 
%% ===== 
%% ===== \begin{flushleft}
%% ===== Description:
%% ===== \end{flushleft}
%% ===== 
%% ===== 
%% ===== \begin{flushleft}
%% ===== An 8-bit physical register located in the operations unit.
%% ===== \end{flushleft}
%% ===== 
%% ===== 
%% ===== \begin{flushleft}
%% ===== Function:
%% ===== \end{flushleft}
%% ===== 
%% ===== 
%% ===== \begin{flushleft}
%% ===== In floating-point binary instructions, holds the exponent.
%% ===== \end{flushleft}
%% ===== 
%% ===== 
%% ===== 
%% ===== 
%% ===== 
%% ===== 28
%% ===== 
%% ===== 
%% ===== 
%% ===== 
%% ===== 
%% ===== \begin{flushleft}
%% ===== \newpage

\subsection{EXPONENT-ACCUMULATOR-QUOTIENT REGISTER (EAQ)}

%% ===== \end{flushleft}
%% ===== 
%% ===== 
%% ===== \begin{flushleft}
%% ===== Format: - 80 bits
%% ===== \end{flushleft}
%% ===== 
%% ===== 
%% ===== 0
%% ===== 
%% ===== 
%% ===== 0
%% ===== 
%% ===== 
%% ===== 
%% ===== 
%% ===== 
%% ===== 0 0
%% ===== 
%% ===== 
%% ===== 7 8
%% ===== 
%% ===== 
%% ===== 
%% ===== 
%% ===== 
%% ===== 7
%% ===== 
%% ===== 
%% ===== 1
%% ===== 
%% ===== 
%% ===== 
%% ===== 
%% ===== 
%% ===== \begin{flushleft}
%% ===== exponent
%% ===== \end{flushleft}
%% ===== 
%% ===== 
%% ===== 
%% ===== 
%% ===== 
%% ===== \begin{flushleft}
%% ===== mantissa
%% ===== \end{flushleft}
%% ===== 
%% ===== 
%% ===== 8
%% ===== 
%% ===== 
%% ===== 
%% ===== 
%% ===== 
%% ===== 64
%% ===== 
%% ===== 
%% ===== 
%% ===== 
%% ===== 
%% ===== \begin{flushleft}
%% ===== Figure 3-5. Exponent-Accumulator-Quotient Register (EAQ) Format
%% ===== \end{flushleft}
%% ===== 
%% ===== 
%% ===== \begin{flushleft}
%% ===== Description:
%% ===== \end{flushleft}
%% ===== 
%% ===== 
%% ===== \begin{flushleft}
%% ===== A combination of the exponent (E), accumulator (A), and quotient (Q) registers. Although
%% ===== \end{flushleft}
%% ===== 
%% ===== 
%% ===== \begin{flushleft}
%% ===== the combined register has a total of 80 bits, only 72 are involved in transfers to and from
%% ===== \end{flushleft}
%% ===== 
%% ===== 
%% ===== \begin{flushleft}
%% ===== main memory. The 8 low-order bits are discarded on store and zero-filled on load.
%% ===== \end{flushleft}
%% ===== 
%% ===== 
%% ===== \begin{flushleft}
%% ===== Function:
%% ===== \end{flushleft}
%% ===== 
%% ===== 
%% ===== \begin{flushleft}
%% ===== In floating-point binary instructions, holds operands and results.
%% ===== \end{flushleft}
%% ===== 
%% ===== 
%% ===== 
%% ===== 
%% ===== 
%% ===== \begin{flushleft}

\subsection{INDEX REGISTERS (Xn)}

%% ===== \end{flushleft}
%% ===== 
%% ===== 
%% ===== \begin{flushleft}
%% ===== Format: - 18 bits each
%% ===== \end{flushleft}
%% ===== 
%% ===== 
%% ===== 0
%% ===== 
%% ===== 
%% ===== 0
%% ===== 
%% ===== 
%% ===== 
%% ===== 
%% ===== 
%% ===== 1
%% ===== 
%% ===== 
%% ===== 7
%% ===== 
%% ===== 
%% ===== 
%% ===== 
%% ===== 
%% ===== 18
%% ===== 
%% ===== 
%% ===== 
%% ===== 
%% ===== 
%% ===== \begin{flushleft}
%% ===== Figure 3-6. Index Register (Xn) Format
%% ===== \end{flushleft}
%% ===== 
%% ===== 
%% ===== \begin{flushleft}
%% ===== Description:
%% ===== \end{flushleft}
%% ===== 
%% ===== 
%% ===== \begin{flushleft}
%% ===== Eight 18-bit physical registers in the operations unit numbered 0 through 7. Index register
%% ===== \end{flushleft}
%% ===== 
%% ===== 
%% ===== \begin{flushleft}
%% ===== data may occupy the position of either an upper or lower 18-bit half-word operand (see
%% ===== \end{flushleft}
%% ===== 
%% ===== 
%% ===== \begin{flushleft}
%% ===== Section 2).
%% ===== \end{flushleft}
%% ===== 
%% ===== 
%% ===== \begin{flushleft}
%% ===== Function:
%% ===== \end{flushleft}
%% ===== 
%% ===== 
%% ===== \begin{flushleft}
%% ===== In fixed-point binary instructions, hold half-word operands and results.
%% ===== \end{flushleft}
%% ===== 
%% ===== 
%% ===== \begin{flushleft}
%% ===== In address preparation, hold word offsets or extended range bit- or character-string lengths.
%% ===== \end{flushleft}
%% ===== 
%% ===== 
%% ===== 
%% ===== 
%% ===== 
%% ===== \begin{flushleft}
%% ===== \newpage

\subsection{INDICATOR REGISTER (IR)}

%% ===== \end{flushleft}
%% ===== 
%% ===== 
%% ===== \begin{flushleft}
%% ===== Format: - 14 bits
%% ===== \end{flushleft}
%% ===== 
%% ===== 
%% ===== 0
%% ===== 
%% ===== 
%% ===== 0
%% ===== 
%% ===== 
%% ===== 
%% ===== 
%% ===== 
%% ===== 1 1 1 2 2 2 2 2 2 2 2 2 2 3 3 3
%% ===== 
%% ===== 
%% ===== 7 8 9 0 1 2 3 4 5 6 7 8 9 0 1 2
%% ===== 
%% ===== 
%% ===== 
%% ===== 
%% ===== 
%% ===== \begin{flushleft}
%% ===== x x x x x x x x x x x x x x x x x x a b c d e f g h i
%% ===== \end{flushleft}
%% ===== 
%% ===== 
%% ===== 
%% ===== 
%% ===== 
%% ===== 3
%% ===== 
%% ===== 
%% ===== 5
%% ===== 
%% ===== 
%% ===== 
%% ===== 
%% ===== 
%% ===== \begin{flushleft}
%% ===== j k l m n o 0 0 0
%% ===== \end{flushleft}
%% ===== 
%% ===== 
%% ===== 
%% ===== 
%% ===== 
%% ===== 18 1 1 1 1 1 1 1 1 1 1 1 1 1 1 1
%% ===== 
%% ===== 
%% ===== 
%% ===== 
%% ===== 
%% ===== 3
%% ===== 
%% ===== 
%% ===== 
%% ===== 
%% ===== 
%% ===== \begin{flushleft}
%% ===== Figure 3-7. Indicator Register (IR) Format
%% ===== \end{flushleft}
%% ===== 
%% ===== 
%% ===== \begin{flushleft}
%% ===== Description:
%% ===== \end{flushleft}
%% ===== 
%% ===== 
%% ===== \begin{flushleft}
%% ===== An assemblage of 15 indicator flags from various units of the processor. The data occupies
%% ===== \end{flushleft}
%% ===== 
%% ===== 
%% ===== \begin{flushleft}
%% ===== the position of a lower 18-bit half word operand (see Section 2). When interpreted as data,
%% ===== \end{flushleft}
%% ===== 
%% ===== 
%% ===== \begin{flushleft}
%% ===== a bit value of 1 corresponds to the ON state of the indicator, a bit value of 0 corresponds to
%% ===== \end{flushleft}
%% ===== 
%% ===== 
%% ===== \begin{flushleft}
%% ===== the OFF state.
%% ===== \end{flushleft}
%% ===== 
%% ===== 
%% ===== \begin{flushleft}
%% ===== Function:
%% ===== \end{flushleft}
%% ===== 
%% ===== 
%% ===== \begin{flushleft}
%% ===== The functions of the individual indicator bits are given below. An {``}x'' in the column headed
%% ===== \end{flushleft}
%% ===== 
%% ===== 
%% ===== \begin{flushleft}
%% ===== {``}L'' indicates that the state of the indicator is not affected by instructions that load the IR.
%% ===== \end{flushleft}
%% ===== 
%% ===== 
%% ===== 
%% ===== 
%% ===== 
%% ===== \begin{flushleft}
%% ===== key L Indicator name Action
%% ===== \end{flushleft}
%% ===== 
%% ===== 
%% ===== \begin{flushleft}
%% ===== a
%% ===== \end{flushleft}
%% ===== 
%% ===== 
%% ===== 
%% ===== 
%% ===== 
%% ===== \begin{flushleft}
%% ===== Zero
%% ===== \end{flushleft}
%% ===== 
%% ===== 
%% ===== 
%% ===== 
%% ===== 
%% ===== \begin{flushleft}
%% ===== This indicator is set ON whenever the output of the main binary
%% ===== \end{flushleft}
%% ===== 
%% ===== 
%% ===== \begin{flushleft}
%% ===== adder consists entirely of zero bits for binary or shifting
%% ===== \end{flushleft}
%% ===== 
%% ===== 
%% ===== \begin{flushleft}
%% ===== instructions or the output of the decimal adder consists entirely
%% ===== \end{flushleft}
%% ===== 
%% ===== 
%% ===== \begin{flushleft}
%% ===== of zero digits for decimal instructions; otherwise, it is set OFF.
%% ===== \end{flushleft}
%% ===== 
%% ===== 
%% ===== 
%% ===== 
%% ===== 
%% ===== \begin{flushleft}
%% ===== b
%% ===== \end{flushleft}
%% ===== 
%% ===== 
%% ===== 
%% ===== 
%% ===== 
%% ===== \begin{flushleft}
%% ===== Negative
%% ===== \end{flushleft}
%% ===== 
%% ===== 
%% ===== 
%% ===== 
%% ===== 
%% ===== \begin{flushleft}
%% ===== This indicator is set ON whenever the output of bit 0 of the main
%% ===== \end{flushleft}
%% ===== 
%% ===== 
%% ===== \begin{flushleft}
%% ===== binary adder has value 1 for binary or shifting instructions or the
%% ===== \end{flushleft}
%% ===== 
%% ===== 
%% ===== \begin{flushleft}
%% ===== sign character of the result of a decimal instruction is the
%% ===== \end{flushleft}
%% ===== 
%% ===== 
%% ===== \begin{flushleft}
%% ===== negative sign character; otherwise, it is set OFF.
%% ===== \end{flushleft}
%% ===== 
%% ===== 
%% ===== 
%% ===== 
%% ===== 
%% ===== \begin{flushleft}
%% ===== c
%% ===== \end{flushleft}
%% ===== 
%% ===== 
%% ===== 
%% ===== 
%% ===== 
%% ===== \begin{flushleft}
%% ===== Carry
%% ===== \end{flushleft}
%% ===== 
%% ===== 
%% ===== 
%% ===== 
%% ===== 
%% ===== \begin{flushleft}
%% ===== This indicator is set ON for any of the following conditions;
%% ===== \end{flushleft}
%% ===== 
%% ===== 
%% ===== \begin{flushleft}
%% ===== otherwise, it is set OFF.
%% ===== \end{flushleft}
%% ===== 
%% ===== 
%% ===== 
%% ===== 
%% ===== 
%% ===== \begin{flushleft}
%% ===== d
%% ===== \end{flushleft}
%% ===== 
%% ===== 
%% ===== 
%% ===== 
%% ===== 
%% ===== \begin{flushleft}
%% ===== Overflow
%% ===== \end{flushleft}
%% ===== 
%% ===== 
%% ===== 
%% ===== 
%% ===== 
%% ===== (1)
%% ===== 
%% ===== 
%% ===== 
%% ===== 
%% ===== 
%% ===== \begin{flushleft}
%% ===== If a bit propagates leftward out of bit 0 of the main binary
%% ===== \end{flushleft}
%% ===== 
%% ===== 
%% ===== \begin{flushleft}
%% ===== adder for any binary or shifting instruction.
%% ===== \end{flushleft}
%% ===== 
%% ===== 
%% ===== 
%% ===== 
%% ===== 
%% ===== (2)
%% ===== 
%% ===== 
%% ===== 
%% ===== 
%% ===== 
%% ===== \begin{flushleft}
%% ===== If | value1 | $<$= | value2 | for a decimal numeric
%% ===== \end{flushleft}
%% ===== 
%% ===== 
%% ===== \begin{flushleft}
%% ===== comparison instruction.
%% ===== \end{flushleft}
%% ===== 
%% ===== 
%% ===== 
%% ===== 
%% ===== 
%% ===== (3)
%% ===== 
%% ===== 
%% ===== 
%% ===== 
%% ===== 
%% ===== \begin{flushleft}
%% ===== If char1 $<$= char2 for a decimal alphanumeric compare
%% ===== \end{flushleft}
%% ===== 
%% ===== 
%% ===== \begin{flushleft}
%% ===== instruction.
%% ===== \end{flushleft}
%% ===== 
%% ===== 
%% ===== 
%% ===== 
%% ===== 
%% ===== \begin{flushleft}
%% ===== This indicator is set ON if the arithmetic range of a register is
%% ===== \end{flushleft}
%% ===== 
%% ===== 
%% ===== \begin{flushleft}
%% ===== exceeded in a fixed-point binary instruction or if the target string
%% ===== \end{flushleft}
%% ===== 
%% ===== 
%% ===== \begin{flushleft}
%% ===== of a decimal numeric instruction is too small to hold the integer
%% ===== \end{flushleft}
%% ===== 
%% ===== 
%% ===== \begin{flushleft}
%% ===== part of the result. It remains ON until reset by the Transfer On
%% ===== \end{flushleft}
%% ===== 
%% ===== 
%% ===== \begin{flushleft}
%% ===== Overflow (tov) instruction or is reset by some other instruction
%% ===== \end{flushleft}
%% ===== 
%% ===== 
%% ===== \begin{flushleft}
%% ===== that loads the IR. The event that sets this indicator ON may also
%% ===== \end{flushleft}
%% ===== 
%% ===== 
%% ===== \begin{flushleft}
%% ===== cause an overflow fault. (See overflow mask indicator below.)
%% ===== \end{flushleft}
%% ===== 
%% ===== 
%% ===== 
%% ===== 
%% ===== 
%% ===== \begin{flushleft}
%% ===== \newpage
%% ===== key L Indicator name Action
%% ===== \end{flushleft}
%% ===== 
%% ===== 
%% ===== \begin{flushleft}
%% ===== e
%% ===== \end{flushleft}
%% ===== 
%% ===== 
%% ===== 
%% ===== 
%% ===== 
%% ===== \begin{flushleft}
%% ===== Exponent
%% ===== \end{flushleft}
%% ===== 
%% ===== 
%% ===== \begin{flushleft}
%% ===== overflow
%% ===== \end{flushleft}
%% ===== 
%% ===== 
%% ===== 
%% ===== 
%% ===== 
%% ===== \begin{flushleft}
%% ===== This indicator is set ON if the exponent of the result of a
%% ===== \end{flushleft}
%% ===== 
%% ===== 
%% ===== \begin{flushleft}
%% ===== floating-point binary or decimal numeric instruction is greater
%% ===== \end{flushleft}
%% ===== 
%% ===== 
%% ===== \begin{flushleft}
%% ===== than +127. It remains ON until reset by the Transfer On
%% ===== \end{flushleft}
%% ===== 
%% ===== 
%% ===== \begin{flushleft}
%% ===== Exponent Overflow (teo) instruction or is reset by some other
%% ===== \end{flushleft}
%% ===== 
%% ===== 
%% ===== \begin{flushleft}
%% ===== instruction that loads the IR. The event that sets this indicator
%% ===== \end{flushleft}
%% ===== 
%% ===== 
%% ===== \begin{flushleft}
%% ===== ON may also cause an overflow fault. (See overflow mask
%% ===== \end{flushleft}
%% ===== 
%% ===== 
%% ===== \begin{flushleft}
%% ===== indicator below.)
%% ===== \end{flushleft}
%% ===== 
%% ===== 
%% ===== 
%% ===== 
%% ===== 
%% ===== \begin{flushleft}
%% ===== f
%% ===== \end{flushleft}
%% ===== 
%% ===== 
%% ===== 
%% ===== 
%% ===== 
%% ===== \begin{flushleft}
%% ===== Exponent
%% ===== \end{flushleft}
%% ===== 
%% ===== 
%% ===== \begin{flushleft}
%% ===== underflow
%% ===== \end{flushleft}
%% ===== 
%% ===== 
%% ===== 
%% ===== 
%% ===== 
%% ===== \begin{flushleft}
%% ===== This indicator is set ON if the exponent of the result of a
%% ===== \end{flushleft}
%% ===== 
%% ===== 
%% ===== \begin{flushleft}
%% ===== floating-point binary or decimal numeric instruction is less than
%% ===== \end{flushleft}
%% ===== 
%% ===== 
%% ===== \begin{flushleft}
%% ===== -128. It remains ON until reset by the Transfer On Exponent
%% ===== \end{flushleft}
%% ===== 
%% ===== 
%% ===== \begin{flushleft}
%% ===== Underflow (teu) instruction or is reset by some other instruction
%% ===== \end{flushleft}
%% ===== 
%% ===== 
%% ===== \begin{flushleft}
%% ===== that loads the IR. The event that sets this indicator ON may also
%% ===== \end{flushleft}
%% ===== 
%% ===== 
%% ===== \begin{flushleft}
%% ===== cause an overflow fault. (See overflow mask indicator below.)
%% ===== \end{flushleft}
%% ===== 
%% ===== 
%% ===== 
%% ===== 
%% ===== 
%% ===== \begin{flushleft}
%% ===== g
%% ===== \end{flushleft}
%% ===== 
%% ===== 
%% ===== 
%% ===== 
%% ===== 
%% ===== \begin{flushleft}
%% ===== Overflow mask
%% ===== \end{flushleft}
%% ===== 
%% ===== 
%% ===== 
%% ===== 
%% ===== 
%% ===== \begin{flushleft}
%% ===== This indicator is set ON or OFF only by the instructions that load
%% ===== \end{flushleft}
%% ===== 
%% ===== 
%% ===== \begin{flushleft}
%% ===== the IR. When set ON, the IR inhibits the generation of the fault
%% ===== \end{flushleft}
%% ===== 
%% ===== 
%% ===== \begin{flushleft}
%% ===== for those events that normally cause an overflow fault. If the
%% ===== \end{flushleft}
%% ===== 
%% ===== 
%% ===== \begin{flushleft}
%% ===== overflow mask indicator is set OFF after occurrence of an
%% ===== \end{flushleft}
%% ===== 
%% ===== 
%% ===== \begin{flushleft}
%% ===== overflow event, an overflow fault does not occur even though the
%% ===== \end{flushleft}
%% ===== 
%% ===== 
%% ===== \begin{flushleft}
%% ===== indicator for that event is still set ON. The state of the overflow
%% ===== \end{flushleft}
%% ===== 
%% ===== 
%% ===== \begin{flushleft}
%% ===== mask indicator does not affect the setting, testing, or storing of
%% ===== \end{flushleft}
%% ===== 
%% ===== 
%% ===== \begin{flushleft}
%% ===== any other indicator.
%% ===== \end{flushleft}
%% ===== 
%% ===== 
%% ===== 
%% ===== 
%% ===== 
%% ===== \begin{flushleft}
%% ===== h
%% ===== \end{flushleft}
%% ===== 
%% ===== 
%% ===== 
%% ===== 
%% ===== 
%% ===== \begin{flushleft}
%% ===== Tally runout
%% ===== \end{flushleft}
%% ===== 
%% ===== 
%% ===== 
%% ===== 
%% ===== 
%% ===== \begin{flushleft}
%% ===== This indicator is set OFF at initialization of any tallying
%% ===== \end{flushleft}
%% ===== 
%% ===== 
%% ===== \begin{flushleft}
%% ===== operation, that is, any repeat instruction or any indirect then
%% ===== \end{flushleft}
%% ===== 
%% ===== 
%% ===== \begin{flushleft}
%% ===== tally address modification. It is then set ON for any of the
%% ===== \end{flushleft}
%% ===== 
%% ===== 
%% ===== \begin{flushleft}
%% ===== following conditions:
%% ===== \end{flushleft}
%% ===== 
%% ===== 
%% ===== (1)
%% ===== 
%% ===== 
%% ===== 
%% ===== 
%% ===== 
%% ===== \begin{flushleft}
%% ===== If any repeat instruction terminates because of tally
%% ===== \end{flushleft}
%% ===== 
%% ===== 
%% ===== \begin{flushleft}
%% ===== exhaust.
%% ===== \end{flushleft}
%% ===== 
%% ===== 
%% ===== 
%% ===== 
%% ===== 
%% ===== (2)
%% ===== 
%% ===== 
%% ===== 
%% ===== 
%% ===== 
%% ===== \begin{flushleft}
%% ===== If a Repeat Link (rpl) instruction terminates because of a
%% ===== \end{flushleft}
%% ===== 
%% ===== 
%% ===== \begin{flushleft}
%% ===== zero link address.
%% ===== \end{flushleft}
%% ===== 
%% ===== 
%% ===== 
%% ===== 
%% ===== 
%% ===== (3)
%% ===== 
%% ===== 
%% ===== 
%% ===== 
%% ===== 
%% ===== \begin{flushleft}
%% ===== If a tally exhaust is detected for an indirect then tally
%% ===== \end{flushleft}
%% ===== 
%% ===== 
%% ===== \begin{flushleft}
%% ===== modifier. The instruction is executed whether or not tally
%% ===== \end{flushleft}
%% ===== 
%% ===== 
%% ===== \begin{flushleft}
%% ===== exhaust occurs.
%% ===== \end{flushleft}
%% ===== 
%% ===== 
%% ===== 
%% ===== 
%% ===== 
%% ===== (4)
%% ===== 
%% ===== 
%% ===== 
%% ===== 
%% ===== 
%% ===== \begin{flushleft}
%% ===== If an EIS string scanning instruction reaches the end of the
%% ===== \end{flushleft}
%% ===== 
%% ===== 
%% ===== \begin{flushleft}
%% ===== string without finding a match condition.
%% ===== \end{flushleft}
%% ===== 
%% ===== 
%% ===== 
%% ===== 
%% ===== 
%% ===== \begin{flushleft}
%% ===== i
%% ===== \end{flushleft}
%% ===== 
%% ===== 
%% ===== 
%% ===== 
%% ===== 
%% ===== \begin{flushleft}
%% ===== Parity error
%% ===== \end{flushleft}
%% ===== 
%% ===== 
%% ===== 
%% ===== 
%% ===== 
%% ===== \begin{flushleft}
%% ===== This indicator is set ON whenever a system controller signals
%% ===== \end{flushleft}
%% ===== 
%% ===== 
%% ===== \begin{flushleft}
%% ===== illegal action with a parity error code or the processor detects an
%% ===== \end{flushleft}
%% ===== 
%% ===== 
%% ===== \begin{flushleft}
%% ===== internal parity error condition. The indicator is set OFF only by
%% ===== \end{flushleft}
%% ===== 
%% ===== 
%% ===== \begin{flushleft}
%% ===== instructions that load the IR.
%% ===== \end{flushleft}
%% ===== 
%% ===== 
%% ===== 
%% ===== 
%% ===== 
%% ===== \begin{flushleft}
%% ===== j
%% ===== \end{flushleft}
%% ===== 
%% ===== 
%% ===== 
%% ===== 
%% ===== 
%% ===== \begin{flushleft}
%% ===== Parity mask
%% ===== \end{flushleft}
%% ===== 
%% ===== 
%% ===== 
%% ===== 
%% ===== 
%% ===== \begin{flushleft}
%% ===== This indicator is set ON or OFF only by the instructions that load
%% ===== \end{flushleft}
%% ===== 
%% ===== 
%% ===== \begin{flushleft}
%% ===== the IR and is changed only when the processor is in privileged or
%% ===== \end{flushleft}
%% ===== 
%% ===== 
%% ===== \begin{flushleft}
%% ===== absolute mode. When it is set ON, the IR inhibits the generation
%% ===== \end{flushleft}
%% ===== 
%% ===== 
%% ===== \begin{flushleft}
%% ===== of the parity fault for all events that set the parity error
%% ===== \end{flushleft}
%% ===== 
%% ===== 
%% ===== \begin{flushleft}
%% ===== indicator. If the parity mask indicator is set OFF after the
%% ===== \end{flushleft}
%% ===== 
%% ===== 
%% ===== \begin{flushleft}
%% ===== occurrence of a parity error event, a parity fault does not occur
%% ===== \end{flushleft}
%% ===== 
%% ===== 
%% ===== \begin{flushleft}
%% ===== even though the parity error indicator may still be set ON. The
%% ===== \end{flushleft}
%% ===== 
%% ===== 
%% ===== \begin{flushleft}
%% ===== state of the parity mask indicator does not affect the loading,
%% ===== \end{flushleft}
%% ===== 
%% ===== 
%% ===== \begin{flushleft}
%% ===== testing, or storing of any other indicator.
%% ===== \end{flushleft}
%% ===== 
%% ===== 
%% ===== 
%% ===== 
%% ===== 
%% ===== \begin{flushleft}
%% ===== \newpage
%% ===== key L Indicator name Action
%% ===== \end{flushleft}
%% ===== 
%% ===== 
%% ===== \begin{flushleft}
%% ===== k
%% ===== \end{flushleft}
%% ===== 
%% ===== 
%% ===== 
%% ===== 
%% ===== 
%% ===== \begin{flushleft}
%% ===== l
%% ===== \end{flushleft}
%% ===== 
%% ===== 
%% ===== 
%% ===== 
%% ===== 
%% ===== \begin{flushleft}
%% ===== m
%% ===== \end{flushleft}
%% ===== 
%% ===== 
%% ===== 
%% ===== 
%% ===== 
%% ===== \begin{flushleft}
%% ===== n
%% ===== \end{flushleft}
%% ===== 
%% ===== 
%% ===== 
%% ===== 
%% ===== 
%% ===== \begin{flushleft}
%% ===== o
%% ===== \end{flushleft}
%% ===== 
%% ===== 
%% ===== 
%% ===== 
%% ===== 
%% ===== \begin{flushleft}
%% ===== x Not BAR mode
%% ===== \end{flushleft}
%% ===== 
%% ===== 
%% ===== 
%% ===== 
%% ===== 
%% ===== \begin{flushleft}
%% ===== This indicator is set OFF (placing the processor in BAR mode)
%% ===== \end{flushleft}
%% ===== 
%% ===== 
%% ===== \begin{flushleft}
%% ===== only by execution of the Transfer and Set Slave (tss) instruction
%% ===== \end{flushleft}
%% ===== 
%% ===== 
%% ===== \begin{flushleft}
%% ===== or by the operand data of the Restore Control Unit (rcu)
%% ===== \end{flushleft}
%% ===== 
%% ===== 
%% ===== \begin{flushleft}
%% ===== instruction and is changed only when the processor is in
%% ===== \end{flushleft}
%% ===== 
%% ===== 
%% ===== \begin{flushleft}
%% ===== privileged or absolute mode. It is set ON (taking the processor
%% ===== \end{flushleft}
%% ===== 
%% ===== 
%% ===== \begin{flushleft}
%% ===== out of BAR node) by the execution of any transfer instruction
%% ===== \end{flushleft}
%% ===== 
%% ===== 
%% ===== \begin{flushleft}
%% ===== other than tss during a fault or interrupt trap. (See Section 7.)
%% ===== \end{flushleft}
%% ===== 
%% ===== 
%% ===== \begin{flushleft}
%% ===== If a fault or interrupt trap occurs while in BAR node and the IR is
%% ===== \end{flushleft}
%% ===== 
%% ===== 
%% ===== \begin{flushleft}
%% ===== stored before any transfer occurs, then a Return (ret) or Restore
%% ===== \end{flushleft}
%% ===== 
%% ===== 
%% ===== \begin{flushleft}
%% ===== Control Unit (rcu) instruction that reloads the stored data will
%% ===== \end{flushleft}
%% ===== 
%% ===== 
%% ===== \begin{flushleft}
%% ===== return the processor to BAR mode.
%% ===== \end{flushleft}
%% ===== 
%% ===== 
%% ===== 
%% ===== 
%% ===== 
%% ===== \begin{flushleft}
%% ===== Truncation
%% ===== \end{flushleft}
%% ===== 
%% ===== 
%% ===== 
%% ===== 
%% ===== 
%% ===== \begin{flushleft}
%% ===== This indicator is set ON whenever the target string of a decimal
%% ===== \end{flushleft}
%% ===== 
%% ===== 
%% ===== \begin{flushleft}
%% ===== numeric instruction is too small to hold all the digits of the result
%% ===== \end{flushleft}
%% ===== 
%% ===== 
%% ===== \begin{flushleft}
%% ===== or the target string of an alphanumeric instruction is too small to
%% ===== \end{flushleft}
%% ===== 
%% ===== 
%% ===== \begin{flushleft}
%% ===== hold all the bits or characters to be stored. (Also see the
%% ===== \end{flushleft}
%% ===== 
%% ===== 
%% ===== \begin{flushleft}
%% ===== overflow indicator for decimal numeric instructions.) The event
%% ===== \end{flushleft}
%% ===== 
%% ===== 
%% ===== \begin{flushleft}
%% ===== that sets this indicator ON may also cause an overflow fault.
%% ===== \end{flushleft}
%% ===== 
%% ===== 
%% ===== \begin{flushleft}
%% ===== (See overflow mask indicator above.)
%% ===== \end{flushleft}
%% ===== 
%% ===== 
%% ===== 
%% ===== 
%% ===== 
%% ===== \begin{flushleft}
%% ===== Mid instruction
%% ===== \end{flushleft}
%% ===== 
%% ===== 
%% ===== \begin{flushleft}
%% ===== interrupt fault
%% ===== \end{flushleft}
%% ===== 
%% ===== 
%% ===== 
%% ===== 
%% ===== 
%% ===== \begin{flushleft}
%% ===== This indicator is set OFF at the start of execution of each
%% ===== \end{flushleft}
%% ===== 
%% ===== 
%% ===== \begin{flushleft}
%% ===== instruction and is set ON by the events described below. The
%% ===== \end{flushleft}
%% ===== 
%% ===== 
%% ===== \begin{flushleft}
%% ===== indicator has meaning only when determining the proper restart
%% ===== \end{flushleft}
%% ===== 
%% ===== 
%% ===== \begin{flushleft}
%% ===== sequence for the interrupted instruction. This indicator can be
%% ===== \end{flushleft}
%% ===== 
%% ===== 
%% ===== \begin{flushleft}
%% ===== set on:
%% ===== \end{flushleft}
%% ===== 
%% ===== 
%% ===== 
%% ===== 
%% ===== 
%% ===== \begin{flushleft}
%% ===== x Absolute mode
%% ===== \end{flushleft}
%% ===== 
%% ===== 
%% ===== 
%% ===== 
%% ===== 
%% ===== \begin{flushleft}
%% ===== Hex mode
%% ===== \end{flushleft}
%% ===== 
%% ===== 
%% ===== 
%% ===== 
%% ===== 
%% ===== (1)
%% ===== 
%% ===== 
%% ===== 
%% ===== 
%% ===== 
%% ===== \begin{flushleft}
%% ===== By any fault during execution of an EIS instruction;
%% ===== \end{flushleft}
%% ===== 
%% ===== 
%% ===== \begin{flushleft}
%% ===== however, the state is safe-stored in the Control Unit Data
%% ===== \end{flushleft}
%% ===== 
%% ===== 
%% ===== \begin{flushleft}
%% ===== only for access violation and directed faults.
%% ===== \end{flushleft}
%% ===== 
%% ===== 
%% ===== 
%% ===== 
%% ===== 
%% ===== (2)
%% ===== 
%% ===== 
%% ===== 
%% ===== 
%% ===== 
%% ===== \begin{flushleft}
%% ===== By an interrupt signal during execution of those EIS
%% ===== \end{flushleft}
%% ===== 
%% ===== 
%% ===== \begin{flushleft}
%% ===== instructions that allow very long operand strings.
%% ===== \end{flushleft}
%% ===== 
%% ===== 
%% ===== 
%% ===== 
%% ===== 
%% ===== (3)
%% ===== 
%% ===== 
%% ===== 
%% ===== 
%% ===== 
%% ===== \begin{flushleft}
%% ===== If the processor is in absolute or privileged mode, by the
%% ===== \end{flushleft}
%% ===== 
%% ===== 
%% ===== \begin{flushleft}
%% ===== execution of a Load Indicator Register (ldi), Return (ret),
%% ===== \end{flushleft}
%% ===== 
%% ===== 
%% ===== \begin{flushleft}
%% ===== or Restore Control Unit (rcu) instruction with bit 30 set to
%% ===== \end{flushleft}
%% ===== 
%% ===== 
%% ===== \begin{flushleft}
%% ===== 1 in the IR data.
%% ===== \end{flushleft}
%% ===== 
%% ===== 
%% ===== 
%% ===== 
%% ===== 
%% ===== \begin{flushleft}
%% ===== This indicator is set ON (placing the processor in absolute mode)
%% ===== \end{flushleft}
%% ===== 
%% ===== 
%% ===== \begin{flushleft}
%% ===== when the processor is initialized and by execution of an
%% ===== \end{flushleft}
%% ===== 
%% ===== 
%% ===== \begin{flushleft}
%% ===== nonappended transfer instruction during a fault or interrupt trap
%% ===== \end{flushleft}
%% ===== 
%% ===== 
%% ===== \begin{flushleft}
%% ===== and is set OFF (placing the processor in append mode) by any
%% ===== \end{flushleft}
%% ===== 
%% ===== 
%% ===== \begin{flushleft}
%% ===== execution of an appended transfer instruction. If the processor
%% ===== \end{flushleft}
%% ===== 
%% ===== 
%% ===== \begin{flushleft}
%% ===== is not in absolute mode when the fault or interrupt occurs and
%% ===== \end{flushleft}
%% ===== 
%% ===== 
%% ===== \begin{flushleft}
%% ===== the transfer instruction is Return (ret) or Restore Control Unit
%% ===== \end{flushleft}
%% ===== 
%% ===== 
%% ===== \begin{flushleft}
%% ===== (rcu) and the appropriate mode bit is properly set in the IR data,
%% ===== \end{flushleft}
%% ===== 
%% ===== 
%% ===== \begin{flushleft}
%% ===== the processor remains in its current mode.
%% ===== \end{flushleft}
%% ===== 
%% ===== 
%% ===== \begin{flushleft}
%% ===== When the hexadecimal permission indicator (bit 33 of the Mode
%% ===== \end{flushleft}
%% ===== 
%% ===== 
%% ===== \begin{flushleft}
%% ===== Register) is set on and this indicator is also on, then the
%% ===== \end{flushleft}
%% ===== 
%% ===== 
%% ===== \begin{flushleft}
%% ===== exponent of a floating point number has a power of 16 rather
%% ===== \end{flushleft}
%% ===== 
%% ===== 
%% ===== \begin{flushleft}
%% ===== than a power of two (binary floating point). The state of the hex
%% ===== \end{flushleft}
%% ===== 
%% ===== 
%% ===== \begin{flushleft}
%% ===== mode indicator can be changed by executing a Load Indicator
%% ===== \end{flushleft}
%% ===== 
%% ===== 
%% ===== \begin{flushleft}
%% ===== Register (ldi), Return (ret), or Restore Control Unit (rcu),
%% ===== \end{flushleft}
%% ===== 
%% ===== 
%% ===== \begin{flushleft}
%% ===== instruction with the desired state (1 or 0) set in bit 32 of the IR
%% ===== \end{flushleft}
%% ===== 
%% ===== 
%% ===== \begin{flushleft}
%% ===== data.
%% ===== \end{flushleft}
%% ===== 
%% ===== 
%% ===== \begin{flushleft}
%% ===== Hexadecimal mode is only available on DPS 8M
%% ===== \end{flushleft}
%% ===== 
%% ===== 
%% ===== \begin{flushleft}
%% ===== processors. Indicator Register bit 32 is set to a zero value on
%% ===== \end{flushleft}
%% ===== 
%% ===== 
%% ===== \begin{flushleft}
%% ===== DPS/L68 processors.
%% ===== \end{flushleft}
%% ===== 
%% ===== 
%% ===== 
%% ===== 
%% ===== 
%% ===== \begin{flushleft}
%% ===== \newpage

\subsection{BASE ADDRESS REGISTER (BAR)}

%% ===== \end{flushleft}
%% ===== 
%% ===== 
%% ===== \begin{flushleft}
%% ===== Format: - 18 bits
%% ===== \end{flushleft}
%% ===== 
%% ===== 
%% ===== 0
%% ===== 
%% ===== 
%% ===== 0
%% ===== 
%% ===== 
%% ===== 
%% ===== 
%% ===== 
%% ===== 0 0
%% ===== 
%% ===== 
%% ===== 8 9
%% ===== 
%% ===== 
%% ===== 
%% ===== 
%% ===== 
%% ===== 1 1
%% ===== 
%% ===== 
%% ===== 7 8
%% ===== 
%% ===== 
%% ===== 
%% ===== 
%% ===== 
%% ===== \begin{flushleft}
%% ===== BASE
%% ===== \end{flushleft}
%% ===== 
%% ===== 
%% ===== 
%% ===== 
%% ===== 
%% ===== \begin{flushleft}
%% ===== BOUND
%% ===== \end{flushleft}
%% ===== 
%% ===== 
%% ===== 9
%% ===== 
%% ===== 
%% ===== 
%% ===== 
%% ===== 
%% ===== 3
%% ===== 
%% ===== 
%% ===== 5
%% ===== 
%% ===== 
%% ===== 
%% ===== 
%% ===== 
%% ===== \begin{flushleft}
%% ===== x x x x x x x x x x x x x x x x x x
%% ===== \end{flushleft}
%% ===== 
%% ===== 
%% ===== 9
%% ===== 
%% ===== 
%% ===== 
%% ===== 
%% ===== 
%% ===== 18
%% ===== 
%% ===== 
%% ===== 
%% ===== 
%% ===== 
%% ===== \begin{flushleft}
%% ===== Figure 3-8. Base Address Register (BAR) Format
%% ===== \end{flushleft}
%% ===== 
%% ===== 
%% ===== \begin{flushleft}
%% ===== Description:
%% ===== \end{flushleft}
%% ===== 
%% ===== 
%% ===== \begin{flushleft}
%% ===== An 18-bit physical register in the control unit.
%% ===== \end{flushleft}
%% ===== 
%% ===== 
%% ===== \begin{flushleft}
%% ===== Function:
%% ===== \end{flushleft}
%% ===== 
%% ===== 
%% ===== \begin{flushleft}
%% ===== The Base Address Register provides automatic hardware Address relocation and Address
%% ===== \end{flushleft}
%% ===== 
%% ===== 
%% ===== \begin{flushleft}
%% ===== range limitation when the processor is in BAR mode.
%% ===== \end{flushleft}
%% ===== 
%% ===== 
%% ===== \begin{flushleft}
%% ===== BAR.BASE
%% ===== \end{flushleft}
%% ===== 
%% ===== 
%% ===== 
%% ===== 
%% ===== 
%% ===== \begin{flushleft}
%% ===== Contains the 9 high-order bits of an 18-bit address relocation constant.
%% ===== \end{flushleft}
%% ===== 
%% ===== 
%% ===== \begin{flushleft}
%% ===== The low-order bits are generated as zeros.
%% ===== \end{flushleft}
%% ===== 
%% ===== 
%% ===== 
%% ===== 
%% ===== 
%% ===== \begin{flushleft}
%% ===== BAR.BOUND
%% ===== \end{flushleft}
%% ===== 
%% ===== 
%% ===== 
%% ===== 
%% ===== 
%% ===== \begin{flushleft}
%% ===== Contains the 9 high-order bits of the unrelocated address limit. The loworder bits are generated as zeros. An attempt to access main memory
%% ===== \end{flushleft}
%% ===== 
%% ===== 
%% ===== \begin{flushleft}
%% ===== beyond this limit causes a store fault, out of bounds. A value of 0 is truly 0,
%% ===== \end{flushleft}
%% ===== 
%% ===== 
%% ===== \begin{flushleft}
%% ===== indicating a null memory range.
%% ===== \end{flushleft}
%% ===== 
%% ===== 
%% ===== 
%% ===== 
%% ===== 
%% ===== \begin{flushleft}

\subsection{TIMER REGISTER (TR)}

%% ===== \end{flushleft}
%% ===== 
%% ===== 
%% ===== \begin{flushleft}
%% ===== Format: - 27 bits
%% ===== \end{flushleft}
%% ===== 
%% ===== 
%% ===== 0
%% ===== 
%% ===== 
%% ===== 0
%% ===== 
%% ===== 
%% ===== 
%% ===== 
%% ===== 
%% ===== 2 2
%% ===== 
%% ===== 
%% ===== 6 7
%% ===== 
%% ===== 
%% ===== \begin{flushleft}
%% ===== Timer value
%% ===== \end{flushleft}
%% ===== 
%% ===== 
%% ===== 
%% ===== 
%% ===== 
%% ===== 3
%% ===== 
%% ===== 
%% ===== 5
%% ===== 
%% ===== 
%% ===== 
%% ===== 
%% ===== 
%% ===== 0 0 0 0 0 0 0 0 0
%% ===== 
%% ===== 
%% ===== 27
%% ===== 
%% ===== 
%% ===== 
%% ===== 
%% ===== 
%% ===== 9
%% ===== 
%% ===== 
%% ===== 
%% ===== 
%% ===== 
%% ===== \begin{flushleft}
%% ===== Figure 3-9. Timer Register (TR) Format
%% ===== \end{flushleft}
%% ===== 
%% ===== 
%% ===== \begin{flushleft}
%% ===== Description:
%% ===== \end{flushleft}
%% ===== 
%% ===== 
%% ===== \begin{flushleft}
%% ===== A 27-bit settable, free-running clock in the control unit. The value decrements at a rate of
%% ===== \end{flushleft}
%% ===== 
%% ===== 
%% ===== \begin{flushleft}
%% ===== 512 kHz. Its range is 1.953125 microseconds to approximately 4.37 minutes.
%% ===== \end{flushleft}
%% ===== 
%% ===== 
%% ===== \begin{flushleft}
%% ===== Function:
%% ===== \end{flushleft}
%% ===== 
%% ===== 
%% ===== \begin{flushleft}
%% ===== The TR may be loaded with any convenient value with the privileged Load Timer (ldt)
%% ===== \end{flushleft}
%% ===== 
%% ===== 
%% ===== \begin{flushleft}
%% ===== instruction. When the value next passes through zero, a timer runout fault is signalled. If
%% ===== \end{flushleft}
%% ===== 
%% ===== 
%% ===== \begin{flushleft}
%% ===== the processor is in normal or BAR mode with interrupts not inhibited or is stopped at an
%% ===== \end{flushleft}
%% ===== 
%% ===== 
%% ===== \begin{flushleft}
%% ===== uninhibited Delay Until Interrupt Signal (dis) instruction, the fault occurs immediately. If
%% ===== \end{flushleft}
%% ===== 
%% ===== 
%% ===== \begin{flushleft}
%% ===== the processor is in absolute or privileged mode or has interrupts inhibited, the fault is
%% ===== \end{flushleft}
%% ===== 
%% ===== 
%% ===== \begin{flushleft}
%% ===== delayed until the processor returns to uninhibited normal or BAR mode or stops at an
%% ===== \end{flushleft}
%% ===== 
%% ===== 
%% ===== \begin{flushleft}
%% ===== uninhibited Delay Until Interrupt Signal (dis) instruction.
%% ===== \end{flushleft}
%% ===== 
%% ===== 
%% ===== 
%% ===== 
%% ===== 
%% ===== \begin{flushleft}
%% ===== \newpage

\subsection{RING ALARM REGISTER (RALR)}

%% ===== \end{flushleft}
%% ===== 
%% ===== 
%% ===== \begin{flushleft}
%% ===== Format: - 3 bits
%% ===== \end{flushleft}
%% ===== 
%% ===== 
%% ===== 0
%% ===== 
%% ===== 
%% ===== 0
%% ===== 
%% ===== 
%% ===== 
%% ===== 
%% ===== 
%% ===== 3 3
%% ===== 
%% ===== 
%% ===== 2 3
%% ===== 
%% ===== 
%% ===== 
%% ===== 
%% ===== 
%% ===== 3
%% ===== 
%% ===== 
%% ===== 5
%% ===== 
%% ===== 
%% ===== 
%% ===== 
%% ===== 
%% ===== 0 0 0 0 0 0 0 0 0 0 0 0 0 0 0 0 0 0 0 0 0 0 0 0 0 0 0 0 0 0 0 0 0
%% ===== 
%% ===== 
%% ===== 
%% ===== 
%% ===== 
%% ===== \begin{flushleft}
%% ===== RALR
%% ===== \end{flushleft}
%% ===== 
%% ===== 
%% ===== 
%% ===== 
%% ===== 
%% ===== 33
%% ===== 
%% ===== 
%% ===== 
%% ===== 
%% ===== 
%% ===== 3
%% ===== 
%% ===== 
%% ===== 
%% ===== 
%% ===== 
%% ===== \begin{flushleft}
%% ===== Figure 3-10. Ring Alarm Register (RALR) Format
%% ===== \end{flushleft}
%% ===== 
%% ===== 
%% ===== \begin{flushleft}
%% ===== Description:
%% ===== \end{flushleft}
%% ===== 
%% ===== 
%% ===== \begin{flushleft}
%% ===== A 3-bit physical register in the appending unit.
%% ===== \end{flushleft}
%% ===== 
%% ===== 
%% ===== \begin{flushleft}
%% ===== Function:
%% ===== \end{flushleft}
%% ===== 
%% ===== 
%% ===== \begin{flushleft}
%% ===== If the RALR contains a value other than zero and the effective ring number (see TPR.TRR
%% ===== \end{flushleft}
%% ===== 
%% ===== 
%% ===== \begin{flushleft}
%% ===== below) is greater than or equal to the contents of the RALR and the instruction for which an
%% ===== \end{flushleft}
%% ===== 
%% ===== 
%% ===== \begin{flushleft}
%% ===== absolute main memory address is being prepared is a transfer instruction, an access
%% ===== \end{flushleft}
%% ===== 
%% ===== 
%% ===== \begin{flushleft}
%% ===== violation, ring alarm, fault occurs. Operating system software may use this register to
%% ===== \end{flushleft}
%% ===== 
%% ===== 
%% ===== \begin{flushleft}
%% ===== detect crossings from inner rings to outer rings.
%% ===== \end{flushleft}
%% ===== 
%% ===== 
%% ===== 
%% ===== 
%% ===== 
%% ===== \begin{flushleft}

\subsection{POINTER REGISTERS (PRn)}

%% ===== \end{flushleft}
%% ===== 
%% ===== 
%% ===== \begin{flushleft}
%% ===== Format: - 42 bits each
%% ===== \end{flushleft}
%% ===== 
%% ===== 
%% ===== \begin{flushleft}
%% ===== Even word of ITS pointer pair
%% ===== \end{flushleft}
%% ===== 
%% ===== 
%% ===== 0
%% ===== 
%% ===== 
%% ===== 0
%% ===== 
%% ===== 
%% ===== 
%% ===== 
%% ===== 
%% ===== 0 0
%% ===== 
%% ===== 
%% ===== 2 3
%% ===== 
%% ===== 
%% ===== 
%% ===== 
%% ===== 
%% ===== 0 0 0
%% ===== 
%% ===== 
%% ===== 
%% ===== 
%% ===== 
%% ===== 1 1
%% ===== 
%% ===== 
%% ===== 7 8
%% ===== 
%% ===== 
%% ===== \begin{flushleft}
%% ===== SNR
%% ===== \end{flushleft}
%% ===== 
%% ===== 
%% ===== 
%% ===== 
%% ===== 
%% ===== 3
%% ===== 
%% ===== 
%% ===== 
%% ===== 
%% ===== 
%% ===== 2 2
%% ===== 
%% ===== 
%% ===== 0 1
%% ===== 
%% ===== 
%% ===== 
%% ===== 
%% ===== 
%% ===== \begin{flushleft}
%% ===== RNR
%% ===== \end{flushleft}
%% ===== 
%% ===== 
%% ===== 15
%% ===== 
%% ===== 
%% ===== 
%% ===== 
%% ===== 
%% ===== 2 3
%% ===== 
%% ===== 
%% ===== 9 0
%% ===== 
%% ===== 
%% ===== 
%% ===== 
%% ===== 
%% ===== 0 0 0 0 0 0 0 0 0
%% ===== 
%% ===== 
%% ===== 
%% ===== 
%% ===== 
%% ===== 3
%% ===== 
%% ===== 
%% ===== 
%% ===== 
%% ===== 
%% ===== 3
%% ===== 
%% ===== 
%% ===== 5
%% ===== 
%% ===== 
%% ===== (43)8
%% ===== 
%% ===== 
%% ===== 
%% ===== 
%% ===== 
%% ===== 9
%% ===== 
%% ===== 
%% ===== 
%% ===== 
%% ===== 
%% ===== 6
%% ===== 
%% ===== 
%% ===== 
%% ===== 
%% ===== 
%% ===== 6 6
%% ===== 
%% ===== 
%% ===== 5 6
%% ===== 
%% ===== 
%% ===== 
%% ===== 
%% ===== 
%% ===== 7
%% ===== 
%% ===== 
%% ===== 1
%% ===== 
%% ===== 
%% ===== 
%% ===== 
%% ===== 
%% ===== \begin{flushleft}
%% ===== Odd word of ITS pointer pair
%% ===== \end{flushleft}
%% ===== 
%% ===== 
%% ===== 3
%% ===== 
%% ===== 
%% ===== 6
%% ===== 
%% ===== 
%% ===== 
%% ===== 
%% ===== 
%% ===== 5 5
%% ===== 
%% ===== 
%% ===== 3 4
%% ===== 
%% ===== 
%% ===== \begin{flushleft}
%% ===== WORDNO
%% ===== \end{flushleft}
%% ===== 
%% ===== 
%% ===== 
%% ===== 
%% ===== 
%% ===== 5 5
%% ===== 
%% ===== 
%% ===== 6 7
%% ===== 
%% ===== 
%% ===== 
%% ===== 
%% ===== 
%% ===== 0 0 0
%% ===== 
%% ===== 
%% ===== 18
%% ===== 
%% ===== 
%% ===== 
%% ===== 
%% ===== 
%% ===== 3
%% ===== 
%% ===== 
%% ===== 
%% ===== 
%% ===== 
%% ===== 6 6
%% ===== 
%% ===== 
%% ===== 2 3
%% ===== 
%% ===== 
%% ===== \begin{flushleft}
%% ===== BITNO
%% ===== \end{flushleft}
%% ===== 
%% ===== 
%% ===== 
%% ===== 
%% ===== 
%% ===== 0 0 0
%% ===== 
%% ===== 
%% ===== 6
%% ===== 
%% ===== 
%% ===== 
%% ===== 
%% ===== 
%% ===== 3
%% ===== 
%% ===== 
%% ===== 
%% ===== 
%% ===== 
%% ===== \begin{flushleft}
%% ===== (TAG)
%% ===== \end{flushleft}
%% ===== 
%% ===== 
%% ===== 6
%% ===== 
%% ===== 
%% ===== 
%% ===== 
%% ===== 
%% ===== \begin{flushleft}
%% ===== \newpage
%% ===== Data as stored by Store Pointer Register n Packed (sprpn)
%% ===== \end{flushleft}
%% ===== 
%% ===== 
%% ===== 0
%% ===== 
%% ===== 
%% ===== 0
%% ===== 
%% ===== 
%% ===== 
%% ===== 
%% ===== 
%% ===== 0 0
%% ===== 
%% ===== 
%% ===== 5 6
%% ===== 
%% ===== 
%% ===== 
%% ===== 
%% ===== 
%% ===== 1 1
%% ===== 
%% ===== 
%% ===== 7 8
%% ===== 
%% ===== 
%% ===== 
%% ===== 
%% ===== 
%% ===== \begin{flushleft}
%% ===== BITNO
%% ===== \end{flushleft}
%% ===== 
%% ===== 
%% ===== 
%% ===== 
%% ===== 
%% ===== 3
%% ===== 
%% ===== 
%% ===== 5
%% ===== 
%% ===== 
%% ===== 
%% ===== 
%% ===== 
%% ===== \begin{flushleft}
%% ===== SNR
%% ===== \end{flushleft}
%% ===== 
%% ===== 
%% ===== 6
%% ===== 
%% ===== 
%% ===== 
%% ===== 
%% ===== 
%% ===== \begin{flushleft}
%% ===== WORDNO
%% ===== \end{flushleft}
%% ===== 
%% ===== 
%% ===== 12
%% ===== 
%% ===== 
%% ===== 
%% ===== 
%% ===== 
%% ===== 18
%% ===== 
%% ===== 
%% ===== 
%% ===== 
%% ===== 
%% ===== \begin{flushleft}
%% ===== Figure 3-11. Pointer Register (PRn) Format
%% ===== \end{flushleft}
%% ===== 
%% ===== 
%% ===== \begin{flushleft}
%% ===== Description:
%% ===== \end{flushleft}
%% ===== 
%% ===== 
%% ===== \begin{flushleft}
%% ===== Eight combinations of physical registers from the appending unit and decimal unit
%% ===== \end{flushleft}
%% ===== 
%% ===== 
%% ===== \begin{flushleft}
%% ===== numbered 0 through 7. PRn.RNR, PRn.SNR, and PRn.BITNO are located in the appending
%% ===== \end{flushleft}
%% ===== 
%% ===== 
%% ===== \begin{flushleft}
%% ===== unit and PRn.WORDNO is located in the decimal unit. The WORDNO registers also form
%% ===== \end{flushleft}
%% ===== 
%% ===== 
%% ===== \begin{flushleft}
%% ===== part of the address registers discussed later in this section.
%% ===== \end{flushleft}
%% ===== 
%% ===== 
%% ===== \begin{flushleft}
%% ===== Function:
%% ===== \end{flushleft}
%% ===== 
%% ===== 
%% ===== \begin{flushleft}
%% ===== The pointer registers hold information relative to the location in main memory of data items
%% ===== \end{flushleft}
%% ===== 
%% ===== 
%% ===== \begin{flushleft}
%% ===== that may be external to the segment containing the procedure being executed. The
%% ===== \end{flushleft}
%% ===== 
%% ===== 
%% ===== \begin{flushleft}
%% ===== functions of the individual constituent registers are:
%% ===== \end{flushleft}
%% ===== 
%% ===== 
%% ===== 
%% ===== 
%% ===== 
%% ===== \begin{flushleft}
%% ===== Register
%% ===== \end{flushleft}
%% ===== 
%% ===== 
%% ===== 
%% ===== 
%% ===== 
%% ===== \begin{flushleft}
%% ===== Function
%% ===== \end{flushleft}
%% ===== 
%% ===== 
%% ===== 
%% ===== 
%% ===== 
%% ===== \begin{flushleft}
%% ===== PRn.SNR
%% ===== \end{flushleft}
%% ===== 
%% ===== 
%% ===== 
%% ===== 
%% ===== 
%% ===== \begin{flushleft}
%% ===== The segment number of the segment containing the data item described
%% ===== \end{flushleft}
%% ===== 
%% ===== 
%% ===== \begin{flushleft}
%% ===== by the pointer register.
%% ===== \end{flushleft}
%% ===== 
%% ===== 
%% ===== 
%% ===== 
%% ===== 
%% ===== \begin{flushleft}
%% ===== PRn.RNR
%% ===== \end{flushleft}
%% ===== 
%% ===== 
%% ===== 
%% ===== 
%% ===== 
%% ===== \begin{flushleft}
%% ===== The final effective ring number value calculated during execution of the
%% ===== \end{flushleft}
%% ===== 
%% ===== 
%% ===== \begin{flushleft}
%% ===== instruction that last loaded the PR.
%% ===== \end{flushleft}
%% ===== 
%% ===== 
%% ===== 
%% ===== 
%% ===== 
%% ===== (43)8
%% ===== 
%% ===== 
%% ===== 
%% ===== 
%% ===== 
%% ===== \begin{flushleft}
%% ===== This field is not part of the PR but is generated each time the PR is stored
%% ===== \end{flushleft}
%% ===== 
%% ===== 
%% ===== \begin{flushleft}
%% ===== as an ITS pair.
%% ===== \end{flushleft}
%% ===== 
%% ===== 
%% ===== 
%% ===== 
%% ===== 
%% ===== \begin{flushleft}
%% ===== PRn.WORDNO
%% ===== \end{flushleft}
%% ===== 
%% ===== 
%% ===== 
%% ===== 
%% ===== 
%% ===== \begin{flushleft}
%% ===== The offset in words from the base or origin of the segment to the data
%% ===== \end{flushleft}
%% ===== 
%% ===== 
%% ===== \begin{flushleft}
%% ===== item.
%% ===== \end{flushleft}
%% ===== 
%% ===== 
%% ===== 
%% ===== 
%% ===== 
%% ===== \begin{flushleft}
%% ===== PRn.BITNO
%% ===== \end{flushleft}
%% ===== 
%% ===== 
%% ===== 
%% ===== 
%% ===== 
%% ===== \begin{flushleft}
%% ===== The number of the bit within PRn.WORDNO that is the first bit of the data
%% ===== \end{flushleft}
%% ===== 
%% ===== 
%% ===== \begin{flushleft}
%% ===== item. Data items aligned on word boundaries always have the value 0.
%% ===== \end{flushleft}
%% ===== 
%% ===== 
%% ===== \begin{flushleft}
%% ===== Unaligned data items may have any value in the range [1,35].
%% ===== \end{flushleft}
%% ===== 
%% ===== 
%% ===== 
%% ===== 
%% ===== 
%% ===== \begin{flushleft}
%% ===== (TAG)
%% ===== \end{flushleft}
%% ===== 
%% ===== 
%% ===== 
%% ===== 
%% ===== 
%% ===== \begin{flushleft}
%% ===== This field is not part of the PR but, in an ITS pointer pair, holds an
%% ===== \end{flushleft}
%% ===== 
%% ===== 
%% ===== \begin{flushleft}
%% ===== address modifier for use in address preparation.
%% ===== \end{flushleft}
%% ===== 
%% ===== 
%% ===== 
%% ===== 
%% ===== 
%% ===== \begin{flushleft}

\subsection{ADDRESS REGISTERS (ARn)}

%% ===== \end{flushleft}
%% ===== 
%% ===== 
%% ===== \begin{flushleft}
%% ===== Format: - 24 bits each
%% ===== \end{flushleft}
%% ===== 
%% ===== 
%% ===== \begin{flushleft}
%% ===== Data as stored by Store Address Register n (sarn)
%% ===== \end{flushleft}
%% ===== 
%% ===== 
%% ===== 0
%% ===== 
%% ===== 
%% ===== 0
%% ===== 
%% ===== 
%% ===== 
%% ===== 
%% ===== 
%% ===== 1 1 1 2
%% ===== 
%% ===== 
%% ===== 7 8 9 0
%% ===== 
%% ===== 
%% ===== \begin{flushleft}
%% ===== WORDNO
%% ===== \end{flushleft}
%% ===== 
%% ===== 
%% ===== 
%% ===== 
%% ===== 
%% ===== \begin{flushleft}
%% ===== a
%% ===== \end{flushleft}
%% ===== 
%% ===== 
%% ===== 18
%% ===== 
%% ===== 
%% ===== 
%% ===== 
%% ===== 
%% ===== 2 2
%% ===== 
%% ===== 
%% ===== 3 4
%% ===== 
%% ===== 
%% ===== \begin{flushleft}
%% ===== BITNO
%% ===== \end{flushleft}
%% ===== 
%% ===== 
%% ===== 
%% ===== 
%% ===== 
%% ===== 2
%% ===== 
%% ===== 
%% ===== 
%% ===== 
%% ===== 
%% ===== 3
%% ===== 
%% ===== 
%% ===== 5
%% ===== 
%% ===== 
%% ===== 
%% ===== 
%% ===== 
%% ===== 0 0 0 0 0 0 0 0 0 0 0 0
%% ===== 
%% ===== 
%% ===== 4
%% ===== 
%% ===== 
%% ===== 
%% ===== 
%% ===== 
%% ===== \begin{flushleft}
%% ===== Figure 3-12. Address Register (ARn) Format
%% ===== \end{flushleft}
%% ===== 
%% ===== 
%% ===== 
%% ===== 
%% ===== 
%% ===== 12
%% ===== 
%% ===== 
%% ===== 
%% ===== 
%% ===== 
%% ===== \begin{flushleft}
%% ===== \newpage
%% ===== Description:
%% ===== \end{flushleft}
%% ===== 
%% ===== 
%% ===== \begin{flushleft}
%% ===== Eight combinations of physical registers from the decimal unit numbered 0 through 7. The
%% ===== \end{flushleft}
%% ===== 
%% ===== 
%% ===== \begin{flushleft}
%% ===== WORDNO registers also form part of the pointer registers discussed earlier in this section.
%% ===== \end{flushleft}
%% ===== 
%% ===== 
%% ===== \begin{flushleft}
%% ===== Function:
%% ===== \end{flushleft}
%% ===== 
%% ===== 
%% ===== \begin{flushleft}
%% ===== The address registers hold information relative to the location in main memory of the next
%% ===== \end{flushleft}
%% ===== 
%% ===== 
%% ===== \begin{flushleft}
%% ===== bit, character, or byte of an EIS operand to be processed by an EIS instruction. The
%% ===== \end{flushleft}
%% ===== 
%% ===== 
%% ===== \begin{flushleft}
%% ===== functions of the individual constituent registers are:
%% ===== \end{flushleft}
%% ===== 
%% ===== 
%% ===== 
%% ===== 
%% ===== 
%% ===== \begin{flushleft}
%% ===== key Register
%% ===== \end{flushleft}
%% ===== 
%% ===== 
%% ===== 
%% ===== 
%% ===== 
%% ===== \begin{flushleft}
%% ===== a
%% ===== \end{flushleft}
%% ===== 
%% ===== 
%% ===== 
%% ===== 
%% ===== 
%% ===== \begin{flushleft}
%% ===== NOTE:
%% ===== \end{flushleft}
%% ===== 
%% ===== 
%% ===== 
%% ===== 
%% ===== 
%% ===== \begin{flushleft}
%% ===== Function
%% ===== \end{flushleft}
%% ===== 
%% ===== 
%% ===== 
%% ===== 
%% ===== 
%% ===== \begin{flushleft}
%% ===== ARn.WORDNO
%% ===== \end{flushleft}
%% ===== 
%% ===== 
%% ===== 
%% ===== 
%% ===== 
%% ===== \begin{flushleft}
%% ===== The offset in words relative to the current addressing base referent
%% ===== \end{flushleft}
%% ===== 
%% ===== 
%% ===== \begin{flushleft}
%% ===== (segment origin, BAR.BASE, or absolute 0 depending on addressing
%% ===== \end{flushleft}
%% ===== 
%% ===== 
%% ===== \begin{flushleft}
%% ===== mode) to the word containing the next data item element.
%% ===== \end{flushleft}
%% ===== 
%% ===== 
%% ===== 
%% ===== 
%% ===== 
%% ===== \begin{flushleft}
%% ===== ARn.CHAR
%% ===== \end{flushleft}
%% ===== 
%% ===== 
%% ===== 
%% ===== 
%% ===== 
%% ===== \begin{flushleft}
%% ===== The number of the 9-bit byte within ARn.WORDNO containing the
%% ===== \end{flushleft}
%% ===== 
%% ===== 
%% ===== \begin{flushleft}
%% ===== first bit of the next data item element.
%% ===== \end{flushleft}
%% ===== 
%% ===== 
%% ===== 
%% ===== 
%% ===== 
%% ===== \begin{flushleft}
%% ===== ARn.BITNO
%% ===== \end{flushleft}
%% ===== 
%% ===== 
%% ===== 
%% ===== 
%% ===== 
%% ===== \begin{flushleft}
%% ===== The number of the bit within ARn.CHAR that is the first bit of the
%% ===== \end{flushleft}
%% ===== 
%% ===== 
%% ===== \begin{flushleft}
%% ===== next data item element.
%% ===== \end{flushleft}
%% ===== 
%% ===== 
%% ===== 
%% ===== 
%% ===== 
%% ===== \begin{flushleft}
%% ===== The reader's attention is directed to the presence of two bit number registers,
%% ===== \end{flushleft}
%% ===== 
%% ===== 
%% ===== \begin{flushleft}
%% ===== PRn.BITNO and ARn.BITNO. Because the Multics processor was implemented as an
%% ===== \end{flushleft}
%% ===== 
%% ===== 
%% ===== \begin{flushleft}
%% ===== enhancement to an existing design, certain apparent anomalies appear. One of these is
%% ===== \end{flushleft}
%% ===== 
%% ===== 
%% ===== \begin{flushleft}
%% ===== the difference in the handling of unaligned data items by the appending unit and decimal
%% ===== \end{flushleft}
%% ===== 
%% ===== 
%% ===== \begin{flushleft}
%% ===== unit. The decimal unit handles all unaligned data items with a 9-bit byte number and bit
%% ===== \end{flushleft}
%% ===== 
%% ===== 
%% ===== \begin{flushleft}
%% ===== offset within the byte. Conversion from the description given in the EIS operand
%% ===== \end{flushleft}
%% ===== 
%% ===== 
%% ===== \begin{flushleft}
%% ===== descriptor is done automatically by the hardware. The appending unit maintains
%% ===== \end{flushleft}
%% ===== 
%% ===== 
%% ===== \begin{flushleft}
%% ===== compatibility with the earlier generation Multics processor by handling all unaligned
%% ===== \end{flushleft}
%% ===== 
%% ===== 
%% ===== \begin{flushleft}
%% ===== data items with a bit offset from the prior word boundary; again with any necessary
%% ===== \end{flushleft}
%% ===== 
%% ===== 
%% ===== \begin{flushleft}
%% ===== conversion done automatically by the hardware. Thus, a pointer register, PRn, may be
%% ===== \end{flushleft}
%% ===== 
%% ===== 
%% ===== \begin{flushleft}
%% ===== loaded from an ITS pointer pair having a pure bit offset and modified by one of the EIS
%% ===== \end{flushleft}
%% ===== 
%% ===== 
%% ===== \begin{flushleft}
%% ===== address register instructions (a4bd, s9bd, etc.) using character displacement counts.
%% ===== \end{flushleft}
%% ===== 
%% ===== 
%% ===== \begin{flushleft}
%% ===== The automatic conversion performed ensures that the pointer register, PR i, and its
%% ===== \end{flushleft}
%% ===== 
%% ===== 
%% ===== \begin{flushleft}
%% ===== matching address register, ARi, both describe the same physical bit in main memory.
%% ===== \end{flushleft}
%% ===== 
%% ===== 
%% ===== 
%% ===== 
%% ===== 
%% ===== \begin{flushleft}
%% ===== SPECIAL NOTICE: The decimal unit has built-in hardware checks for illegal bit offset values but
%% ===== \end{flushleft}
%% ===== 
%% ===== 
%% ===== \begin{flushleft}
%% ===== the appending unit does not except for a single case for packed pointers. See NOTES for
%% ===== \end{flushleft}
%% ===== 
%% ===== 
%% ===== \begin{flushleft}
%% ===== Load Packed Pointers (lprpn) in Section 4.
%% ===== \end{flushleft}
%% ===== 
%% ===== 
%% ===== 
%% ===== 
%% ===== 
%% ===== \begin{flushleft}

\subsection{PROCEDURE POINTER REGISTER (PPR)}

%% ===== \end{flushleft}
%% ===== 
%% ===== 
%% ===== \begin{flushleft}
%% ===== Format: - 37 bits
%% ===== \end{flushleft}
%% ===== 
%% ===== 
%% ===== \begin{flushleft}
%% ===== Shown as part of word 0 of control unit data
%% ===== \end{flushleft}
%% ===== 
%% ===== 
%% ===== 0
%% ===== 
%% ===== 
%% ===== 0
%% ===== 
%% ===== 
%% ===== 
%% ===== 
%% ===== 
%% ===== 0 0
%% ===== 
%% ===== 
%% ===== 2 3
%% ===== 
%% ===== 
%% ===== 
%% ===== 
%% ===== 
%% ===== \begin{flushleft}
%% ===== PRR
%% ===== \end{flushleft}
%% ===== 
%% ===== 
%% ===== 
%% ===== 
%% ===== 
%% ===== 1 1
%% ===== 
%% ===== 
%% ===== 7 8
%% ===== 
%% ===== 
%% ===== \begin{flushleft}
%% ===== PSR
%% ===== \end{flushleft}
%% ===== 
%% ===== 
%% ===== 
%% ===== 
%% ===== 
%% ===== 3
%% ===== 
%% ===== 
%% ===== 
%% ===== 
%% ===== 
%% ===== \begin{flushleft}
%% ===== P
%% ===== \end{flushleft}
%% ===== 
%% ===== 
%% ===== 15 1
%% ===== 
%% ===== 
%% ===== 
%% ===== 
%% ===== 
%% ===== \begin{flushleft}
%% ===== Other control unit data
%% ===== \end{flushleft}
%% ===== 
%% ===== 
%% ===== 
%% ===== 
%% ===== 
%% ===== \begin{flushleft}
%% ===== \newpage
%% ===== Shown as part of word 4 of control unit data
%% ===== \end{flushleft}
%% ===== 
%% ===== 
%% ===== 0
%% ===== 
%% ===== 
%% ===== 0
%% ===== 
%% ===== 
%% ===== 
%% ===== 
%% ===== 
%% ===== 1
%% ===== 
%% ===== 
%% ===== 7
%% ===== 
%% ===== 
%% ===== \begin{flushleft}
%% ===== Other control unit data
%% ===== \end{flushleft}
%% ===== 
%% ===== 
%% ===== 
%% ===== 
%% ===== 
%% ===== \begin{flushleft}
%% ===== IC
%% ===== \end{flushleft}
%% ===== 
%% ===== 
%% ===== 18
%% ===== 
%% ===== 
%% ===== 
%% ===== 
%% ===== 
%% ===== \begin{flushleft}
%% ===== Figure 3-13. Procedure Pointer Register (PPR) Format
%% ===== \end{flushleft}
%% ===== 
%% ===== 
%% ===== \begin{flushleft}
%% ===== Description:
%% ===== \end{flushleft}
%% ===== 
%% ===== 
%% ===== \begin{flushleft}
%% ===== A combination of physical registers from the appending unit and the control unit. PPR.PRR,
%% ===== \end{flushleft}
%% ===== 
%% ===== 
%% ===== \begin{flushleft}
%% ===== PPR.PSR, and PPR.P are located in the appending unit and PPR.IC is located in the control
%% ===== \end{flushleft}
%% ===== 
%% ===== 
%% ===== \begin{flushleft}
%% ===== unit. The PPR is not explicitly addressable but its data is extracted and stored as part of the
%% ===== \end{flushleft}
%% ===== 
%% ===== 
%% ===== \begin{flushleft}
%% ===== data stored with the Store Control Unit (scu) and Store Control Double (stcd) instructions.
%% ===== \end{flushleft}
%% ===== 
%% ===== 
%% ===== \begin{flushleft}
%% ===== It is loaded from the control unit data with the Restore Control Unit (rcu) instruction.
%% ===== \end{flushleft}
%% ===== 
%% ===== 
%% ===== \begin{flushleft}
%% ===== Function:
%% ===== \end{flushleft}
%% ===== 
%% ===== 
%% ===== \begin{flushleft}
%% ===== The Procedure Pointer Register holds information relative to the location in main memory
%% ===== \end{flushleft}
%% ===== 
%% ===== 
%% ===== \begin{flushleft}
%% ===== of the procedure segment in execution and the location of the current instruction within
%% ===== \end{flushleft}
%% ===== 
%% ===== 
%% ===== \begin{flushleft}
%% ===== that segment. The functions of the individual constituent registers are:
%% ===== \end{flushleft}
%% ===== 
%% ===== 
%% ===== 
%% ===== 
%% ===== 
%% ===== \begin{flushleft}
%% ===== Register
%% ===== \end{flushleft}
%% ===== 
%% ===== 
%% ===== 
%% ===== 
%% ===== 
%% ===== \begin{flushleft}
%% ===== Function
%% ===== \end{flushleft}
%% ===== 
%% ===== 
%% ===== 
%% ===== 
%% ===== 
%% ===== \begin{flushleft}
%% ===== PPR.PRR
%% ===== \end{flushleft}
%% ===== 
%% ===== 
%% ===== 
%% ===== 
%% ===== 
%% ===== \begin{flushleft}
%% ===== The number of the ring in which the process is executing. It is set to the
%% ===== \end{flushleft}
%% ===== 
%% ===== 
%% ===== \begin{flushleft}
%% ===== effective ring number of the procedure segment when control is transferred
%% ===== \end{flushleft}
%% ===== 
%% ===== 
%% ===== \begin{flushleft}
%% ===== to the procedure.
%% ===== \end{flushleft}
%% ===== 
%% ===== 
%% ===== 
%% ===== 
%% ===== 
%% ===== \begin{flushleft}
%% ===== PPR.PSR
%% ===== \end{flushleft}
%% ===== 
%% ===== 
%% ===== 
%% ===== 
%% ===== 
%% ===== \begin{flushleft}
%% ===== The segment number of the procedure being executed.
%% ===== \end{flushleft}
%% ===== 
%% ===== 
%% ===== 
%% ===== 
%% ===== 
%% ===== \begin{flushleft}
%% ===== PPR.P
%% ===== \end{flushleft}
%% ===== 
%% ===== 
%% ===== 
%% ===== 
%% ===== 
%% ===== \begin{flushleft}
%% ===== A flag controlling execution of privileged instructions.
%% ===== \end{flushleft}
%% ===== 
%% ===== 
%% ===== \begin{flushleft}
%% ===== Its value is 1
%% ===== \end{flushleft}
%% ===== 
%% ===== 
%% ===== \begin{flushleft}
%% ===== (permitting execution of privileged instructions) if PPR.PRR is 0 and the
%% ===== \end{flushleft}
%% ===== 
%% ===== 
%% ===== \begin{flushleft}
%% ===== privileged bit in the segment descriptor word (SDW.P) for the procedure is 1;
%% ===== \end{flushleft}
%% ===== 
%% ===== 
%% ===== \begin{flushleft}
%% ===== otherwise, its value is 0.
%% ===== \end{flushleft}
%% ===== 
%% ===== 
%% ===== 
%% ===== 
%% ===== 
%% ===== \begin{flushleft}
%% ===== PPR.IC
%% ===== \end{flushleft}
%% ===== 
%% ===== 
%% ===== 
%% ===== 
%% ===== 
%% ===== \begin{flushleft}
%% ===== The word offset from the origin of the procedure segment to the current
%% ===== \end{flushleft}
%% ===== 
%% ===== 
%% ===== \begin{flushleft}
%% ===== instruction.
%% ===== \end{flushleft}
%% ===== 
%% ===== 
%% ===== 
%% ===== 
%% ===== 
%% ===== \begin{flushleft}

\subsection{TEMPORARY POINTER REGISTER (TPR)}

%% ===== \end{flushleft}
%% ===== 
%% ===== 
%% ===== \begin{flushleft}
%% ===== Format: - 42 bits
%% ===== \end{flushleft}
%% ===== 
%% ===== 
%% ===== \begin{flushleft}
%% ===== Shown as part of word 2 of control unit data
%% ===== \end{flushleft}
%% ===== 
%% ===== 
%% ===== 0
%% ===== 
%% ===== 
%% ===== 0
%% ===== 
%% ===== 
%% ===== 
%% ===== 
%% ===== 
%% ===== 0 0
%% ===== 
%% ===== 
%% ===== 2 3
%% ===== 
%% ===== 
%% ===== 
%% ===== 
%% ===== 
%% ===== \begin{flushleft}
%% ===== TRR
%% ===== \end{flushleft}
%% ===== 
%% ===== 
%% ===== 
%% ===== 
%% ===== 
%% ===== 1
%% ===== 
%% ===== 
%% ===== 7
%% ===== 
%% ===== 
%% ===== \begin{flushleft}
%% ===== Other control unit data
%% ===== \end{flushleft}
%% ===== 
%% ===== 
%% ===== 
%% ===== 
%% ===== 
%% ===== \begin{flushleft}
%% ===== TSR
%% ===== \end{flushleft}
%% ===== 
%% ===== 
%% ===== 3
%% ===== 
%% ===== 
%% ===== 
%% ===== 
%% ===== 
%% ===== 15
%% ===== 
%% ===== 
%% ===== 
%% ===== 
%% ===== 
%% ===== \begin{flushleft}
%% ===== \newpage
%% ===== Shown as part of word 3 of control unit data
%% ===== \end{flushleft}
%% ===== 
%% ===== 
%% ===== 3
%% ===== 
%% ===== 
%% ===== 0
%% ===== 
%% ===== 
%% ===== \begin{flushleft}
%% ===== Other control unit data
%% ===== \end{flushleft}
%% ===== 
%% ===== 
%% ===== 
%% ===== 
%% ===== 
%% ===== 3
%% ===== 
%% ===== 
%% ===== 5
%% ===== 
%% ===== 
%% ===== \begin{flushleft}
%% ===== TBR
%% ===== \end{flushleft}
%% ===== 
%% ===== 
%% ===== 6
%% ===== 
%% ===== 
%% ===== 
%% ===== 
%% ===== 
%% ===== \begin{flushleft}
%% ===== Shown as part of word 5 of control unit data
%% ===== \end{flushleft}
%% ===== 
%% ===== 
%% ===== 0
%% ===== 
%% ===== 
%% ===== 0
%% ===== 
%% ===== 
%% ===== 
%% ===== 
%% ===== 
%% ===== 1
%% ===== 
%% ===== 
%% ===== 7
%% ===== 
%% ===== 
%% ===== \begin{flushleft}
%% ===== Other control unit data
%% ===== \end{flushleft}
%% ===== 
%% ===== 
%% ===== 
%% ===== 
%% ===== 
%% ===== \begin{flushleft}
%% ===== CA
%% ===== \end{flushleft}
%% ===== 
%% ===== 
%% ===== 18
%% ===== 
%% ===== 
%% ===== 
%% ===== 
%% ===== 
%% ===== \begin{flushleft}
%% ===== Figure 3-14. Temporary Pointer Register (TPR) Format
%% ===== \end{flushleft}
%% ===== 
%% ===== 
%% ===== \begin{flushleft}
%% ===== Description:
%% ===== \end{flushleft}
%% ===== 
%% ===== 
%% ===== \begin{flushleft}
%% ===== A combination of physical registers from the appending unit and the control unit. TPR.TRR,
%% ===== \end{flushleft}
%% ===== 
%% ===== 
%% ===== \begin{flushleft}
%% ===== TPR.TSR, and TPR.TBR are located in the appending unit and TPR.CA is located in the
%% ===== \end{flushleft}
%% ===== 
%% ===== 
%% ===== \begin{flushleft}
%% ===== control unit. The TPR is not explicitly addressable but its data is extracted and stored as
%% ===== \end{flushleft}
%% ===== 
%% ===== 
%% ===== \begin{flushleft}
%% ===== part of the data stored with the Store Control Unit (scu) instruction. It is loaded from the
%% ===== \end{flushleft}
%% ===== 
%% ===== 
%% ===== \begin{flushleft}
%% ===== control unit data with the Restore Control Unit (rcu) instruction.
%% ===== \end{flushleft}
%% ===== 
%% ===== 
%% ===== \begin{flushleft}
%% ===== Function:
%% ===== \end{flushleft}
%% ===== 
%% ===== 
%% ===== \begin{flushleft}
%% ===== The temporary pointer register holds the current virtual address used by the processor in
%% ===== \end{flushleft}
%% ===== 
%% ===== 
%% ===== \begin{flushleft}
%% ===== performing address preparation for operands, indirect words, and instructions. At the
%% ===== \end{flushleft}
%% ===== 
%% ===== 
%% ===== \begin{flushleft}
%% ===== completion of address preparation, the contents of the TPR is presented to the appending
%% ===== \end{flushleft}
%% ===== 
%% ===== 
%% ===== \begin{flushleft}
%% ===== unit associative memories for translation into the 24-bit absolute main memory address.
%% ===== \end{flushleft}
%% ===== 
%% ===== 
%% ===== \begin{flushleft}
%% ===== The functions of the individual constituent registers are:
%% ===== \end{flushleft}
%% ===== 
%% ===== 
%% ===== 
%% ===== 
%% ===== 
%% ===== \begin{flushleft}
%% ===== Register
%% ===== \end{flushleft}
%% ===== 
%% ===== 
%% ===== 
%% ===== 
%% ===== 
%% ===== \begin{flushleft}
%% ===== Function
%% ===== \end{flushleft}
%% ===== 
%% ===== 
%% ===== 
%% ===== 
%% ===== 
%% ===== \begin{flushleft}
%% ===== TPR.TRR
%% ===== \end{flushleft}
%% ===== 
%% ===== 
%% ===== 
%% ===== 
%% ===== 
%% ===== \begin{flushleft}
%% ===== The current effective ring number (see Section 8).
%% ===== \end{flushleft}
%% ===== 
%% ===== 
%% ===== 
%% ===== 
%% ===== 
%% ===== \begin{flushleft}
%% ===== TPR.TSR
%% ===== \end{flushleft}
%% ===== 
%% ===== 
%% ===== 
%% ===== 
%% ===== 
%% ===== \begin{flushleft}
%% ===== The current effective segment number (see Section 8).
%% ===== \end{flushleft}
%% ===== 
%% ===== 
%% ===== 
%% ===== 
%% ===== 
%% ===== \begin{flushleft}
%% ===== TPR.TBR
%% ===== \end{flushleft}
%% ===== 
%% ===== 
%% ===== 
%% ===== 
%% ===== 
%% ===== \begin{flushleft}
%% ===== The current bit offset as calculated from ITS and ITP pointer pairs. (See
%% ===== \end{flushleft}
%% ===== 
%% ===== 
%% ===== \begin{flushleft}
%% ===== Section 8.)
%% ===== \end{flushleft}
%% ===== 
%% ===== 
%% ===== 
%% ===== 
%% ===== 
%% ===== \begin{flushleft}
%% ===== TPR.CA
%% ===== \end{flushleft}
%% ===== 
%% ===== 
%% ===== 
%% ===== 
%% ===== 
%% ===== \begin{flushleft}
%% ===== The current computed address relative to the origin of the segment whose
%% ===== \end{flushleft}
%% ===== 
%% ===== 
%% ===== \begin{flushleft}
%% ===== segment number is in TPR.TSR. (See Section 8.)
%% ===== \end{flushleft}
%% ===== 
%% ===== 
%% ===== 
%% ===== 
%% ===== 
%% ===== \begin{flushleft}
%% ===== \newpage

\subsection{DESCRIPTOR SEGMENT BASE REGISTER (DSBR)}

%% ===== \end{flushleft}
%% ===== 
%% ===== 
%% ===== \begin{flushleft}
%% ===== Format: - 51 bits
%% ===== \end{flushleft}
%% ===== 
%% ===== 
%% ===== \begin{flushleft}
%% ===== Even word of Y-pair as stored by Store Descriptor Base Register (sdbr)
%% ===== \end{flushleft}
%% ===== 
%% ===== 
%% ===== 0
%% ===== 
%% ===== 
%% ===== 0
%% ===== 
%% ===== 
%% ===== 
%% ===== 
%% ===== 
%% ===== 2 2
%% ===== 
%% ===== 
%% ===== 3 4
%% ===== 
%% ===== 
%% ===== \begin{flushleft}
%% ===== ADDR
%% ===== \end{flushleft}
%% ===== 
%% ===== 
%% ===== 
%% ===== 
%% ===== 
%% ===== 3
%% ===== 
%% ===== 
%% ===== 5
%% ===== 
%% ===== 
%% ===== 
%% ===== 
%% ===== 
%% ===== 0 0 0 0 0 0 0 0 0 0 0 0
%% ===== 
%% ===== 
%% ===== 24
%% ===== 
%% ===== 
%% ===== 
%% ===== 
%% ===== 
%% ===== 12
%% ===== 
%% ===== 
%% ===== 
%% ===== 
%% ===== 
%% ===== \begin{flushleft}
%% ===== Odd word of Y-pair as stored by Store Descriptor Base Register (sdbr)
%% ===== \end{flushleft}
%% ===== 
%% ===== 
%% ===== 3 3
%% ===== 
%% ===== 
%% ===== 6 7
%% ===== 
%% ===== 
%% ===== 
%% ===== 
%% ===== 
%% ===== 5 5
%% ===== 
%% ===== 
%% ===== 0 1
%% ===== 
%% ===== 
%% ===== 
%% ===== 
%% ===== 
%% ===== 0
%% ===== 
%% ===== 
%% ===== 
%% ===== 
%% ===== 
%% ===== \begin{flushleft}
%% ===== BND
%% ===== \end{flushleft}
%% ===== 
%% ===== 
%% ===== 
%% ===== 
%% ===== 
%% ===== 1
%% ===== 
%% ===== 
%% ===== 
%% ===== 
%% ===== 
%% ===== 5 5 5
%% ===== 
%% ===== 
%% ===== 4 5 6
%% ===== 
%% ===== 
%% ===== 
%% ===== 
%% ===== 
%% ===== 5 6
%% ===== 
%% ===== 
%% ===== 9 0
%% ===== 
%% ===== 
%% ===== 
%% ===== 
%% ===== 
%% ===== \begin{flushleft}
%% ===== 0 0 0 0 U 0 0 0 0
%% ===== \end{flushleft}
%% ===== 
%% ===== 
%% ===== 14
%% ===== 
%% ===== 
%% ===== 
%% ===== 
%% ===== 
%% ===== 4 1
%% ===== 
%% ===== 
%% ===== 
%% ===== 
%% ===== 
%% ===== 7
%% ===== 
%% ===== 
%% ===== 1
%% ===== 
%% ===== 
%% ===== \begin{flushleft}
%% ===== STACK
%% ===== \end{flushleft}
%% ===== 
%% ===== 
%% ===== 
%% ===== 
%% ===== 
%% ===== 4
%% ===== 
%% ===== 
%% ===== 
%% ===== 
%% ===== 
%% ===== 12
%% ===== 
%% ===== 
%% ===== 
%% ===== 
%% ===== 
%% ===== \begin{flushleft}
%% ===== Figure 3-15. Descriptor Segment Base Register (DSBR) Format
%% ===== \end{flushleft}
%% ===== 
%% ===== 
%% ===== \begin{flushleft}
%% ===== Description:
%% ===== \end{flushleft}
%% ===== 
%% ===== 
%% ===== \begin{flushleft}
%% ===== A physical register in the appending unit.
%% ===== \end{flushleft}
%% ===== 
%% ===== 
%% ===== \begin{flushleft}
%% ===== Function:
%% ===== \end{flushleft}
%% ===== 
%% ===== 
%% ===== \begin{flushleft}
%% ===== The Descriptor Segment Base Register contains information concerning the descriptor
%% ===== \end{flushleft}
%% ===== 
%% ===== 
%% ===== \begin{flushleft}
%% ===== segment being used by the processor.
%% ===== \end{flushleft}
%% ===== 
%% ===== 
%% ===== \begin{flushleft}
%% ===== The descriptor segment holds the segment
%% ===== \end{flushleft}
%% ===== 
%% ===== 
%% ===== \begin{flushleft}
%% ===== descriptor words (SDWs) for all segments accessible by the processor, that is, the currently
%% ===== \end{flushleft}
%% ===== 
%% ===== 
%% ===== \begin{flushleft}
%% ===== defined virtual address space. The functions of its individual constituent registers are:
%% ===== \end{flushleft}
%% ===== 
%% ===== 
%% ===== 
%% ===== 
%% ===== 
%% ===== \begin{flushleft}
%% ===== Register
%% ===== \end{flushleft}
%% ===== 
%% ===== 
%% ===== 
%% ===== 
%% ===== 
%% ===== \begin{flushleft}
%% ===== Function
%% ===== \end{flushleft}
%% ===== 
%% ===== 
%% ===== 
%% ===== 
%% ===== 
%% ===== \begin{flushleft}
%% ===== DSBR.ADDR
%% ===== \end{flushleft}
%% ===== 
%% ===== 
%% ===== 
%% ===== 
%% ===== 
%% ===== \begin{flushleft}
%% ===== If DSBR.U = 1, the 24-bit absolute main memory address of the origin
%% ===== \end{flushleft}
%% ===== 
%% ===== 
%% ===== \begin{flushleft}
%% ===== of the current descriptor segment; otherwise, the 24-bit absolute main
%% ===== \end{flushleft}
%% ===== 
%% ===== 
%% ===== \begin{flushleft}
%% ===== memory address of the page table for the current descriptor segment.
%% ===== \end{flushleft}
%% ===== 
%% ===== 
%% ===== 
%% ===== 
%% ===== 
%% ===== \begin{flushleft}
%% ===== DSBR.BND
%% ===== \end{flushleft}
%% ===== 
%% ===== 
%% ===== 
%% ===== 
%% ===== 
%% ===== \begin{flushleft}
%% ===== The 14 most significant bits of the highest Y-block16 address of the
%% ===== \end{flushleft}
%% ===== 
%% ===== 
%% ===== \begin{flushleft}
%% ===== descriptor segment that can be addressed without causing an access
%% ===== \end{flushleft}
%% ===== 
%% ===== 
%% ===== \begin{flushleft}
%% ===== violation, out of segment bounds, fault.
%% ===== \end{flushleft}
%% ===== 
%% ===== 
%% ===== 
%% ===== 
%% ===== 
%% ===== \begin{flushleft}
%% ===== DSBR.U
%% ===== \end{flushleft}
%% ===== 
%% ===== 
%% ===== 
%% ===== 
%% ===== 
%% ===== \begin{flushleft}
%% ===== A flag specifying whether the descriptor segment is unpaged (U = 1) or
%% ===== \end{flushleft}
%% ===== 
%% ===== 
%% ===== \begin{flushleft}
%% ===== paged (U = 0).
%% ===== \end{flushleft}
%% ===== 
%% ===== 
%% ===== 
%% ===== 
%% ===== 
%% ===== \begin{flushleft}
%% ===== DSBR.STACK
%% ===== \end{flushleft}
%% ===== 
%% ===== 
%% ===== 
%% ===== 
%% ===== 
%% ===== \begin{flushleft}
%% ===== The upper 12 bits of the 15-bit stack base segment number. It is used
%% ===== \end{flushleft}
%% ===== 
%% ===== 
%% ===== \begin{flushleft}
%% ===== only during the execution of the call6 instruction. (See Section 8 for a
%% ===== \end{flushleft}
%% ===== 
%% ===== 
%% ===== \begin{flushleft}
%% ===== discussion of generation of the stack segment number.)
%% ===== \end{flushleft}
%% ===== 
%% ===== 
%% ===== 
%% ===== 
%% ===== 
%% ===== \begin{flushleft}
%% ===== \newpage

\subsection{SEGMENT DESCRIPTOR WORD ASSOCIATIVE MEMORY (SDWAM)}

%% ===== \end{flushleft}
%% ===== 
%% ===== 
%% ===== \begin{flushleft}
%% ===== Format: - 88 bits each
%% ===== \end{flushleft}
%% ===== 
%% ===== 
%% ===== \begin{flushleft}
%% ===== Even word of Y-pairs as stored by Store Segment Descriptor Registers (ssdr)
%% ===== \end{flushleft}
%% ===== 
%% ===== 
%% ===== 0
%% ===== 
%% ===== 
%% ===== 0
%% ===== 
%% ===== 
%% ===== 
%% ===== 
%% ===== 
%% ===== 2 2
%% ===== 
%% ===== 
%% ===== 3 4
%% ===== 
%% ===== 
%% ===== \begin{flushleft}
%% ===== ADDR
%% ===== \end{flushleft}
%% ===== 
%% ===== 
%% ===== 
%% ===== 
%% ===== 
%% ===== 2 2
%% ===== 
%% ===== 
%% ===== 6 7
%% ===== 
%% ===== 
%% ===== \begin{flushleft}
%% ===== R1
%% ===== \end{flushleft}
%% ===== 
%% ===== 
%% ===== 
%% ===== 
%% ===== 
%% ===== 24
%% ===== 
%% ===== 
%% ===== 
%% ===== 
%% ===== 
%% ===== 2 3
%% ===== 
%% ===== 
%% ===== 9 0
%% ===== 
%% ===== 
%% ===== \begin{flushleft}
%% ===== R2
%% ===== \end{flushleft}
%% ===== 
%% ===== 
%% ===== 
%% ===== 
%% ===== 
%% ===== 3
%% ===== 
%% ===== 
%% ===== 
%% ===== 
%% ===== 
%% ===== 3 3
%% ===== 
%% ===== 
%% ===== 2 3
%% ===== 
%% ===== 
%% ===== \begin{flushleft}
%% ===== R3
%% ===== \end{flushleft}
%% ===== 
%% ===== 
%% ===== 
%% ===== 
%% ===== 
%% ===== 3
%% ===== 
%% ===== 
%% ===== 
%% ===== 
%% ===== 
%% ===== 3
%% ===== 
%% ===== 
%% ===== 5
%% ===== 
%% ===== 
%% ===== 
%% ===== 
%% ===== 
%% ===== 0 0 0
%% ===== 
%% ===== 
%% ===== 3
%% ===== 
%% ===== 
%% ===== 
%% ===== 
%% ===== 
%% ===== 3
%% ===== 
%% ===== 
%% ===== 
%% ===== 
%% ===== 
%% ===== \begin{flushleft}
%% ===== Odd word of Y-pairs as stored by Store Segment Descriptor Registers (ssdr)
%% ===== \end{flushleft}
%% ===== 
%% ===== 
%% ===== 3 3
%% ===== 
%% ===== 
%% ===== 6 7
%% ===== 
%% ===== 
%% ===== 0
%% ===== 
%% ===== 
%% ===== 
%% ===== 
%% ===== 
%% ===== 5 5 5 5 5 5 5 5 5
%% ===== 
%% ===== 
%% ===== 0 1 2 3 4 5 6 7 8
%% ===== 
%% ===== 
%% ===== \begin{flushleft}
%% ===== BOUND
%% ===== \end{flushleft}
%% ===== 
%% ===== 
%% ===== 
%% ===== 
%% ===== 
%% ===== 1
%% ===== 
%% ===== 
%% ===== 
%% ===== 
%% ===== 
%% ===== 7
%% ===== 
%% ===== 
%% ===== 1
%% ===== 
%% ===== 
%% ===== 
%% ===== 
%% ===== 
%% ===== \begin{flushleft}
%% ===== R E W P U G C
%% ===== \end{flushleft}
%% ===== 
%% ===== 
%% ===== 
%% ===== 
%% ===== 
%% ===== \begin{flushleft}
%% ===== CL
%% ===== \end{flushleft}
%% ===== 
%% ===== 
%% ===== 
%% ===== 
%% ===== 
%% ===== 14 1 1 1 1 1 1 1
%% ===== 
%% ===== 
%% ===== 
%% ===== 
%% ===== 
%% ===== 14
%% ===== 
%% ===== 
%% ===== 
%% ===== 
%% ===== 
%% ===== \begin{flushleft}
%% ===== Data as stored by Store Segment Descriptor Pointers (ssdp)
%% ===== \end{flushleft}
%% ===== 
%% ===== 
%% ===== 0
%% ===== 
%% ===== 
%% ===== 0
%% ===== 
%% ===== 
%% ===== 
%% ===== 
%% ===== 
%% ===== 1 1
%% ===== 
%% ===== 
%% ===== 4 5
%% ===== 
%% ===== 
%% ===== \begin{flushleft}
%% ===== POINTER
%% ===== \end{flushleft}
%% ===== 
%% ===== 
%% ===== 
%% ===== 
%% ===== 
%% ===== 2 2 2 2 3 3 3
%% ===== 
%% ===== 
%% ===== 6 7 8 9 0 1 2
%% ===== 
%% ===== 
%% ===== 
%% ===== 
%% ===== 
%% ===== \begin{flushleft}
%% ===== 0 0 0 0 0 0 0 0 0 0 0 0 F 0 0
%% ===== \end{flushleft}
%% ===== 
%% ===== 
%% ===== 15
%% ===== 
%% ===== 
%% ===== 
%% ===== 
%% ===== 
%% ===== 12 1
%% ===== 
%% ===== 
%% ===== 
%% ===== 
%% ===== 
%% ===== 2
%% ===== 
%% ===== 
%% ===== 
%% ===== 
%% ===== 
%% ===== 3
%% ===== 
%% ===== 
%% ===== 5
%% ===== 
%% ===== 
%% ===== 
%% ===== 
%% ===== 
%% ===== \begin{flushleft}
%% ===== 0 0 USE L68
%% ===== \end{flushleft}
%% ===== 
%% ===== 
%% ===== \begin{flushleft}
%% ===== USE DPS 8M
%% ===== \end{flushleft}
%% ===== 
%% ===== 
%% ===== 2
%% ===== 
%% ===== 
%% ===== 4
%% ===== 
%% ===== 
%% ===== 
%% ===== 
%% ===== 
%% ===== \begin{flushleft}
%% ===== Figure 3-16. Segment Descriptor Word Associative Memory (SDWAM) Format
%% ===== \end{flushleft}
%% ===== 
%% ===== 
%% ===== \begin{flushleft}
%% ===== Description:
%% ===== \end{flushleft}
%% ===== 
%% ===== 
%% ===== \begin{flushleft}
%% ===== A combination of 16 registers and flags from the appending unit constitute the Segment
%% ===== \end{flushleft}
%% ===== 
%% ===== 
%% ===== \begin{flushleft}
%% ===== Descriptor Word Associative Memory (SDWAM). The registers are numbered consecutively
%% ===== \end{flushleft}
%% ===== 
%% ===== 
%% ===== \begin{flushleft}
%% ===== from 0 through 15 but are not explicitly addressable by number.
%% ===== \end{flushleft}
%% ===== 
%% ===== 
%% ===== \begin{flushleft}
%% ===== For the DPS/L68 processors, the SDW associative memory will hold the 16 most recently
%% ===== \end{flushleft}
%% ===== 
%% ===== 
%% ===== \begin{flushleft}
%% ===== used (MRU) SDWs and have a full associative organization with least recently used (LRU)
%% ===== \end{flushleft}
%% ===== 
%% ===== 
%% ===== \begin{flushleft}
%% ===== replacement.
%% ===== \end{flushleft}
%% ===== 
%% ===== 
%% ===== \begin{flushleft}
%% ===== For the DPS 8M processor, the SDW associative memory will hold the 64 MRU SDWs and
%% ===== \end{flushleft}
%% ===== 
%% ===== 
%% ===== \begin{flushleft}
%% ===== have a 4-way set associative organization with LRU replacement.
%% ===== \end{flushleft}
%% ===== 
%% ===== 
%% ===== \begin{flushleft}
%% ===== Function:
%% ===== \end{flushleft}
%% ===== 
%% ===== 
%% ===== \begin{flushleft}
%% ===== Hardware segmentation in the processor is implemented by the appending unit (see Section
%% ===== \end{flushleft}
%% ===== 
%% ===== 
%% ===== \begin{flushleft}
%% ===== 5). In order to permit addressing by segment number and offset as prepared in the
%% ===== \end{flushleft}
%% ===== 
%% ===== 
%% ===== \begin{flushleft}
%% ===== temporary pointer register (described earlier), a table containing the location and status of
%% ===== \end{flushleft}
%% ===== 
%% ===== 
%% ===== \begin{flushleft}
%% ===== each accessible segment must be kept. This table is the descriptor segment. The
%% ===== \end{flushleft}
%% ===== 
%% ===== 
%% ===== \begin{flushleft}
%% ===== descriptor segment is located by information held in the descriptor segment base register
%% ===== \end{flushleft}
%% ===== 
%% ===== 
%% ===== \begin{flushleft}
%% ===== (DSBR) described earlier.
%% ===== \end{flushleft}
%% ===== 
%% ===== 
%% ===== 
%% ===== 
%% ===== 
%% ===== \begin{flushleft}
%% ===== \newpage
%% ===== Every time an effective segment number (TPR.TSR) is prepared, it is used as an index into
%% ===== \end{flushleft}
%% ===== 
%% ===== 
%% ===== \begin{flushleft}
%% ===== the descriptor segment to retrieve the segment descriptor word (SDW) for the target
%% ===== \end{flushleft}
%% ===== 
%% ===== 
%% ===== \begin{flushleft}
%% ===== segment. To reduce the number of main memory references required for segment
%% ===== \end{flushleft}
%% ===== 
%% ===== 
%% ===== \begin{flushleft}
%% ===== addressing, the SDWAM provides a content addressable memory to hold the sixteen most
%% ===== \end{flushleft}
%% ===== 
%% ===== 
%% ===== \begin{flushleft}
%% ===== recently referenced SDWs.
%% ===== \end{flushleft}
%% ===== 
%% ===== 
%% ===== \begin{flushleft}
%% ===== Whenever a reference to the SDW for a segment is required, the effective segment number
%% ===== \end{flushleft}
%% ===== 
%% ===== 
%% ===== \begin{flushleft}
%% ===== (TPR.TSR) is matched associatively against all 16 SDWAM.POINTER registers (described
%% ===== \end{flushleft}
%% ===== 
%% ===== 
%% ===== \begin{flushleft}
%% ===== below). If the SDWAM match logic circuitry indicates a hit, all usage counts (SDWAM.USE)
%% ===== \end{flushleft}
%% ===== 
%% ===== 
%% ===== \begin{flushleft}
%% ===== greater than the usage count of the register hit are decremented by one, the usage count of
%% ===== \end{flushleft}
%% ===== 
%% ===== 
%% ===== \begin{flushleft}
%% ===== the register hit is set to 15, and the contents of the register hit are read out into the address
%% ===== \end{flushleft}
%% ===== 
%% ===== 
%% ===== \begin{flushleft}
%% ===== preparation circuitry. If the SDWAM match logic does not indicate a hit, the SDW is fetched
%% ===== \end{flushleft}
%% ===== 
%% ===== 
%% ===== \begin{flushleft}
%% ===== from the descriptor segment in main memory and loaded into the SDWAM register with
%% ===== \end{flushleft}
%% ===== 
%% ===== 
%% ===== \begin{flushleft}
%% ===== usage count 0 (the oldest), all usage counts are decremented by one with the newly loaded
%% ===== \end{flushleft}
%% ===== 
%% ===== 
%% ===== \begin{flushleft}
%% ===== register rolling over from 0 to 15, and the newly loaded register is read out into the address
%% ===== \end{flushleft}
%% ===== 
%% ===== 
%% ===== \begin{flushleft}
%% ===== preparation circuitry. Faulted SDWs are not loaded into the SDWAM.
%% ===== \end{flushleft}
%% ===== 
%% ===== 
%% ===== \begin{flushleft}
%% ===== The functions of the constituent registers and flags of each SDWAM register are as follows:
%% ===== \end{flushleft}
%% ===== 
%% ===== 
%% ===== 
%% ===== 
%% ===== 
%% ===== \begin{flushleft}
%% ===== Register
%% ===== \end{flushleft}
%% ===== 
%% ===== 
%% ===== 
%% ===== 
%% ===== 
%% ===== \begin{flushleft}
%% ===== Function
%% ===== \end{flushleft}
%% ===== 
%% ===== 
%% ===== 
%% ===== 
%% ===== 
%% ===== \begin{flushleft}
%% ===== SDWAM.ADDR
%% ===== \end{flushleft}
%% ===== 
%% ===== 
%% ===== 
%% ===== 
%% ===== 
%% ===== \begin{flushleft}
%% ===== The 24-bit absolute main memory address of the page table for the
%% ===== \end{flushleft}
%% ===== 
%% ===== 
%% ===== \begin{flushleft}
%% ===== target segment if SDWAM.U = 0; otherwise, the 24-bit absolute main
%% ===== \end{flushleft}
%% ===== 
%% ===== 
%% ===== \begin{flushleft}
%% ===== memory address of the origin of the target segment.
%% ===== \end{flushleft}
%% ===== 
%% ===== 
%% ===== 
%% ===== 
%% ===== 
%% ===== \begin{flushleft}
%% ===== SDWAM.R1
%% ===== \end{flushleft}
%% ===== 
%% ===== 
%% ===== 
%% ===== 
%% ===== 
%% ===== \begin{flushleft}
%% ===== Upper limit of read/write ring bracket (see Section 8).
%% ===== \end{flushleft}
%% ===== 
%% ===== 
%% ===== 
%% ===== 
%% ===== 
%% ===== \begin{flushleft}
%% ===== SDWAM.R2
%% ===== \end{flushleft}
%% ===== 
%% ===== 
%% ===== 
%% ===== 
%% ===== 
%% ===== \begin{flushleft}
%% ===== Upper limit of read/execute ring bracket (see Section 8).
%% ===== \end{flushleft}
%% ===== 
%% ===== 
%% ===== 
%% ===== 
%% ===== 
%% ===== \begin{flushleft}
%% ===== SDWAM.R3
%% ===== \end{flushleft}
%% ===== 
%% ===== 
%% ===== 
%% ===== 
%% ===== 
%% ===== \begin{flushleft}
%% ===== Upper limit of call ring bracket (see Section 8).
%% ===== \end{flushleft}
%% ===== 
%% ===== 
%% ===== 
%% ===== 
%% ===== 
%% ===== \begin{flushleft}
%% ===== SDWAM.BOUND
%% ===== \end{flushleft}
%% ===== 
%% ===== 
%% ===== 
%% ===== 
%% ===== 
%% ===== \begin{flushleft}
%% ===== The 14 high-order bits of the last Y-block16 address within the
%% ===== \end{flushleft}
%% ===== 
%% ===== 
%% ===== \begin{flushleft}
%% ===== segment that can be referenced without an access violation, out of
%% ===== \end{flushleft}
%% ===== 
%% ===== 
%% ===== \begin{flushleft}
%% ===== segment bound, fault.
%% ===== \end{flushleft}
%% ===== 
%% ===== 
%% ===== 
%% ===== 
%% ===== 
%% ===== \begin{flushleft}
%% ===== SDWAM.R
%% ===== \end{flushleft}
%% ===== 
%% ===== 
%% ===== 
%% ===== 
%% ===== 
%% ===== \begin{flushleft}
%% ===== Read permission bit. If this bit is set ON, read access requests are
%% ===== \end{flushleft}
%% ===== 
%% ===== 
%% ===== \begin{flushleft}
%% ===== allowed.
%% ===== \end{flushleft}
%% ===== 
%% ===== 
%% ===== 
%% ===== 
%% ===== 
%% ===== \begin{flushleft}
%% ===== SDWAM.E
%% ===== \end{flushleft}
%% ===== 
%% ===== 
%% ===== 
%% ===== 
%% ===== 
%% ===== \begin{flushleft}
%% ===== Execute permission bit. If this bit is set ON, the SDW may be loaded
%% ===== \end{flushleft}
%% ===== 
%% ===== 
%% ===== \begin{flushleft}
%% ===== into the procedure pointer register (PPR) and instructions fetched
%% ===== \end{flushleft}
%% ===== 
%% ===== 
%% ===== \begin{flushleft}
%% ===== from the segment for execution.
%% ===== \end{flushleft}
%% ===== 
%% ===== 
%% ===== 
%% ===== 
%% ===== 
%% ===== \begin{flushleft}
%% ===== SDWAM.W
%% ===== \end{flushleft}
%% ===== 
%% ===== 
%% ===== 
%% ===== 
%% ===== 
%% ===== \begin{flushleft}
%% ===== Write permission bit. If this bit is set ON, write access requests are
%% ===== \end{flushleft}
%% ===== 
%% ===== 
%% ===== \begin{flushleft}
%% ===== allowed.
%% ===== \end{flushleft}
%% ===== 
%% ===== 
%% ===== 
%% ===== 
%% ===== 
%% ===== \begin{flushleft}
%% ===== SDWAM.P
%% ===== \end{flushleft}
%% ===== 
%% ===== 
%% ===== 
%% ===== 
%% ===== 
%% ===== \begin{flushleft}
%% ===== Privileged flag bit. If this bit is set ON, privileged instructions from
%% ===== \end{flushleft}
%% ===== 
%% ===== 
%% ===== \begin{flushleft}
%% ===== the segment may be executed if PPR.PRR is 0.
%% ===== \end{flushleft}
%% ===== 
%% ===== 
%% ===== 
%% ===== 
%% ===== 
%% ===== \begin{flushleft}
%% ===== SDWAM.U
%% ===== \end{flushleft}
%% ===== 
%% ===== 
%% ===== 
%% ===== 
%% ===== 
%% ===== \begin{flushleft}
%% ===== Unpaged flag bit. If this bit
%% ===== \end{flushleft}
%% ===== 
%% ===== 
%% ===== \begin{flushleft}
%% ===== SDWAM.ADDR is the 24-bit
%% ===== \end{flushleft}
%% ===== 
%% ===== 
%% ===== \begin{flushleft}
%% ===== origin of the segment. If this
%% ===== \end{flushleft}
%% ===== 
%% ===== 
%% ===== \begin{flushleft}
%% ===== SDWAM.ADDR is the 24-bit
%% ===== \end{flushleft}
%% ===== 
%% ===== 
%% ===== \begin{flushleft}
%% ===== page table for the segment.
%% ===== \end{flushleft}
%% ===== 
%% ===== 
%% ===== 
%% ===== 
%% ===== 
%% ===== \begin{flushleft}
%% ===== SDWAM.G
%% ===== \end{flushleft}
%% ===== 
%% ===== 
%% ===== 
%% ===== 
%% ===== 
%% ===== \begin{flushleft}
%% ===== Gate control bit. If this bit is set OFF, calls and transfers into the
%% ===== \end{flushleft}
%% ===== 
%% ===== 
%% ===== \begin{flushleft}
%% ===== segment must be to an offset no greater than the value of SDWAM.CL
%% ===== \end{flushleft}
%% ===== 
%% ===== 
%% ===== \begin{flushleft}
%% ===== as described below.
%% ===== \end{flushleft}
%% ===== 
%% ===== 
%% ===== 
%% ===== 
%% ===== 
%% ===== \begin{flushleft}
%% ===== SDWAM.C
%% ===== \end{flushleft}
%% ===== 
%% ===== 
%% ===== 
%% ===== 
%% ===== 
%% ===== \begin{flushleft}
%% ===== Cache control bit. If this bit is set ON, data and/or instructions from
%% ===== \end{flushleft}
%% ===== 
%% ===== 
%% ===== \begin{flushleft}
%% ===== the segment may be placed in the cache memory.
%% ===== \end{flushleft}
%% ===== 
%% ===== 
%% ===== 
%% ===== 
%% ===== 
%% ===== \begin{flushleft}
%% ===== SDWAM.CL
%% ===== \end{flushleft}
%% ===== 
%% ===== 
%% ===== 
%% ===== 
%% ===== 
%% ===== \begin{flushleft}
%% ===== Call limiter (entry bound) value. If SDWAM.G is set OFF, transfers of
%% ===== \end{flushleft}
%% ===== 
%% ===== 
%% ===== \begin{flushleft}
%% ===== control into the segment must be to segment addresses no greater
%% ===== \end{flushleft}
%% ===== 
%% ===== 
%% ===== \begin{flushleft}
%% ===== than this value.
%% ===== \end{flushleft}
%% ===== 
%% ===== 
%% ===== 
%% ===== 
%% ===== 
%% ===== \begin{flushleft}
%% ===== SDWAM.POINTER
%% ===== \end{flushleft}
%% ===== 
%% ===== 
%% ===== 
%% ===== 
%% ===== 
%% ===== \begin{flushleft}
%% ===== The effective segment number used to fetch this SDW from main
%% ===== \end{flushleft}
%% ===== 
%% ===== 
%% ===== \begin{flushleft}
%% ===== memory.
%% ===== \end{flushleft}
%% ===== 
%% ===== 
%% ===== 
%% ===== 
%% ===== 
%% ===== \begin{flushleft}
%% ===== is set ON, the segment is unpaged and
%% ===== \end{flushleft}
%% ===== 
%% ===== 
%% ===== \begin{flushleft}
%% ===== absolute main memory address of the
%% ===== \end{flushleft}
%% ===== 
%% ===== 
%% ===== \begin{flushleft}
%% ===== bit is set OFF, the segment is paged and
%% ===== \end{flushleft}
%% ===== 
%% ===== 
%% ===== \begin{flushleft}
%% ===== absolute main memory address of the
%% ===== \end{flushleft}
%% ===== 
%% ===== 
%% ===== 
%% ===== 
%% ===== 
%% ===== \begin{flushleft}
%% ===== \newpage
%% ===== Register
%% ===== \end{flushleft}
%% ===== 
%% ===== 
%% ===== 
%% ===== 
%% ===== 
%% ===== \begin{flushleft}
%% ===== Function
%% ===== \end{flushleft}
%% ===== 
%% ===== 
%% ===== 
%% ===== 
%% ===== 
%% ===== \begin{flushleft}
%% ===== SDWAM.F
%% ===== \end{flushleft}
%% ===== 
%% ===== 
%% ===== 
%% ===== 
%% ===== 
%% ===== \begin{flushleft}
%% ===== Full/empty bit. If this bit is set ON, the SDW in the register is valid.
%% ===== \end{flushleft}
%% ===== 
%% ===== 
%% ===== \begin{flushleft}
%% ===== If this bit is set OFF, a hit is not possible. All SDWAM.F bits are set
%% ===== \end{flushleft}
%% ===== 
%% ===== 
%% ===== \begin{flushleft}
%% ===== OFF by the instructions that clear the SDWAM.
%% ===== \end{flushleft}
%% ===== 
%% ===== 
%% ===== 
%% ===== 
%% ===== 
%% ===== \begin{flushleft}
%% ===== SDWAM.USE
%% ===== \end{flushleft}
%% ===== 
%% ===== 
%% ===== 
%% ===== 
%% ===== 
%% ===== \begin{flushleft}
%% ===== Usage count for the register. The SDWAM.USE field is used to
%% ===== \end{flushleft}
%% ===== 
%% ===== 
%% ===== \begin{flushleft}
%% ===== maintain a strict FIFO queue order among the SDWs. When an SDW
%% ===== \end{flushleft}
%% ===== 
%% ===== 
%% ===== \begin{flushleft}
%% ===== is matched, its USE value is set to 15 (newest) on the DPS/L68 and to
%% ===== \end{flushleft}
%% ===== 
%% ===== 
%% ===== \begin{flushleft}
%% ===== 63 on the DPS 8M, and the queue is reordered. SDWs newly fetched
%% ===== \end{flushleft}
%% ===== 
%% ===== 
%% ===== \begin{flushleft}
%% ===== from main memory replace the SDW with USE value 0 (oldest) and
%% ===== \end{flushleft}
%% ===== 
%% ===== 
%% ===== \begin{flushleft}
%% ===== the queue is reordered.
%% ===== \end{flushleft}
%% ===== 
%% ===== 
%% ===== 
%% ===== 
%% ===== 
%% ===== \begin{flushleft}

\subsection{PAGE TABLE WORD ASSOCIATIVE MEMORY (PTWAM)}

%% ===== \end{flushleft}
%% ===== 
%% ===== 
%% ===== \begin{flushleft}
%% ===== Format: - 51 bits each
%% ===== \end{flushleft}
%% ===== 
%% ===== 
%% ===== \begin{flushleft}
%% ===== Data as stored by Store Page Table Registers (sptr)
%% ===== \end{flushleft}
%% ===== 
%% ===== 
%% ===== 0
%% ===== 
%% ===== 
%% ===== 0
%% ===== 
%% ===== 
%% ===== 
%% ===== 
%% ===== 
%% ===== 1 1
%% ===== 
%% ===== 
%% ===== 7 8
%% ===== 
%% ===== 
%% ===== \begin{flushleft}
%% ===== ADDR
%% ===== \end{flushleft}
%% ===== 
%% ===== 
%% ===== 
%% ===== 
%% ===== 
%% ===== 2 2 3
%% ===== 
%% ===== 
%% ===== 8 9 0
%% ===== 
%% ===== 
%% ===== 
%% ===== 
%% ===== 
%% ===== 3
%% ===== 
%% ===== 
%% ===== 5
%% ===== 
%% ===== 
%% ===== 
%% ===== 
%% ===== 
%% ===== \begin{flushleft}
%% ===== 0 0 0 0 0 0 0 0 0 0 0 M 0 0 0 0 0 0
%% ===== \end{flushleft}
%% ===== 
%% ===== 
%% ===== 18
%% ===== 
%% ===== 
%% ===== 
%% ===== 
%% ===== 
%% ===== 11 1
%% ===== 
%% ===== 
%% ===== 
%% ===== 
%% ===== 
%% ===== 6
%% ===== 
%% ===== 
%% ===== 
%% ===== 
%% ===== 
%% ===== \begin{flushleft}
%% ===== Data as stored by Store Page Table Pointers (sptp)
%% ===== \end{flushleft}
%% ===== 
%% ===== 
%% ===== 0
%% ===== 
%% ===== 
%% ===== 0
%% ===== 
%% ===== 
%% ===== 
%% ===== 
%% ===== 
%% ===== 1 1
%% ===== 
%% ===== 
%% ===== 4 5
%% ===== 
%% ===== 
%% ===== \begin{flushleft}
%% ===== POINTER
%% ===== \end{flushleft}
%% ===== 
%% ===== 
%% ===== 
%% ===== 
%% ===== 
%% ===== 2 2 2 2 3 3 3
%% ===== 
%% ===== 
%% ===== 6 7 8 9 0 1 2
%% ===== 
%% ===== 
%% ===== \begin{flushleft}
%% ===== PAGENO
%% ===== \end{flushleft}
%% ===== 
%% ===== 
%% ===== 
%% ===== 
%% ===== 
%% ===== 15
%% ===== 
%% ===== 
%% ===== 
%% ===== 
%% ===== 
%% ===== \begin{flushleft}
%% ===== F 0 0
%% ===== \end{flushleft}
%% ===== 
%% ===== 
%% ===== 12 1
%% ===== 
%% ===== 
%% ===== 
%% ===== 
%% ===== 
%% ===== 2
%% ===== 
%% ===== 
%% ===== 
%% ===== 
%% ===== 
%% ===== 3
%% ===== 
%% ===== 
%% ===== 5
%% ===== 
%% ===== 
%% ===== 
%% ===== 
%% ===== 
%% ===== \begin{flushleft}
%% ===== 0 0 USE L68
%% ===== \end{flushleft}
%% ===== 
%% ===== 
%% ===== \begin{flushleft}
%% ===== USE DPS 8M
%% ===== \end{flushleft}
%% ===== 
%% ===== 
%% ===== 2
%% ===== 
%% ===== 
%% ===== 4
%% ===== 
%% ===== 
%% ===== 
%% ===== 
%% ===== 
%% ===== \begin{flushleft}
%% ===== Figure 3-17. Page Table Word Associative Memory (PTWAM) Format
%% ===== \end{flushleft}
%% ===== 
%% ===== 
%% ===== \begin{flushleft}
%% ===== Description:
%% ===== \end{flushleft}
%% ===== 
%% ===== 
%% ===== \begin{flushleft}
%% ===== A combination of 16 registers and flags from the appending unit constitute the Page Table
%% ===== \end{flushleft}
%% ===== 
%% ===== 
%% ===== \begin{flushleft}
%% ===== Word Associative Memory (PTWAM). The registers are numbered consecutively from 0
%% ===== \end{flushleft}
%% ===== 
%% ===== 
%% ===== \begin{flushleft}
%% ===== through 15 but are not explicitly addressable by number.
%% ===== \end{flushleft}
%% ===== 
%% ===== 
%% ===== \begin{flushleft}
%% ===== For the DPS/L68 processors, the PTW associative memory will hold the 16 most recently
%% ===== \end{flushleft}
%% ===== 
%% ===== 
%% ===== \begin{flushleft}
%% ===== used (MRU) PTWs and have a full associative organization with least recently used (LRU)
%% ===== \end{flushleft}
%% ===== 
%% ===== 
%% ===== \begin{flushleft}
%% ===== replacement.
%% ===== \end{flushleft}
%% ===== 
%% ===== 
%% ===== \begin{flushleft}
%% ===== For the DPS 8M processors, the PTW associative memory will hold the 64 MRU PTWs and
%% ===== \end{flushleft}
%% ===== 
%% ===== 
%% ===== \begin{flushleft}
%% ===== have a 4-way set associative organization with LRU replacement.
%% ===== \end{flushleft}
%% ===== 
%% ===== 
%% ===== \begin{flushleft}
%% ===== Function:
%% ===== \end{flushleft}
%% ===== 
%% ===== 
%% ===== \begin{flushleft}
%% ===== Hardware paging in the Multics processor is implemented by the appending unit (see
%% ===== \end{flushleft}
%% ===== 
%% ===== 
%% ===== \begin{flushleft}
%% ===== Section 5 for details). In order to permit segment addressing by page number and page
%% ===== \end{flushleft}
%% ===== 
%% ===== 
%% ===== \begin{flushleft}
%% ===== offset as derived from the computed address prepared in the temporary pointer register
%% ===== \end{flushleft}
%% ===== 
%% ===== 
%% ===== \begin{flushleft}
%% ===== (TPR.CA described above), a table containing the location and status of each page of an
%% ===== \end{flushleft}
%% ===== 
%% ===== 
%% ===== \begin{flushleft}
%% ===== accessible segment must be kept. This table is the page table for the segment. The page
%% ===== \end{flushleft}
%% ===== 
%% ===== 
%% ===== 
%% ===== 
%% ===== 
%% ===== \begin{flushleft}
%% ===== \newpage
%% ===== table for an accessible paged segment is located by information held in the segment
%% ===== \end{flushleft}
%% ===== 
%% ===== 
%% ===== \begin{flushleft}
%% ===== descriptor word (SDW) for the segment.
%% ===== \end{flushleft}
%% ===== 
%% ===== 
%% ===== \begin{flushleft}
%% ===== Every time a computed address (TPR.CA) for a paged segment is prepared, it is separated
%% ===== \end{flushleft}
%% ===== 
%% ===== 
%% ===== \begin{flushleft}
%% ===== into a page number and a page offset. The page number is used as an index into the page
%% ===== \end{flushleft}
%% ===== 
%% ===== 
%% ===== \begin{flushleft}
%% ===== table to retrieve the page table word (PTW) for the target page. To reduce the number of
%% ===== \end{flushleft}
%% ===== 
%% ===== 
%% ===== \begin{flushleft}
%% ===== main memory references required for paging, the PTWAM provides a content addressable
%% ===== \end{flushleft}
%% ===== 
%% ===== 
%% ===== \begin{flushleft}
%% ===== memory to hold the 16 most recently referenced PTWs.
%% ===== \end{flushleft}
%% ===== 
%% ===== 
%% ===== \begin{flushleft}
%% ===== Whenever a reference to the PTW for a page of a paged segment is required, the page
%% ===== \end{flushleft}
%% ===== 
%% ===== 
%% ===== \begin{flushleft}
%% ===== number (as derived from TPR.CA) is matched associatively against all 16 PTWAM.PAGENO
%% ===== \end{flushleft}
%% ===== 
%% ===== 
%% ===== \begin{flushleft}
%% ===== registers (described below) and, simultaneously, TPR.TSR is matched against
%% ===== \end{flushleft}
%% ===== 
%% ===== 
%% ===== \begin{flushleft}
%% ===== PTWAM.POINTER (described below). If the PTWAM match logic circuitry indicates a hit,
%% ===== \end{flushleft}
%% ===== 
%% ===== 
%% ===== \begin{flushleft}
%% ===== all usage counts (PTWAM.USE) greater than the usage count of the register hit are
%% ===== \end{flushleft}
%% ===== 
%% ===== 
%% ===== \begin{flushleft}
%% ===== decremented by one, the usage count of the register hit is set to 15, and the contents of the
%% ===== \end{flushleft}
%% ===== 
%% ===== 
%% ===== \begin{flushleft}
%% ===== register hit are read out into the address preparation circuitry. If the PTWAM match logic
%% ===== \end{flushleft}
%% ===== 
%% ===== 
%% ===== \begin{flushleft}
%% ===== does not indicate a hit, the PTW is fetched from main memory and loaded into the PTWAM
%% ===== \end{flushleft}
%% ===== 
%% ===== 
%% ===== \begin{flushleft}
%% ===== register with usage count 0 (the oldest), all usage counts are decremented by one with the
%% ===== \end{flushleft}
%% ===== 
%% ===== 
%% ===== \begin{flushleft}
%% ===== newly loaded register rolling over from 0 to 15, and the newly loaded register is read out
%% ===== \end{flushleft}
%% ===== 
%% ===== 
%% ===== \begin{flushleft}
%% ===== into the address preparation circuitry. Faulted PTWs are not loaded into the PTWAM.
%% ===== \end{flushleft}
%% ===== 
%% ===== 
%% ===== \begin{flushleft}
%% ===== The functions of the constituent registers and flags of each PTWAM register are: (See
%% ===== \end{flushleft}
%% ===== 
%% ===== 
%% ===== \begin{flushleft}
%% ===== Section 8 for additional details.)
%% ===== \end{flushleft}
%% ===== 
%% ===== 
%% ===== 
%% ===== 
%% ===== 
%% ===== \begin{flushleft}
%% ===== Register
%% ===== \end{flushleft}
%% ===== 
%% ===== 
%% ===== 
%% ===== 
%% ===== 
%% ===== \begin{flushleft}
%% ===== Function
%% ===== \end{flushleft}
%% ===== 
%% ===== 
%% ===== 
%% ===== 
%% ===== 
%% ===== \begin{flushleft}
%% ===== PTWAM.ADDR
%% ===== \end{flushleft}
%% ===== 
%% ===== 
%% ===== 
%% ===== 
%% ===== 
%% ===== \begin{flushleft}
%% ===== The 18 high-order bits of the 24-bit absolute main memory address of
%% ===== \end{flushleft}
%% ===== 
%% ===== 
%% ===== \begin{flushleft}
%% ===== the page. The hardware ignores low-order bits of this page address
%% ===== \end{flushleft}
%% ===== 
%% ===== 
%% ===== \begin{flushleft}
%% ===== according to page size based on the following:
%% ===== \end{flushleft}
%% ===== 
%% ===== 
%% ===== 
%% ===== 
%% ===== 
%% ===== \begin{flushleft}
%% ===== Page size in words
%% ===== \end{flushleft}
%% ===== 
%% ===== 
%% ===== 64
%% ===== 
%% ===== 
%% ===== 128
%% ===== 
%% ===== 
%% ===== 256
%% ===== 
%% ===== 
%% ===== 512
%% ===== 
%% ===== 
%% ===== 1024
%% ===== 
%% ===== 
%% ===== 2048
%% ===== 
%% ===== 
%% ===== 4096
%% ===== 
%% ===== 
%% ===== 
%% ===== 
%% ===== 
%% ===== \begin{flushleft}
%% ===== ADDR bits ignored
%% ===== \end{flushleft}
%% ===== 
%% ===== 
%% ===== \begin{flushleft}
%% ===== none
%% ===== \end{flushleft}
%% ===== 
%% ===== 
%% ===== 17
%% ===== 
%% ===== 
%% ===== 16-17
%% ===== 
%% ===== 
%% ===== 15-17
%% ===== 
%% ===== 
%% ===== 14-17
%% ===== 
%% ===== 
%% ===== 13-17
%% ===== 
%% ===== 
%% ===== 12-17
%% ===== 
%% ===== 
%% ===== 
%% ===== 
%% ===== 
%% ===== \begin{flushleft}
%% ===== PTWAM.M
%% ===== \end{flushleft}
%% ===== 
%% ===== 
%% ===== 
%% ===== 
%% ===== 
%% ===== \begin{flushleft}
%% ===== Page modified flag bit. This bit is set ON whenever the PTW is used
%% ===== \end{flushleft}
%% ===== 
%% ===== 
%% ===== \begin{flushleft}
%% ===== for a store type instruction. When the bit changes value from 0 to 1, a
%% ===== \end{flushleft}
%% ===== 
%% ===== 
%% ===== \begin{flushleft}
%% ===== special extra cycle is generated to write it back into the PTW in the
%% ===== \end{flushleft}
%% ===== 
%% ===== 
%% ===== \begin{flushleft}
%% ===== page table in main memory.
%% ===== \end{flushleft}
%% ===== 
%% ===== 
%% ===== 
%% ===== 
%% ===== 
%% ===== \begin{flushleft}
%% ===== PTWAM.POINTER
%% ===== \end{flushleft}
%% ===== 
%% ===== 
%% ===== 
%% ===== 
%% ===== 
%% ===== \begin{flushleft}
%% ===== The effective segment number used to fetch this PTW from main
%% ===== \end{flushleft}
%% ===== 
%% ===== 
%% ===== \begin{flushleft}
%% ===== memory.
%% ===== \end{flushleft}
%% ===== 
%% ===== 
%% ===== 
%% ===== 
%% ===== 
%% ===== \begin{flushleft}
%% ===== PTWAM.PAGENO
%% ===== \end{flushleft}
%% ===== 
%% ===== 
%% ===== 
%% ===== 
%% ===== 
%% ===== \begin{flushleft}
%% ===== The 12 high-order bits of the 18-bit computed address (TPR.CA) used
%% ===== \end{flushleft}
%% ===== 
%% ===== 
%% ===== \begin{flushleft}
%% ===== to fetch this PTW from main memory. Low-order bits are forced to
%% ===== \end{flushleft}
%% ===== 
%% ===== 
%% ===== \begin{flushleft}
%% ===== zero by the hardware and not used as part of the page table index
%% ===== \end{flushleft}
%% ===== 
%% ===== 
%% ===== \begin{flushleft}
%% ===== according to page size based on the following:
%% ===== \end{flushleft}
%% ===== 
%% ===== 
%% ===== 
%% ===== 
%% ===== 
%% ===== \begin{flushleft}
%% ===== Page size in words
%% ===== \end{flushleft}
%% ===== 
%% ===== 
%% ===== 64
%% ===== 
%% ===== 
%% ===== 128
%% ===== 
%% ===== 
%% ===== 256
%% ===== 
%% ===== 
%% ===== 512
%% ===== 
%% ===== 
%% ===== 1024
%% ===== 
%% ===== 
%% ===== 2048
%% ===== 
%% ===== 
%% ===== 4096
%% ===== 
%% ===== 
%% ===== \begin{flushleft}
%% ===== PTWAM.F
%% ===== \end{flushleft}
%% ===== 
%% ===== 
%% ===== 
%% ===== 
%% ===== 
%% ===== \begin{flushleft}
%% ===== PAGENO bits forced
%% ===== \end{flushleft}
%% ===== 
%% ===== 
%% ===== \begin{flushleft}
%% ===== none
%% ===== \end{flushleft}
%% ===== 
%% ===== 
%% ===== 11
%% ===== 
%% ===== 
%% ===== 10-11
%% ===== 
%% ===== 
%% ===== 09-11
%% ===== 
%% ===== 
%% ===== 08-11
%% ===== 
%% ===== 
%% ===== 07-11
%% ===== 
%% ===== 
%% ===== 06-11
%% ===== 
%% ===== 
%% ===== 
%% ===== 
%% ===== 
%% ===== \begin{flushleft}
%% ===== Full/empty bit. If this bit is set ON, the PTW in the register is valid. If
%% ===== \end{flushleft}
%% ===== 
%% ===== 
%% ===== \begin{flushleft}
%% ===== this bit is set OFF, a hit is not possible. All PTWAM.F bits are set OFF
%% ===== \end{flushleft}
%% ===== 
%% ===== 
%% ===== \begin{flushleft}
%% ===== by the instructions that clear the PTWAM.
%% ===== \end{flushleft}
%% ===== 
%% ===== 
%% ===== 
%% ===== 
%% ===== 
%% ===== \begin{flushleft}
%% ===== \newpage
%% ===== Register
%% ===== \end{flushleft}
%% ===== 
%% ===== 
%% ===== 
%% ===== 
%% ===== 
%% ===== \begin{flushleft}
%% ===== Function
%% ===== \end{flushleft}
%% ===== 
%% ===== 
%% ===== 
%% ===== 
%% ===== 
%% ===== \begin{flushleft}
%% ===== PTWAM.USE
%% ===== \end{flushleft}
%% ===== 
%% ===== 
%% ===== 
%% ===== 
%% ===== 
%% ===== \begin{flushleft}
%% ===== Usage count for the register. The PTWAM.USE field is used to
%% ===== \end{flushleft}
%% ===== 
%% ===== 
%% ===== \begin{flushleft}
%% ===== maintain a strict FIFO queue order among the PTWs. When an PTW is
%% ===== \end{flushleft}
%% ===== 
%% ===== 
%% ===== \begin{flushleft}
%% ===== matched its USE value is set to 15 (newest) on the DPS/L68 and to 63
%% ===== \end{flushleft}
%% ===== 
%% ===== 
%% ===== \begin{flushleft}
%% ===== on the DPS 8M, and the queue is reordered. PTWs newly fetched
%% ===== \end{flushleft}
%% ===== 
%% ===== 
%% ===== \begin{flushleft}
%% ===== from main memory replace the PTW with USE value 0 (oldest) and the
%% ===== \end{flushleft}
%% ===== 
%% ===== 
%% ===== \begin{flushleft}
%% ===== queue is reordered.
%% ===== \end{flushleft}
%% ===== 
%% ===== 
%% ===== 
%% ===== 
%% ===== 
%% ===== \begin{flushleft}

\subsection{FAULT REGISTER (FR) -- DPS AND L68}

%% ===== \end{flushleft}
%% ===== 
%% ===== 
%% ===== \begin{flushleft}
%% ===== Format: - 72 bits
%% ===== \end{flushleft}
%% ===== 
%% ===== 
%% ===== \begin{flushleft}
%% ===== Even word of Y-pair as stored by Store Central Processor Register (scpr), TAG = 01
%% ===== \end{flushleft}
%% ===== 
%% ===== 
%% ===== 0 0 0 0 0 0 0 0 0 0 1 1 1 1 1 1 1
%% ===== 
%% ===== 
%% ===== 0 1 2 3 4 5 6 7 8 9 0 1 2 3 4 5 6
%% ===== 
%% ===== 
%% ===== \begin{flushleft}
%% ===== a b c d e f g h i
%% ===== \end{flushleft}
%% ===== 
%% ===== 
%% ===== 
%% ===== 
%% ===== 
%% ===== \begin{flushleft}
%% ===== j k l m n o 0
%% ===== \end{flushleft}
%% ===== 
%% ===== 
%% ===== 
%% ===== 
%% ===== 
%% ===== 1 1 1 1 1 1 1 1 1 1 1 1 1 1 1 1
%% ===== 
%% ===== 
%% ===== 
%% ===== 
%% ===== 
%% ===== 1 2
%% ===== 
%% ===== 
%% ===== 9 0
%% ===== 
%% ===== 
%% ===== \begin{flushleft}
%% ===== IAA
%% ===== \end{flushleft}
%% ===== 
%% ===== 
%% ===== 
%% ===== 
%% ===== 
%% ===== 2 2
%% ===== 
%% ===== 
%% ===== 3 4
%% ===== 
%% ===== 
%% ===== \begin{flushleft}
%% ===== IAB
%% ===== \end{flushleft}
%% ===== 
%% ===== 
%% ===== 
%% ===== 
%% ===== 
%% ===== 4
%% ===== 
%% ===== 
%% ===== 
%% ===== 
%% ===== 
%% ===== 2 2
%% ===== 
%% ===== 
%% ===== 7 8
%% ===== 
%% ===== 
%% ===== \begin{flushleft}
%% ===== IAC
%% ===== \end{flushleft}
%% ===== 
%% ===== 
%% ===== 
%% ===== 
%% ===== 
%% ===== 4
%% ===== 
%% ===== 
%% ===== 
%% ===== 
%% ===== 
%% ===== 3 3 3 3 3
%% ===== 
%% ===== 
%% ===== 1 2 3 4 5
%% ===== 
%% ===== 
%% ===== \begin{flushleft}
%% ===== IAD
%% ===== \end{flushleft}
%% ===== 
%% ===== 
%% ===== 
%% ===== 
%% ===== 
%% ===== 4
%% ===== 
%% ===== 
%% ===== 
%% ===== 
%% ===== 
%% ===== \begin{flushleft}
%% ===== p q r s
%% ===== \end{flushleft}
%% ===== 
%% ===== 
%% ===== 4 1 1 1 1
%% ===== 
%% ===== 
%% ===== 
%% ===== 
%% ===== 
%% ===== \begin{flushleft}
%% ===== Odd word of Y-pair as stored by Store Central Processor Register (scpr), TAG = 01
%% ===== \end{flushleft}
%% ===== 
%% ===== 
%% ===== 3
%% ===== 
%% ===== 
%% ===== 6
%% ===== 
%% ===== 
%% ===== 
%% ===== 
%% ===== 
%% ===== 7
%% ===== 
%% ===== 
%% ===== 1
%% ===== 
%% ===== 
%% ===== 
%% ===== 
%% ===== 
%% ===== 0 0 0 0 0 0 0 0 0 0 0 0 0 0 0 0 0 0 0 0 0 0 0 0 0 0 0 0 0 0 0 0 0 0 0 0
%% ===== 
%% ===== 
%% ===== 36
%% ===== 
%% ===== 
%% ===== 
%% ===== 
%% ===== 
%% ===== \begin{flushleft}
%% ===== Figure 3-18. Fault Register (FR) Format - DPS and L68
%% ===== \end{flushleft}
%% ===== 
%% ===== 
%% ===== \begin{flushleft}
%% ===== Description:
%% ===== \end{flushleft}
%% ===== 
%% ===== 
%% ===== \begin{flushleft}
%% ===== A combination of flags and registers all located in the control unit. The register is stored
%% ===== \end{flushleft}
%% ===== 
%% ===== 
%% ===== \begin{flushleft}
%% ===== and cleared by the Store Central Processor Register (scpr), TAG = 01, instruction. Note
%% ===== \end{flushleft}
%% ===== 
%% ===== 
%% ===== \begin{flushleft}
%% ===== that the data is stored into the word pair at location Y. The Fault Register cannot be
%% ===== \end{flushleft}
%% ===== 
%% ===== 
%% ===== \begin{flushleft}
%% ===== loaded.
%% ===== \end{flushleft}
%% ===== 
%% ===== 
%% ===== \begin{flushleft}
%% ===== Function:
%% ===== \end{flushleft}
%% ===== 
%% ===== 
%% ===== \begin{flushleft}
%% ===== The Fault Register contains the conditions in the processor for several of the hardware
%% ===== \end{flushleft}
%% ===== 
%% ===== 
%% ===== \begin{flushleft}
%% ===== faults. Data is strobed into the Fault Register during a fault sequence. Once a bit or field in
%% ===== \end{flushleft}
%% ===== 
%% ===== 
%% ===== \begin{flushleft}
%% ===== the Fault register is set, it remains set until the register is stored and cleared. The data is
%% ===== \end{flushleft}
%% ===== 
%% ===== 
%% ===== \begin{flushleft}
%% ===== not overwritten during subsequent fault events.
%% ===== \end{flushleft}
%% ===== 
%% ===== 
%% ===== \begin{flushleft}
%% ===== The functions of the constituent flags and registers are:
%% ===== \end{flushleft}
%% ===== 
%% ===== 
%% ===== 
%% ===== 
%% ===== 
%% ===== \begin{flushleft}
%% ===== Flag or
%% ===== \end{flushleft}
%% ===== 
%% ===== 
%% ===== \begin{flushleft}
%% ===== key register
%% ===== \end{flushleft}
%% ===== 
%% ===== 
%% ===== 
%% ===== 
%% ===== 
%% ===== \begin{flushleft}
%% ===== Function
%% ===== \end{flushleft}
%% ===== 
%% ===== 
%% ===== 
%% ===== 
%% ===== 
%% ===== \begin{flushleft}
%% ===== a
%% ===== \end{flushleft}
%% ===== 
%% ===== 
%% ===== 
%% ===== 
%% ===== 
%% ===== \begin{flushleft}
%% ===== ILL OP
%% ===== \end{flushleft}
%% ===== 
%% ===== 
%% ===== 
%% ===== 
%% ===== 
%% ===== \begin{flushleft}
%% ===== An illegal operation code has been detected.
%% ===== \end{flushleft}
%% ===== 
%% ===== 
%% ===== 
%% ===== 
%% ===== 
%% ===== \begin{flushleft}
%% ===== b
%% ===== \end{flushleft}
%% ===== 
%% ===== 
%% ===== 
%% ===== 
%% ===== 
%% ===== \begin{flushleft}
%% ===== ILL MOD
%% ===== \end{flushleft}
%% ===== 
%% ===== 
%% ===== 
%% ===== 
%% ===== 
%% ===== \begin{flushleft}
%% ===== An illegal address modifier has been detected.
%% ===== \end{flushleft}
%% ===== 
%% ===== 
%% ===== 
%% ===== 
%% ===== 
%% ===== \begin{flushleft}
%% ===== c
%% ===== \end{flushleft}
%% ===== 
%% ===== 
%% ===== 
%% ===== 
%% ===== 
%% ===== \begin{flushleft}
%% ===== ILL SLV
%% ===== \end{flushleft}
%% ===== 
%% ===== 
%% ===== 
%% ===== 
%% ===== 
%% ===== \begin{flushleft}
%% ===== An illegal BAR mode procedure has been encountered.
%% ===== \end{flushleft}
%% ===== 
%% ===== 
%% ===== 
%% ===== 
%% ===== 
%% ===== \begin{flushleft}
%% ===== \newpage
%% ===== Flag or
%% ===== \end{flushleft}
%% ===== 
%% ===== 
%% ===== \begin{flushleft}
%% ===== key register
%% ===== \end{flushleft}
%% ===== 
%% ===== 
%% ===== 
%% ===== 
%% ===== 
%% ===== \begin{flushleft}
%% ===== Function
%% ===== \end{flushleft}
%% ===== 
%% ===== 
%% ===== 
%% ===== 
%% ===== 
%% ===== \begin{flushleft}
%% ===== d
%% ===== \end{flushleft}
%% ===== 
%% ===== 
%% ===== 
%% ===== 
%% ===== 
%% ===== \begin{flushleft}
%% ===== ILL PROC
%% ===== \end{flushleft}
%% ===== 
%% ===== 
%% ===== 
%% ===== 
%% ===== 
%% ===== \begin{flushleft}
%% ===== An illegal procedure other than the three above has been
%% ===== \end{flushleft}
%% ===== 
%% ===== 
%% ===== \begin{flushleft}
%% ===== encountered.
%% ===== \end{flushleft}
%% ===== 
%% ===== 
%% ===== 
%% ===== 
%% ===== 
%% ===== \begin{flushleft}
%% ===== e
%% ===== \end{flushleft}
%% ===== 
%% ===== 
%% ===== 
%% ===== 
%% ===== 
%% ===== \begin{flushleft}
%% ===== NEM
%% ===== \end{flushleft}
%% ===== 
%% ===== 
%% ===== 
%% ===== 
%% ===== 
%% ===== \begin{flushleft}
%% ===== A nonexistent main memory address has been requested.
%% ===== \end{flushleft}
%% ===== 
%% ===== 
%% ===== 
%% ===== 
%% ===== 
%% ===== \begin{flushleft}
%% ===== f
%% ===== \end{flushleft}
%% ===== 
%% ===== 
%% ===== 
%% ===== 
%% ===== 
%% ===== \begin{flushleft}
%% ===== OOB
%% ===== \end{flushleft}
%% ===== 
%% ===== 
%% ===== 
%% ===== 
%% ===== 
%% ===== \begin{flushleft}
%% ===== A BAR mode boundary violation has occurred.
%% ===== \end{flushleft}
%% ===== 
%% ===== 
%% ===== 
%% ===== 
%% ===== 
%% ===== \begin{flushleft}
%% ===== g
%% ===== \end{flushleft}
%% ===== 
%% ===== 
%% ===== 
%% ===== 
%% ===== 
%% ===== \begin{flushleft}
%% ===== ILL DIG
%% ===== \end{flushleft}
%% ===== 
%% ===== 
%% ===== 
%% ===== 
%% ===== 
%% ===== \begin{flushleft}
%% ===== An illegal decimal digit or sign has been detected by the decimal
%% ===== \end{flushleft}
%% ===== 
%% ===== 
%% ===== \begin{flushleft}
%% ===== unit.
%% ===== \end{flushleft}
%% ===== 
%% ===== 
%% ===== 
%% ===== 
%% ===== 
%% ===== \begin{flushleft}
%% ===== h
%% ===== \end{flushleft}
%% ===== 
%% ===== 
%% ===== 
%% ===== 
%% ===== 
%% ===== \begin{flushleft}
%% ===== PROC PARU
%% ===== \end{flushleft}
%% ===== 
%% ===== 
%% ===== 
%% ===== 
%% ===== 
%% ===== \begin{flushleft}
%% ===== A parity error has been detected in the upper 36 bits of data.
%% ===== \end{flushleft}
%% ===== 
%% ===== 
%% ===== 
%% ===== 
%% ===== 
%% ===== \begin{flushleft}
%% ===== i
%% ===== \end{flushleft}
%% ===== 
%% ===== 
%% ===== 
%% ===== 
%% ===== 
%% ===== \begin{flushleft}
%% ===== PROC PARL
%% ===== \end{flushleft}
%% ===== 
%% ===== 
%% ===== 
%% ===== 
%% ===== 
%% ===== \begin{flushleft}
%% ===== A parity error has been detected in the lower 36 bits of data.
%% ===== \end{flushleft}
%% ===== 
%% ===== 
%% ===== 
%% ===== 
%% ===== 
%% ===== \begin{flushleft}
%% ===== j
%% ===== \end{flushleft}
%% ===== 
%% ===== 
%% ===== 
%% ===== 
%% ===== 
%% ===== \begin{flushleft}
%% ===== \$CON A
%% ===== \end{flushleft}
%% ===== 
%% ===== 
%% ===== 
%% ===== 
%% ===== 
%% ===== \begin{flushleft}
%% ===== A \$CONNECT signal has been received through port A.
%% ===== \end{flushleft}
%% ===== 
%% ===== 
%% ===== 
%% ===== 
%% ===== 
%% ===== \begin{flushleft}
%% ===== k
%% ===== \end{flushleft}
%% ===== 
%% ===== 
%% ===== 
%% ===== 
%% ===== 
%% ===== \begin{flushleft}
%% ===== \$CON B
%% ===== \end{flushleft}
%% ===== 
%% ===== 
%% ===== 
%% ===== 
%% ===== 
%% ===== \begin{flushleft}
%% ===== A \$CONNECT signal has been received through port B.
%% ===== \end{flushleft}
%% ===== 
%% ===== 
%% ===== 
%% ===== 
%% ===== 
%% ===== 1
%% ===== 
%% ===== 
%% ===== 
%% ===== 
%% ===== 
%% ===== \begin{flushleft}
%% ===== \$CON C
%% ===== \end{flushleft}
%% ===== 
%% ===== 
%% ===== 
%% ===== 
%% ===== 
%% ===== \begin{flushleft}
%% ===== A \$CONNECT signal has been received through port C.
%% ===== \end{flushleft}
%% ===== 
%% ===== 
%% ===== 
%% ===== 
%% ===== 
%% ===== \begin{flushleft}
%% ===== m
%% ===== \end{flushleft}
%% ===== 
%% ===== 
%% ===== 
%% ===== 
%% ===== 
%% ===== \begin{flushleft}
%% ===== \$CON D
%% ===== \end{flushleft}
%% ===== 
%% ===== 
%% ===== 
%% ===== 
%% ===== 
%% ===== \begin{flushleft}
%% ===== A \$CONNECT signal has been received through port D.
%% ===== \end{flushleft}
%% ===== 
%% ===== 
%% ===== 
%% ===== 
%% ===== 
%% ===== \begin{flushleft}
%% ===== n
%% ===== \end{flushleft}
%% ===== 
%% ===== 
%% ===== 
%% ===== 
%% ===== 
%% ===== \begin{flushleft}
%% ===== DA ERR1
%% ===== \end{flushleft}
%% ===== 
%% ===== 
%% ===== 
%% ===== 
%% ===== 
%% ===== \begin{flushleft}
%% ===== Operation not complete. Processor/system controller interface
%% ===== \end{flushleft}
%% ===== 
%% ===== 
%% ===== \begin{flushleft}
%% ===== sequence error 1 has been detected. (\$DATA-AVAIL received with
%% ===== \end{flushleft}
%% ===== 
%% ===== 
%% ===== \begin{flushleft}
%% ===== no prior \$INTERRUPT sent.)
%% ===== \end{flushleft}
%% ===== 
%% ===== 
%% ===== 
%% ===== 
%% ===== 
%% ===== \begin{flushleft}
%% ===== o
%% ===== \end{flushleft}
%% ===== 
%% ===== 
%% ===== 
%% ===== 
%% ===== 
%% ===== \begin{flushleft}
%% ===== DA ERR2
%% ===== \end{flushleft}
%% ===== 
%% ===== 
%% ===== 
%% ===== 
%% ===== 
%% ===== \begin{flushleft}
%% ===== Operation not complete. Processor/system controller interface
%% ===== \end{flushleft}
%% ===== 
%% ===== 
%% ===== \begin{flushleft}
%% ===== sequence error 2 has been detected. (Multiple \$DATA-AVAIL
%% ===== \end{flushleft}
%% ===== 
%% ===== 
%% ===== \begin{flushleft}
%% ===== received or \$DATA-AVAIL received out of order.)
%% ===== \end{flushleft}
%% ===== 
%% ===== 
%% ===== 
%% ===== 
%% ===== 
%% ===== \begin{flushleft}
%% ===== IAA
%% ===== \end{flushleft}
%% ===== 
%% ===== 
%% ===== 
%% ===== 
%% ===== 
%% ===== \begin{flushleft}
%% ===== Coded illegal action, port A. (see Table 3-2)
%% ===== \end{flushleft}
%% ===== 
%% ===== 
%% ===== 
%% ===== 
%% ===== 
%% ===== \begin{flushleft}
%% ===== IAB
%% ===== \end{flushleft}
%% ===== 
%% ===== 
%% ===== 
%% ===== 
%% ===== 
%% ===== \begin{flushleft}
%% ===== Coded illegal action, port B. (See Table 3-2)
%% ===== \end{flushleft}
%% ===== 
%% ===== 
%% ===== 
%% ===== 
%% ===== 
%% ===== \begin{flushleft}
%% ===== IAC
%% ===== \end{flushleft}
%% ===== 
%% ===== 
%% ===== 
%% ===== 
%% ===== 
%% ===== \begin{flushleft}
%% ===== Coded illegal action, port C. (See Table 3-2)
%% ===== \end{flushleft}
%% ===== 
%% ===== 
%% ===== 
%% ===== 
%% ===== 
%% ===== \begin{flushleft}
%% ===== IAD
%% ===== \end{flushleft}
%% ===== 
%% ===== 
%% ===== 
%% ===== 
%% ===== 
%% ===== \begin{flushleft}
%% ===== Coded illegal action, port D. (See Table 3-2)
%% ===== \end{flushleft}
%% ===== 
%% ===== 
%% ===== 
%% ===== 
%% ===== 
%% ===== \begin{flushleft}
%% ===== p
%% ===== \end{flushleft}
%% ===== 
%% ===== 
%% ===== 
%% ===== 
%% ===== 
%% ===== \begin{flushleft}
%% ===== CPAR DIR
%% ===== \end{flushleft}
%% ===== 
%% ===== 
%% ===== 
%% ===== 
%% ===== 
%% ===== \begin{flushleft}
%% ===== A parity error has been detected in the cache memory directory.
%% ===== \end{flushleft}
%% ===== 
%% ===== 
%% ===== 
%% ===== 
%% ===== 
%% ===== \begin{flushleft}
%% ===== q
%% ===== \end{flushleft}
%% ===== 
%% ===== 
%% ===== 
%% ===== 
%% ===== 
%% ===== \begin{flushleft}
%% ===== CPAR STR
%% ===== \end{flushleft}
%% ===== 
%% ===== 
%% ===== 
%% ===== 
%% ===== 
%% ===== \begin{flushleft}
%% ===== A data parity error has been detected in the cache memory.
%% ===== \end{flushleft}
%% ===== 
%% ===== 
%% ===== 
%% ===== 
%% ===== 
%% ===== \begin{flushleft}
%% ===== r
%% ===== \end{flushleft}
%% ===== 
%% ===== 
%% ===== 
%% ===== 
%% ===== 
%% ===== \begin{flushleft}
%% ===== CPAR IA
%% ===== \end{flushleft}
%% ===== 
%% ===== 
%% ===== 
%% ===== 
%% ===== 
%% ===== \begin{flushleft}
%% ===== An illegal action has been received from a system controller during
%% ===== \end{flushleft}
%% ===== 
%% ===== 
%% ===== \begin{flushleft}
%% ===== a store operation with cache memory enabled. This implies that the
%% ===== \end{flushleft}
%% ===== 
%% ===== 
%% ===== \begin{flushleft}
%% ===== data are correct in cache memory and incorrect in main memory.
%% ===== \end{flushleft}
%% ===== 
%% ===== 
%% ===== 
%% ===== 
%% ===== 
%% ===== \begin{flushleft}
%% ===== s
%% ===== \end{flushleft}
%% ===== 
%% ===== 
%% ===== 
%% ===== 
%% ===== 
%% ===== \begin{flushleft}
%% ===== CPAR BLK
%% ===== \end{flushleft}
%% ===== 
%% ===== 
%% ===== 
%% ===== 
%% ===== 
%% ===== \begin{flushleft}
%% ===== A cache memory parity error has occurred during a cache memory
%% ===== \end{flushleft}
%% ===== 
%% ===== 
%% ===== \begin{flushleft}
%% ===== data block load.
%% ===== \end{flushleft}
%% ===== 
%% ===== 
%% ===== 
%% ===== 
%% ===== 
%% ===== \begin{flushleft}
%% ===== Table 3-2. System Controller Illegal Action Codes
%% ===== \end{flushleft}
%% ===== 
%% ===== 
%% ===== \begin{flushleft}
%% ===== Code
%% ===== \end{flushleft}
%% ===== 
%% ===== 
%% ===== 
%% ===== 
%% ===== 
%% ===== \begin{flushleft}
%% ===== Priority Fault
%% ===== \end{flushleft}
%% ===== 
%% ===== 
%% ===== 
%% ===== 
%% ===== 
%% ===== \begin{flushleft}
%% ===== Reason
%% ===== \end{flushleft}
%% ===== 
%% ===== 
%% ===== 
%% ===== 
%% ===== 
%% ===== 00
%% ===== 
%% ===== 
%% ===== 
%% ===== 
%% ===== 
%% ===== --
%% ===== 
%% ===== 
%% ===== 
%% ===== 
%% ===== 
%% ===== \begin{flushleft}
%% ===== No illegal action
%% ===== \end{flushleft}
%% ===== 
%% ===== 
%% ===== 
%% ===== 
%% ===== 
%% ===== 01
%% ===== 
%% ===== 
%% ===== 
%% ===== 
%% ===== 
%% ===== --
%% ===== 
%% ===== 
%% ===== 
%% ===== 
%% ===== 
%% ===== \begin{flushleft}
%% ===== Command
%% ===== \end{flushleft}
%% ===== 
%% ===== 
%% ===== 
%% ===== 
%% ===== 
%% ===== \begin{flushleft}
%% ===== Unassigned
%% ===== \end{flushleft}
%% ===== 
%% ===== 
%% ===== 
%% ===== 
%% ===== 
%% ===== 02
%% ===== 
%% ===== 
%% ===== 
%% ===== 
%% ===== 
%% ===== 05
%% ===== 
%% ===== 
%% ===== 
%% ===== 
%% ===== 
%% ===== \begin{flushleft}
%% ===== Store
%% ===== \end{flushleft}
%% ===== 
%% ===== 
%% ===== 
%% ===== 
%% ===== 
%% ===== \begin{flushleft}
%% ===== Nonexistent address
%% ===== \end{flushleft}
%% ===== 
%% ===== 
%% ===== 
%% ===== 
%% ===== 
%% ===== 03
%% ===== 
%% ===== 
%% ===== 
%% ===== 
%% ===== 
%% ===== 01
%% ===== 
%% ===== 
%% ===== 
%% ===== 
%% ===== 
%% ===== \begin{flushleft}
%% ===== Command
%% ===== \end{flushleft}
%% ===== 
%% ===== 
%% ===== 
%% ===== 
%% ===== 
%% ===== \begin{flushleft}
%% ===== Stop on condition
%% ===== \end{flushleft}
%% ===== 
%% ===== 
%% ===== 
%% ===== 
%% ===== 
%% ===== 04
%% ===== 
%% ===== 
%% ===== 
%% ===== 
%% ===== 
%% ===== --
%% ===== 
%% ===== 
%% ===== 
%% ===== 
%% ===== 
%% ===== \begin{flushleft}
%% ===== Command
%% ===== \end{flushleft}
%% ===== 
%% ===== 
%% ===== 
%% ===== 
%% ===== 
%% ===== \begin{flushleft}
%% ===== Unassigned
%% ===== \end{flushleft}
%% ===== 
%% ===== 
%% ===== 
%% ===== 
%% ===== 
%% ===== 05
%% ===== 
%% ===== 
%% ===== 
%% ===== 
%% ===== 
%% ===== 12
%% ===== 
%% ===== 
%% ===== 
%% ===== 
%% ===== 
%% ===== \begin{flushleft}
%% ===== Parity
%% ===== \end{flushleft}
%% ===== 
%% ===== 
%% ===== 
%% ===== 
%% ===== 
%% ===== \begin{flushleft}
%% ===== Data parity, store unit to system controller
%% ===== \end{flushleft}
%% ===== 
%% ===== 
%% ===== 
%% ===== 
%% ===== 
%% ===== 06
%% ===== 
%% ===== 
%% ===== 
%% ===== 
%% ===== 
%% ===== 11
%% ===== 
%% ===== 
%% ===== 
%% ===== 
%% ===== 
%% ===== \begin{flushleft}
%% ===== Parity
%% ===== \end{flushleft}
%% ===== 
%% ===== 
%% ===== 
%% ===== 
%% ===== 
%% ===== \begin{flushleft}
%% ===== Data parity in store unit
%% ===== \end{flushleft}
%% ===== 
%% ===== 
%% ===== 
%% ===== 
%% ===== 
%% ===== 07
%% ===== 
%% ===== 
%% ===== 
%% ===== 
%% ===== 
%% ===== 10
%% ===== 
%% ===== 
%% ===== 
%% ===== 
%% ===== 
%% ===== \begin{flushleft}
%% ===== Parity
%% ===== \end{flushleft}
%% ===== 
%% ===== 
%% ===== 
%% ===== 
%% ===== 
%% ===== \begin{flushleft}
%% ===== Data parity in store unit and store unit to system controller
%% ===== \end{flushleft}
%% ===== 
%% ===== 
%% ===== 
%% ===== 
%% ===== 
%% ===== 10
%% ===== 
%% ===== 
%% ===== 
%% ===== 
%% ===== 
%% ===== 04
%% ===== 
%% ===== 
%% ===== 
%% ===== 
%% ===== 
%% ===== \begin{flushleft}
%% ===== Command
%% ===== \end{flushleft}
%% ===== 
%% ===== 
%% ===== 
%% ===== 
%% ===== 
%% ===== \begin{flushleft}
%% ===== Not control
%% ===== \end{flushleft}
%% ===== 
%% ===== 
%% ===== 
%% ===== 
%% ===== 
%% ===== \begin{flushleft}
%% ===== (a)
%% ===== \end{flushleft}
%% ===== 
%% ===== 
%% ===== 
%% ===== 
%% ===== 
%% ===== \begin{flushleft}
%% ===== \newpage
%% ===== Code
%% ===== \end{flushleft}
%% ===== 
%% ===== 
%% ===== 
%% ===== 
%% ===== 
%% ===== \begin{flushleft}
%% ===== Priority Fault
%% ===== \end{flushleft}
%% ===== 
%% ===== 
%% ===== 
%% ===== 
%% ===== 
%% ===== \begin{flushleft}
%% ===== Reason
%% ===== \end{flushleft}
%% ===== 
%% ===== 
%% ===== 
%% ===== 
%% ===== 
%% ===== 11
%% ===== 
%% ===== 
%% ===== 
%% ===== 
%% ===== 
%% ===== 13
%% ===== 
%% ===== 
%% ===== 
%% ===== 
%% ===== 
%% ===== \begin{flushleft}
%% ===== Command
%% ===== \end{flushleft}
%% ===== 
%% ===== 
%% ===== 
%% ===== 
%% ===== 
%% ===== \begin{flushleft}
%% ===== Port not enabled
%% ===== \end{flushleft}
%% ===== 
%% ===== 
%% ===== 
%% ===== 
%% ===== 
%% ===== 12
%% ===== 
%% ===== 
%% ===== 
%% ===== 
%% ===== 
%% ===== 03
%% ===== 
%% ===== 
%% ===== 
%% ===== 
%% ===== 
%% ===== \begin{flushleft}
%% ===== Command
%% ===== \end{flushleft}
%% ===== 
%% ===== 
%% ===== 
%% ===== 
%% ===== 
%% ===== \begin{flushleft}
%% ===== Illegal command
%% ===== \end{flushleft}
%% ===== 
%% ===== 
%% ===== 
%% ===== 
%% ===== 
%% ===== 13
%% ===== 
%% ===== 
%% ===== 
%% ===== 
%% ===== 
%% ===== 07
%% ===== 
%% ===== 
%% ===== 
%% ===== 
%% ===== 
%% ===== \begin{flushleft}
%% ===== Store
%% ===== \end{flushleft}
%% ===== 
%% ===== 
%% ===== 
%% ===== 
%% ===== 
%% ===== \begin{flushleft}
%% ===== Store unit not ready
%% ===== \end{flushleft}
%% ===== 
%% ===== 
%% ===== 
%% ===== 
%% ===== 
%% ===== 14
%% ===== 
%% ===== 
%% ===== 
%% ===== 
%% ===== 
%% ===== 02
%% ===== 
%% ===== 
%% ===== 
%% ===== 
%% ===== 
%% ===== \begin{flushleft}
%% ===== Parity
%% ===== \end{flushleft}
%% ===== 
%% ===== 
%% ===== 
%% ===== 
%% ===== 
%% ===== \begin{flushleft}
%% ===== Zone-address-command parity, processor to system controller
%% ===== \end{flushleft}
%% ===== 
%% ===== 
%% ===== 
%% ===== 
%% ===== 
%% ===== 15
%% ===== 
%% ===== 
%% ===== 
%% ===== 
%% ===== 
%% ===== 06
%% ===== 
%% ===== 
%% ===== 
%% ===== 
%% ===== 
%% ===== \begin{flushleft}
%% ===== Parity
%% ===== \end{flushleft}
%% ===== 
%% ===== 
%% ===== 
%% ===== 
%% ===== 
%% ===== \begin{flushleft}
%% ===== Data parity, processor to system controller
%% ===== \end{flushleft}
%% ===== 
%% ===== 
%% ===== 
%% ===== 
%% ===== 
%% ===== 16
%% ===== 
%% ===== 
%% ===== 
%% ===== 
%% ===== 
%% ===== 08
%% ===== 
%% ===== 
%% ===== 
%% ===== 
%% ===== 
%% ===== \begin{flushleft}
%% ===== Parity
%% ===== \end{flushleft}
%% ===== 
%% ===== 
%% ===== 
%% ===== 
%% ===== 
%% ===== \begin{flushleft}
%% ===== Zone-address-command parity, system controller to store unit
%% ===== \end{flushleft}
%% ===== 
%% ===== 
%% ===== 
%% ===== 
%% ===== 
%% ===== 17
%% ===== 
%% ===== 
%% ===== 
%% ===== 
%% ===== 
%% ===== 09
%% ===== 
%% ===== 
%% ===== 
%% ===== 
%% ===== 
%% ===== \begin{flushleft}
%% ===== Parity
%% ===== \end{flushleft}
%% ===== 
%% ===== 
%% ===== 
%% ===== 
%% ===== 
%% ===== \begin{flushleft}
%% ===== Data parity, system controller to store unit
%% ===== \end{flushleft}
%% ===== 
%% ===== 
%% ===== 
%% ===== 
%% ===== 
%% ===== \begin{flushleft}
%% ===== (a) This illegal action code not relevant to later model system controllers.
%% ===== \end{flushleft}
%% ===== 
%% ===== 
%% ===== 
%% ===== 
%% ===== 
%% ===== \begin{flushleft}

\subsection{FAULT REGISTER (FR) - DPS 8M}

%% ===== \end{flushleft}
%% ===== 
%% ===== 
%% ===== \begin{flushleft}
%% ===== Format: - 72 bits
%% ===== \end{flushleft}
%% ===== 
%% ===== 
%% ===== \begin{flushleft}
%% ===== Even word of Y-pair as stored by Store Central Processor Register (scpr), TAG = 01
%% ===== \end{flushleft}
%% ===== 
%% ===== 
%% ===== 0 0 0 0 0 0 0 0 0 0 1 1 1 1 1 1 1
%% ===== 
%% ===== 
%% ===== 0 1 2 3 4 5 6 7 8 9 0 1 2 3 4 5 6
%% ===== 
%% ===== 
%% ===== \begin{flushleft}
%% ===== a b c d e f g h i
%% ===== \end{flushleft}
%% ===== 
%% ===== 
%% ===== 
%% ===== 
%% ===== 
%% ===== \begin{flushleft}
%% ===== j k l m n o 0
%% ===== \end{flushleft}
%% ===== 
%% ===== 
%% ===== 
%% ===== 
%% ===== 
%% ===== 1 1 1 1 1 1 1 1 1 1 1 1 1 1 1 1
%% ===== 
%% ===== 
%% ===== 
%% ===== 
%% ===== 
%% ===== 1 2
%% ===== 
%% ===== 
%% ===== 9 0
%% ===== 
%% ===== 
%% ===== \begin{flushleft}
%% ===== IAA
%% ===== \end{flushleft}
%% ===== 
%% ===== 
%% ===== 
%% ===== 
%% ===== 
%% ===== 2 2
%% ===== 
%% ===== 
%% ===== 3 4
%% ===== 
%% ===== 
%% ===== \begin{flushleft}
%% ===== IAB
%% ===== \end{flushleft}
%% ===== 
%% ===== 
%% ===== 
%% ===== 
%% ===== 
%% ===== 4
%% ===== 
%% ===== 
%% ===== 
%% ===== 
%% ===== 
%% ===== 2 2
%% ===== 
%% ===== 
%% ===== 7 8
%% ===== 
%% ===== 
%% ===== \begin{flushleft}
%% ===== IAC
%% ===== \end{flushleft}
%% ===== 
%% ===== 
%% ===== 
%% ===== 
%% ===== 
%% ===== 4
%% ===== 
%% ===== 
%% ===== 
%% ===== 
%% ===== 
%% ===== 3 3 3 3 3
%% ===== 
%% ===== 
%% ===== 1 2 3 4 5
%% ===== 
%% ===== 
%% ===== \begin{flushleft}
%% ===== IAD
%% ===== \end{flushleft}
%% ===== 
%% ===== 
%% ===== 
%% ===== 
%% ===== 
%% ===== 4
%% ===== 
%% ===== 
%% ===== 
%% ===== 
%% ===== 
%% ===== \begin{flushleft}
%% ===== p q r s
%% ===== \end{flushleft}
%% ===== 
%% ===== 
%% ===== 4 1 1 1 1
%% ===== 
%% ===== 
%% ===== 
%% ===== 
%% ===== 
%% ===== \begin{flushleft}
%% ===== Odd word of Y-pair as stored by Store Central Processor Register (scpr), TAG = 01
%% ===== \end{flushleft}
%% ===== 
%% ===== 
%% ===== 3 3 3 3 4 4 4 4 4 4 4 4 4
%% ===== 
%% ===== 
%% ===== 6 7 8 9 0 1 2 3 4 5 6 7 8
%% ===== 
%% ===== 
%% ===== 
%% ===== 
%% ===== 
%% ===== 7
%% ===== 
%% ===== 
%% ===== 1
%% ===== 
%% ===== 
%% ===== 
%% ===== 
%% ===== 
%% ===== \begin{flushleft}
%% ===== t u v w x y z A B C D E F 0 0 0 0 0 0 0 0 0 0 0 0 0 0 0 0 0 0 0 0 0 0 0
%% ===== \end{flushleft}
%% ===== 
%% ===== 
%% ===== 1 1 1 1 1 1 1 1 1 1 1 1 1
%% ===== 
%% ===== 
%% ===== 
%% ===== 
%% ===== 
%% ===== 25
%% ===== 
%% ===== 
%% ===== 
%% ===== 
%% ===== 
%% ===== \begin{flushleft}
%% ===== Figure 3-19. Fault Register (FR) Format - DPS 8M
%% ===== \end{flushleft}
%% ===== 
%% ===== 
%% ===== \begin{flushleft}
%% ===== Function:
%% ===== \end{flushleft}
%% ===== 
%% ===== 
%% ===== \begin{flushleft}
%% ===== The Fault Register contains the conditions in the processor for several of the hardware
%% ===== \end{flushleft}
%% ===== 
%% ===== 
%% ===== \begin{flushleft}
%% ===== faults on the DPS 8M CPU and cache directory buffer overflows. Data is strobed into the
%% ===== \end{flushleft}
%% ===== 
%% ===== 
%% ===== \begin{flushleft}
%% ===== Fault Register during a fault or buffer overflow fault sequence. Once a bit or field in the
%% ===== \end{flushleft}
%% ===== 
%% ===== 
%% ===== \begin{flushleft}
%% ===== Fault Register is set, it remains set until the register is stored and cleared. The data is not
%% ===== \end{flushleft}
%% ===== 
%% ===== 
%% ===== \begin{flushleft}
%% ===== overwritten during subsequent fault events.
%% ===== \end{flushleft}
%% ===== 
%% ===== 
%% ===== \begin{flushleft}
%% ===== The functions of the constituent flags and registers are:
%% ===== \end{flushleft}
%% ===== 
%% ===== 
%% ===== 
%% ===== 
%% ===== 
%% ===== \begin{flushleft}
%% ===== Flag or
%% ===== \end{flushleft}
%% ===== 
%% ===== 
%% ===== \begin{flushleft}
%% ===== key register
%% ===== \end{flushleft}
%% ===== 
%% ===== 
%% ===== 
%% ===== 
%% ===== 
%% ===== \begin{flushleft}
%% ===== Fault
%% ===== \end{flushleft}
%% ===== 
%% ===== 
%% ===== 
%% ===== 
%% ===== 
%% ===== \begin{flushleft}
%% ===== Function
%% ===== \end{flushleft}
%% ===== 
%% ===== 
%% ===== 
%% ===== 
%% ===== 
%% ===== \begin{flushleft}
%% ===== a
%% ===== \end{flushleft}
%% ===== 
%% ===== 
%% ===== 
%% ===== 
%% ===== 
%% ===== \begin{flushleft}
%% ===== ILL OP
%% ===== \end{flushleft}
%% ===== 
%% ===== 
%% ===== 
%% ===== 
%% ===== 
%% ===== \begin{flushleft}
%% ===== IPR
%% ===== \end{flushleft}
%% ===== 
%% ===== 
%% ===== 
%% ===== 
%% ===== 
%% ===== \begin{flushleft}
%% ===== An illegal operation code has been detected.
%% ===== \end{flushleft}
%% ===== 
%% ===== 
%% ===== 
%% ===== 
%% ===== 
%% ===== \begin{flushleft}
%% ===== b
%% ===== \end{flushleft}
%% ===== 
%% ===== 
%% ===== 
%% ===== 
%% ===== 
%% ===== \begin{flushleft}
%% ===== ILL MOD
%% ===== \end{flushleft}
%% ===== 
%% ===== 
%% ===== 
%% ===== 
%% ===== 
%% ===== \begin{flushleft}
%% ===== IPR
%% ===== \end{flushleft}
%% ===== 
%% ===== 
%% ===== 
%% ===== 
%% ===== 
%% ===== \begin{flushleft}
%% ===== An illegal address modifier has been detected.
%% ===== \end{flushleft}
%% ===== 
%% ===== 
%% ===== 
%% ===== 
%% ===== 
%% ===== \begin{flushleft}
%% ===== c
%% ===== \end{flushleft}
%% ===== 
%% ===== 
%% ===== 
%% ===== 
%% ===== 
%% ===== \begin{flushleft}
%% ===== ILL SLV
%% ===== \end{flushleft}
%% ===== 
%% ===== 
%% ===== 
%% ===== 
%% ===== 
%% ===== \begin{flushleft}
%% ===== IPR
%% ===== \end{flushleft}
%% ===== 
%% ===== 
%% ===== 
%% ===== 
%% ===== 
%% ===== \begin{flushleft}
%% ===== An illegal BAR mode procedure has been encountered.
%% ===== \end{flushleft}
%% ===== 
%% ===== 
%% ===== 
%% ===== 
%% ===== 
%% ===== \begin{flushleft}
%% ===== \newpage
%% ===== Flag or
%% ===== \end{flushleft}
%% ===== 
%% ===== 
%% ===== \begin{flushleft}
%% ===== key register
%% ===== \end{flushleft}
%% ===== 
%% ===== 
%% ===== 
%% ===== 
%% ===== 
%% ===== \begin{flushleft}
%% ===== Fault
%% ===== \end{flushleft}
%% ===== 
%% ===== 
%% ===== 
%% ===== 
%% ===== 
%% ===== \begin{flushleft}
%% ===== Function
%% ===== \end{flushleft}
%% ===== 
%% ===== 
%% ===== 
%% ===== 
%% ===== 
%% ===== \begin{flushleft}
%% ===== d
%% ===== \end{flushleft}
%% ===== 
%% ===== 
%% ===== 
%% ===== 
%% ===== 
%% ===== \begin{flushleft}
%% ===== ILL PROC
%% ===== \end{flushleft}
%% ===== 
%% ===== 
%% ===== 
%% ===== 
%% ===== 
%% ===== \begin{flushleft}
%% ===== IPR
%% ===== \end{flushleft}
%% ===== 
%% ===== 
%% ===== 
%% ===== 
%% ===== 
%% ===== \begin{flushleft}
%% ===== An illegal procedure other than the three above has been
%% ===== \end{flushleft}
%% ===== 
%% ===== 
%% ===== \begin{flushleft}
%% ===== encountered.
%% ===== \end{flushleft}
%% ===== 
%% ===== 
%% ===== 
%% ===== 
%% ===== 
%% ===== \begin{flushleft}
%% ===== e
%% ===== \end{flushleft}
%% ===== 
%% ===== 
%% ===== 
%% ===== 
%% ===== 
%% ===== \begin{flushleft}
%% ===== NEM
%% ===== \end{flushleft}
%% ===== 
%% ===== 
%% ===== 
%% ===== 
%% ===== 
%% ===== \begin{flushleft}
%% ===== ONC
%% ===== \end{flushleft}
%% ===== 
%% ===== 
%% ===== 
%% ===== 
%% ===== 
%% ===== \begin{flushleft}
%% ===== A nonexistent main memory address has been requested.
%% ===== \end{flushleft}
%% ===== 
%% ===== 
%% ===== 
%% ===== 
%% ===== 
%% ===== \begin{flushleft}
%% ===== f
%% ===== \end{flushleft}
%% ===== 
%% ===== 
%% ===== 
%% ===== 
%% ===== 
%% ===== \begin{flushleft}
%% ===== OOB
%% ===== \end{flushleft}
%% ===== 
%% ===== 
%% ===== 
%% ===== 
%% ===== 
%% ===== \begin{flushleft}
%% ===== STR
%% ===== \end{flushleft}
%% ===== 
%% ===== 
%% ===== 
%% ===== 
%% ===== 
%% ===== \begin{flushleft}
%% ===== A BAR mode boundary violation has occurred.
%% ===== \end{flushleft}
%% ===== 
%% ===== 
%% ===== 
%% ===== 
%% ===== 
%% ===== \begin{flushleft}
%% ===== g
%% ===== \end{flushleft}
%% ===== 
%% ===== 
%% ===== 
%% ===== 
%% ===== 
%% ===== \begin{flushleft}
%% ===== ILL DIG
%% ===== \end{flushleft}
%% ===== 
%% ===== 
%% ===== 
%% ===== 
%% ===== 
%% ===== \begin{flushleft}
%% ===== IPR
%% ===== \end{flushleft}
%% ===== 
%% ===== 
%% ===== 
%% ===== 
%% ===== 
%% ===== \begin{flushleft}
%% ===== An illegal decimal digit or sign has been detected by the
%% ===== \end{flushleft}
%% ===== 
%% ===== 
%% ===== \begin{flushleft}
%% ===== decimal unit.
%% ===== \end{flushleft}
%% ===== 
%% ===== 
%% ===== 
%% ===== 
%% ===== 
%% ===== \begin{flushleft}
%% ===== h
%% ===== \end{flushleft}
%% ===== 
%% ===== 
%% ===== 
%% ===== 
%% ===== 
%% ===== \begin{flushleft}
%% ===== PROC PARU
%% ===== \end{flushleft}
%% ===== 
%% ===== 
%% ===== 
%% ===== 
%% ===== 
%% ===== \begin{flushleft}
%% ===== PAR
%% ===== \end{flushleft}
%% ===== 
%% ===== 
%% ===== 
%% ===== 
%% ===== 
%% ===== \begin{flushleft}
%% ===== A parity error has been detected in the upper 36 bits of data.
%% ===== \end{flushleft}
%% ===== 
%% ===== 
%% ===== 
%% ===== 
%% ===== 
%% ===== \begin{flushleft}
%% ===== i
%% ===== \end{flushleft}
%% ===== 
%% ===== 
%% ===== 
%% ===== 
%% ===== 
%% ===== \begin{flushleft}
%% ===== PROC PARL
%% ===== \end{flushleft}
%% ===== 
%% ===== 
%% ===== 
%% ===== 
%% ===== 
%% ===== \begin{flushleft}
%% ===== PAR
%% ===== \end{flushleft}
%% ===== 
%% ===== 
%% ===== 
%% ===== 
%% ===== 
%% ===== \begin{flushleft}
%% ===== A parity error has been detected in the lower 36 bits of data.
%% ===== \end{flushleft}
%% ===== 
%% ===== 
%% ===== 
%% ===== 
%% ===== 
%% ===== \begin{flushleft}
%% ===== j
%% ===== \end{flushleft}
%% ===== 
%% ===== 
%% ===== 
%% ===== 
%% ===== 
%% ===== \begin{flushleft}
%% ===== \$CON A
%% ===== \end{flushleft}
%% ===== 
%% ===== 
%% ===== 
%% ===== 
%% ===== 
%% ===== \begin{flushleft}
%% ===== CON
%% ===== \end{flushleft}
%% ===== 
%% ===== 
%% ===== 
%% ===== 
%% ===== 
%% ===== \begin{flushleft}
%% ===== A \$CONNECT signal has been received through port A.
%% ===== \end{flushleft}
%% ===== 
%% ===== 
%% ===== 
%% ===== 
%% ===== 
%% ===== \begin{flushleft}
%% ===== k
%% ===== \end{flushleft}
%% ===== 
%% ===== 
%% ===== 
%% ===== 
%% ===== 
%% ===== \begin{flushleft}
%% ===== \$CON B
%% ===== \end{flushleft}
%% ===== 
%% ===== 
%% ===== 
%% ===== 
%% ===== 
%% ===== \begin{flushleft}
%% ===== CON
%% ===== \end{flushleft}
%% ===== 
%% ===== 
%% ===== 
%% ===== 
%% ===== 
%% ===== \begin{flushleft}
%% ===== A \$CONNECT signal has been received through port B.
%% ===== \end{flushleft}
%% ===== 
%% ===== 
%% ===== 
%% ===== 
%% ===== 
%% ===== \begin{flushleft}
%% ===== l
%% ===== \end{flushleft}
%% ===== 
%% ===== 
%% ===== 
%% ===== 
%% ===== 
%% ===== \begin{flushleft}
%% ===== \$CON C
%% ===== \end{flushleft}
%% ===== 
%% ===== 
%% ===== 
%% ===== 
%% ===== 
%% ===== \begin{flushleft}
%% ===== CON
%% ===== \end{flushleft}
%% ===== 
%% ===== 
%% ===== 
%% ===== 
%% ===== 
%% ===== \begin{flushleft}
%% ===== A \$CONNECT signal has been received through port C.
%% ===== \end{flushleft}
%% ===== 
%% ===== 
%% ===== 
%% ===== 
%% ===== 
%% ===== \begin{flushleft}
%% ===== m
%% ===== \end{flushleft}
%% ===== 
%% ===== 
%% ===== 
%% ===== 
%% ===== 
%% ===== \begin{flushleft}
%% ===== \$CON D
%% ===== \end{flushleft}
%% ===== 
%% ===== 
%% ===== 
%% ===== 
%% ===== 
%% ===== \begin{flushleft}
%% ===== CON
%% ===== \end{flushleft}
%% ===== 
%% ===== 
%% ===== 
%% ===== 
%% ===== 
%% ===== \begin{flushleft}
%% ===== A \$CONNECT signal has been received through port D.
%% ===== \end{flushleft}
%% ===== 
%% ===== 
%% ===== 
%% ===== 
%% ===== 
%% ===== \begin{flushleft}
%% ===== n
%% ===== \end{flushleft}
%% ===== 
%% ===== 
%% ===== 
%% ===== 
%% ===== 
%% ===== \begin{flushleft}
%% ===== DA ERR
%% ===== \end{flushleft}
%% ===== 
%% ===== 
%% ===== 
%% ===== 
%% ===== 
%% ===== \begin{flushleft}
%% ===== ONC
%% ===== \end{flushleft}
%% ===== 
%% ===== 
%% ===== 
%% ===== 
%% ===== 
%% ===== \begin{flushleft}
%% ===== Operation not complete.
%% ===== \end{flushleft}
%% ===== 
%% ===== 
%% ===== \begin{flushleft}
%% ===== Processor/system controller
%% ===== \end{flushleft}
%% ===== 
%% ===== 
%% ===== \begin{flushleft}
%% ===== interface sequence error 1 has been detected. (\$DATA-AVAIL
%% ===== \end{flushleft}
%% ===== 
%% ===== 
%% ===== \begin{flushleft}
%% ===== received with no prior \$INTERRUPT sent.)
%% ===== \end{flushleft}
%% ===== 
%% ===== 
%% ===== 
%% ===== 
%% ===== 
%% ===== \begin{flushleft}
%% ===== o
%% ===== \end{flushleft}
%% ===== 
%% ===== 
%% ===== 
%% ===== 
%% ===== 
%% ===== \begin{flushleft}
%% ===== DA ERR2
%% ===== \end{flushleft}
%% ===== 
%% ===== 
%% ===== 
%% ===== 
%% ===== 
%% ===== \begin{flushleft}
%% ===== ONC
%% ===== \end{flushleft}
%% ===== 
%% ===== 
%% ===== 
%% ===== 
%% ===== 
%% ===== \begin{flushleft}
%% ===== Operation not completed.
%% ===== \end{flushleft}
%% ===== 
%% ===== 
%% ===== \begin{flushleft}
%% ===== Processor/system controller
%% ===== \end{flushleft}
%% ===== 
%% ===== 
%% ===== \begin{flushleft}
%% ===== interface sequence error 2 has been detected. (Multiple
%% ===== \end{flushleft}
%% ===== 
%% ===== 
%% ===== \begin{flushleft}
%% ===== \$DATA-AVAIL received or \$DATA-AVAIL received out of
%% ===== \end{flushleft}
%% ===== 
%% ===== 
%% ===== \begin{flushleft}
%% ===== order.)
%% ===== \end{flushleft}
%% ===== 
%% ===== 
%% ===== 
%% ===== 
%% ===== 
%% ===== \begin{flushleft}
%% ===== IAA
%% ===== \end{flushleft}
%% ===== 
%% ===== 
%% ===== 
%% ===== 
%% ===== 
%% ===== \begin{flushleft}
%% ===== Coded illegal action, port A. (See Table 3-2)
%% ===== \end{flushleft}
%% ===== 
%% ===== 
%% ===== 
%% ===== 
%% ===== 
%% ===== \begin{flushleft}
%% ===== IAB
%% ===== \end{flushleft}
%% ===== 
%% ===== 
%% ===== 
%% ===== 
%% ===== 
%% ===== \begin{flushleft}
%% ===== Coded illegal action, port B. (See Table 3-2)
%% ===== \end{flushleft}
%% ===== 
%% ===== 
%% ===== 
%% ===== 
%% ===== 
%% ===== \begin{flushleft}
%% ===== IAC
%% ===== \end{flushleft}
%% ===== 
%% ===== 
%% ===== 
%% ===== 
%% ===== 
%% ===== \begin{flushleft}
%% ===== Coded illegal action, port C. (See Table 3-2)
%% ===== \end{flushleft}
%% ===== 
%% ===== 
%% ===== 
%% ===== 
%% ===== 
%% ===== \begin{flushleft}
%% ===== IAD
%% ===== \end{flushleft}
%% ===== 
%% ===== 
%% ===== 
%% ===== 
%% ===== 
%% ===== \begin{flushleft}
%% ===== Coded illegal action, port D. (See Table 3-2)
%% ===== \end{flushleft}
%% ===== 
%% ===== 
%% ===== 
%% ===== 
%% ===== 
%% ===== \begin{flushleft}
%% ===== p
%% ===== \end{flushleft}
%% ===== 
%% ===== 
%% ===== 
%% ===== 
%% ===== 
%% ===== \begin{flushleft}
%% ===== CPAR DIR
%% ===== \end{flushleft}
%% ===== 
%% ===== 
%% ===== 
%% ===== 
%% ===== 
%% ===== \begin{flushleft}
%% ===== None
%% ===== \end{flushleft}
%% ===== 
%% ===== 
%% ===== 
%% ===== 
%% ===== 
%% ===== \begin{flushleft}
%% ===== A parity error has been detected in the cache memory
%% ===== \end{flushleft}
%% ===== 
%% ===== 
%% ===== \begin{flushleft}
%% ===== directory.
%% ===== \end{flushleft}
%% ===== 
%% ===== 
%% ===== 
%% ===== 
%% ===== 
%% ===== \begin{flushleft}
%% ===== q
%% ===== \end{flushleft}
%% ===== 
%% ===== 
%% ===== 
%% ===== 
%% ===== 
%% ===== \begin{flushleft}
%% ===== CPAR STR
%% ===== \end{flushleft}
%% ===== 
%% ===== 
%% ===== 
%% ===== 
%% ===== 
%% ===== \begin{flushleft}
%% ===== PAR
%% ===== \end{flushleft}
%% ===== 
%% ===== 
%% ===== 
%% ===== 
%% ===== 
%% ===== \begin{flushleft}
%% ===== A data parity error has been detected in the cache memory.
%% ===== \end{flushleft}
%% ===== 
%% ===== 
%% ===== 
%% ===== 
%% ===== 
%% ===== \begin{flushleft}
%% ===== r
%% ===== \end{flushleft}
%% ===== 
%% ===== 
%% ===== 
%% ===== 
%% ===== 
%% ===== \begin{flushleft}
%% ===== CPAR IA
%% ===== \end{flushleft}
%% ===== 
%% ===== 
%% ===== 
%% ===== 
%% ===== 
%% ===== \begin{flushleft}
%% ===== PAR
%% ===== \end{flushleft}
%% ===== 
%% ===== 
%% ===== 
%% ===== 
%% ===== 
%% ===== \begin{flushleft}
%% ===== An illegal action has been received from a system controller
%% ===== \end{flushleft}
%% ===== 
%% ===== 
%% ===== \begin{flushleft}
%% ===== during a store operation with cache memory enabled. This
%% ===== \end{flushleft}
%% ===== 
%% ===== 
%% ===== \begin{flushleft}
%% ===== implies that the data are correct in cache memory and
%% ===== \end{flushleft}
%% ===== 
%% ===== 
%% ===== \begin{flushleft}
%% ===== incorrect in main memory.
%% ===== \end{flushleft}
%% ===== 
%% ===== 
%% ===== 
%% ===== 
%% ===== 
%% ===== \begin{flushleft}
%% ===== s
%% ===== \end{flushleft}
%% ===== 
%% ===== 
%% ===== 
%% ===== 
%% ===== 
%% ===== \begin{flushleft}
%% ===== CPAR BLK
%% ===== \end{flushleft}
%% ===== 
%% ===== 
%% ===== 
%% ===== 
%% ===== 
%% ===== \begin{flushleft}
%% ===== PAR
%% ===== \end{flushleft}
%% ===== 
%% ===== 
%% ===== 
%% ===== 
%% ===== 
%% ===== \begin{flushleft}
%% ===== A cache memory parity error has occurred during a cache
%% ===== \end{flushleft}
%% ===== 
%% ===== 
%% ===== \begin{flushleft}
%% ===== memory data block load.
%% ===== \end{flushleft}
%% ===== 
%% ===== 
%% ===== \begin{flushleft}
%% ===== Cache Duplicate Directory WNO Buffer Overflow
%% ===== \end{flushleft}
%% ===== 
%% ===== 
%% ===== 
%% ===== 
%% ===== 
%% ===== \begin{flushleft}
%% ===== t
%% ===== \end{flushleft}
%% ===== 
%% ===== 
%% ===== 
%% ===== 
%% ===== 
%% ===== \begin{flushleft}
%% ===== None
%% ===== \end{flushleft}
%% ===== 
%% ===== 
%% ===== 
%% ===== 
%% ===== 
%% ===== \begin{flushleft}
%% ===== Port A
%% ===== \end{flushleft}
%% ===== 
%% ===== 
%% ===== 
%% ===== 
%% ===== 
%% ===== \begin{flushleft}
%% ===== u
%% ===== \end{flushleft}
%% ===== 
%% ===== 
%% ===== 
%% ===== 
%% ===== 
%% ===== \begin{flushleft}
%% ===== None
%% ===== \end{flushleft}
%% ===== 
%% ===== 
%% ===== 
%% ===== 
%% ===== 
%% ===== \begin{flushleft}
%% ===== Port B
%% ===== \end{flushleft}
%% ===== 
%% ===== 
%% ===== 
%% ===== 
%% ===== 
%% ===== \begin{flushleft}
%% ===== v
%% ===== \end{flushleft}
%% ===== 
%% ===== 
%% ===== 
%% ===== 
%% ===== 
%% ===== \begin{flushleft}
%% ===== None
%% ===== \end{flushleft}
%% ===== 
%% ===== 
%% ===== 
%% ===== 
%% ===== 
%% ===== \begin{flushleft}
%% ===== Port C
%% ===== \end{flushleft}
%% ===== 
%% ===== 
%% ===== 
%% ===== 
%% ===== 
%% ===== \begin{flushleft}
%% ===== w
%% ===== \end{flushleft}
%% ===== 
%% ===== 
%% ===== 
%% ===== 
%% ===== 
%% ===== \begin{flushleft}
%% ===== None
%% ===== \end{flushleft}
%% ===== 
%% ===== 
%% ===== 
%% ===== 
%% ===== 
%% ===== \begin{flushleft}
%% ===== Port D
%% ===== \end{flushleft}
%% ===== 
%% ===== 
%% ===== 
%% ===== 
%% ===== 
%% ===== \begin{flushleft}
%% ===== x
%% ===== \end{flushleft}
%% ===== 
%% ===== 
%% ===== 
%% ===== 
%% ===== 
%% ===== \begin{flushleft}
%% ===== None
%% ===== \end{flushleft}
%% ===== 
%% ===== 
%% ===== 
%% ===== 
%% ===== 
%% ===== \begin{flushleft}
%% ===== Cache Primary Directory WNO Buffer Overflow
%% ===== \end{flushleft}
%% ===== 
%% ===== 
%% ===== 
%% ===== 
%% ===== 
%% ===== \begin{flushleft}
%% ===== y
%% ===== \end{flushleft}
%% ===== 
%% ===== 
%% ===== 
%% ===== 
%% ===== 
%% ===== \begin{flushleft}
%% ===== None
%% ===== \end{flushleft}
%% ===== 
%% ===== 
%% ===== 
%% ===== 
%% ===== 
%% ===== \begin{flushleft}
%% ===== Write Notify (WNO) Parity Error on Port A, B, C, or D.
%% ===== \end{flushleft}
%% ===== 
%% ===== 
%% ===== 
%% ===== 
%% ===== 
%% ===== \begin{flushleft}
%% ===== None
%% ===== \end{flushleft}
%% ===== 
%% ===== 
%% ===== 
%% ===== 
%% ===== 
%% ===== \begin{flushleft}
%% ===== Cache Duplicate Directory Parity Error
%% ===== \end{flushleft}
%% ===== 
%% ===== 
%% ===== 
%% ===== 
%% ===== 
%% ===== \begin{flushleft}
%% ===== z
%% ===== \end{flushleft}
%% ===== 
%% ===== 
%% ===== 
%% ===== 
%% ===== 
%% ===== \begin{flushleft}
%% ===== None
%% ===== \end{flushleft}
%% ===== 
%% ===== 
%% ===== 
%% ===== 
%% ===== 
%% ===== \begin{flushleft}
%% ===== Level 0
%% ===== \end{flushleft}
%% ===== 
%% ===== 
%% ===== 
%% ===== 
%% ===== 
%% ===== \begin{flushleft}
%% ===== A
%% ===== \end{flushleft}
%% ===== 
%% ===== 
%% ===== 
%% ===== 
%% ===== 
%% ===== \begin{flushleft}
%% ===== None
%% ===== \end{flushleft}
%% ===== 
%% ===== 
%% ===== 
%% ===== 
%% ===== 
%% ===== \begin{flushleft}
%% ===== Level 1
%% ===== \end{flushleft}
%% ===== 
%% ===== 
%% ===== 
%% ===== 
%% ===== 
%% ===== \begin{flushleft}
%% ===== B
%% ===== \end{flushleft}
%% ===== 
%% ===== 
%% ===== 
%% ===== 
%% ===== 
%% ===== \begin{flushleft}
%% ===== None
%% ===== \end{flushleft}
%% ===== 
%% ===== 
%% ===== 
%% ===== 
%% ===== 
%% ===== \begin{flushleft}
%% ===== Level 2
%% ===== \end{flushleft}
%% ===== 
%% ===== 
%% ===== 
%% ===== 
%% ===== 
%% ===== \begin{flushleft}
%% ===== \newpage
%% ===== Flag or
%% ===== \end{flushleft}
%% ===== 
%% ===== 
%% ===== \begin{flushleft}
%% ===== key register
%% ===== \end{flushleft}
%% ===== 
%% ===== 
%% ===== 
%% ===== 
%% ===== 
%% ===== \begin{flushleft}
%% ===== Fault
%% ===== \end{flushleft}
%% ===== 
%% ===== 
%% ===== 
%% ===== 
%% ===== 
%% ===== \begin{flushleft}
%% ===== C
%% ===== \end{flushleft}
%% ===== 
%% ===== 
%% ===== 
%% ===== 
%% ===== 
%% ===== \begin{flushleft}
%% ===== Function
%% ===== \end{flushleft}
%% ===== 
%% ===== 
%% ===== 
%% ===== 
%% ===== 
%% ===== \begin{flushleft}
%% ===== None
%% ===== \end{flushleft}
%% ===== 
%% ===== 
%% ===== 
%% ===== 
%% ===== 
%% ===== \begin{flushleft}
%% ===== D
%% ===== \end{flushleft}
%% ===== 
%% ===== 
%% ===== 
%% ===== 
%% ===== 
%% ===== \begin{flushleft}
%% ===== Level 3
%% ===== \end{flushleft}
%% ===== 
%% ===== 
%% ===== \begin{flushleft}
%% ===== Cache Duplicate Directory Multiple Match
%% ===== \end{flushleft}
%% ===== 
%% ===== 
%% ===== 
%% ===== 
%% ===== 
%% ===== \begin{flushleft}
%% ===== E
%% ===== \end{flushleft}
%% ===== 
%% ===== 
%% ===== 
%% ===== 
%% ===== 
%% ===== \begin{flushleft}
%% ===== None
%% ===== \end{flushleft}
%% ===== 
%% ===== 
%% ===== 
%% ===== 
%% ===== 
%% ===== \begin{flushleft}
%% ===== A parity error has been detected in the SDWAM.
%% ===== \end{flushleft}
%% ===== 
%% ===== 
%% ===== 
%% ===== 
%% ===== 
%% ===== \begin{flushleft}
%% ===== F
%% ===== \end{flushleft}
%% ===== 
%% ===== 
%% ===== 
%% ===== 
%% ===== 
%% ===== \begin{flushleft}
%% ===== None
%% ===== \end{flushleft}
%% ===== 
%% ===== 
%% ===== 
%% ===== 
%% ===== 
%% ===== \begin{flushleft}
%% ===== A parity error has been detected in the PTWAM.
%% ===== \end{flushleft}
%% ===== 
%% ===== 
%% ===== 
%% ===== 
%% ===== 
%% ===== \begin{flushleft}

\subsection{MODE REGISTER (MR) - DPS AND L68}

%% ===== \end{flushleft}
%% ===== 
%% ===== 
%% ===== \begin{flushleft}
%% ===== Format: - 33 bits
%% ===== \end{flushleft}
%% ===== 
%% ===== 
%% ===== \begin{flushleft}
%% ===== Even word of Y-pair as stored by Store Central Processor Register (scpr), TAG = 06
%% ===== \end{flushleft}
%% ===== 
%% ===== 
%% ===== 0
%% ===== 
%% ===== 
%% ===== 0
%% ===== 
%% ===== 
%% ===== 
%% ===== 
%% ===== 
%% ===== 1 1 1 1 1 1 2 2 2 2 2 2 2 2 2 2 3 3 3 3 3 3
%% ===== 
%% ===== 
%% ===== 4 5 6 7 8 9 0 1 2 3 4 5 6 7 8 9 0 1 2 3 4 5
%% ===== 
%% ===== 
%% ===== \begin{flushleft}
%% ===== FFV
%% ===== \end{flushleft}
%% ===== 
%% ===== 
%% ===== 
%% ===== 
%% ===== 
%% ===== \begin{flushleft}
%% ===== 0 a b
%% ===== \end{flushleft}
%% ===== 
%% ===== 
%% ===== 
%% ===== 
%% ===== 
%% ===== \begin{flushleft}
%% ===== OPCODE
%% ===== \end{flushleft}
%% ===== 
%% ===== 
%% ===== 
%% ===== 
%% ===== 
%% ===== \begin{flushleft}
%% ===== c d e f
%% ===== \end{flushleft}
%% ===== 
%% ===== 
%% ===== 15 1 1 1 1 1 1 1
%% ===== 
%% ===== 
%% ===== 
%% ===== 
%% ===== 
%% ===== \begin{flushleft}
%% ===== g
%% ===== \end{flushleft}
%% ===== 
%% ===== 
%% ===== 
%% ===== 
%% ===== 
%% ===== \begin{flushleft}
%% ===== h
%% ===== \end{flushleft}
%% ===== 
%% ===== 
%% ===== 2
%% ===== 
%% ===== 
%% ===== 
%% ===== 
%% ===== 
%% ===== \begin{flushleft}
%% ===== i j k l m 0 0 n
%% ===== \end{flushleft}
%% ===== 
%% ===== 
%% ===== 0 0
%% ===== 
%% ===== 
%% ===== 2
%% ===== 
%% ===== 
%% ===== 2 1 1 1 1 1
%% ===== 
%% ===== 
%% ===== 2 1
%% ===== 
%% ===== 
%% ===== 
%% ===== 
%% ===== 
%% ===== \begin{flushleft}
%% ===== Figure 3-20. Mode Register (MR) Format - DPS and L68
%% ===== \end{flushleft}
%% ===== 
%% ===== 
%% ===== \begin{flushleft}
%% ===== Description:
%% ===== \end{flushleft}
%% ===== 
%% ===== 
%% ===== \begin{flushleft}
%% ===== An assemblage of flags and registers from the control unit. The Mode Register and the
%% ===== \end{flushleft}
%% ===== 
%% ===== 
%% ===== \begin{flushleft}
%% ===== Cache Mode Register are both stored into the Y-pair by the Store Central Processor
%% ===== \end{flushleft}
%% ===== 
%% ===== 
%% ===== \begin{flushleft}
%% ===== Register (scpr), TAG = 06. The Mode Register is loaded with the Load Central Processor
%% ===== \end{flushleft}
%% ===== 
%% ===== 
%% ===== \begin{flushleft}
%% ===== Register (lcpr), TAG = 04, instruction.
%% ===== \end{flushleft}
%% ===== 
%% ===== 
%% ===== \begin{flushleft}
%% ===== Function:
%% ===== \end{flushleft}
%% ===== 
%% ===== 
%% ===== \begin{flushleft}
%% ===== The Mode Register controls the operation of those features of the processor that are
%% ===== \end{flushleft}
%% ===== 
%% ===== 
%% ===== \begin{flushleft}
%% ===== capable of being enabled and disabled.
%% ===== \end{flushleft}
%% ===== 
%% ===== 
%% ===== \begin{flushleft}
%% ===== The functions of the constituent flags and registers are:
%% ===== \end{flushleft}
%% ===== 
%% ===== 
%% ===== 
%% ===== 
%% ===== 
%% ===== \begin{flushleft}
%% ===== Flag or
%% ===== \end{flushleft}
%% ===== 
%% ===== 
%% ===== \begin{flushleft}
%% ===== key register
%% ===== \end{flushleft}
%% ===== 
%% ===== 
%% ===== 
%% ===== 
%% ===== 
%% ===== \begin{flushleft}
%% ===== Function
%% ===== \end{flushleft}
%% ===== 
%% ===== 
%% ===== 
%% ===== 
%% ===== 
%% ===== \begin{flushleft}
%% ===== FFV
%% ===== \end{flushleft}
%% ===== 
%% ===== 
%% ===== 
%% ===== 
%% ===== 
%% ===== \begin{flushleft}
%% ===== A floating-fault vector address. The 15 high-order bits of the Yblock8 address of four word pairs constituting a floating-fault vector.
%% ===== \end{flushleft}
%% ===== 
%% ===== 
%% ===== \begin{flushleft}
%% ===== Traps to these floating faults are generated by other conditions the
%% ===== \end{flushleft}
%% ===== 
%% ===== 
%% ===== \begin{flushleft}
%% ===== mode register sets.
%% ===== \end{flushleft}
%% ===== 
%% ===== 
%% ===== 
%% ===== 
%% ===== 
%% ===== \begin{flushleft}
%% ===== a
%% ===== \end{flushleft}
%% ===== 
%% ===== 
%% ===== 
%% ===== 
%% ===== 
%% ===== \begin{flushleft}
%% ===== OC TRAP
%% ===== \end{flushleft}
%% ===== 
%% ===== 
%% ===== 
%% ===== 
%% ===== 
%% ===== \begin{flushleft}
%% ===== Trap on OPCODE match. If this bit is set ON and OPCODE matches
%% ===== \end{flushleft}
%% ===== 
%% ===== 
%% ===== \begin{flushleft}
%% ===== the operation code of the instruction for which an address is being
%% ===== \end{flushleft}
%% ===== 
%% ===== 
%% ===== \begin{flushleft}
%% ===== prepared (including indirect cycles), generate the second floating
%% ===== \end{flushleft}
%% ===== 
%% ===== 
%% ===== \begin{flushleft}
%% ===== fault (xed FFV+2). See NOTE below.
%% ===== \end{flushleft}
%% ===== 
%% ===== 
%% ===== 
%% ===== 
%% ===== 
%% ===== \begin{flushleft}
%% ===== b
%% ===== \end{flushleft}
%% ===== 
%% ===== 
%% ===== 
%% ===== 
%% ===== 
%% ===== \begin{flushleft}
%% ===== ADR TRAP
%% ===== \end{flushleft}
%% ===== 
%% ===== 
%% ===== 
%% ===== 
%% ===== 
%% ===== \begin{flushleft}
%% ===== Trap on ADDRESS match. If this bit is set ON and the computed
%% ===== \end{flushleft}
%% ===== 
%% ===== 
%% ===== \begin{flushleft}
%% ===== address (TPR.CA) matches the setting of the address switches on the
%% ===== \end{flushleft}
%% ===== 
%% ===== 
%% ===== \begin{flushleft}
%% ===== processor maintenance panel, generate the fourth floating fault (xed
%% ===== \end{flushleft}
%% ===== 
%% ===== 
%% ===== \begin{flushleft}
%% ===== FFV+6). See NOTE below.
%% ===== \end{flushleft}
%% ===== 
%% ===== 
%% ===== 
%% ===== 
%% ===== 
%% ===== \begin{flushleft}
%% ===== \newpage
%% ===== Flag or
%% ===== \end{flushleft}
%% ===== 
%% ===== 
%% ===== \begin{flushleft}
%% ===== key register
%% ===== \end{flushleft}
%% ===== 
%% ===== 
%% ===== \begin{flushleft}
%% ===== OPCODE
%% ===== \end{flushleft}
%% ===== 
%% ===== 
%% ===== 
%% ===== 
%% ===== 
%% ===== \begin{flushleft}
%% ===== Function
%% ===== \end{flushleft}
%% ===== 
%% ===== 
%% ===== \begin{flushleft}
%% ===== The operation code on which to trap if OC TRAP (bit 16, key a) is set
%% ===== \end{flushleft}
%% ===== 
%% ===== 
%% ===== \begin{flushleft}
%% ===== ON or for which to strobe all control unit cycles into the control unit
%% ===== \end{flushleft}
%% ===== 
%% ===== 
%% ===== \begin{flushleft}
%% ===== %% cac history registers if O.C\$¢ (bit 29, key j) is set ON.
%% ===== history registers if O.C\$\textcent (bit 29, key j) is set ON.
%% ===== \end{flushleft}
%% ===== 
%% ===== 
%% ===== \begin{flushleft}
%% ===== or
%% ===== \end{flushleft}
%% ===== 
%% ===== 
%% ===== \begin{flushleft}
%% ===== Processor conditions codes as follows if OC TRAP (bit 16, key a) and
%% ===== \end{flushleft}
%% ===== 
%% ===== 
%% ===== \begin{flushleft}
%% ===== O.C\$¢ (bit 29, key j) are set OFF and ¢ VOLT (bit 32, key m) is set
%% ===== \end{flushleft}
%% ===== 
%% ===== 
%% ===== \begin{flushleft}
%% ===== ON.
%% ===== \end{flushleft}
%% ===== 
%% ===== 
%% ===== 
%% ===== 
%% ===== 
%% ===== \begin{flushleft}
%% ===== c
%% ===== \end{flushleft}
%% ===== 
%% ===== 
%% ===== 
%% ===== 
%% ===== 
%% ===== \begin{flushleft}
%% ===== Set control unit overlap inhibit if set ON. The control unit waits
%% ===== \end{flushleft}
%% ===== 
%% ===== 
%% ===== \begin{flushleft}
%% ===== for the operations unit to complete execution of the even
%% ===== \end{flushleft}
%% ===== 
%% ===== 
%% ===== \begin{flushleft}
%% ===== instruction of the current instruction pair before it begins
%% ===== \end{flushleft}
%% ===== 
%% ===== 
%% ===== \begin{flushleft}
%% ===== address preparation for the associated odd instruction. The
%% ===== \end{flushleft}
%% ===== 
%% ===== 
%% ===== \begin{flushleft}
%% ===== control unit also waits for the operations unit to complete
%% ===== \end{flushleft}
%% ===== 
%% ===== 
%% ===== \begin{flushleft}
%% ===== execution of the odd instruction before it fetches the next
%% ===== \end{flushleft}
%% ===== 
%% ===== 
%% ===== \begin{flushleft}
%% ===== instruction pair.
%% ===== \end{flushleft}
%% ===== 
%% ===== 
%% ===== 
%% ===== 
%% ===== 
%% ===== \begin{flushleft}
%% ===== d
%% ===== \end{flushleft}
%% ===== 
%% ===== 
%% ===== 
%% ===== 
%% ===== 
%% ===== \begin{flushleft}
%% ===== Set store overlap inhibit if set ON. The control unit waits for
%% ===== \end{flushleft}
%% ===== 
%% ===== 
%% ===== \begin{flushleft}
%% ===== completion of a current main memory fetch (read cycles only)
%% ===== \end{flushleft}
%% ===== 
%% ===== 
%% ===== \begin{flushleft}
%% ===== before requesting a main memory access for another fetch.
%% ===== \end{flushleft}
%% ===== 
%% ===== 
%% ===== 
%% ===== 
%% ===== 
%% ===== \begin{flushleft}
%% ===== e
%% ===== \end{flushleft}
%% ===== 
%% ===== 
%% ===== 
%% ===== 
%% ===== 
%% ===== \begin{flushleft}
%% ===== Set store incorrect data parity if set ON. The control unit causes
%% ===== \end{flushleft}
%% ===== 
%% ===== 
%% ===== \begin{flushleft}
%% ===== incorrect data parity to be sent to the system controller for the
%% ===== \end{flushleft}
%% ===== 
%% ===== 
%% ===== \begin{flushleft}
%% ===== next store instruction and then resets bit 20.
%% ===== \end{flushleft}
%% ===== 
%% ===== 
%% ===== 
%% ===== 
%% ===== 
%% ===== \begin{flushleft}
%% ===== f
%% ===== \end{flushleft}
%% ===== 
%% ===== 
%% ===== 
%% ===== 
%% ===== 
%% ===== \begin{flushleft}
%% ===== Set store incorrect zone-address-command (ZAC) parity if set
%% ===== \end{flushleft}
%% ===== 
%% ===== 
%% ===== \begin{flushleft}
%% ===== ON. The control unit causes incorrect zone-address-command
%% ===== \end{flushleft}
%% ===== 
%% ===== 
%% ===== \begin{flushleft}
%% ===== (ZAC) parity to be sent to the system controller for each main
%% ===== \end{flushleft}
%% ===== 
%% ===== 
%% ===== \begin{flushleft}
%% ===== memory cycle of the next store instruction and resets bit 21 at
%% ===== \end{flushleft}
%% ===== 
%% ===== 
%% ===== \begin{flushleft}
%% ===== the end of the instruction.
%% ===== \end{flushleft}
%% ===== 
%% ===== 
%% ===== 
%% ===== 
%% ===== 
%% ===== \begin{flushleft}
%% ===== g
%% ===== \end{flushleft}
%% ===== 
%% ===== 
%% ===== 
%% ===== 
%% ===== 
%% ===== \begin{flushleft}
%% ===== Set timing margins if set ON. If ¢ VOLT (bit 32, key m) is set ON
%% ===== \end{flushleft}
%% ===== 
%% ===== 
%% ===== \begin{flushleft}
%% ===== and the margin control switch on the processor maintenance
%% ===== \end{flushleft}
%% ===== 
%% ===== 
%% ===== \begin{flushleft}
%% ===== panel is in PROG position, set processor timing margins as
%% ===== \end{flushleft}
%% ===== 
%% ===== 
%% ===== \begin{flushleft}
%% ===== follows:
%% ===== \end{flushleft}
%% ===== 
%% ===== 
%% ===== 
%% ===== 
%% ===== 
%% ===== 22,23
%% ===== 
%% ===== 
%% ===== 0,0
%% ===== 
%% ===== 
%% ===== 0,1
%% ===== 
%% ===== 
%% ===== 1,0
%% ===== 
%% ===== 
%% ===== 1,1
%% ===== 
%% ===== 
%% ===== \begin{flushleft}
%% ===== h
%% ===== \end{flushleft}
%% ===== 
%% ===== 
%% ===== 
%% ===== 
%% ===== 
%% ===== \begin{flushleft}
%% ===== Set +5 voltage margins if set ON. If ¢ VOLT (bit 32, key m) is set
%% ===== \end{flushleft}
%% ===== 
%% ===== 
%% ===== \begin{flushleft}
%% ===== ON and the margin control switch on the processor maintenance
%% ===== \end{flushleft}
%% ===== 
%% ===== 
%% ===== \begin{flushleft}
%% ===== panel is in the PROG position, set +5 voltage margins as follows:
%% ===== \end{flushleft}
%% ===== 
%% ===== 
%% ===== 
%% ===== 
%% ===== 
%% ===== 24,25
%% ===== 
%% ===== 
%% ===== 0,0
%% ===== 
%% ===== 
%% ===== 0,1
%% ===== 
%% ===== 
%% ===== 1,0
%% ===== 
%% ===== 
%% ===== 1,1
%% ===== 
%% ===== 
%% ===== \begin{flushleft}
%% ===== i
%% ===== \end{flushleft}
%% ===== 
%% ===== 
%% ===== 
%% ===== 
%% ===== 
%% ===== \begin{flushleft}
%% ===== margin
%% ===== \end{flushleft}
%% ===== 
%% ===== 
%% ===== \begin{flushleft}
%% ===== normal
%% ===== \end{flushleft}
%% ===== 
%% ===== 
%% ===== \begin{flushleft}
%% ===== slow
%% ===== \end{flushleft}
%% ===== 
%% ===== 
%% ===== \begin{flushleft}
%% ===== normal
%% ===== \end{flushleft}
%% ===== 
%% ===== 
%% ===== \begin{flushleft}
%% ===== fast
%% ===== \end{flushleft}
%% ===== 
%% ===== 
%% ===== 
%% ===== 
%% ===== 
%% ===== \begin{flushleft}
%% ===== margin
%% ===== \end{flushleft}
%% ===== 
%% ===== 
%% ===== \begin{flushleft}
%% ===== normal
%% ===== \end{flushleft}
%% ===== 
%% ===== 
%% ===== \begin{flushleft}
%% ===== low
%% ===== \end{flushleft}
%% ===== 
%% ===== 
%% ===== \begin{flushleft}
%% ===== high
%% ===== \end{flushleft}
%% ===== 
%% ===== 
%% ===== \begin{flushleft}
%% ===== normal
%% ===== \end{flushleft}
%% ===== 
%% ===== 
%% ===== 
%% ===== 
%% ===== 
%% ===== \begin{flushleft}
%% ===== Trap on control unit history register count overflow if set ON. If this
%% ===== \end{flushleft}
%% ===== 
%% ===== 
%% ===== \begin{flushleft}
%% ===== bit and STROBE ¢ (bit 30, key k) are set ON and the control unit
%% ===== \end{flushleft}
%% ===== 
%% ===== 
%% ===== \begin{flushleft}
%% ===== history register counter overflows, generate the third floating fault
%% ===== \end{flushleft}
%% ===== 
%% ===== 
%% ===== \begin{flushleft}
%% ===== (xed FFV+4). Further, if FAULT RESET (bit 31, key 1) is set, reset
%% ===== \end{flushleft}
%% ===== 
%% ===== 
%% ===== \begin{flushleft}
%% ===== STROBE ¢ (bit 30, key k), locking the history registers. A Load
%% ===== \end{flushleft}
%% ===== 
%% ===== 
%% ===== \begin{flushleft}
%% ===== Central Processor Register (lcpr), TAG = 04, instruction setting bit
%% ===== \end{flushleft}
%% ===== 
%% ===== 
%% ===== \begin{flushleft}
%% ===== 28 ON resets the control unit history register counter to zero. (See
%% ===== \end{flushleft}
%% ===== 
%% ===== 
%% ===== \begin{flushleft}
%% ===== NOTE below.)
%% ===== \end{flushleft}
%% ===== 
%% ===== 
%% ===== 
%% ===== 
%% ===== 
%% ===== \begin{flushleft}
%% ===== \newpage
%% ===== Flag or
%% ===== \end{flushleft}
%% ===== 
%% ===== 
%% ===== \begin{flushleft}
%% ===== key register
%% ===== \end{flushleft}
%% ===== 
%% ===== 
%% ===== 
%% ===== 
%% ===== 
%% ===== \begin{flushleft}
%% ===== NOTE:
%% ===== \end{flushleft}
%% ===== 
%% ===== 
%% ===== 
%% ===== 
%% ===== 
%% ===== \begin{flushleft}
%% ===== Function
%% ===== \end{flushleft}
%% ===== 
%% ===== 
%% ===== 
%% ===== 
%% ===== 
%% ===== \begin{flushleft}
%% ===== j
%% ===== \end{flushleft}
%% ===== 
%% ===== 
%% ===== 
%% ===== 
%% ===== 
%% ===== \begin{flushleft}
%% ===== O.C\$¢
%% ===== \end{flushleft}
%% ===== 
%% ===== 
%% ===== 
%% ===== 
%% ===== 
%% ===== \begin{flushleft}
%% ===== Strobe control unit history registers on OPCODE match. If this bit
%% ===== \end{flushleft}
%% ===== 
%% ===== 
%% ===== \begin{flushleft}
%% ===== and STROBE ¢ (bit 30, key k) are set ON and the operation code of
%% ===== \end{flushleft}
%% ===== 
%% ===== 
%% ===== \begin{flushleft}
%% ===== the current instruction matches OPCODE, strobe the control unit
%% ===== \end{flushleft}
%% ===== 
%% ===== 
%% ===== \begin{flushleft}
%% ===== history registers on all control unit cycles (including indirect cycles).
%% ===== \end{flushleft}
%% ===== 
%% ===== 
%% ===== 
%% ===== 
%% ===== 
%% ===== \begin{flushleft}
%% ===== k
%% ===== \end{flushleft}
%% ===== 
%% ===== 
%% ===== 
%% ===== 
%% ===== 
%% ===== \begin{flushleft}
%% ===== STROBE ¢
%% ===== \end{flushleft}
%% ===== 
%% ===== 
%% ===== 
%% ===== 
%% ===== 
%% ===== \begin{flushleft}
%% ===== Enable history registers. If this bit is set ON, all history registers are
%% ===== \end{flushleft}
%% ===== 
%% ===== 
%% ===== \begin{flushleft}
%% ===== strobed at appropriate points in the various processor cycles. If this
%% ===== \end{flushleft}
%% ===== 
%% ===== 
%% ===== \begin{flushleft}
%% ===== bit is set OFF or MR ENABLE (bit 35, key n) is set OFF, all history
%% ===== \end{flushleft}
%% ===== 
%% ===== 
%% ===== \begin{flushleft}
%% ===== registers are locked. This bit is set OFF with a Load Central
%% ===== \end{flushleft}
%% ===== 
%% ===== 
%% ===== \begin{flushleft}
%% ===== Processor Register (lcpr), TAG = 04, instruction providing a 0 bit, by
%% ===== \end{flushleft}
%% ===== 
%% ===== 
%% ===== \begin{flushleft}
%% ===== an operation not complete fault, and, conditionally, by other faults
%% ===== \end{flushleft}
%% ===== 
%% ===== 
%% ===== \begin{flushleft}
%% ===== (see FAULT RESET (bit 31, key 1) below). Once set OFF, this bit
%% ===== \end{flushleft}
%% ===== 
%% ===== 
%% ===== \begin{flushleft}
%% ===== must be set ON with a Load Central Processor Register (lcpr), TAG
%% ===== \end{flushleft}
%% ===== 
%% ===== 
%% ===== \begin{flushleft}
%% ===== = 04, instruction providing a 1 bit to re-enable the history registers.
%% ===== \end{flushleft}
%% ===== 
%% ===== 
%% ===== 
%% ===== 
%% ===== 
%% ===== \begin{flushleft}
%% ===== l
%% ===== \end{flushleft}
%% ===== 
%% ===== 
%% ===== 
%% ===== 
%% ===== 
%% ===== \begin{flushleft}
%% ===== FAULT RESET
%% ===== \end{flushleft}
%% ===== 
%% ===== 
%% ===== 
%% ===== 
%% ===== 
%% ===== \begin{flushleft}
%% ===== History register lock control. If this bit is set ON, set STROBE ¢ (bit
%% ===== \end{flushleft}
%% ===== 
%% ===== 
%% ===== \begin{flushleft}
%% ===== 30, key k) OFF, locking the history registers for all faults including
%% ===== \end{flushleft}
%% ===== 
%% ===== 
%% ===== \begin{flushleft}
%% ===== the floating faults. See NOTE below.
%% ===== \end{flushleft}
%% ===== 
%% ===== 
%% ===== 
%% ===== 
%% ===== 
%% ===== \begin{flushleft}
%% ===== m
%% ===== \end{flushleft}
%% ===== 
%% ===== 
%% ===== 
%% ===== 
%% ===== 
%% ===== \begin{flushleft}
%% ===== ¢ VOLT
%% ===== \end{flushleft}
%% ===== 
%% ===== 
%% ===== 
%% ===== 
%% ===== 
%% ===== \begin{flushleft}
%% ===== Test mode indicator. This bit is set ON whenever the TEST/NORMAL
%% ===== \end{flushleft}
%% ===== 
%% ===== 
%% ===== \begin{flushleft}
%% ===== switch on the processor maintenance panel is in TEST position;
%% ===== \end{flushleft}
%% ===== 
%% ===== 
%% ===== \begin{flushleft}
%% ===== otherwise, it is set OFF. It serves to enable the program control of
%% ===== \end{flushleft}
%% ===== 
%% ===== 
%% ===== \begin{flushleft}
%% ===== voltage and timing margins.
%% ===== \end{flushleft}
%% ===== 
%% ===== 
%% ===== 
%% ===== 
%% ===== 
%% ===== \begin{flushleft}
%% ===== n
%% ===== \end{flushleft}
%% ===== 
%% ===== 
%% ===== 
%% ===== 
%% ===== 
%% ===== \begin{flushleft}
%% ===== MR ENABLE
%% ===== \end{flushleft}
%% ===== 
%% ===== 
%% ===== 
%% ===== 
%% ===== 
%% ===== \begin{flushleft}
%% ===== Enable mode register. When this bit is set ON, all other bits and
%% ===== \end{flushleft}
%% ===== 
%% ===== 
%% ===== \begin{flushleft}
%% ===== controls of the mode register are active. When this bit is set OFF,
%% ===== \end{flushleft}
%% ===== 
%% ===== 
%% ===== \begin{flushleft}
%% ===== the mode register controls are disabled.
%% ===== \end{flushleft}
%% ===== 
%% ===== 
%% ===== 
%% ===== 
%% ===== 
%% ===== \begin{flushleft}
%% ===== The traps described above (address match, OPCODE match, control unit history register
%% ===== \end{flushleft}
%% ===== 
%% ===== 
%% ===== \begin{flushleft}
%% ===== counter overflow) occur after completion of the next odd instruction following their
%% ===== \end{flushleft}
%% ===== 
%% ===== 
%% ===== \begin{flushleft}
%% ===== detection. They are handled as Group 7 faults in regard to servicing and inhibition. (See
%% ===== \end{flushleft}
%% ===== 
%% ===== 
%% ===== \begin{flushleft}
%% ===== Section 7 for descriptions of these faults.) The complete Group 7 priority sequence (in
%% ===== \end{flushleft}
%% ===== 
%% ===== 
%% ===== \begin{flushleft}
%% ===== increasing order) is:
%% ===== \end{flushleft}
%% ===== 
%% ===== 
%% ===== \begin{flushleft}
%% ===== 1 - Connect
%% ===== \end{flushleft}
%% ===== 
%% ===== 
%% ===== \begin{flushleft}
%% ===== 2 - Time runout
%% ===== \end{flushleft}
%% ===== 
%% ===== 
%% ===== \begin{flushleft}
%% ===== 3 - Shutdown
%% ===== \end{flushleft}
%% ===== 
%% ===== 
%% ===== \begin{flushleft}
%% ===== 4 - OPCODE trap
%% ===== \end{flushleft}
%% ===== 
%% ===== 
%% ===== \begin{flushleft}
%% ===== 5 - Control unit history register counter overflow
%% ===== \end{flushleft}
%% ===== 
%% ===== 
%% ===== \begin{flushleft}
%% ===== 6 - Address match trap
%% ===== \end{flushleft}
%% ===== 
%% ===== 
%% ===== \begin{flushleft}
%% ===== 7 - External interrupts
%% ===== \end{flushleft}
%% ===== 
%% ===== 
%% ===== 
%% ===== 
%% ===== 
%% ===== \begin{flushleft}
%% ===== \newpage

\subsection{MODE REGISTER (MR) - DPS 8M}

%% ===== \end{flushleft}
%% ===== 
%% ===== 
%% ===== \begin{flushleft}
%% ===== Format: - 36 bits
%% ===== \end{flushleft}
%% ===== 
%% ===== 
%% ===== \begin{flushleft}
%% ===== Even word of Y-pair as stored by Store Central Processor Register (scpr), TAG = 06
%% ===== \end{flushleft}
%% ===== 
%% ===== 
%% ===== 0
%% ===== 
%% ===== 
%% ===== 0
%% ===== 
%% ===== 
%% ===== 
%% ===== 
%% ===== 
%% ===== 1 1 1 2 2 2 2 2 2 2 2 2 2 3 3 3 3 3 3
%% ===== 
%% ===== 
%% ===== 7 8 9 0 1 2 3 4 5 6 7 8 9 0 1 2 3 4 5
%% ===== 
%% ===== 
%% ===== 
%% ===== 
%% ===== 
%% ===== \begin{flushleft}
%% ===== 0 0 0 0 0 0 0 0 0 0 0 0 0 0 0 0 0 0 a b c d
%% ===== \end{flushleft}
%% ===== 
%% ===== 
%% ===== 18 1 1 1 1
%% ===== 
%% ===== 
%% ===== 
%% ===== 
%% ===== 
%% ===== \begin{flushleft}
%% ===== e
%% ===== \end{flushleft}
%% ===== 
%% ===== 
%% ===== 
%% ===== 
%% ===== 
%% ===== \begin{flushleft}
%% ===== f
%% ===== \end{flushleft}
%% ===== 
%% ===== 
%% ===== 2
%% ===== 
%% ===== 
%% ===== 
%% ===== 
%% ===== 
%% ===== \begin{flushleft}
%% ===== 0 0 g h i
%% ===== \end{flushleft}
%% ===== 
%% ===== 
%% ===== 2
%% ===== 
%% ===== 
%% ===== 
%% ===== 
%% ===== 
%% ===== \begin{flushleft}
%% ===== j k l 0 m
%% ===== \end{flushleft}
%% ===== 
%% ===== 
%% ===== 
%% ===== 
%% ===== 
%% ===== 2 1 1 1 1 1 1 1 1
%% ===== 
%% ===== 
%% ===== 
%% ===== 
%% ===== 
%% ===== \begin{flushleft}
%% ===== Figure 3-21. Mode Register (MR) Format - DPS 8M
%% ===== \end{flushleft}
%% ===== 
%% ===== 
%% ===== \begin{flushleft}
%% ===== Description:
%% ===== \end{flushleft}
%% ===== 
%% ===== 
%% ===== \begin{flushleft}
%% ===== An assemblage of flags and registers from the control unit. The Mode Register and the
%% ===== \end{flushleft}
%% ===== 
%% ===== 
%% ===== \begin{flushleft}
%% ===== Cache Mode Register are both stored into the Y-pair by the Store Central Processor
%% ===== \end{flushleft}
%% ===== 
%% ===== 
%% ===== \begin{flushleft}
%% ===== Register (scpr), TAG = 06. The Mode Register is loaded with the Load Central Processor
%% ===== \end{flushleft}
%% ===== 
%% ===== 
%% ===== \begin{flushleft}
%% ===== Register (lcpr), TAG = 04, instruction.
%% ===== \end{flushleft}
%% ===== 
%% ===== 
%% ===== \begin{flushleft}
%% ===== Function:
%% ===== \end{flushleft}
%% ===== 
%% ===== 
%% ===== \begin{flushleft}
%% ===== The mode register controls the operation of those features of the processor that are capable
%% ===== \end{flushleft}
%% ===== 
%% ===== 
%% ===== \begin{flushleft}
%% ===== of being enabled and disabled.
%% ===== \end{flushleft}
%% ===== 
%% ===== 
%% ===== \begin{flushleft}
%% ===== The functions of the constituent flags and registers are:
%% ===== \end{flushleft}
%% ===== 
%% ===== 
%% ===== 
%% ===== 
%% ===== 
%% ===== \begin{flushleft}
%% ===== Flag or
%% ===== \end{flushleft}
%% ===== 
%% ===== 
%% ===== \begin{flushleft}
%% ===== key register
%% ===== \end{flushleft}
%% ===== 
%% ===== 
%% ===== 
%% ===== 
%% ===== 
%% ===== \begin{flushleft}
%% ===== Function
%% ===== \end{flushleft}
%% ===== 
%% ===== 
%% ===== 
%% ===== 
%% ===== 
%% ===== \begin{flushleft}
%% ===== a
%% ===== \end{flushleft}
%% ===== 
%% ===== 
%% ===== 
%% ===== 
%% ===== 
%% ===== \begin{flushleft}
%% ===== cuolin
%% ===== \end{flushleft}
%% ===== 
%% ===== 
%% ===== 
%% ===== 
%% ===== 
%% ===== \begin{flushleft}
%% ===== Set CU overlap inhibit. The CU waits for the OU to complete execution of
%% ===== \end{flushleft}
%% ===== 
%% ===== 
%% ===== \begin{flushleft}
%% ===== the even instruction before it begins address preparation for the
%% ===== \end{flushleft}
%% ===== 
%% ===== 
%% ===== \begin{flushleft}
%% ===== associated odd instruction. The CU also waits for the OU to complete
%% ===== \end{flushleft}
%% ===== 
%% ===== 
%% ===== \begin{flushleft}
%% ===== execution of the odd instruction before it fetches the next instruction pair.
%% ===== \end{flushleft}
%% ===== 
%% ===== 
%% ===== 
%% ===== 
%% ===== 
%% ===== \begin{flushleft}
%% ===== b
%% ===== \end{flushleft}
%% ===== 
%% ===== 
%% ===== 
%% ===== 
%% ===== 
%% ===== \begin{flushleft}
%% ===== solin
%% ===== \end{flushleft}
%% ===== 
%% ===== 
%% ===== 
%% ===== 
%% ===== 
%% ===== \begin{flushleft}
%% ===== Set store overlap inhibit. The CU waits for completion of a current
%% ===== \end{flushleft}
%% ===== 
%% ===== 
%% ===== \begin{flushleft}
%% ===== memory fetch (read cycles only) before requesting a memory access for
%% ===== \end{flushleft}
%% ===== 
%% ===== 
%% ===== \begin{flushleft}
%% ===== another fetch.
%% ===== \end{flushleft}
%% ===== 
%% ===== 
%% ===== 
%% ===== 
%% ===== 
%% ===== \begin{flushleft}
%% ===== c
%% ===== \end{flushleft}
%% ===== 
%% ===== 
%% ===== 
%% ===== 
%% ===== 
%% ===== \begin{flushleft}
%% ===== sdpap
%% ===== \end{flushleft}
%% ===== 
%% ===== 
%% ===== 
%% ===== 
%% ===== 
%% ===== \begin{flushleft}
%% ===== Set store incorrect data parity. The CU causes incorrect data parity to be
%% ===== \end{flushleft}
%% ===== 
%% ===== 
%% ===== \begin{flushleft}
%% ===== sent to the SC for the next data store instruction and then resets bit 20.
%% ===== \end{flushleft}
%% ===== 
%% ===== 
%% ===== 
%% ===== 
%% ===== 
%% ===== \begin{flushleft}
%% ===== d
%% ===== \end{flushleft}
%% ===== 
%% ===== 
%% ===== 
%% ===== 
%% ===== 
%% ===== \begin{flushleft}
%% ===== separ
%% ===== \end{flushleft}
%% ===== 
%% ===== 
%% ===== 
%% ===== 
%% ===== 
%% ===== \begin{flushleft}
%% ===== Set store incorrect ZAC parity. The CU causes incorrect zone-addresscommand (ZAC) parity to be sent to the SC for each memory cycle of the
%% ===== \end{flushleft}
%% ===== 
%% ===== 
%% ===== \begin{flushleft}
%% ===== next data store instruction and resets bit 21 at the end of the instruction.
%% ===== \end{flushleft}
%% ===== 
%% ===== 
%% ===== 
%% ===== 
%% ===== 
%% ===== \begin{flushleft}
%% ===== e
%% ===== \end{flushleft}
%% ===== 
%% ===== 
%% ===== 
%% ===== 
%% ===== 
%% ===== \begin{flushleft}
%% ===== tm
%% ===== \end{flushleft}
%% ===== 
%% ===== 
%% ===== 
%% ===== 
%% ===== 
%% ===== \begin{flushleft}
%% ===== Set timing margins. If bit 32 key (k) is set and the margin control switch
%% ===== \end{flushleft}
%% ===== 
%% ===== 
%% ===== \begin{flushleft}
%% ===== on the CPU maintenance panel is in program position, set CPU timing
%% ===== \end{flushleft}
%% ===== 
%% ===== 
%% ===== \begin{flushleft}
%% ===== margins as follows:
%% ===== \end{flushleft}
%% ===== 
%% ===== 
%% ===== 
%% ===== 
%% ===== 
%% ===== 22,23
%% ===== 
%% ===== 
%% ===== 0,0
%% ===== 
%% ===== 
%% ===== 0,1
%% ===== 
%% ===== 
%% ===== 1,0
%% ===== 
%% ===== 
%% ===== 1,1
%% ===== 
%% ===== 
%% ===== \begin{flushleft}
%% ===== f
%% ===== \end{flushleft}
%% ===== 
%% ===== 
%% ===== 
%% ===== 
%% ===== 
%% ===== \begin{flushleft}
%% ===== vm
%% ===== \end{flushleft}
%% ===== 
%% ===== 
%% ===== 
%% ===== 
%% ===== 
%% ===== \begin{flushleft}
%% ===== margin
%% ===== \end{flushleft}
%% ===== 
%% ===== 
%% ===== \begin{flushleft}
%% ===== normal
%% ===== \end{flushleft}
%% ===== 
%% ===== 
%% ===== \begin{flushleft}
%% ===== slow
%% ===== \end{flushleft}
%% ===== 
%% ===== 
%% ===== \begin{flushleft}
%% ===== normal
%% ===== \end{flushleft}
%% ===== 
%% ===== 
%% ===== \begin{flushleft}
%% ===== fast
%% ===== \end{flushleft}
%% ===== 
%% ===== 
%% ===== 
%% ===== 
%% ===== 
%% ===== \begin{flushleft}
%% ===== Set +5 voltage margins. If bit 32 (key k) is set and the margin control
%% ===== \end{flushleft}
%% ===== 
%% ===== 
%% ===== \begin{flushleft}
%% ===== switch on the CPU maintenance panel is in the program position, set +5
%% ===== \end{flushleft}
%% ===== 
%% ===== 
%% ===== \begin{flushleft}
%% ===== voltage margins as follows:
%% ===== \end{flushleft}
%% ===== 
%% ===== 
%% ===== 
%% ===== 
%% ===== 
%% ===== \begin{flushleft}
%% ===== \newpage
%% ===== Flag or
%% ===== \end{flushleft}
%% ===== 
%% ===== 
%% ===== \begin{flushleft}
%% ===== key register
%% ===== \end{flushleft}
%% ===== 
%% ===== 
%% ===== 
%% ===== 
%% ===== 
%% ===== \begin{flushleft}
%% ===== Function
%% ===== \end{flushleft}
%% ===== 
%% ===== 
%% ===== 24,25
%% ===== 
%% ===== 
%% ===== 0,0
%% ===== 
%% ===== 
%% ===== 0,1
%% ===== 
%% ===== 
%% ===== 1,0
%% ===== 
%% ===== 
%% ===== 1,1
%% ===== 
%% ===== 
%% ===== 
%% ===== 
%% ===== 
%% ===== \begin{flushleft}
%% ===== margin
%% ===== \end{flushleft}
%% ===== 
%% ===== 
%% ===== \begin{flushleft}
%% ===== normal
%% ===== \end{flushleft}
%% ===== 
%% ===== 
%% ===== \begin{flushleft}
%% ===== low
%% ===== \end{flushleft}
%% ===== 
%% ===== 
%% ===== \begin{flushleft}
%% ===== high
%% ===== \end{flushleft}
%% ===== 
%% ===== 
%% ===== \begin{flushleft}
%% ===== normal
%% ===== \end{flushleft}
%% ===== 
%% ===== 
%% ===== 
%% ===== 
%% ===== 
%% ===== \begin{flushleft}
%% ===== g
%% ===== \end{flushleft}
%% ===== 
%% ===== 
%% ===== 
%% ===== 
%% ===== 
%% ===== \begin{flushleft}
%% ===== hrhlt
%% ===== \end{flushleft}
%% ===== 
%% ===== 
%% ===== 
%% ===== 
%% ===== 
%% ===== \begin{flushleft}
%% ===== Stop HR Strobe on HR Counter Overflow. (Setting bit 28 shall cause the
%% ===== \end{flushleft}
%% ===== 
%% ===== 
%% ===== \begin{flushleft}
%% ===== HR counter to be reset to zero.)
%% ===== \end{flushleft}
%% ===== 
%% ===== 
%% ===== 
%% ===== 
%% ===== 
%% ===== \begin{flushleft}
%% ===== h
%% ===== \end{flushleft}
%% ===== 
%% ===== 
%% ===== 
%% ===== 
%% ===== 
%% ===== \begin{flushleft}
%% ===== hrxfr
%% ===== \end{flushleft}
%% ===== 
%% ===== 
%% ===== 
%% ===== 
%% ===== 
%% ===== \begin{flushleft}
%% ===== Strobe the HR on Transfer Made. If bits 29,30, and 35 are = 1, the HR
%% ===== \end{flushleft}
%% ===== 
%% ===== 
%% ===== \begin{flushleft}
%% ===== will be strobed on all Transfers Made. Bits 36-53 of the OU/DU register
%% ===== \end{flushleft}
%% ===== 
%% ===== 
%% ===== \begin{flushleft}
%% ===== will indicate the {``}From'' location and bits 36-59 of the CU register will
%% ===== \end{flushleft}
%% ===== 
%% ===== 
%% ===== \begin{flushleft}
%% ===== contain the real address of the final {``}To'' location.
%% ===== \end{flushleft}
%% ===== 
%% ===== 
%% ===== 
%% ===== 
%% ===== 
%% ===== \begin{flushleft}
%% ===== i
%% ===== \end{flushleft}
%% ===== 
%% ===== 
%% ===== 
%% ===== 
%% ===== 
%% ===== \begin{flushleft}
%% ===== ihr
%% ===== \end{flushleft}
%% ===== 
%% ===== 
%% ===== 
%% ===== 
%% ===== 
%% ===== \begin{flushleft}
%% ===== Enable History Registers. If bit 30 = 1, the HRs may be strobed. If bit 30
%% ===== \end{flushleft}
%% ===== 
%% ===== 
%% ===== \begin{flushleft}
%% ===== = 0 or bit 35 = 0, they will be locked out. This bit will be reset by either
%% ===== \end{flushleft}
%% ===== 
%% ===== 
%% ===== \begin{flushleft}
%% ===== an LCPR with the bit corresponding to 30 = 0 or by an Op Not Complete
%% ===== \end{flushleft}
%% ===== 
%% ===== 
%% ===== \begin{flushleft}
%% ===== fault. It may be reset by other faults (see bit 31). After being reset, it
%% ===== \end{flushleft}
%% ===== 
%% ===== 
%% ===== \begin{flushleft}
%% ===== must be enabled by another LCPR instruction before the History Registers
%% ===== \end{flushleft}
%% ===== 
%% ===== 
%% ===== \begin{flushleft}
%% ===== may be strobed again.
%% ===== \end{flushleft}
%% ===== 
%% ===== 
%% ===== 
%% ===== 
%% ===== 
%% ===== \begin{flushleft}
%% ===== j
%% ===== \end{flushleft}
%% ===== 
%% ===== 
%% ===== 
%% ===== 
%% ===== 
%% ===== \begin{flushleft}
%% ===== ihrrs
%% ===== \end{flushleft}
%% ===== 
%% ===== 
%% ===== 
%% ===== 
%% ===== 
%% ===== \begin{flushleft}
%% ===== Additional resetting of bit 30. If bit 31 = 1, the following faults also reset
%% ===== \end{flushleft}
%% ===== 
%% ===== 
%% ===== \begin{flushleft}
%% ===== bit 30:
%% ===== \end{flushleft}
%% ===== 
%% ===== 
%% ===== -
%% ===== 
%% ===== 
%% ===== 
%% ===== 
%% ===== 
%% ===== \begin{flushleft}
%% ===== Lock Up
%% ===== \end{flushleft}
%% ===== 
%% ===== 
%% ===== \begin{flushleft}
%% ===== Parity
%% ===== \end{flushleft}
%% ===== 
%% ===== 
%% ===== \begin{flushleft}
%% ===== Command
%% ===== \end{flushleft}
%% ===== 
%% ===== 
%% ===== \begin{flushleft}
%% ===== Store
%% ===== \end{flushleft}
%% ===== 
%% ===== 
%% ===== \begin{flushleft}
%% ===== Illegal Procedure
%% ===== \end{flushleft}
%% ===== 
%% ===== 
%% ===== \begin{flushleft}
%% ===== Shutdown
%% ===== \end{flushleft}
%% ===== 
%% ===== 
%% ===== 
%% ===== 
%% ===== 
%% ===== \begin{flushleft}
%% ===== k
%% ===== \end{flushleft}
%% ===== 
%% ===== 
%% ===== 
%% ===== 
%% ===== 
%% ===== \begin{flushleft}
%% ===== mrgctl
%% ===== \end{flushleft}
%% ===== 
%% ===== 
%% ===== 
%% ===== 
%% ===== 
%% ===== \begin{flushleft}
%% ===== Margin Control. Bit 32 informs the software when it can control margins.
%% ===== \end{flushleft}
%% ===== 
%% ===== 
%% ===== \begin{flushleft}
%% ===== A one indicates that software has control. When the LOCAL/REMOTE
%% ===== \end{flushleft}
%% ===== 
%% ===== 
%% ===== \begin{flushleft}
%% ===== switch on the power supply is in REMOTE and bit 35 = 1, bit 32 is set to 1
%% ===== \end{flushleft}
%% ===== 
%% ===== 
%% ===== \begin{flushleft}
%% ===== by occurrence of the following conditions: the NORMAL/TEST switch is in
%% ===== \end{flushleft}
%% ===== 
%% ===== 
%% ===== \begin{flushleft}
%% ===== the TEST position, the Memory and CU Overlap Inhibit switches are OFF,
%% ===== \end{flushleft}
%% ===== 
%% ===== 
%% ===== \begin{flushleft}
%% ===== the Timing Margins for the OU, CU, DU and VU are NORMAL, and the
%% ===== \end{flushleft}
%% ===== 
%% ===== 
%% ===== \begin{flushleft}
%% ===== Forced Data and ZAC Parity are OFF.
%% ===== \end{flushleft}
%% ===== 
%% ===== 
%% ===== 
%% ===== 
%% ===== 
%% ===== \begin{flushleft}
%% ===== l
%% ===== \end{flushleft}
%% ===== 
%% ===== 
%% ===== 
%% ===== 
%% ===== 
%% ===== \begin{flushleft}
%% ===== hexfp
%% ===== \end{flushleft}
%% ===== 
%% ===== 
%% ===== 
%% ===== 
%% ===== 
%% ===== \begin{flushleft}
%% ===== Hexadecimal Exponent Floating Point Arithmetic Mode can be set. When
%% ===== \end{flushleft}
%% ===== 
%% ===== 
%% ===== \begin{flushleft}
%% ===== this bit is set, the Hex mode becomes effective when the Indicator
%% ===== \end{flushleft}
%% ===== 
%% ===== 
%% ===== \begin{flushleft}
%% ===== Register bit 32 is set to 1.
%% ===== \end{flushleft}
%% ===== 
%% ===== 
%% ===== 
%% ===== 
%% ===== 
%% ===== \begin{flushleft}
%% ===== emr
%% ===== \end{flushleft}
%% ===== 
%% ===== 
%% ===== 
%% ===== 
%% ===== 
%% ===== \begin{flushleft}
%% ===== Enable Mode Register. Unless bit 35 = 1, all other bits in the Mode
%% ===== \end{flushleft}
%% ===== 
%% ===== 
%% ===== \begin{flushleft}
%% ===== Register are ignored and the History Register is ignored and locked.
%% ===== \end{flushleft}
%% ===== 
%% ===== 
%% ===== 
%% ===== 
%% ===== 
%% ===== \begin{flushleft}
%% ===== m
%% ===== \end{flushleft}
%% ===== 
%% ===== 
%% ===== 
%% ===== 
%% ===== 
%% ===== \begin{flushleft}

\subsection{CACHE MODE REGISTER (CMR) - DPS AND L68}

%% ===== \end{flushleft}
%% ===== 
%% ===== 
%% ===== \begin{flushleft}
%% ===== Format: - 28 bits
%% ===== \end{flushleft}
%% ===== 
%% ===== 
%% ===== \begin{flushleft}
%% ===== Odd word of Y-pair as stored by Store Central Processor Register (scpr), TAG = 06
%% ===== \end{flushleft}
%% ===== 
%% ===== 
%% ===== 3
%% ===== 
%% ===== 
%% ===== 6
%% ===== 
%% ===== 
%% ===== 
%% ===== 
%% ===== 
%% ===== 5 5 5 5 5 5 5 5 5 5 6 6 6 6 6
%% ===== 
%% ===== 
%% ===== 0 1 2 3 4 5 6 7 8 9 0 1 2 3 4
%% ===== 
%% ===== 
%% ===== \begin{flushleft}
%% ===== CACHE DIR ADDRESS
%% ===== \end{flushleft}
%% ===== 
%% ===== 
%% ===== 
%% ===== 
%% ===== 
%% ===== \begin{flushleft}
%% ===== a b 0 c d e f 0 g h i
%% ===== \end{flushleft}
%% ===== 
%% ===== 
%% ===== 1 1 1 1 1 1 1 1 1 1 1 1
%% ===== 
%% ===== 
%% ===== 
%% ===== 
%% ===== 
%% ===== \begin{flushleft}
%% ===== j
%% ===== \end{flushleft}
%% ===== 
%% ===== 
%% ===== 
%% ===== 
%% ===== 
%% ===== 6 7 7
%% ===== 
%% ===== 
%% ===== 9 0 1
%% ===== 
%% ===== 
%% ===== 
%% ===== 
%% ===== 
%% ===== 0 0 0 0 0 0
%% ===== 
%% ===== 
%% ===== 2
%% ===== 
%% ===== 
%% ===== 
%% ===== 
%% ===== 
%% ===== 6
%% ===== 
%% ===== 
%% ===== 
%% ===== 
%% ===== 
%% ===== \begin{flushleft}
%% ===== Figure 3-22. Cache Mode Register (CMR) Format - DPS and L68
%% ===== \end{flushleft}
%% ===== 
%% ===== 
%% ===== 
%% ===== 
%% ===== 
%% ===== \begin{flushleft}
%% ===== k
%% ===== \end{flushleft}
%% ===== 
%% ===== 
%% ===== 2
%% ===== 
%% ===== 
%% ===== 
%% ===== 
%% ===== 
%% ===== \begin{flushleft}
%% ===== \newpage
%% ===== Description:
%% ===== \end{flushleft}
%% ===== 
%% ===== 
%% ===== \begin{flushleft}
%% ===== An assemblage of flags and registers from the control unit. The Mode Register and Cache
%% ===== \end{flushleft}
%% ===== 
%% ===== 
%% ===== \begin{flushleft}
%% ===== Mode Register are both stored into the Y-pair by the Store Central Processor Register
%% ===== \end{flushleft}
%% ===== 
%% ===== 
%% ===== \begin{flushleft}
%% ===== (scpr), TAG = 06, instruction. The Cache Mode Register is loaded with the Load Central
%% ===== \end{flushleft}
%% ===== 
%% ===== 
%% ===== \begin{flushleft}
%% ===== Processor Register (lcpr), TAG = 02, instruction.
%% ===== \end{flushleft}
%% ===== 
%% ===== 
%% ===== \begin{flushleft}
%% ===== The data stored from the cache mode register is address-dependent. The algorithm used to
%% ===== \end{flushleft}
%% ===== 
%% ===== 
%% ===== \begin{flushleft}
%% ===== map main memory into the cache memory (see Section 9) is effective for the Store Central
%% ===== \end{flushleft}
%% ===== 
%% ===== 
%% ===== \begin{flushleft}
%% ===== Processor Register (scpr) instruction. In general, the user may read out data from the
%% ===== \end{flushleft}
%% ===== 
%% ===== 
%% ===== \begin{flushleft}
%% ===== directory entry for any cache memory block by proper selection of certain subfields in the
%% ===== \end{flushleft}
%% ===== 
%% ===== 
%% ===== \begin{flushleft}
%% ===== 24-bit absolute main memory address. In particular, the user may read out the directory
%% ===== \end{flushleft}
%% ===== 
%% ===== 
%% ===== \begin{flushleft}
%% ===== entry for the cache memory block involved in a suspected cache memory error by ensuring
%% ===== \end{flushleft}
%% ===== 
%% ===== 
%% ===== \begin{flushleft}
%% ===== that the required 24-bit absolute main memory address subfields are the same as those for
%% ===== \end{flushleft}
%% ===== 
%% ===== 
%% ===== \begin{flushleft}
%% ===== the access which produced the suspected error.
%% ===== \end{flushleft}
%% ===== 
%% ===== 
%% ===== \begin{flushleft}
%% ===== The fault handling procedure(s) should be unencacheable (SDW.C = 0) and the history
%% ===== \end{flushleft}
%% ===== 
%% ===== 
%% ===== \begin{flushleft}
%% ===== registers and cache memory should be disabled as quickly as possible in order that vital
%% ===== \end{flushleft}
%% ===== 
%% ===== 
%% ===== \begin{flushleft}
%% ===== information concerning the suspected error not be lost.
%% ===== \end{flushleft}
%% ===== 
%% ===== 
%% ===== \begin{flushleft}
%% ===== Function:
%% ===== \end{flushleft}
%% ===== 
%% ===== 
%% ===== \begin{flushleft}
%% ===== The Cache Mode register provides configuration information and software control over the
%% ===== \end{flushleft}
%% ===== 
%% ===== 
%% ===== \begin{flushleft}
%% ===== operation of the cache memory. Those items with an {``}x'' in the column headed L are not
%% ===== \end{flushleft}
%% ===== 
%% ===== 
%% ===== \begin{flushleft}
%% ===== loaded by the Load Central Processor Register (lcpr), TAG = 02, instruction.
%% ===== \end{flushleft}
%% ===== 
%% ===== 
%% ===== \begin{flushleft}
%% ===== The functions of the constituent flags and registers are:
%% ===== \end{flushleft}
%% ===== 
%% ===== 
%% ===== 
%% ===== 
%% ===== 
%% ===== \begin{flushleft}
%% ===== key L Register
%% ===== \end{flushleft}
%% ===== 
%% ===== 
%% ===== 
%% ===== 
%% ===== 
%% ===== \begin{flushleft}
%% ===== Function
%% ===== \end{flushleft}
%% ===== 
%% ===== 
%% ===== 
%% ===== 
%% ===== 
%% ===== \begin{flushleft}
%% ===== x CACHE DIR
%% ===== \end{flushleft}
%% ===== 
%% ===== 
%% ===== \begin{flushleft}
%% ===== ADDRESS
%% ===== \end{flushleft}
%% ===== 
%% ===== 
%% ===== 
%% ===== 
%% ===== 
%% ===== \begin{flushleft}
%% ===== 15 high-order bits of the cache memory block address from the
%% ===== \end{flushleft}
%% ===== 
%% ===== 
%% ===== \begin{flushleft}
%% ===== cache directory.
%% ===== \end{flushleft}
%% ===== 
%% ===== 
%% ===== 
%% ===== 
%% ===== 
%% ===== \begin{flushleft}
%% ===== a
%% ===== \end{flushleft}
%% ===== 
%% ===== 
%% ===== 
%% ===== 
%% ===== 
%% ===== \begin{flushleft}
%% ===== x PAR BIT
%% ===== \end{flushleft}
%% ===== 
%% ===== 
%% ===== 
%% ===== 
%% ===== 
%% ===== \begin{flushleft}
%% ===== Cache memory directory parity bit.
%% ===== \end{flushleft}
%% ===== 
%% ===== 
%% ===== 
%% ===== 
%% ===== 
%% ===== \begin{flushleft}
%% ===== b
%% ===== \end{flushleft}
%% ===== 
%% ===== 
%% ===== 
%% ===== 
%% ===== 
%% ===== \begin{flushleft}
%% ===== x LEV FUL
%% ===== \end{flushleft}
%% ===== 
%% ===== 
%% ===== 
%% ===== 
%% ===== 
%% ===== \begin{flushleft}
%% ===== The selected column and level is loaded with active data.
%% ===== \end{flushleft}
%% ===== 
%% ===== 
%% ===== 
%% ===== 
%% ===== 
%% ===== \begin{flushleft}
%% ===== c
%% ===== \end{flushleft}
%% ===== 
%% ===== 
%% ===== 
%% ===== 
%% ===== 
%% ===== \begin{flushleft}
%% ===== CSH1 ON
%% ===== \end{flushleft}
%% ===== 
%% ===== 
%% ===== 
%% ===== 
%% ===== 
%% ===== \begin{flushleft}
%% ===== Enable the upper 1024 words of the cache memory (see Section
%% ===== \end{flushleft}
%% ===== 
%% ===== 
%% ===== 9).
%% ===== 
%% ===== 
%% ===== 
%% ===== 
%% ===== 
%% ===== \begin{flushleft}
%% ===== d
%% ===== \end{flushleft}
%% ===== 
%% ===== 
%% ===== 
%% ===== 
%% ===== 
%% ===== \begin{flushleft}
%% ===== CSH2 ON
%% ===== \end{flushleft}
%% ===== 
%% ===== 
%% ===== 
%% ===== 
%% ===== 
%% ===== \begin{flushleft}
%% ===== Enable the lower 1024 words of the cache memory (see Section
%% ===== \end{flushleft}
%% ===== 
%% ===== 
%% ===== 9).
%% ===== 
%% ===== 
%% ===== 
%% ===== 
%% ===== 
%% ===== \begin{flushleft}
%% ===== e
%% ===== \end{flushleft}
%% ===== 
%% ===== 
%% ===== 
%% ===== 
%% ===== 
%% ===== \begin{flushleft}
%% ===== OPND ON
%% ===== \end{flushleft}
%% ===== 
%% ===== 
%% ===== 
%% ===== 
%% ===== 
%% ===== \begin{flushleft}
%% ===== Enable the cache memory for operands (see Section 9).
%% ===== \end{flushleft}
%% ===== 
%% ===== 
%% ===== 
%% ===== 
%% ===== 
%% ===== \begin{flushleft}
%% ===== f
%% ===== \end{flushleft}
%% ===== 
%% ===== 
%% ===== 
%% ===== 
%% ===== 
%% ===== \begin{flushleft}
%% ===== INST ON
%% ===== \end{flushleft}
%% ===== 
%% ===== 
%% ===== 
%% ===== 
%% ===== 
%% ===== \begin{flushleft}
%% ===== Enable the cache memory for instructions (see Section 9).
%% ===== \end{flushleft}
%% ===== 
%% ===== 
%% ===== 
%% ===== 
%% ===== 
%% ===== \begin{flushleft}
%% ===== g
%% ===== \end{flushleft}
%% ===== 
%% ===== 
%% ===== 
%% ===== 
%% ===== 
%% ===== \begin{flushleft}
%% ===== CSH REG
%% ===== \end{flushleft}
%% ===== 
%% ===== 
%% ===== 
%% ===== 
%% ===== 
%% ===== \begin{flushleft}
%% ===== Enable cache-to-register (dump) mode. When this bit is set ON,
%% ===== \end{flushleft}
%% ===== 
%% ===== 
%% ===== \begin{flushleft}
%% ===== double-precision operations unit read operands (e.g., Load AQ
%% ===== \end{flushleft}
%% ===== 
%% ===== 
%% ===== \begin{flushleft}
%% ===== (ldaq) operands) are read from the cache memory according to
%% ===== \end{flushleft}
%% ===== 
%% ===== 
%% ===== \begin{flushleft}
%% ===== the mapping algorithm and without regard to matching of the full
%% ===== \end{flushleft}
%% ===== 
%% ===== 
%% ===== \begin{flushleft}
%% ===== 24-bit absolute main memory address.
%% ===== \end{flushleft}
%% ===== 
%% ===== 
%% ===== \begin{flushleft}
%% ===== All other operands
%% ===== \end{flushleft}
%% ===== 
%% ===== 
%% ===== \begin{flushleft}
%% ===== address main memory as though the cache memory were
%% ===== \end{flushleft}
%% ===== 
%% ===== 
%% ===== \begin{flushleft}
%% ===== disabled. This bit is reset automatically by the hardware for any
%% ===== \end{flushleft}
%% ===== 
%% ===== 
%% ===== \begin{flushleft}
%% ===== fault or interrupt.
%% ===== \end{flushleft}
%% ===== 
%% ===== 
%% ===== 
%% ===== 
%% ===== 
%% ===== \begin{flushleft}
%% ===== h
%% ===== \end{flushleft}
%% ===== 
%% ===== 
%% ===== 
%% ===== 
%% ===== 
%% ===== \begin{flushleft}
%% ===== x STR ASD
%% ===== \end{flushleft}
%% ===== 
%% ===== 
%% ===== 
%% ===== 
%% ===== 
%% ===== \begin{flushleft}
%% ===== Enable store aside. When this bit is set ON, the processor does
%% ===== \end{flushleft}
%% ===== 
%% ===== 
%% ===== \begin{flushleft}
%% ===== not wait for main memory cycle completion after a store
%% ===== \end{flushleft}
%% ===== 
%% ===== 
%% ===== \begin{flushleft}
%% ===== operation but proceeds after the cache memory cycle is complete.
%% ===== \end{flushleft}
%% ===== 
%% ===== 
%% ===== 
%% ===== 
%% ===== 
%% ===== \begin{flushleft}
%% ===== i
%% ===== \end{flushleft}
%% ===== 
%% ===== 
%% ===== 
%% ===== 
%% ===== 
%% ===== \begin{flushleft}
%% ===== x COL FUL
%% ===== \end{flushleft}
%% ===== 
%% ===== 
%% ===== 
%% ===== 
%% ===== 
%% ===== \begin{flushleft}
%% ===== Selected cache memory column is full.
%% ===== \end{flushleft}
%% ===== 
%% ===== 
%% ===== 
%% ===== 
%% ===== 
%% ===== \begin{flushleft}
%% ===== j
%% ===== \end{flushleft}
%% ===== 
%% ===== 
%% ===== 
%% ===== 
%% ===== 
%% ===== \begin{flushleft}
%% ===== x RRO A,B
%% ===== \end{flushleft}
%% ===== 
%% ===== 
%% ===== 
%% ===== 
%% ===== 
%% ===== \begin{flushleft}
%% ===== Cache round robin counter (see Section 9).
%% ===== \end{flushleft}
%% ===== 
%% ===== 
%% ===== 
%% ===== 
%% ===== 
%% ===== \begin{flushleft}
%% ===== k
%% ===== \end{flushleft}
%% ===== 
%% ===== 
%% ===== 
%% ===== 
%% ===== 
%% ===== \begin{flushleft}
%% ===== LUF MSB,LSB
%% ===== \end{flushleft}
%% ===== 
%% ===== 
%% ===== 
%% ===== 
%% ===== 
%% ===== \begin{flushleft}
%% ===== Lockup timer setting. The lockup timer may be set to four
%% ===== \end{flushleft}
%% ===== 
%% ===== 
%% ===== \begin{flushleft}
%% ===== different values according to the value of this field.
%% ===== \end{flushleft}
%% ===== 
%% ===== 
%% ===== 
%% ===== 
%% ===== 
%% ===== \begin{flushleft}
%% ===== \newpage
%% ===== LUF value
%% ===== \end{flushleft}
%% ===== 
%% ===== 
%% ===== 0
%% ===== 
%% ===== 
%% ===== 1
%% ===== 
%% ===== 
%% ===== 2
%% ===== 
%% ===== 
%% ===== 3
%% ===== 
%% ===== 
%% ===== 
%% ===== 
%% ===== 
%% ===== \begin{flushleft}
%% ===== Lockup time
%% ===== \end{flushleft}
%% ===== 
%% ===== 
%% ===== \begin{flushleft}
%% ===== 2ms
%% ===== \end{flushleft}
%% ===== 
%% ===== 
%% ===== \begin{flushleft}
%% ===== 4ms
%% ===== \end{flushleft}
%% ===== 
%% ===== 
%% ===== \begin{flushleft}
%% ===== 8ms
%% ===== \end{flushleft}
%% ===== 
%% ===== 
%% ===== \begin{flushleft}
%% ===== 16ms
%% ===== \end{flushleft}
%% ===== 
%% ===== 
%% ===== 
%% ===== 
%% ===== 
%% ===== \begin{flushleft}
%% ===== The lockup timer is set to 16ms when the processor is initialized.
%% ===== \end{flushleft}
%% ===== 
%% ===== 
%% ===== 
%% ===== 
%% ===== 
%% ===== \begin{flushleft}

\subsection{CACHE MODE REGISTER (CMR) - DPS 8M}

%% ===== \end{flushleft}
%% ===== 
%% ===== 
%% ===== \begin{flushleft}
%% ===== Format: - 36 bits
%% ===== \end{flushleft}
%% ===== 
%% ===== 
%% ===== \begin{flushleft}
%% ===== Odd word of Y-pair as stored by Store Central Processor Register (scpr), TAG = 06.
%% ===== \end{flushleft}
%% ===== 
%% ===== 
%% ===== 3
%% ===== 
%% ===== 
%% ===== 6
%% ===== 
%% ===== 
%% ===== 
%% ===== 
%% ===== 
%% ===== 5 5 5 5 5 5 5 5 5 5 6 6 6 6 6
%% ===== 
%% ===== 
%% ===== 0 1 2 3 4 5 6 7 8 9 0 1 2 3 4
%% ===== 
%% ===== 
%% ===== \begin{flushleft}
%% ===== CACHE DIR ADDRESS
%% ===== \end{flushleft}
%% ===== 
%% ===== 
%% ===== 
%% ===== 
%% ===== 
%% ===== \begin{flushleft}
%% ===== a b 0 c d 0 e 0 f g h
%% ===== \end{flushleft}
%% ===== 
%% ===== 
%% ===== 15 1 1 1 1 1 1 1 1 1 1 1
%% ===== 
%% ===== 
%% ===== 
%% ===== 
%% ===== 
%% ===== \begin{flushleft}
%% ===== i
%% ===== \end{flushleft}
%% ===== 
%% ===== 
%% ===== 
%% ===== 
%% ===== 
%% ===== 6 6 6 7 7
%% ===== 
%% ===== 
%% ===== 7 8 9 0 1
%% ===== 
%% ===== 
%% ===== 
%% ===== 
%% ===== 
%% ===== \begin{flushleft}
%% ===== 0 0 0 0 j 0
%% ===== \end{flushleft}
%% ===== 
%% ===== 
%% ===== 2
%% ===== 
%% ===== 
%% ===== 
%% ===== 
%% ===== 
%% ===== \begin{flushleft}
%% ===== k
%% ===== \end{flushleft}
%% ===== 
%% ===== 
%% ===== 
%% ===== 
%% ===== 
%% ===== 4 1 1
%% ===== 
%% ===== 
%% ===== 
%% ===== 
%% ===== 
%% ===== 2
%% ===== 
%% ===== 
%% ===== 
%% ===== 
%% ===== 
%% ===== \begin{flushleft}
%% ===== Figure 3-23. Cache Mode Register (CMR) Format - DPS 8M
%% ===== \end{flushleft}
%% ===== 
%% ===== 
%% ===== \begin{flushleft}
%% ===== Description:
%% ===== \end{flushleft}
%% ===== 
%% ===== 
%% ===== \begin{flushleft}
%% ===== An assemblage of flags and registers from the control unit. The Mode Register and Cache
%% ===== \end{flushleft}
%% ===== 
%% ===== 
%% ===== \begin{flushleft}
%% ===== Mode Register are both stored into the Y-pair by the Store Central Processor Register
%% ===== \end{flushleft}
%% ===== 
%% ===== 
%% ===== \begin{flushleft}
%% ===== (scpr), TAG = 06, instruction. The Cache Mode Register is loaded with the Load Central
%% ===== \end{flushleft}
%% ===== 
%% ===== 
%% ===== \begin{flushleft}
%% ===== Processor Register (lcpr), TAG = 02, instruction.
%% ===== \end{flushleft}
%% ===== 
%% ===== 
%% ===== \begin{flushleft}
%% ===== The data stored from the Cache Mode register is address-dependent. The algorithm used to
%% ===== \end{flushleft}
%% ===== 
%% ===== 
%% ===== \begin{flushleft}
%% ===== map main memory into the cache memory (see Section 9) is effective for the Store central
%% ===== \end{flushleft}
%% ===== 
%% ===== 
%% ===== \begin{flushleft}
%% ===== Processor Register (scpr) instruction. In general, the user may read out data from the
%% ===== \end{flushleft}
%% ===== 
%% ===== 
%% ===== \begin{flushleft}
%% ===== directory entry for any cache memory block by proper selection of certain subfields in the
%% ===== \end{flushleft}
%% ===== 
%% ===== 
%% ===== \begin{flushleft}
%% ===== 24-bit absolute main memory address. In particular, the user may read out the directory
%% ===== \end{flushleft}
%% ===== 
%% ===== 
%% ===== \begin{flushleft}
%% ===== entry for the cache memory block involved in a suspected cache memory error by ensuring
%% ===== \end{flushleft}
%% ===== 
%% ===== 
%% ===== \begin{flushleft}
%% ===== that the required 24-bit absolute main memory address subfields are the same as those for
%% ===== \end{flushleft}
%% ===== 
%% ===== 
%% ===== \begin{flushleft}
%% ===== the access which produced the suspected error.
%% ===== \end{flushleft}
%% ===== 
%% ===== 
%% ===== \begin{flushleft}
%% ===== The fault handling procedure(s) should be unencacheable (SDW.D = 0) and the history
%% ===== \end{flushleft}
%% ===== 
%% ===== 
%% ===== \begin{flushleft}
%% ===== registers and cache memory should be disabled as quickly as possible in order that vital
%% ===== \end{flushleft}
%% ===== 
%% ===== 
%% ===== \begin{flushleft}
%% ===== information concerning the suspected error not be lost.
%% ===== \end{flushleft}
%% ===== 
%% ===== 
%% ===== \begin{flushleft}
%% ===== Function:
%% ===== \end{flushleft}
%% ===== 
%% ===== 
%% ===== \begin{flushleft}
%% ===== The Cache Mode Register provides configuration information and software control over the
%% ===== \end{flushleft}
%% ===== 
%% ===== 
%% ===== \begin{flushleft}
%% ===== operation of the cache memory. Those items with an {``}x'' in the column headed L are not
%% ===== \end{flushleft}
%% ===== 
%% ===== 
%% ===== \begin{flushleft}
%% ===== loaded by the Load Central Processor Register (lcpr), TAG = 02, instruction.
%% ===== \end{flushleft}
%% ===== 
%% ===== 
%% ===== \begin{flushleft}
%% ===== The functions of the constituent flags and registers are:
%% ===== \end{flushleft}
%% ===== 
%% ===== 
%% ===== 
%% ===== 
%% ===== 
%% ===== \begin{flushleft}
%% ===== key L Register
%% ===== \end{flushleft}
%% ===== 
%% ===== 
%% ===== 
%% ===== 
%% ===== 
%% ===== \begin{flushleft}
%% ===== Function
%% ===== \end{flushleft}
%% ===== 
%% ===== 
%% ===== 
%% ===== 
%% ===== 
%% ===== \begin{flushleft}
%% ===== x CACHE DIR
%% ===== \end{flushleft}
%% ===== 
%% ===== 
%% ===== \begin{flushleft}
%% ===== ADDRESS
%% ===== \end{flushleft}
%% ===== 
%% ===== 
%% ===== 
%% ===== 
%% ===== 
%% ===== \begin{flushleft}
%% ===== 15 high-order bits of the cache memory block address from the
%% ===== \end{flushleft}
%% ===== 
%% ===== 
%% ===== \begin{flushleft}
%% ===== cache directory.
%% ===== \end{flushleft}
%% ===== 
%% ===== 
%% ===== 
%% ===== 
%% ===== 
%% ===== \begin{flushleft}
%% ===== a
%% ===== \end{flushleft}
%% ===== 
%% ===== 
%% ===== 
%% ===== 
%% ===== 
%% ===== \begin{flushleft}
%% ===== x PAR BIT
%% ===== \end{flushleft}
%% ===== 
%% ===== 
%% ===== 
%% ===== 
%% ===== 
%% ===== \begin{flushleft}
%% ===== Cache memory directory parity bit.
%% ===== \end{flushleft}
%% ===== 
%% ===== 
%% ===== 
%% ===== 
%% ===== 
%% ===== \begin{flushleft}
%% ===== b
%% ===== \end{flushleft}
%% ===== 
%% ===== 
%% ===== 
%% ===== 
%% ===== 
%% ===== \begin{flushleft}
%% ===== x LEV FUL
%% ===== \end{flushleft}
%% ===== 
%% ===== 
%% ===== 
%% ===== 
%% ===== 
%% ===== \begin{flushleft}
%% ===== The selected column and level is loaded with active data.
%% ===== \end{flushleft}
%% ===== 
%% ===== 
%% ===== 
%% ===== 
%% ===== 
%% ===== \begin{flushleft}
%% ===== \newpage
%% ===== key L Register
%% ===== \end{flushleft}
%% ===== 
%% ===== 
%% ===== 
%% ===== 
%% ===== 
%% ===== \begin{flushleft}
%% ===== Function
%% ===== \end{flushleft}
%% ===== 
%% ===== 
%% ===== 
%% ===== 
%% ===== 
%% ===== \begin{flushleft}
%% ===== c
%% ===== \end{flushleft}
%% ===== 
%% ===== 
%% ===== 
%% ===== 
%% ===== 
%% ===== \begin{flushleft}
%% ===== CSH1 ON
%% ===== \end{flushleft}
%% ===== 
%% ===== 
%% ===== 
%% ===== 
%% ===== 
%% ===== \begin{flushleft}
%% ===== Enable the upper 4096 words of the cache memory (see Section
%% ===== \end{flushleft}
%% ===== 
%% ===== 
%% ===== 9).
%% ===== 
%% ===== 
%% ===== 
%% ===== 
%% ===== 
%% ===== \begin{flushleft}
%% ===== d
%% ===== \end{flushleft}
%% ===== 
%% ===== 
%% ===== 
%% ===== 
%% ===== 
%% ===== \begin{flushleft}
%% ===== CSH2 ON
%% ===== \end{flushleft}
%% ===== 
%% ===== 
%% ===== 
%% ===== 
%% ===== 
%% ===== \begin{flushleft}
%% ===== Enable the lower 4096 words of the cache memory (see Section
%% ===== \end{flushleft}
%% ===== 
%% ===== 
%% ===== 9).
%% ===== 
%% ===== 
%% ===== 
%% ===== 
%% ===== 
%% ===== \begin{flushleft}
%% ===== e
%% ===== \end{flushleft}
%% ===== 
%% ===== 
%% ===== 
%% ===== 
%% ===== 
%% ===== \begin{flushleft}
%% ===== INST ON
%% ===== \end{flushleft}
%% ===== 
%% ===== 
%% ===== 
%% ===== 
%% ===== 
%% ===== \begin{flushleft}
%% ===== Enable the cache memory for instructions (see Section 9).
%% ===== \end{flushleft}
%% ===== 
%% ===== 
%% ===== 
%% ===== 
%% ===== 
%% ===== \begin{flushleft}
%% ===== f
%% ===== \end{flushleft}
%% ===== 
%% ===== 
%% ===== 
%% ===== 
%% ===== 
%% ===== \begin{flushleft}
%% ===== CSH REG
%% ===== \end{flushleft}
%% ===== 
%% ===== 
%% ===== 
%% ===== 
%% ===== 
%% ===== \begin{flushleft}
%% ===== Enable cache-to-register (dump) mode. When this bit is set ON,
%% ===== \end{flushleft}
%% ===== 
%% ===== 
%% ===== \begin{flushleft}
%% ===== double-precision operations unit read operands (e.g., Load AQ
%% ===== \end{flushleft}
%% ===== 
%% ===== 
%% ===== \begin{flushleft}
%% ===== (ldaq) operands) are read from the cache memory according to
%% ===== \end{flushleft}
%% ===== 
%% ===== 
%% ===== \begin{flushleft}
%% ===== the mapping algorithm and without regard to matching of the
%% ===== \end{flushleft}
%% ===== 
%% ===== 
%% ===== \begin{flushleft}
%% ===== full 24-bit absolute main memory address. All other operands
%% ===== \end{flushleft}
%% ===== 
%% ===== 
%% ===== \begin{flushleft}
%% ===== address main memory as though the cache memory were
%% ===== \end{flushleft}
%% ===== 
%% ===== 
%% ===== \begin{flushleft}
%% ===== disabled. This bit is reset automatically by the hardware for any
%% ===== \end{flushleft}
%% ===== 
%% ===== 
%% ===== \begin{flushleft}
%% ===== fault or interrupt.
%% ===== \end{flushleft}
%% ===== 
%% ===== 
%% ===== 
%% ===== 
%% ===== 
%% ===== \begin{flushleft}
%% ===== g
%% ===== \end{flushleft}
%% ===== 
%% ===== 
%% ===== 
%% ===== 
%% ===== 
%% ===== \begin{flushleft}
%% ===== x STR ASD
%% ===== \end{flushleft}
%% ===== 
%% ===== 
%% ===== 
%% ===== 
%% ===== 
%% ===== \begin{flushleft}
%% ===== Enable store aside. When this bit is set ON, the processor does
%% ===== \end{flushleft}
%% ===== 
%% ===== 
%% ===== \begin{flushleft}
%% ===== not wait for main memory cycle completion after a store
%% ===== \end{flushleft}
%% ===== 
%% ===== 
%% ===== \begin{flushleft}
%% ===== operation but proceeds after the cache memory cycle is
%% ===== \end{flushleft}
%% ===== 
%% ===== 
%% ===== \begin{flushleft}
%% ===== complete.
%% ===== \end{flushleft}
%% ===== 
%% ===== 
%% ===== 
%% ===== 
%% ===== 
%% ===== \begin{flushleft}
%% ===== h
%% ===== \end{flushleft}
%% ===== 
%% ===== 
%% ===== 
%% ===== 
%% ===== 
%% ===== \begin{flushleft}
%% ===== x COL FUL
%% ===== \end{flushleft}
%% ===== 
%% ===== 
%% ===== 
%% ===== 
%% ===== 
%% ===== \begin{flushleft}
%% ===== Selected cache memory column is full.
%% ===== \end{flushleft}
%% ===== 
%% ===== 
%% ===== 
%% ===== 
%% ===== 
%% ===== \begin{flushleft}
%% ===== i
%% ===== \end{flushleft}
%% ===== 
%% ===== 
%% ===== 
%% ===== 
%% ===== 
%% ===== \begin{flushleft}
%% ===== x RRO A,B
%% ===== \end{flushleft}
%% ===== 
%% ===== 
%% ===== 
%% ===== 
%% ===== 
%% ===== \begin{flushleft}
%% ===== Cache round-robin counter (see Section 9).
%% ===== \end{flushleft}
%% ===== 
%% ===== 
%% ===== 
%% ===== 
%% ===== 
%% ===== \begin{flushleft}
%% ===== j
%% ===== \end{flushleft}
%% ===== 
%% ===== 
%% ===== \begin{flushleft}
%% ===== k
%% ===== \end{flushleft}
%% ===== 
%% ===== 
%% ===== 
%% ===== 
%% ===== 
%% ===== \begin{flushleft}
%% ===== Bypass cache bit. Enables the bypass option of SDW.C when set
%% ===== \end{flushleft}
%% ===== 
%% ===== 
%% ===== \begin{flushleft}
%% ===== OFF. See Notes below for further information.
%% ===== \end{flushleft}
%% ===== 
%% ===== 
%% ===== \begin{flushleft}
%% ===== LUF MSB,LSB
%% ===== \end{flushleft}
%% ===== 
%% ===== 
%% ===== 
%% ===== 
%% ===== 
%% ===== \begin{flushleft}
%% ===== Lockup timer setting. The lockup timer may be set to four
%% ===== \end{flushleft}
%% ===== 
%% ===== 
%% ===== \begin{flushleft}
%% ===== different values according to the value of this field.
%% ===== \end{flushleft}
%% ===== 
%% ===== 
%% ===== 
%% ===== 
%% ===== 
%% ===== \begin{flushleft}
%% ===== LUF value
%% ===== \end{flushleft}
%% ===== 
%% ===== 
%% ===== 0
%% ===== 
%% ===== 
%% ===== 1
%% ===== 
%% ===== 
%% ===== 2
%% ===== 
%% ===== 
%% ===== 3
%% ===== 
%% ===== 
%% ===== 
%% ===== 
%% ===== 
%% ===== \begin{flushleft}
%% ===== Lockup time
%% ===== \end{flushleft}
%% ===== 
%% ===== 
%% ===== \begin{flushleft}
%% ===== 2ms
%% ===== \end{flushleft}
%% ===== 
%% ===== 
%% ===== \begin{flushleft}
%% ===== 4ms
%% ===== \end{flushleft}
%% ===== 
%% ===== 
%% ===== \begin{flushleft}
%% ===== 8ms
%% ===== \end{flushleft}
%% ===== 
%% ===== 
%% ===== \begin{flushleft}
%% ===== 16ms
%% ===== \end{flushleft}
%% ===== 
%% ===== 
%% ===== 
%% ===== 
%% ===== 
%% ===== \begin{flushleft}
%% ===== The lockup timer is set to 16ms when the processor is
%% ===== \end{flushleft}
%% ===== 
%% ===== 
%% ===== \begin{flushleft}
%% ===== initialized.
%% ===== \end{flushleft}
%% ===== 
%% ===== 
%% ===== \begin{flushleft}
%% ===== Notes
%% ===== \end{flushleft}
%% ===== 
%% ===== 
%% ===== \begin{flushleft}
%% ===== 1. The COL FUL, RRO A, RRO B, and CACHE DIR ADDRESS fields reflect different locations in
%% ===== \end{flushleft}
%% ===== 
%% ===== 
%% ===== \begin{flushleft}
%% ===== cache depending on the final (absolute) address of the scpr instruction storing this data.
%% ===== \end{flushleft}
%% ===== 
%% ===== 
%% ===== \begin{flushleft}
%% ===== 2. If either cache enable bit c or d changes from disable state to enable state, the entire cache
%% ===== \end{flushleft}
%% ===== 
%% ===== 
%% ===== \begin{flushleft}
%% ===== is cleared.
%% ===== \end{flushleft}
%% ===== 
%% ===== 
%% ===== \begin{flushleft}
%% ===== 3. The DPS 8M processors contain an 8k hardware-controlled cache memory. When running a
%% ===== \end{flushleft}
%% ===== 
%% ===== 
%% ===== \begin{flushleft}
%% ===== mixed configuration of DPS 8M and DPS/L68 processors, bit 68 of the CMR (reference j)
%% ===== \end{flushleft}
%% ===== 
%% ===== 
%% ===== \begin{flushleft}
%% ===== allows the DPS 8M processor to utilize software compatible with the older 2k software
%% ===== \end{flushleft}
%% ===== 
%% ===== 
%% ===== \begin{flushleft}
%% ===== controlled by the DPS/L68 and DPS processors. The following summarizes the operation of
%% ===== \end{flushleft}
%% ===== 
%% ===== 
%% ===== \begin{flushleft}
%% ===== the DPS 8M hardware-controlled cache.
%% ===== \end{flushleft}
%% ===== 
%% ===== 
%% ===== \begin{flushleft}
%% ===== a. The cache bypass option in the segment descriptor word is retained. An overriding
%% ===== \end{flushleft}
%% ===== 
%% ===== 
%% ===== \begin{flushleft}
%% ===== bypass enable, bit 68 of the Cache Mode Register, is added. The cache mode is set
%% ===== \end{flushleft}
%% ===== 
%% ===== 
%% ===== \begin{flushleft}
%% ===== as follows:
%% ===== \end{flushleft}
%% ===== 
%% ===== 
%% ===== 
%% ===== 
%% ===== 
%% ===== \begin{flushleft}
%% ===== \newpage
%% ===== SDW.C
%% ===== \end{flushleft}
%% ===== 
%% ===== 
%% ===== \begin{flushleft}
%% ===== Use Cache
%% ===== \end{flushleft}
%% ===== 
%% ===== 
%% ===== 
%% ===== 
%% ===== 
%% ===== \begin{flushleft}
%% ===== CMR68
%% ===== \end{flushleft}
%% ===== 
%% ===== 
%% ===== \begin{flushleft}
%% ===== X
%% ===== \end{flushleft}
%% ===== 
%% ===== 
%% ===== 
%% ===== 
%% ===== 
%% ===== \begin{flushleft}
%% ===== RESULTANT
%% ===== \end{flushleft}
%% ===== 
%% ===== 
%% ===== \begin{flushleft}
%% ===== CACHE MODE
%% ===== \end{flushleft}
%% ===== 
%% ===== 
%% ===== \begin{flushleft}
%% ===== Use Cache
%% ===== \end{flushleft}
%% ===== 
%% ===== 
%% ===== 
%% ===== 
%% ===== 
%% ===== \begin{flushleft}
%% ===== Bypass Cache
%% ===== \end{flushleft}
%% ===== 
%% ===== 
%% ===== 
%% ===== 
%% ===== 
%% ===== \begin{flushleft}
%% ===== Bypass Cache
%% ===== \end{flushleft}
%% ===== 
%% ===== 
%% ===== 
%% ===== 
%% ===== 
%% ===== \begin{flushleft}
%% ===== Bypass Cache
%% ===== \end{flushleft}
%% ===== 
%% ===== 
%% ===== 
%% ===== 
%% ===== 
%% ===== \begin{flushleft}
%% ===== Bypass Cache
%% ===== \end{flushleft}
%% ===== 
%% ===== 
%% ===== 
%% ===== 
%% ===== 
%% ===== \begin{flushleft}
%% ===== Use Cache
%% ===== \end{flushleft}
%% ===== 
%% ===== 
%% ===== 
%% ===== 
%% ===== 
%% ===== \begin{flushleft}
%% ===== Use Cache
%% ===== \end{flushleft}
%% ===== 
%% ===== 
%% ===== 
%% ===== 
%% ===== 
%% ===== \begin{flushleft}
%% ===== b. All close gate instructions, LDAC, LDQC, STAC, STACQ, and SZNC automatically bypass
%% ===== \end{flushleft}
%% ===== 
%% ===== 
%% ===== \begin{flushleft}
%% ===== cache. Two features are added to ensure integrity of gated shared data; one is added
%% ===== \end{flushleft}
%% ===== 
%% ===== 
%% ===== \begin{flushleft}
%% ===== during the close gate operation and the other during the open gate operation. The
%% ===== \end{flushleft}
%% ===== 
%% ===== 
%% ===== \begin{flushleft}
%% ===== instruction following the close gate instruction bypasses cache if the instruction is a
%% ===== \end{flushleft}
%% ===== 
%% ===== 
%% ===== \begin{flushleft}
%% ===== Read or a Read-alter-rewrite. The open gate operation must be performed with
%% ===== \end{flushleft}
%% ===== 
%% ===== 
%% ===== \begin{flushleft}
%% ===== either a STC2 or STACQ, which includes the synchronizing function.
%% ===== \end{flushleft}
%% ===== 
%% ===== 
%% ===== \begin{flushleft}
%% ===== The
%% ===== \end{flushleft}
%% ===== 
%% ===== 
%% ===== \begin{flushleft}
%% ===== synchronizing function forces the processor to delay the open gate operation until it
%% ===== \end{flushleft}
%% ===== 
%% ===== 
%% ===== \begin{flushleft}
%% ===== is notified by the SCU that write completes have occurred and write notifications
%% ===== \end{flushleft}
%% ===== 
%% ===== 
%% ===== \begin{flushleft}
%% ===== requesting cache block clears have been sent to the other processors for all write
%% ===== \end{flushleft}
%% ===== 
%% ===== 
%% ===== \begin{flushleft}
%% ===== instructions that the processor previously issued.
%% ===== \end{flushleft}
%% ===== 
%% ===== 
%% ===== \begin{flushleft}
%% ===== c. Read-alter-rewrite instructions no longer automatically bypass cache.
%% ===== \end{flushleft}
%% ===== 
%% ===== 
%% ===== \begin{flushleft}
%% ===== Cache
%% ===== \end{flushleft}
%% ===== 
%% ===== 
%% ===== \begin{flushleft}
%% ===== behavior for these instructions is determined fully by SDW.C. If the bypass cache
%% ===== \end{flushleft}
%% ===== 
%% ===== 
%% ===== \begin{flushleft}
%% ===== mode is set, these instructions bypass cache and issue read-lock-write-unlock
%% ===== \end{flushleft}
%% ===== 
%% ===== 
%% ===== \begin{flushleft}
%% ===== commands to memory. If a cache directory match occurs, the location is cleared.
%% ===== \end{flushleft}
%% ===== 
%% ===== 
%% ===== \begin{flushleft}
%% ===== d. All accesses to memory by SDW and PTW associative memory hardware continue to
%% ===== \end{flushleft}
%% ===== 
%% ===== 
%% ===== \begin{flushleft}
%% ===== bypass cache. Operations are Reads for SDWs, Read-alter-rewrites with lock for
%% ===== \end{flushleft}
%% ===== 
%% ===== 
%% ===== \begin{flushleft}
%% ===== PTWs and setting the page Used bit, and Writes for setting the page Modified and
%% ===== \end{flushleft}
%% ===== 
%% ===== 
%% ===== \begin{flushleft}
%% ===== Used bits. For Writes, the hardware also disables the key line so that the SCU lock is
%% ===== \end{flushleft}
%% ===== 
%% ===== 
%% ===== \begin{flushleft}
%% ===== honored. This is consistent with dynamic PTW modification by software, which also
%% ===== \end{flushleft}
%% ===== 
%% ===== 
%% ===== \begin{flushleft}
%% ===== bypasses cache and uses Read-alter-rewrite instructions.
%% ===== \end{flushleft}
%% ===== 
%% ===== 
%% ===== \begin{flushleft}
%% ===== e. The instructions that cleared the associative memories and also cleared cache or
%% ===== \end{flushleft}
%% ===== 
%% ===== 
%% ===== \begin{flushleft}
%% ===== selective portions of cache are changed to eliminate the cache clear function. Bit C
%% ===== \end{flushleft}
%% ===== 
%% ===== 
%% ===== \begin{flushleft}
%% ===== (TPR.CA)15, is ignored. These instructions also include disable/enable capabilities
%% ===== \end{flushleft}
%% ===== 
%% ===== 
%% ===== \begin{flushleft}
%% ===== for each half of the associative memories.
%% ===== \end{flushleft}
%% ===== 
%% ===== 
%% ===== \begin{flushleft}
%% ===== f. Cache mode register bit 56, which had previously controlled cache bypass for
%% ===== \end{flushleft}
%% ===== 
%% ===== 
%% ===== \begin{flushleft}
%% ===== operands, is disregarded. All other cache control bits are continued. However,
%% ===== \end{flushleft}
%% ===== 
%% ===== 
%% ===== \begin{flushleft}
%% ===== maintenance panel cache control function is restricted to cache half enable/disable
%% ===== \end{flushleft}
%% ===== 
%% ===== 
%% ===== \begin{flushleft}
%% ===== functions.
%% ===== \end{flushleft}
%% ===== 
%% ===== 
%% ===== 
%% ===== 
%% ===== 
%% ===== \begin{flushleft}

\subsection{CONTROL UNIT (CU) HISTORY REGISTERS - DPS AND L68}

%% ===== \end{flushleft}
%% ===== 
%% ===== 
%% ===== \begin{flushleft}
%% ===== The L68 and DPS processors have four sets of 16 history requests. There is one set for each
%% ===== \end{flushleft}
%% ===== 
%% ===== 
%% ===== \begin{flushleft}
%% ===== major unit: the Control Unit, CU; the Operations Unit, OU; the Decimal Unit, DU; and the
%% ===== \end{flushleft}
%% ===== 
%% ===== 
%% ===== \begin{flushleft}
%% ===== Appending Unit, APU. The DPS 8M Processor has four sets of 64 history registers. There is one
%% ===== \end{flushleft}
%% ===== 
%% ===== 
%% ===== \begin{flushleft}
%% ===== set for the CU, two sets for the APU, and one set that combines the history of the OU and DU.
%% ===== \end{flushleft}
%% ===== 
%% ===== 
%% ===== \begin{flushleft}
%% ===== Because the history registers for the L68 and DPS and the DPS 8M are different in number
%% ===== \end{flushleft}
%% ===== 
%% ===== 
%% ===== \begin{flushleft}
%% ===== and content, they are described separately. The following section describes the L68 and DPS
%% ===== \end{flushleft}
%% ===== 
%% ===== 
%% ===== \begin{flushleft}
%% ===== history registers first, followed by a description of the DPS 8M history registers.
%% ===== \end{flushleft}
%% ===== 
%% ===== 
%% ===== 
%% ===== 
%% ===== 
%% ===== \begin{flushleft}
%% ===== \newpage
%% ===== Format: - 72 bits each
%% ===== \end{flushleft}
%% ===== 
%% ===== 
%% ===== \begin{flushleft}
%% ===== Even word of Y-pair as stored by Store Central Processor Register (scpr), TAG = 20
%% ===== \end{flushleft}
%% ===== 
%% ===== 
%% ===== 0 0 0 0 0 0 0 0 0 0 1 1 1 1 1 1 1 1 1
%% ===== 
%% ===== 
%% ===== 0 1 2 3 4 5 6 7 8 9 0 1 2 3 4 5 6 7 8
%% ===== 
%% ===== 
%% ===== \begin{flushleft}
%% ===== a b c d e f g h i
%% ===== \end{flushleft}
%% ===== 
%% ===== 
%% ===== 
%% ===== 
%% ===== 
%% ===== 2 2 3
%% ===== 
%% ===== 
%% ===== 8 9 0
%% ===== 
%% ===== 
%% ===== 
%% ===== 
%% ===== 
%% ===== \begin{flushleft}
%% ===== j k l m n o p q r
%% ===== \end{flushleft}
%% ===== 
%% ===== 
%% ===== 
%% ===== 
%% ===== 
%% ===== \begin{flushleft}
%% ===== OPCODE
%% ===== \end{flushleft}
%% ===== 
%% ===== 
%% ===== 
%% ===== 
%% ===== 
%% ===== 1 1 1 1 1 1 1 1 1 1 1 1 1 1 1 1 1 1
%% ===== 
%% ===== 
%% ===== 
%% ===== 
%% ===== 
%% ===== \begin{flushleft}
%% ===== I P
%% ===== \end{flushleft}
%% ===== 
%% ===== 
%% ===== 
%% ===== 
%% ===== 
%% ===== 3
%% ===== 
%% ===== 
%% ===== 5
%% ===== 
%% ===== 
%% ===== \begin{flushleft}
%% ===== TAG
%% ===== \end{flushleft}
%% ===== 
%% ===== 
%% ===== 
%% ===== 
%% ===== 
%% ===== 10 1 1
%% ===== 
%% ===== 
%% ===== 
%% ===== 
%% ===== 
%% ===== 6
%% ===== 
%% ===== 
%% ===== 
%% ===== 
%% ===== 
%% ===== \begin{flushleft}
%% ===== Odd word of Y-pair as stored by Store Central Processor Register (scpr), TAG = 20
%% ===== \end{flushleft}
%% ===== 
%% ===== 
%% ===== 3
%% ===== 
%% ===== 
%% ===== 6
%% ===== 
%% ===== 
%% ===== 
%% ===== 
%% ===== 
%% ===== 5 5
%% ===== 
%% ===== 
%% ===== 3 4
%% ===== 
%% ===== 
%% ===== \begin{flushleft}
%% ===== ADDRESS
%% ===== \end{flushleft}
%% ===== 
%% ===== 
%% ===== 
%% ===== 
%% ===== 
%% ===== 5 5
%% ===== 
%% ===== 
%% ===== 8 9
%% ===== 
%% ===== 
%% ===== \begin{flushleft}
%% ===== CMD
%% ===== \end{flushleft}
%% ===== 
%% ===== 
%% ===== 
%% ===== 
%% ===== 
%% ===== 18
%% ===== 
%% ===== 
%% ===== 
%% ===== 
%% ===== 
%% ===== 6 6 6 6 6 6 6 6 7 7
%% ===== 
%% ===== 
%% ===== 2 3 4 5 6 7 8 9 0 1
%% ===== 
%% ===== 
%% ===== \begin{flushleft}
%% ===== SEL
%% ===== \end{flushleft}
%% ===== 
%% ===== 
%% ===== 
%% ===== 
%% ===== 
%% ===== 5
%% ===== 
%% ===== 
%% ===== 
%% ===== 
%% ===== 
%% ===== \begin{flushleft}
%% ===== s t u v w x y z *
%% ===== \end{flushleft}
%% ===== 
%% ===== 
%% ===== 4 1 1 1 1 1 1 1 1 1
%% ===== 
%% ===== 
%% ===== 
%% ===== 
%% ===== 
%% ===== \begin{flushleft}
%% ===== Figure 3-24. Control Unit (CU) History Register Format - DPS and L68
%% ===== \end{flushleft}
%% ===== 
%% ===== 
%% ===== \begin{flushleft}
%% ===== Description:
%% ===== \end{flushleft}
%% ===== 
%% ===== 
%% ===== \begin{flushleft}
%% ===== A combination of 16 flags and registers from the control unit. The 16 registers are handled
%% ===== \end{flushleft}
%% ===== 
%% ===== 
%% ===== \begin{flushleft}
%% ===== as a rotating queue controlled by the Control Unit History Register counter. The counter is
%% ===== \end{flushleft}
%% ===== 
%% ===== 
%% ===== \begin{flushleft}
%% ===== always set to the number of the oldest entry and advances by one for each history register
%% ===== \end{flushleft}
%% ===== 
%% ===== 
%% ===== \begin{flushleft}
%% ===== reference (data entry or Store Central Processor Register (scpr) instruction). Multicycle
%% ===== \end{flushleft}
%% ===== 
%% ===== 
%% ===== \begin{flushleft}
%% ===== instructions (such as Load Pointer Registers from ITS Pairs (lpri), Load Registers (lreg),
%% ===== \end{flushleft}
%% ===== 
%% ===== 
%% ===== \begin{flushleft}
%% ===== Restore Control Unit (rcu), etc.) have an entry for each of their cycles.
%% ===== \end{flushleft}
%% ===== 
%% ===== 
%% ===== \begin{flushleft}
%% ===== Function:
%% ===== \end{flushleft}
%% ===== 
%% ===== 
%% ===== \begin{flushleft}
%% ===== A control unit history register entry shows the conditions at the end of the control unit cycle
%% ===== \end{flushleft}
%% ===== 
%% ===== 
%% ===== \begin{flushleft}
%% ===== to which it applies. The 16 registers hold the conditions for the last 16 control unit cycles.
%% ===== \end{flushleft}
%% ===== 
%% ===== 
%% ===== \begin{flushleft}
%% ===== Entries are made according to controls set in the Mode Register. (See Mode Register
%% ===== \end{flushleft}
%% ===== 
%% ===== 
%% ===== \begin{flushleft}
%% ===== earlier in this section.)
%% ===== \end{flushleft}
%% ===== 
%% ===== 
%% ===== \begin{flushleft}
%% ===== The meanings of the constituent flags and registers are:
%% ===== \end{flushleft}
%% ===== 
%% ===== 
%% ===== 
%% ===== 
%% ===== 
%% ===== \begin{flushleft}
%% ===== key Flag Name
%% ===== \end{flushleft}
%% ===== 
%% ===== 
%% ===== 
%% ===== 
%% ===== 
%% ===== \begin{flushleft}
%% ===== Meaning
%% ===== \end{flushleft}
%% ===== 
%% ===== 
%% ===== 
%% ===== 
%% ===== 
%% ===== \begin{flushleft}
%% ===== a
%% ===== \end{flushleft}
%% ===== 
%% ===== 
%% ===== 
%% ===== 
%% ===== 
%% ===== \begin{flushleft}
%% ===== PIA
%% ===== \end{flushleft}
%% ===== 
%% ===== 
%% ===== 
%% ===== 
%% ===== 
%% ===== \begin{flushleft}
%% ===== Prepare instruction address
%% ===== \end{flushleft}
%% ===== 
%% ===== 
%% ===== 
%% ===== 
%% ===== 
%% ===== \begin{flushleft}
%% ===== b
%% ===== \end{flushleft}
%% ===== 
%% ===== 
%% ===== 
%% ===== 
%% ===== 
%% ===== \begin{flushleft}
%% ===== POA
%% ===== \end{flushleft}
%% ===== 
%% ===== 
%% ===== 
%% ===== 
%% ===== 
%% ===== \begin{flushleft}
%% ===== Prepare operand address
%% ===== \end{flushleft}
%% ===== 
%% ===== 
%% ===== 
%% ===== 
%% ===== 
%% ===== \begin{flushleft}
%% ===== c
%% ===== \end{flushleft}
%% ===== 
%% ===== 
%% ===== 
%% ===== 
%% ===== 
%% ===== \begin{flushleft}
%% ===== RIW
%% ===== \end{flushleft}
%% ===== 
%% ===== 
%% ===== 
%% ===== 
%% ===== 
%% ===== \begin{flushleft}
%% ===== Request indirect word
%% ===== \end{flushleft}
%% ===== 
%% ===== 
%% ===== 
%% ===== 
%% ===== 
%% ===== \begin{flushleft}
%% ===== d
%% ===== \end{flushleft}
%% ===== 
%% ===== 
%% ===== 
%% ===== 
%% ===== 
%% ===== \begin{flushleft}
%% ===== SIW
%% ===== \end{flushleft}
%% ===== 
%% ===== 
%% ===== 
%% ===== 
%% ===== 
%% ===== \begin{flushleft}
%% ===== Restore indirect word
%% ===== \end{flushleft}
%% ===== 
%% ===== 
%% ===== 
%% ===== 
%% ===== 
%% ===== \begin{flushleft}
%% ===== e
%% ===== \end{flushleft}
%% ===== 
%% ===== 
%% ===== 
%% ===== 
%% ===== 
%% ===== \begin{flushleft}
%% ===== POT
%% ===== \end{flushleft}
%% ===== 
%% ===== 
%% ===== 
%% ===== 
%% ===== 
%% ===== \begin{flushleft}
%% ===== Prepare operand tally (indirect tally chain)
%% ===== \end{flushleft}
%% ===== 
%% ===== 
%% ===== 
%% ===== 
%% ===== 
%% ===== \begin{flushleft}
%% ===== f
%% ===== \end{flushleft}
%% ===== 
%% ===== 
%% ===== 
%% ===== 
%% ===== 
%% ===== \begin{flushleft}
%% ===== PON
%% ===== \end{flushleft}
%% ===== 
%% ===== 
%% ===== 
%% ===== 
%% ===== 
%% ===== \begin{flushleft}
%% ===== Prepare operand no tally (as for POT except no chain)
%% ===== \end{flushleft}
%% ===== 
%% ===== 
%% ===== 
%% ===== 
%% ===== 
%% ===== \begin{flushleft}
%% ===== g
%% ===== \end{flushleft}
%% ===== 
%% ===== 
%% ===== 
%% ===== 
%% ===== 
%% ===== \begin{flushleft}
%% ===== RAW
%% ===== \end{flushleft}
%% ===== 
%% ===== 
%% ===== 
%% ===== 
%% ===== 
%% ===== \begin{flushleft}
%% ===== Request read-alter-rewrite word
%% ===== \end{flushleft}
%% ===== 
%% ===== 
%% ===== 
%% ===== 
%% ===== 
%% ===== \begin{flushleft}
%% ===== h
%% ===== \end{flushleft}
%% ===== 
%% ===== 
%% ===== 
%% ===== 
%% ===== 
%% ===== \begin{flushleft}
%% ===== SAW
%% ===== \end{flushleft}
%% ===== 
%% ===== 
%% ===== 
%% ===== 
%% ===== 
%% ===== \begin{flushleft}
%% ===== Restore read-alter-rewrite word
%% ===== \end{flushleft}
%% ===== 
%% ===== 
%% ===== 
%% ===== 
%% ===== 
%% ===== \begin{flushleft}
%% ===== i
%% ===== \end{flushleft}
%% ===== 
%% ===== 
%% ===== 
%% ===== 
%% ===== 
%% ===== \begin{flushleft}
%% ===== TRGO
%% ===== \end{flushleft}
%% ===== 
%% ===== 
%% ===== 
%% ===== 
%% ===== 
%% ===== \begin{flushleft}
%% ===== Transfer GO (conditions met)
%% ===== \end{flushleft}
%% ===== 
%% ===== 
%% ===== 
%% ===== 
%% ===== 
%% ===== \begin{flushleft}
%% ===== j
%% ===== \end{flushleft}
%% ===== 
%% ===== 
%% ===== 
%% ===== 
%% ===== 
%% ===== \begin{flushleft}
%% ===== XDE
%% ===== \end{flushleft}
%% ===== 
%% ===== 
%% ===== 
%% ===== 
%% ===== 
%% ===== \begin{flushleft}
%% ===== Execute even instruction from Execute Double (xed) pair
%% ===== \end{flushleft}
%% ===== 
%% ===== 
%% ===== 
%% ===== 
%% ===== 
%% ===== \begin{flushleft}
%% ===== k
%% ===== \end{flushleft}
%% ===== 
%% ===== 
%% ===== 
%% ===== 
%% ===== 
%% ===== \begin{flushleft}
%% ===== XDO
%% ===== \end{flushleft}
%% ===== 
%% ===== 
%% ===== 
%% ===== 
%% ===== 
%% ===== \begin{flushleft}
%% ===== Execute odd instruction from Execute Double (xed) pair
%% ===== \end{flushleft}
%% ===== 
%% ===== 
%% ===== 
%% ===== 
%% ===== 
%% ===== \begin{flushleft}
%% ===== \newpage
%% ===== key Flag Name
%% ===== \end{flushleft}
%% ===== 
%% ===== 
%% ===== \begin{flushleft}
%% ===== l
%% ===== \end{flushleft}
%% ===== 
%% ===== 
%% ===== 
%% ===== 
%% ===== 
%% ===== \begin{flushleft}
%% ===== Meaning
%% ===== \end{flushleft}
%% ===== 
%% ===== 
%% ===== 
%% ===== 
%% ===== 
%% ===== \begin{flushleft}
%% ===== IC
%% ===== \end{flushleft}
%% ===== 
%% ===== 
%% ===== 
%% ===== 
%% ===== 
%% ===== \begin{flushleft}
%% ===== Execute odd instruction of the current pair
%% ===== \end{flushleft}
%% ===== 
%% ===== 
%% ===== 
%% ===== 
%% ===== 
%% ===== \begin{flushleft}
%% ===== m
%% ===== \end{flushleft}
%% ===== 
%% ===== 
%% ===== 
%% ===== 
%% ===== 
%% ===== \begin{flushleft}
%% ===== RPTS
%% ===== \end{flushleft}
%% ===== 
%% ===== 
%% ===== 
%% ===== 
%% ===== 
%% ===== \begin{flushleft}
%% ===== Execute a repeat instruction
%% ===== \end{flushleft}
%% ===== 
%% ===== 
%% ===== 
%% ===== 
%% ===== 
%% ===== \begin{flushleft}
%% ===== n
%% ===== \end{flushleft}
%% ===== 
%% ===== 
%% ===== 
%% ===== 
%% ===== 
%% ===== \begin{flushleft}
%% ===== WI
%% ===== \end{flushleft}
%% ===== 
%% ===== 
%% ===== 
%% ===== 
%% ===== 
%% ===== \begin{flushleft}
%% ===== Wait for instruction fetch
%% ===== \end{flushleft}
%% ===== 
%% ===== 
%% ===== 
%% ===== 
%% ===== 
%% ===== \begin{flushleft}
%% ===== o
%% ===== \end{flushleft}
%% ===== 
%% ===== 
%% ===== 
%% ===== 
%% ===== 
%% ===== \begin{flushleft}
%% ===== AR F/E
%% ===== \end{flushleft}
%% ===== 
%% ===== 
%% ===== 
%% ===== 
%% ===== 
%% ===== \begin{flushleft}
%% ===== 1 = ADDRESS has valid data
%% ===== \end{flushleft}
%% ===== 
%% ===== 
%% ===== 
%% ===== 
%% ===== 
%% ===== \begin{flushleft}
%% ===== p
%% ===== \end{flushleft}
%% ===== 
%% ===== 
%% ===== 
%% ===== 
%% ===== 
%% ===== \begin{flushleft}
%% ===== -XIP
%% ===== \end{flushleft}
%% ===== 
%% ===== 
%% ===== 
%% ===== 
%% ===== 
%% ===== \begin{flushleft}
%% ===== NOT prepare interrupt address
%% ===== \end{flushleft}
%% ===== 
%% ===== 
%% ===== 
%% ===== 
%% ===== 
%% ===== \begin{flushleft}
%% ===== q
%% ===== \end{flushleft}
%% ===== 
%% ===== 
%% ===== 
%% ===== 
%% ===== 
%% ===== \begin{flushleft}
%% ===== -FLT
%% ===== \end{flushleft}
%% ===== 
%% ===== 
%% ===== 
%% ===== 
%% ===== 
%% ===== \begin{flushleft}
%% ===== NOT prepare fault address
%% ===== \end{flushleft}
%% ===== 
%% ===== 
%% ===== 
%% ===== 
%% ===== 
%% ===== \begin{flushleft}
%% ===== r
%% ===== \end{flushleft}
%% ===== 
%% ===== 
%% ===== 
%% ===== 
%% ===== 
%% ===== \begin{flushleft}
%% ===== -BASE
%% ===== \end{flushleft}
%% ===== 
%% ===== 
%% ===== 
%% ===== 
%% ===== 
%% ===== \begin{flushleft}
%% ===== NOT BAR mode
%% ===== \end{flushleft}
%% ===== 
%% ===== 
%% ===== 
%% ===== 
%% ===== 
%% ===== \begin{flushleft}
%% ===== OPCODE
%% ===== \end{flushleft}
%% ===== 
%% ===== 
%% ===== 
%% ===== 
%% ===== 
%% ===== \begin{flushleft}
%% ===== Operation code from current instruction word
%% ===== \end{flushleft}
%% ===== 
%% ===== 
%% ===== 
%% ===== 
%% ===== 
%% ===== \begin{flushleft}
%% ===== I
%% ===== \end{flushleft}
%% ===== 
%% ===== 
%% ===== 
%% ===== 
%% ===== 
%% ===== \begin{flushleft}
%% ===== Interrupt inhibit bit from current instruction word
%% ===== \end{flushleft}
%% ===== 
%% ===== 
%% ===== 
%% ===== 
%% ===== 
%% ===== \begin{flushleft}
%% ===== P
%% ===== \end{flushleft}
%% ===== 
%% ===== 
%% ===== 
%% ===== 
%% ===== 
%% ===== \begin{flushleft}
%% ===== Pointer register flag bit from current instruction word
%% ===== \end{flushleft}
%% ===== 
%% ===== 
%% ===== 
%% ===== 
%% ===== 
%% ===== \begin{flushleft}
%% ===== TAG
%% ===== \end{flushleft}
%% ===== 
%% ===== 
%% ===== 
%% ===== 
%% ===== 
%% ===== \begin{flushleft}
%% ===== Current address modifier. This modifier is replaced by the contents of
%% ===== \end{flushleft}
%% ===== 
%% ===== 
%% ===== \begin{flushleft}
%% ===== the TAG fields of indirect words as they are fetched during indirect
%% ===== \end{flushleft}
%% ===== 
%% ===== 
%% ===== \begin{flushleft}
%% ===== chains.
%% ===== \end{flushleft}
%% ===== 
%% ===== 
%% ===== 
%% ===== 
%% ===== 
%% ===== \begin{flushleft}
%% ===== ADDRESS
%% ===== \end{flushleft}
%% ===== 
%% ===== 
%% ===== 
%% ===== 
%% ===== 
%% ===== \begin{flushleft}
%% ===== Current computed address (TPR.CA)
%% ===== \end{flushleft}
%% ===== 
%% ===== 
%% ===== 
%% ===== 
%% ===== 
%% ===== \begin{flushleft}
%% ===== CMD
%% ===== \end{flushleft}
%% ===== 
%% ===== 
%% ===== 
%% ===== 
%% ===== 
%% ===== \begin{flushleft}
%% ===== System controller command
%% ===== \end{flushleft}
%% ===== 
%% ===== 
%% ===== 
%% ===== 
%% ===== 
%% ===== \begin{flushleft}
%% ===== SEL
%% ===== \end{flushleft}
%% ===== 
%% ===== 
%% ===== 
%% ===== 
%% ===== 
%% ===== \begin{flushleft}
%% ===== Port select bits. (Valid only if port A-D is selected)
%% ===== \end{flushleft}
%% ===== 
%% ===== 
%% ===== 
%% ===== 
%% ===== 
%% ===== \begin{flushleft}
%% ===== s
%% ===== \end{flushleft}
%% ===== 
%% ===== 
%% ===== 
%% ===== 
%% ===== 
%% ===== \begin{flushleft}
%% ===== XEC-INT
%% ===== \end{flushleft}
%% ===== 
%% ===== 
%% ===== 
%% ===== 
%% ===== 
%% ===== \begin{flushleft}
%% ===== An interrupt is present
%% ===== \end{flushleft}
%% ===== 
%% ===== 
%% ===== 
%% ===== 
%% ===== 
%% ===== \begin{flushleft}
%% ===== t
%% ===== \end{flushleft}
%% ===== 
%% ===== 
%% ===== 
%% ===== 
%% ===== 
%% ===== \begin{flushleft}
%% ===== INS-FETCH
%% ===== \end{flushleft}
%% ===== 
%% ===== 
%% ===== 
%% ===== 
%% ===== 
%% ===== \begin{flushleft}
%% ===== Perform an instruction fetch
%% ===== \end{flushleft}
%% ===== 
%% ===== 
%% ===== 
%% ===== 
%% ===== 
%% ===== \begin{flushleft}
%% ===== u
%% ===== \end{flushleft}
%% ===== 
%% ===== 
%% ===== 
%% ===== 
%% ===== 
%% ===== \begin{flushleft}
%% ===== CU-STORE
%% ===== \end{flushleft}
%% ===== 
%% ===== 
%% ===== 
%% ===== 
%% ===== 
%% ===== \begin{flushleft}
%% ===== Control unit store cycle
%% ===== \end{flushleft}
%% ===== 
%% ===== 
%% ===== 
%% ===== 
%% ===== 
%% ===== \begin{flushleft}
%% ===== v
%% ===== \end{flushleft}
%% ===== 
%% ===== 
%% ===== 
%% ===== 
%% ===== 
%% ===== \begin{flushleft}
%% ===== OU-STORE
%% ===== \end{flushleft}
%% ===== 
%% ===== 
%% ===== 
%% ===== 
%% ===== 
%% ===== \begin{flushleft}
%% ===== Operations unit store cycle
%% ===== \end{flushleft}
%% ===== 
%% ===== 
%% ===== 
%% ===== 
%% ===== 
%% ===== \begin{flushleft}
%% ===== w
%% ===== \end{flushleft}
%% ===== 
%% ===== 
%% ===== 
%% ===== 
%% ===== 
%% ===== \begin{flushleft}
%% ===== CU-LOAD
%% ===== \end{flushleft}
%% ===== 
%% ===== 
%% ===== 
%% ===== 
%% ===== 
%% ===== \begin{flushleft}
%% ===== Control unit load cycle
%% ===== \end{flushleft}
%% ===== 
%% ===== 
%% ===== 
%% ===== 
%% ===== 
%% ===== \begin{flushleft}
%% ===== x
%% ===== \end{flushleft}
%% ===== 
%% ===== 
%% ===== 
%% ===== 
%% ===== 
%% ===== \begin{flushleft}
%% ===== OU-LOAD
%% ===== \end{flushleft}
%% ===== 
%% ===== 
%% ===== 
%% ===== 
%% ===== 
%% ===== \begin{flushleft}
%% ===== Operations unit load cycle
%% ===== \end{flushleft}
%% ===== 
%% ===== 
%% ===== 
%% ===== 
%% ===== 
%% ===== \begin{flushleft}
%% ===== y
%% ===== \end{flushleft}
%% ===== 
%% ===== 
%% ===== 
%% ===== 
%% ===== 
%% ===== \begin{flushleft}
%% ===== DIRECT
%% ===== \end{flushleft}
%% ===== 
%% ===== 
%% ===== 
%% ===== 
%% ===== 
%% ===== \begin{flushleft}
%% ===== Direct cycle
%% ===== \end{flushleft}
%% ===== 
%% ===== 
%% ===== 
%% ===== 
%% ===== 
%% ===== \begin{flushleft}
%% ===== z
%% ===== \end{flushleft}
%% ===== 
%% ===== 
%% ===== 
%% ===== 
%% ===== 
%% ===== \begin{flushleft}
%% ===== -PC-BUSY
%% ===== \end{flushleft}
%% ===== 
%% ===== 
%% ===== 
%% ===== 
%% ===== 
%% ===== \begin{flushleft}
%% ===== Port control logic not busy
%% ===== \end{flushleft}
%% ===== 
%% ===== 
%% ===== 
%% ===== 
%% ===== 
%% ===== *
%% ===== 
%% ===== 
%% ===== 
%% ===== 
%% ===== 
%% ===== \begin{flushleft}
%% ===== BUSY
%% ===== \end{flushleft}
%% ===== 
%% ===== 
%% ===== 
%% ===== 
%% ===== 
%% ===== \begin{flushleft}
%% ===== Port interface busy
%% ===== \end{flushleft}
%% ===== 
%% ===== 
%% ===== 
%% ===== 
%% ===== 
%% ===== \begin{flushleft}

\subsection{CONTROL UNIT (CU) HISTORY REGISTERS - DPS 8M}

%% ===== \end{flushleft}
%% ===== 
%% ===== 
%% ===== \begin{flushleft}
%% ===== Format: - 72 bits each
%% ===== \end{flushleft}
%% ===== 
%% ===== 
%% ===== \begin{flushleft}
%% ===== Even word of Y-pair as stored by Store Central Processor Register (scpr), TAG = 20
%% ===== \end{flushleft}
%% ===== 
%% ===== 
%% ===== 0 0 0 0 0 0 0 0 0 0 1 1 1 1 1 1 1 1 1
%% ===== 
%% ===== 
%% ===== 0 1 2 3 4 5 6 7 8 9 0 1 2 3 4 5 6 7 8
%% ===== 
%% ===== 
%% ===== \begin{flushleft}
%% ===== a b c d e f g h i
%% ===== \end{flushleft}
%% ===== 
%% ===== 
%% ===== 
%% ===== 
%% ===== 
%% ===== \begin{flushleft}
%% ===== j k l m n o p q r
%% ===== \end{flushleft}
%% ===== 
%% ===== 
%% ===== 
%% ===== 
%% ===== 
%% ===== 1 1 1 1 1 1 1 1 1 1 1 1 1 1 1 1 1 1
%% ===== 
%% ===== 
%% ===== 
%% ===== 
%% ===== 
%% ===== 2 2 3
%% ===== 
%% ===== 
%% ===== 8 9 0
%% ===== 
%% ===== 
%% ===== \begin{flushleft}
%% ===== OPCODE
%% ===== \end{flushleft}
%% ===== 
%% ===== 
%% ===== 
%% ===== 
%% ===== 
%% ===== \begin{flushleft}
%% ===== I P
%% ===== \end{flushleft}
%% ===== 
%% ===== 
%% ===== 10 1 1
%% ===== 
%% ===== 
%% ===== 
%% ===== 
%% ===== 
%% ===== 3
%% ===== 
%% ===== 
%% ===== 5
%% ===== 
%% ===== 
%% ===== \begin{flushleft}
%% ===== TAG
%% ===== \end{flushleft}
%% ===== 
%% ===== 
%% ===== 6
%% ===== 
%% ===== 
%% ===== 
%% ===== 
%% ===== 
%% ===== \begin{flushleft}
%% ===== \newpage
%% ===== Odd word of Y-pair as stored by Store Central Processor Register (scpr), TAG = 20
%% ===== \end{flushleft}
%% ===== 
%% ===== 
%% ===== 3
%% ===== 
%% ===== 
%% ===== 6
%% ===== 
%% ===== 
%% ===== 
%% ===== 
%% ===== 
%% ===== 5 6
%% ===== 
%% ===== 
%% ===== 9 0
%% ===== 
%% ===== 
%% ===== \begin{flushleft}
%% ===== ADDRESS
%% ===== \end{flushleft}
%% ===== 
%% ===== 
%% ===== 
%% ===== 
%% ===== 
%% ===== 6 6 6 6 6 6 7 7
%% ===== 
%% ===== 
%% ===== 4 5 6 7 8 9 0 1
%% ===== 
%% ===== 
%% ===== \begin{flushleft}
%% ===== CMD
%% ===== \end{flushleft}
%% ===== 
%% ===== 
%% ===== 
%% ===== 
%% ===== 
%% ===== 24
%% ===== 
%% ===== 
%% ===== 
%% ===== 
%% ===== 
%% ===== \begin{flushleft}
%% ===== s t u v w x 0
%% ===== \end{flushleft}
%% ===== 
%% ===== 
%% ===== 5 1 1 1 1 1 1 1
%% ===== 
%% ===== 
%% ===== 
%% ===== 
%% ===== 
%% ===== \begin{flushleft}
%% ===== Figure 3-25. Control Unit (CU) History Register Format - DPS 8M
%% ===== \end{flushleft}
%% ===== 
%% ===== 
%% ===== \begin{flushleft}
%% ===== Description:
%% ===== \end{flushleft}
%% ===== 
%% ===== 
%% ===== \begin{flushleft}
%% ===== A combination of 64 flags and registers from the control unit. The 64 registers are handled
%% ===== \end{flushleft}
%% ===== 
%% ===== 
%% ===== \begin{flushleft}
%% ===== as a rotating queue, controlled by the control unit history register counter, in which only the
%% ===== \end{flushleft}
%% ===== 
%% ===== 
%% ===== \begin{flushleft}
%% ===== 16 most recently used are stored (except in the event of a system crash in which case all 64
%% ===== \end{flushleft}
%% ===== 
%% ===== 
%% ===== \begin{flushleft}
%% ===== will be saved). The counter is always set to the number of the oldest entry and advances by
%% ===== \end{flushleft}
%% ===== 
%% ===== 
%% ===== \begin{flushleft}
%% ===== one for each history register reference (data entry or Store Central Processor Register
%% ===== \end{flushleft}
%% ===== 
%% ===== 
%% ===== \begin{flushleft}
%% ===== (scpr) instruction). Multicycle instructions (such as Load Pointer Registers from ITS Pairs
%% ===== \end{flushleft}
%% ===== 
%% ===== 
%% ===== \begin{flushleft}
%% ===== (lpri), Load Registers (lreg), Restore Control Unit (rcu), etc.) have an entry for each of
%% ===== \end{flushleft}
%% ===== 
%% ===== 
%% ===== \begin{flushleft}
%% ===== their cycles.
%% ===== \end{flushleft}
%% ===== 
%% ===== 
%% ===== \begin{flushleft}
%% ===== Function:
%% ===== \end{flushleft}
%% ===== 
%% ===== 
%% ===== \begin{flushleft}
%% ===== A control unit history register entry shows the conditions at the end of the control unit cycle
%% ===== \end{flushleft}
%% ===== 
%% ===== 
%% ===== \begin{flushleft}
%% ===== to which it applies. The 16 registers hold the conditions for the last 16 control unit cycles.
%% ===== \end{flushleft}
%% ===== 
%% ===== 
%% ===== \begin{flushleft}
%% ===== Entries are made according to controls set in the Mode Register. (See Mode Register
%% ===== \end{flushleft}
%% ===== 
%% ===== 
%% ===== \begin{flushleft}
%% ===== earlier in this section.)
%% ===== \end{flushleft}
%% ===== 
%% ===== 
%% ===== \begin{flushleft}
%% ===== The meanings of the constituent flags and registers are:
%% ===== \end{flushleft}
%% ===== 
%% ===== 
%% ===== 
%% ===== 
%% ===== 
%% ===== \begin{flushleft}
%% ===== key Flag Name
%% ===== \end{flushleft}
%% ===== 
%% ===== 
%% ===== 
%% ===== 
%% ===== 
%% ===== \begin{flushleft}
%% ===== Meaning
%% ===== \end{flushleft}
%% ===== 
%% ===== 
%% ===== 
%% ===== 
%% ===== 
%% ===== \begin{flushleft}
%% ===== a
%% ===== \end{flushleft}
%% ===== 
%% ===== 
%% ===== 
%% ===== 
%% ===== 
%% ===== \begin{flushleft}
%% ===== PIA
%% ===== \end{flushleft}
%% ===== 
%% ===== 
%% ===== 
%% ===== 
%% ===== 
%% ===== \begin{flushleft}
%% ===== Prepare instruction address
%% ===== \end{flushleft}
%% ===== 
%% ===== 
%% ===== 
%% ===== 
%% ===== 
%% ===== \begin{flushleft}
%% ===== b
%% ===== \end{flushleft}
%% ===== 
%% ===== 
%% ===== 
%% ===== 
%% ===== 
%% ===== \begin{flushleft}
%% ===== POA
%% ===== \end{flushleft}
%% ===== 
%% ===== 
%% ===== 
%% ===== 
%% ===== 
%% ===== \begin{flushleft}
%% ===== Prepare operand address
%% ===== \end{flushleft}
%% ===== 
%% ===== 
%% ===== 
%% ===== 
%% ===== 
%% ===== \begin{flushleft}
%% ===== c
%% ===== \end{flushleft}
%% ===== 
%% ===== 
%% ===== 
%% ===== 
%% ===== 
%% ===== \begin{flushleft}
%% ===== RIW
%% ===== \end{flushleft}
%% ===== 
%% ===== 
%% ===== 
%% ===== 
%% ===== 
%% ===== \begin{flushleft}
%% ===== Request indirect word
%% ===== \end{flushleft}
%% ===== 
%% ===== 
%% ===== 
%% ===== 
%% ===== 
%% ===== \begin{flushleft}
%% ===== d
%% ===== \end{flushleft}
%% ===== 
%% ===== 
%% ===== 
%% ===== 
%% ===== 
%% ===== \begin{flushleft}
%% ===== SIW
%% ===== \end{flushleft}
%% ===== 
%% ===== 
%% ===== 
%% ===== 
%% ===== 
%% ===== \begin{flushleft}
%% ===== Restore indirect word
%% ===== \end{flushleft}
%% ===== 
%% ===== 
%% ===== 
%% ===== 
%% ===== 
%% ===== \begin{flushleft}
%% ===== e
%% ===== \end{flushleft}
%% ===== 
%% ===== 
%% ===== 
%% ===== 
%% ===== 
%% ===== \begin{flushleft}
%% ===== POT
%% ===== \end{flushleft}
%% ===== 
%% ===== 
%% ===== 
%% ===== 
%% ===== 
%% ===== \begin{flushleft}
%% ===== Prepare operand tally
%% ===== \end{flushleft}
%% ===== 
%% ===== 
%% ===== 
%% ===== 
%% ===== 
%% ===== \begin{flushleft}
%% ===== f
%% ===== \end{flushleft}
%% ===== 
%% ===== 
%% ===== 
%% ===== 
%% ===== 
%% ===== \begin{flushleft}
%% ===== PON
%% ===== \end{flushleft}
%% ===== 
%% ===== 
%% ===== 
%% ===== 
%% ===== 
%% ===== \begin{flushleft}
%% ===== Prepare operand no tally
%% ===== \end{flushleft}
%% ===== 
%% ===== 
%% ===== 
%% ===== 
%% ===== 
%% ===== \begin{flushleft}
%% ===== g
%% ===== \end{flushleft}
%% ===== 
%% ===== 
%% ===== 
%% ===== 
%% ===== 
%% ===== \begin{flushleft}
%% ===== RAW
%% ===== \end{flushleft}
%% ===== 
%% ===== 
%% ===== 
%% ===== 
%% ===== 
%% ===== \begin{flushleft}
%% ===== Request read-alter-rewrite word
%% ===== \end{flushleft}
%% ===== 
%% ===== 
%% ===== 
%% ===== 
%% ===== 
%% ===== \begin{flushleft}
%% ===== h
%% ===== \end{flushleft}
%% ===== 
%% ===== 
%% ===== 
%% ===== 
%% ===== 
%% ===== \begin{flushleft}
%% ===== SAW
%% ===== \end{flushleft}
%% ===== 
%% ===== 
%% ===== 
%% ===== 
%% ===== 
%% ===== \begin{flushleft}
%% ===== Restore read-alter-rewrite word
%% ===== \end{flushleft}
%% ===== 
%% ===== 
%% ===== 
%% ===== 
%% ===== 
%% ===== \begin{flushleft}
%% ===== i
%% ===== \end{flushleft}
%% ===== 
%% ===== 
%% ===== 
%% ===== 
%% ===== 
%% ===== \begin{flushleft}
%% ===== RTRGO
%% ===== \end{flushleft}
%% ===== 
%% ===== 
%% ===== 
%% ===== 
%% ===== 
%% ===== \begin{flushleft}
%% ===== Remember transfer GO (condition met)
%% ===== \end{flushleft}
%% ===== 
%% ===== 
%% ===== 
%% ===== 
%% ===== 
%% ===== \begin{flushleft}
%% ===== j
%% ===== \end{flushleft}
%% ===== 
%% ===== 
%% ===== 
%% ===== 
%% ===== 
%% ===== \begin{flushleft}
%% ===== XDE
%% ===== \end{flushleft}
%% ===== 
%% ===== 
%% ===== 
%% ===== 
%% ===== 
%% ===== \begin{flushleft}
%% ===== XED from even location
%% ===== \end{flushleft}
%% ===== 
%% ===== 
%% ===== 
%% ===== 
%% ===== 
%% ===== \begin{flushleft}
%% ===== k
%% ===== \end{flushleft}
%% ===== 
%% ===== 
%% ===== 
%% ===== 
%% ===== 
%% ===== \begin{flushleft}
%% ===== XDO
%% ===== \end{flushleft}
%% ===== 
%% ===== 
%% ===== 
%% ===== 
%% ===== 
%% ===== \begin{flushleft}
%% ===== XED from odd location
%% ===== \end{flushleft}
%% ===== 
%% ===== 
%% ===== 
%% ===== 
%% ===== 
%% ===== \begin{flushleft}
%% ===== l
%% ===== \end{flushleft}
%% ===== 
%% ===== 
%% ===== 
%% ===== 
%% ===== 
%% ===== \begin{flushleft}
%% ===== IC
%% ===== \end{flushleft}
%% ===== 
%% ===== 
%% ===== 
%% ===== 
%% ===== 
%% ===== \begin{flushleft}
%% ===== Even/odd instruction pair
%% ===== \end{flushleft}
%% ===== 
%% ===== 
%% ===== 
%% ===== 
%% ===== 
%% ===== \begin{flushleft}
%% ===== m
%% ===== \end{flushleft}
%% ===== 
%% ===== 
%% ===== 
%% ===== 
%% ===== 
%% ===== \begin{flushleft}
%% ===== RPTS
%% ===== \end{flushleft}
%% ===== 
%% ===== 
%% ===== 
%% ===== 
%% ===== 
%% ===== \begin{flushleft}
%% ===== Repeat operation
%% ===== \end{flushleft}
%% ===== 
%% ===== 
%% ===== 
%% ===== 
%% ===== 
%% ===== \begin{flushleft}
%% ===== n
%% ===== \end{flushleft}
%% ===== 
%% ===== 
%% ===== 
%% ===== 
%% ===== 
%% ===== \begin{flushleft}
%% ===== PORTF
%% ===== \end{flushleft}
%% ===== 
%% ===== 
%% ===== 
%% ===== 
%% ===== 
%% ===== \begin{flushleft}
%% ===== Memory cycle to port on previous cycle
%% ===== \end{flushleft}
%% ===== 
%% ===== 
%% ===== 
%% ===== 
%% ===== 
%% ===== \begin{flushleft}
%% ===== o
%% ===== \end{flushleft}
%% ===== 
%% ===== 
%% ===== 
%% ===== 
%% ===== 
%% ===== \begin{flushleft}
%% ===== INTERNAL
%% ===== \end{flushleft}
%% ===== 
%% ===== 
%% ===== 
%% ===== 
%% ===== 
%% ===== \begin{flushleft}
%% ===== Memory cycle to cache or direct on previous cycle
%% ===== \end{flushleft}
%% ===== 
%% ===== 
%% ===== 
%% ===== 
%% ===== 
%% ===== \begin{flushleft}
%% ===== p
%% ===== \end{flushleft}
%% ===== 
%% ===== 
%% ===== 
%% ===== 
%% ===== 
%% ===== \begin{flushleft}
%% ===== PAI
%% ===== \end{flushleft}
%% ===== 
%% ===== 
%% ===== 
%% ===== 
%% ===== 
%% ===== \begin{flushleft}
%% ===== Prepare interrupt address
%% ===== \end{flushleft}
%% ===== 
%% ===== 
%% ===== 
%% ===== 
%% ===== 
%% ===== \begin{flushleft}
%% ===== q
%% ===== \end{flushleft}
%% ===== 
%% ===== 
%% ===== 
%% ===== 
%% ===== 
%% ===== \begin{flushleft}
%% ===== PFA
%% ===== \end{flushleft}
%% ===== 
%% ===== 
%% ===== 
%% ===== 
%% ===== 
%% ===== \begin{flushleft}
%% ===== Prepare fault address
%% ===== \end{flushleft}
%% ===== 
%% ===== 
%% ===== 
%% ===== 
%% ===== 
%% ===== \begin{flushleft}
%% ===== r
%% ===== \end{flushleft}
%% ===== 
%% ===== 
%% ===== 
%% ===== 
%% ===== 
%% ===== \begin{flushleft}
%% ===== PRIV
%% ===== \end{flushleft}
%% ===== 
%% ===== 
%% ===== 
%% ===== 
%% ===== 
%% ===== \begin{flushleft}
%% ===== In privileged mode
%% ===== \end{flushleft}
%% ===== 
%% ===== 
%% ===== 
%% ===== 
%% ===== 
%% ===== \begin{flushleft}
%% ===== \newpage
%% ===== key Flag Name
%% ===== \end{flushleft}
%% ===== 
%% ===== 
%% ===== 
%% ===== 
%% ===== 
%% ===== \begin{flushleft}
%% ===== Meaning
%% ===== \end{flushleft}
%% ===== 
%% ===== 
%% ===== 
%% ===== 
%% ===== 
%% ===== \begin{flushleft}
%% ===== OPCODE
%% ===== \end{flushleft}
%% ===== 
%% ===== 
%% ===== 
%% ===== 
%% ===== 
%% ===== \begin{flushleft}
%% ===== Opcode of instruction
%% ===== \end{flushleft}
%% ===== 
%% ===== 
%% ===== 
%% ===== 
%% ===== 
%% ===== \begin{flushleft}
%% ===== I
%% ===== \end{flushleft}
%% ===== 
%% ===== 
%% ===== 
%% ===== 
%% ===== 
%% ===== \begin{flushleft}
%% ===== Inhibit interrupt bit
%% ===== \end{flushleft}
%% ===== 
%% ===== 
%% ===== 
%% ===== 
%% ===== 
%% ===== \begin{flushleft}
%% ===== P
%% ===== \end{flushleft}
%% ===== 
%% ===== 
%% ===== 
%% ===== 
%% ===== 
%% ===== \begin{flushleft}
%% ===== AR reg mod flag
%% ===== \end{flushleft}
%% ===== 
%% ===== 
%% ===== 
%% ===== 
%% ===== 
%% ===== \begin{flushleft}
%% ===== TAG
%% ===== \end{flushleft}
%% ===== 
%% ===== 
%% ===== 
%% ===== 
%% ===== 
%% ===== \begin{flushleft}
%% ===== Tag field of instruction
%% ===== \end{flushleft}
%% ===== 
%% ===== 
%% ===== 
%% ===== 
%% ===== 
%% ===== \begin{flushleft}
%% ===== ADDRESS
%% ===== \end{flushleft}
%% ===== 
%% ===== 
%% ===== 
%% ===== 
%% ===== 
%% ===== \begin{flushleft}
%% ===== Absolute mean address of instruction
%% ===== \end{flushleft}
%% ===== 
%% ===== 
%% ===== 
%% ===== 
%% ===== 
%% ===== \begin{flushleft}
%% ===== CMD
%% ===== \end{flushleft}
%% ===== 
%% ===== 
%% ===== 
%% ===== 
%% ===== 
%% ===== \begin{flushleft}
%% ===== Processor command register
%% ===== \end{flushleft}
%% ===== 
%% ===== 
%% ===== 
%% ===== 
%% ===== 
%% ===== \begin{flushleft}
%% ===== s
%% ===== \end{flushleft}
%% ===== 
%% ===== 
%% ===== 
%% ===== 
%% ===== 
%% ===== \begin{flushleft}
%% ===== XINT
%% ===== \end{flushleft}
%% ===== 
%% ===== 
%% ===== 
%% ===== 
%% ===== 
%% ===== \begin{flushleft}
%% ===== Execute instruction
%% ===== \end{flushleft}
%% ===== 
%% ===== 
%% ===== 
%% ===== 
%% ===== 
%% ===== \begin{flushleft}
%% ===== t
%% ===== \end{flushleft}
%% ===== 
%% ===== 
%% ===== 
%% ===== 
%% ===== 
%% ===== \begin{flushleft}
%% ===== IFT
%% ===== \end{flushleft}
%% ===== 
%% ===== 
%% ===== 
%% ===== 
%% ===== 
%% ===== \begin{flushleft}
%% ===== Instruction fetch
%% ===== \end{flushleft}
%% ===== 
%% ===== 
%% ===== 
%% ===== 
%% ===== 
%% ===== \begin{flushleft}
%% ===== u
%% ===== \end{flushleft}
%% ===== 
%% ===== 
%% ===== 
%% ===== 
%% ===== 
%% ===== \begin{flushleft}
%% ===== CRD
%% ===== \end{flushleft}
%% ===== 
%% ===== 
%% ===== 
%% ===== 
%% ===== 
%% ===== \begin{flushleft}
%% ===== Cache read, this CU cycle
%% ===== \end{flushleft}
%% ===== 
%% ===== 
%% ===== 
%% ===== 
%% ===== 
%% ===== \begin{flushleft}
%% ===== v
%% ===== \end{flushleft}
%% ===== 
%% ===== 
%% ===== 
%% ===== 
%% ===== 
%% ===== \begin{flushleft}
%% ===== MRD
%% ===== \end{flushleft}
%% ===== 
%% ===== 
%% ===== 
%% ===== 
%% ===== 
%% ===== \begin{flushleft}
%% ===== Memory read, this CU cycle
%% ===== \end{flushleft}
%% ===== 
%% ===== 
%% ===== 
%% ===== 
%% ===== 
%% ===== \begin{flushleft}
%% ===== w
%% ===== \end{flushleft}
%% ===== 
%% ===== 
%% ===== 
%% ===== 
%% ===== 
%% ===== \begin{flushleft}
%% ===== MSTO
%% ===== \end{flushleft}
%% ===== 
%% ===== 
%% ===== 
%% ===== 
%% ===== 
%% ===== \begin{flushleft}
%% ===== Memory store, this CU cycle
%% ===== \end{flushleft}
%% ===== 
%% ===== 
%% ===== 
%% ===== 
%% ===== 
%% ===== \begin{flushleft}
%% ===== x
%% ===== \end{flushleft}
%% ===== 
%% ===== 
%% ===== 
%% ===== 
%% ===== 
%% ===== \begin{flushleft}
%% ===== PIB
%% ===== \end{flushleft}
%% ===== 
%% ===== 
%% ===== 
%% ===== 
%% ===== 
%% ===== \begin{flushleft}
%% ===== Memory port interface busy
%% ===== \end{flushleft}
%% ===== 
%% ===== 
%% ===== 
%% ===== 
%% ===== 
%% ===== \begin{flushleft}

\subsection{OPERATIONS UNIT (OU) HISTORY REGISTERS - DPS AND L68}

%% ===== \end{flushleft}
%% ===== 
%% ===== 
%% ===== \begin{flushleft}
%% ===== Format: - 72 bits each
%% ===== \end{flushleft}
%% ===== 
%% ===== 
%% ===== \begin{flushleft}
%% ===== Even word of Y-pair as stored by Store Central Processor Register (scpr), TAG = 40
%% ===== \end{flushleft}
%% ===== 
%% ===== 
%% ===== 0
%% ===== 
%% ===== 
%% ===== 0
%% ===== 
%% ===== 
%% ===== 
%% ===== 
%% ===== 
%% ===== 0 0 1
%% ===== 
%% ===== 
%% ===== 8 9 0
%% ===== 
%% ===== 
%% ===== 
%% ===== 
%% ===== 
%% ===== 1 1 1 1 1 1 1
%% ===== 
%% ===== 
%% ===== 2 3 4 5 6 7 8
%% ===== 
%% ===== 
%% ===== 
%% ===== 
%% ===== 
%% ===== \begin{flushleft}
%% ===== RP REG
%% ===== \end{flushleft}
%% ===== 
%% ===== 
%% ===== \begin{flushleft}
%% ===== OP CODE
%% ===== \end{flushleft}
%% ===== 
%% ===== 
%% ===== 
%% ===== 
%% ===== 
%% ===== \begin{flushleft}
%% ===== a
%% ===== \end{flushleft}
%% ===== 
%% ===== 
%% ===== 9 1
%% ===== 
%% ===== 
%% ===== 
%% ===== 
%% ===== 
%% ===== \begin{flushleft}
%% ===== b
%% ===== \end{flushleft}
%% ===== 
%% ===== 
%% ===== 
%% ===== 
%% ===== 
%% ===== 0
%% ===== 
%% ===== 
%% ===== \begin{flushleft}
%% ===== c d EAC
%% ===== \end{flushleft}
%% ===== 
%% ===== 
%% ===== 3 1 1
%% ===== 
%% ===== 
%% ===== 2 1
%% ===== 
%% ===== 
%% ===== 
%% ===== 
%% ===== 
%% ===== 2 2 2 2 3 3 3 3 3 3
%% ===== 
%% ===== 
%% ===== 6 7 8 9 0 1 2 3 4 5
%% ===== 
%% ===== 
%% ===== \begin{flushleft}
%% ===== RS REG
%% ===== \end{flushleft}
%% ===== 
%% ===== 
%% ===== 
%% ===== 
%% ===== 
%% ===== \begin{flushleft}
%% ===== e f g h i
%% ===== \end{flushleft}
%% ===== 
%% ===== 
%% ===== 
%% ===== 
%% ===== 
%% ===== \begin{flushleft}
%% ===== j k l m
%% ===== \end{flushleft}
%% ===== 
%% ===== 
%% ===== 
%% ===== 
%% ===== 
%% ===== 9 1 1 1 1 1 1 1 1 1
%% ===== 
%% ===== 
%% ===== 
%% ===== 
%% ===== 
%% ===== \begin{flushleft}
%% ===== Odd word of Y-pair as stored by Store Central Processor Register (scpr), TAG = 40
%% ===== \end{flushleft}
%% ===== 
%% ===== 
%% ===== 3 3 3 3 4 4 4 4 4 4 4 4 4 4 5 5
%% ===== 
%% ===== 
%% ===== 6 7 8 9 0 1 2 3 4 5 6 7 8 9 0 1
%% ===== 
%% ===== 
%% ===== 
%% ===== 
%% ===== 
%% ===== 5 5
%% ===== 
%% ===== 
%% ===== 3 4
%% ===== 
%% ===== 
%% ===== 
%% ===== 
%% ===== 
%% ===== \begin{flushleft}
%% ===== n o p q r s t u v w x y z A B 0 0 0
%% ===== \end{flushleft}
%% ===== 
%% ===== 
%% ===== 1 1 1 1 1 1 1 1 1 1 1 1 1 1 1
%% ===== 
%% ===== 
%% ===== 
%% ===== 
%% ===== 
%% ===== 7
%% ===== 
%% ===== 
%% ===== 1
%% ===== 
%% ===== 
%% ===== \begin{flushleft}
%% ===== ICT TRACKER
%% ===== \end{flushleft}
%% ===== 
%% ===== 
%% ===== 
%% ===== 
%% ===== 
%% ===== 3
%% ===== 
%% ===== 
%% ===== 
%% ===== 
%% ===== 
%% ===== 18
%% ===== 
%% ===== 
%% ===== 
%% ===== 
%% ===== 
%% ===== \begin{flushleft}
%% ===== Figure 3-26. Operations Unit (OU) History Register Format
%% ===== \end{flushleft}
%% ===== 
%% ===== 
%% ===== \begin{flushleft}
%% ===== Description:
%% ===== \end{flushleft}
%% ===== 
%% ===== 
%% ===== \begin{flushleft}
%% ===== A combination of 16 flags and registers from the operation unit and control unit. The 16
%% ===== \end{flushleft}
%% ===== 
%% ===== 
%% ===== \begin{flushleft}
%% ===== registers are handled as a rotating queue controlled by the operations unit history register
%% ===== \end{flushleft}
%% ===== 
%% ===== 
%% ===== \begin{flushleft}
%% ===== counter. The counter is always set to the number of the oldest entry and advances by one
%% ===== \end{flushleft}
%% ===== 
%% ===== 
%% ===== \begin{flushleft}
%% ===== for each history register reference (data entry or Store Central Processor Register (scpr)
%% ===== \end{flushleft}
%% ===== 
%% ===== 
%% ===== \begin{flushleft}
%% ===== instruction).
%% ===== \end{flushleft}
%% ===== 
%% ===== 
%% ===== \begin{flushleft}
%% ===== Function:
%% ===== \end{flushleft}
%% ===== 
%% ===== 
%% ===== \begin{flushleft}
%% ===== An Operations Unit History Register entry shows the conditions at the end of the operations
%% ===== \end{flushleft}
%% ===== 
%% ===== 
%% ===== \begin{flushleft}
%% ===== unit cycle to which it applies. The 16 registers hold the conditions for the last 16 operations
%% ===== \end{flushleft}
%% ===== 
%% ===== 
%% ===== \begin{flushleft}
%% ===== unit cycles. As the operations unit performs various cycles in the execution of an
%% ===== \end{flushleft}
%% ===== 
%% ===== 
%% ===== 
%% ===== 
%% ===== 
%% ===== \begin{flushleft}
%% ===== \newpage
%% ===== instruction, it does not advance the counter for each such cycle. The counter is advanced
%% ===== \end{flushleft}
%% ===== 
%% ===== 
%% ===== \begin{flushleft}
%% ===== only at successful completion of the instruction or if the instruction is aborted for a fault
%% ===== \end{flushleft}
%% ===== 
%% ===== 
%% ===== \begin{flushleft}
%% ===== condition. Entries are made according to controls set in the Mode Register. (See Mode
%% ===== \end{flushleft}
%% ===== 
%% ===== 
%% ===== \begin{flushleft}
%% ===== Register earlier in this section.)
%% ===== \end{flushleft}
%% ===== 
%% ===== 
%% ===== \begin{flushleft}
%% ===== The meanings of the constituent flags and registers are:
%% ===== \end{flushleft}
%% ===== 
%% ===== 
%% ===== 
%% ===== 
%% ===== 
%% ===== \begin{flushleft}
%% ===== key Flag Name
%% ===== \end{flushleft}
%% ===== 
%% ===== 
%% ===== \begin{flushleft}
%% ===== RP REG
%% ===== \end{flushleft}
%% ===== 
%% ===== 
%% ===== 
%% ===== 
%% ===== 
%% ===== \begin{flushleft}
%% ===== Meaning
%% ===== \end{flushleft}
%% ===== 
%% ===== 
%% ===== \begin{flushleft}
%% ===== Primary operations unit operation register. RP REG receives the
%% ===== \end{flushleft}
%% ===== 
%% ===== 
%% ===== \begin{flushleft}
%% ===== operation code and other data for the next instruction from the control
%% ===== \end{flushleft}
%% ===== 
%% ===== 
%% ===== \begin{flushleft}
%% ===== unit during the control unit instruction fetch cycle while the operations
%% ===== \end{flushleft}
%% ===== 
%% ===== 
%% ===== \begin{flushleft}
%% ===== unit may be busy with a prior instruction.
%% ===== \end{flushleft}
%% ===== 
%% ===== 
%% ===== \begin{flushleft}
%% ===== RP REG is further
%% ===== \end{flushleft}
%% ===== 
%% ===== 
%% ===== \begin{flushleft}
%% ===== substructured as:
%% ===== \end{flushleft}
%% ===== 
%% ===== 
%% ===== 
%% ===== 
%% ===== 
%% ===== \begin{flushleft}
%% ===== OP CODE
%% ===== \end{flushleft}
%% ===== 
%% ===== 
%% ===== 
%% ===== 
%% ===== 
%% ===== \begin{flushleft}
%% ===== The 9 high-order bits of the 10-bit operation code from the
%% ===== \end{flushleft}
%% ===== 
%% ===== 
%% ===== \begin{flushleft}
%% ===== instruction word. Note that basic (non EIS) instructions do not
%% ===== \end{flushleft}
%% ===== 
%% ===== 
%% ===== \begin{flushleft}
%% ===== involve bit 27 hence the 9-bit field is sufficient to determine the
%% ===== \end{flushleft}
%% ===== 
%% ===== 
%% ===== \begin{flushleft}
%% ===== instruction.
%% ===== \end{flushleft}
%% ===== 
%% ===== 
%% ===== 
%% ===== 
%% ===== 
%% ===== \begin{flushleft}
%% ===== a
%% ===== \end{flushleft}
%% ===== 
%% ===== 
%% ===== 
%% ===== 
%% ===== 
%% ===== \begin{flushleft}
%% ===== 9 CHAR
%% ===== \end{flushleft}
%% ===== 
%% ===== 
%% ===== 
%% ===== 
%% ===== 
%% ===== \begin{flushleft}
%% ===== Character size for indirect then tally address modifiers
%% ===== \end{flushleft}
%% ===== 
%% ===== 
%% ===== \begin{flushleft}
%% ===== 0 = 6-bit
%% ===== \end{flushleft}
%% ===== 
%% ===== 
%% ===== \begin{flushleft}
%% ===== 1 = 9-bit
%% ===== \end{flushleft}
%% ===== 
%% ===== 
%% ===== 
%% ===== 
%% ===== 
%% ===== \begin{flushleft}
%% ===== b
%% ===== \end{flushleft}
%% ===== 
%% ===== 
%% ===== 
%% ===== 
%% ===== 
%% ===== \begin{flushleft}
%% ===== TAG1,2,3
%% ===== \end{flushleft}
%% ===== 
%% ===== 
%% ===== 
%% ===== 
%% ===== 
%% ===== \begin{flushleft}
%% ===== The 3 low-order bits of the address modifier from the instruction
%% ===== \end{flushleft}
%% ===== 
%% ===== 
%% ===== \begin{flushleft}
%% ===== word. This field may contain a character position for an indirect
%% ===== \end{flushleft}
%% ===== 
%% ===== 
%% ===== \begin{flushleft}
%% ===== then tally address modifier.
%% ===== \end{flushleft}
%% ===== 
%% ===== 
%% ===== 
%% ===== 
%% ===== 
%% ===== \begin{flushleft}
%% ===== c
%% ===== \end{flushleft}
%% ===== 
%% ===== 
%% ===== 
%% ===== 
%% ===== 
%% ===== \begin{flushleft}
%% ===== CR FLG
%% ===== \end{flushleft}
%% ===== 
%% ===== 
%% ===== 
%% ===== 
%% ===== 
%% ===== \begin{flushleft}
%% ===== Character operation flag
%% ===== \end{flushleft}
%% ===== 
%% ===== 
%% ===== 
%% ===== 
%% ===== 
%% ===== \begin{flushleft}
%% ===== d
%% ===== \end{flushleft}
%% ===== 
%% ===== 
%% ===== 
%% ===== 
%% ===== 
%% ===== \begin{flushleft}
%% ===== DR FLG
%% ===== \end{flushleft}
%% ===== 
%% ===== 
%% ===== 
%% ===== 
%% ===== 
%% ===== \begin{flushleft}
%% ===== Direct operation flag
%% ===== \end{flushleft}
%% ===== 
%% ===== 
%% ===== 
%% ===== 
%% ===== 
%% ===== \begin{flushleft}
%% ===== EAC
%% ===== \end{flushleft}
%% ===== 
%% ===== 
%% ===== 
%% ===== 
%% ===== 
%% ===== \begin{flushleft}
%% ===== Address counter for lreg/sreg instructions
%% ===== \end{flushleft}
%% ===== 
%% ===== 
%% ===== 
%% ===== 
%% ===== 
%% ===== \begin{flushleft}
%% ===== RS REG
%% ===== \end{flushleft}
%% ===== 
%% ===== 
%% ===== 
%% ===== 
%% ===== 
%% ===== \begin{flushleft}
%% ===== Secondary operations unit operation register. OP CODE is moved from
%% ===== \end{flushleft}
%% ===== 
%% ===== 
%% ===== \begin{flushleft}
%% ===== RP REG to RS REG during the operand fetch cycle and is held until
%% ===== \end{flushleft}
%% ===== 
%% ===== 
%% ===== \begin{flushleft}
%% ===== completion of the instruction.
%% ===== \end{flushleft}
%% ===== 
%% ===== 
%% ===== 
%% ===== 
%% ===== 
%% ===== \begin{flushleft}
%% ===== e
%% ===== \end{flushleft}
%% ===== 
%% ===== 
%% ===== 
%% ===== 
%% ===== 
%% ===== \begin{flushleft}
%% ===== RB1 FULL
%% ===== \end{flushleft}
%% ===== 
%% ===== 
%% ===== 
%% ===== 
%% ===== 
%% ===== \begin{flushleft}
%% ===== OP CODE buffer is loaded
%% ===== \end{flushleft}
%% ===== 
%% ===== 
%% ===== 
%% ===== 
%% ===== 
%% ===== \begin{flushleft}
%% ===== f
%% ===== \end{flushleft}
%% ===== 
%% ===== 
%% ===== 
%% ===== 
%% ===== 
%% ===== \begin{flushleft}
%% ===== RP FULL
%% ===== \end{flushleft}
%% ===== 
%% ===== 
%% ===== 
%% ===== 
%% ===== 
%% ===== \begin{flushleft}
%% ===== RP REG is loaded
%% ===== \end{flushleft}
%% ===== 
%% ===== 
%% ===== 
%% ===== 
%% ===== 
%% ===== \begin{flushleft}
%% ===== g
%% ===== \end{flushleft}
%% ===== 
%% ===== 
%% ===== 
%% ===== 
%% ===== 
%% ===== \begin{flushleft}
%% ===== RS FULL
%% ===== \end{flushleft}
%% ===== 
%% ===== 
%% ===== 
%% ===== 
%% ===== 
%% ===== \begin{flushleft}
%% ===== RS REG is loaded
%% ===== \end{flushleft}
%% ===== 
%% ===== 
%% ===== 
%% ===== 
%% ===== 
%% ===== \begin{flushleft}
%% ===== h
%% ===== \end{flushleft}
%% ===== 
%% ===== 
%% ===== 
%% ===== 
%% ===== 
%% ===== \begin{flushleft}
%% ===== GIN
%% ===== \end{flushleft}
%% ===== 
%% ===== 
%% ===== 
%% ===== 
%% ===== 
%% ===== \begin{flushleft}
%% ===== First cycle for all OU instructions
%% ===== \end{flushleft}
%% ===== 
%% ===== 
%% ===== 
%% ===== 
%% ===== 
%% ===== \begin{flushleft}
%% ===== i
%% ===== \end{flushleft}
%% ===== 
%% ===== 
%% ===== 
%% ===== 
%% ===== 
%% ===== \begin{flushleft}
%% ===== GOS
%% ===== \end{flushleft}
%% ===== 
%% ===== 
%% ===== 
%% ===== 
%% ===== 
%% ===== \begin{flushleft}
%% ===== Second cycle for multicycle OU instructions
%% ===== \end{flushleft}
%% ===== 
%% ===== 
%% ===== 
%% ===== 
%% ===== 
%% ===== \begin{flushleft}
%% ===== j
%% ===== \end{flushleft}
%% ===== 
%% ===== 
%% ===== 
%% ===== 
%% ===== 
%% ===== \begin{flushleft}
%% ===== GD1
%% ===== \end{flushleft}
%% ===== 
%% ===== 
%% ===== 
%% ===== 
%% ===== 
%% ===== \begin{flushleft}
%% ===== First divide cycle
%% ===== \end{flushleft}
%% ===== 
%% ===== 
%% ===== 
%% ===== 
%% ===== 
%% ===== \begin{flushleft}
%% ===== k
%% ===== \end{flushleft}
%% ===== 
%% ===== 
%% ===== 
%% ===== 
%% ===== 
%% ===== \begin{flushleft}
%% ===== GD2
%% ===== \end{flushleft}
%% ===== 
%% ===== 
%% ===== 
%% ===== 
%% ===== 
%% ===== \begin{flushleft}
%% ===== Second divide cycle
%% ===== \end{flushleft}
%% ===== 
%% ===== 
%% ===== 
%% ===== 
%% ===== 
%% ===== \begin{flushleft}
%% ===== l
%% ===== \end{flushleft}
%% ===== 
%% ===== 
%% ===== 
%% ===== 
%% ===== 
%% ===== \begin{flushleft}
%% ===== GOE
%% ===== \end{flushleft}
%% ===== 
%% ===== 
%% ===== 
%% ===== 
%% ===== 
%% ===== \begin{flushleft}
%% ===== Exponent compare cycle
%% ===== \end{flushleft}
%% ===== 
%% ===== 
%% ===== 
%% ===== 
%% ===== 
%% ===== \begin{flushleft}
%% ===== m
%% ===== \end{flushleft}
%% ===== 
%% ===== 
%% ===== 
%% ===== 
%% ===== 
%% ===== \begin{flushleft}
%% ===== GOA
%% ===== \end{flushleft}
%% ===== 
%% ===== 
%% ===== 
%% ===== 
%% ===== 
%% ===== \begin{flushleft}
%% ===== Mantissa alignment cycle
%% ===== \end{flushleft}
%% ===== 
%% ===== 
%% ===== 
%% ===== 
%% ===== 
%% ===== \begin{flushleft}
%% ===== n
%% ===== \end{flushleft}
%% ===== 
%% ===== 
%% ===== 
%% ===== 
%% ===== 
%% ===== \begin{flushleft}
%% ===== GOM
%% ===== \end{flushleft}
%% ===== 
%% ===== 
%% ===== 
%% ===== 
%% ===== 
%% ===== \begin{flushleft}
%% ===== General operations unit cycle
%% ===== \end{flushleft}
%% ===== 
%% ===== 
%% ===== 
%% ===== 
%% ===== 
%% ===== \begin{flushleft}
%% ===== o
%% ===== \end{flushleft}
%% ===== 
%% ===== 
%% ===== 
%% ===== 
%% ===== 
%% ===== \begin{flushleft}
%% ===== GON
%% ===== \end{flushleft}
%% ===== 
%% ===== 
%% ===== 
%% ===== 
%% ===== 
%% ===== \begin{flushleft}
%% ===== Normalize cycle
%% ===== \end{flushleft}
%% ===== 
%% ===== 
%% ===== 
%% ===== 
%% ===== 
%% ===== \begin{flushleft}
%% ===== p
%% ===== \end{flushleft}
%% ===== 
%% ===== 
%% ===== 
%% ===== 
%% ===== 
%% ===== \begin{flushleft}
%% ===== GOF
%% ===== \end{flushleft}
%% ===== 
%% ===== 
%% ===== 
%% ===== 
%% ===== 
%% ===== \begin{flushleft}
%% ===== Final operations unit cycle
%% ===== \end{flushleft}
%% ===== 
%% ===== 
%% ===== 
%% ===== 
%% ===== 
%% ===== \begin{flushleft}
%% ===== q
%% ===== \end{flushleft}
%% ===== 
%% ===== 
%% ===== 
%% ===== 
%% ===== 
%% ===== \begin{flushleft}
%% ===== STR OP
%% ===== \end{flushleft}
%% ===== 
%% ===== 
%% ===== 
%% ===== 
%% ===== 
%% ===== \begin{flushleft}
%% ===== Store (output) data available
%% ===== \end{flushleft}
%% ===== 
%% ===== 
%% ===== 
%% ===== 
%% ===== 
%% ===== \begin{flushleft}
%% ===== r
%% ===== \end{flushleft}
%% ===== 
%% ===== 
%% ===== 
%% ===== 
%% ===== 
%% ===== \begin{flushleft}
%% ===== -DA-AV
%% ===== \end{flushleft}
%% ===== 
%% ===== 
%% ===== 
%% ===== 
%% ===== 
%% ===== \begin{flushleft}
%% ===== Data not available
%% ===== \end{flushleft}
%% ===== 
%% ===== 
%% ===== 
%% ===== 
%% ===== 
%% ===== \begin{flushleft}
%% ===== s
%% ===== \end{flushleft}
%% ===== 
%% ===== 
%% ===== 
%% ===== 
%% ===== 
%% ===== \begin{flushleft}
%% ===== -A-REG
%% ===== \end{flushleft}
%% ===== 
%% ===== 
%% ===== 
%% ===== 
%% ===== 
%% ===== \begin{flushleft}
%% ===== A register not in use
%% ===== \end{flushleft}
%% ===== 
%% ===== 
%% ===== 
%% ===== 
%% ===== 
%% ===== \begin{flushleft}
%% ===== t
%% ===== \end{flushleft}
%% ===== 
%% ===== 
%% ===== 
%% ===== 
%% ===== 
%% ===== \begin{flushleft}
%% ===== -Q-REG
%% ===== \end{flushleft}
%% ===== 
%% ===== 
%% ===== 
%% ===== 
%% ===== 
%% ===== \begin{flushleft}
%% ===== Q register not in use
%% ===== \end{flushleft}
%% ===== 
%% ===== 
%% ===== 
%% ===== 
%% ===== 
%% ===== \begin{flushleft}
%% ===== u
%% ===== \end{flushleft}
%% ===== 
%% ===== 
%% ===== 
%% ===== 
%% ===== 
%% ===== \begin{flushleft}
%% ===== -X0-RG
%% ===== \end{flushleft}
%% ===== 
%% ===== 
%% ===== 
%% ===== 
%% ===== 
%% ===== \begin{flushleft}
%% ===== X0 not in use
%% ===== \end{flushleft}
%% ===== 
%% ===== 
%% ===== 
%% ===== 
%% ===== 
%% ===== \begin{flushleft}
%% ===== v
%% ===== \end{flushleft}
%% ===== 
%% ===== 
%% ===== 
%% ===== 
%% ===== 
%% ===== \begin{flushleft}
%% ===== -X1-RG
%% ===== \end{flushleft}
%% ===== 
%% ===== 
%% ===== 
%% ===== 
%% ===== 
%% ===== \begin{flushleft}
%% ===== X1 not in use
%% ===== \end{flushleft}
%% ===== 
%% ===== 
%% ===== 
%% ===== 
%% ===== 
%% ===== \begin{flushleft}
%% ===== \newpage
%% ===== w
%% ===== \end{flushleft}
%% ===== 
%% ===== 
%% ===== 
%% ===== 
%% ===== 
%% ===== \begin{flushleft}
%% ===== -X2-RG
%% ===== \end{flushleft}
%% ===== 
%% ===== 
%% ===== 
%% ===== 
%% ===== 
%% ===== \begin{flushleft}
%% ===== X2 not in use
%% ===== \end{flushleft}
%% ===== 
%% ===== 
%% ===== 
%% ===== 
%% ===== 
%% ===== \begin{flushleft}
%% ===== x
%% ===== \end{flushleft}
%% ===== 
%% ===== 
%% ===== 
%% ===== 
%% ===== 
%% ===== \begin{flushleft}
%% ===== -X3-RG
%% ===== \end{flushleft}
%% ===== 
%% ===== 
%% ===== 
%% ===== 
%% ===== 
%% ===== \begin{flushleft}
%% ===== X3 not in use
%% ===== \end{flushleft}
%% ===== 
%% ===== 
%% ===== 
%% ===== 
%% ===== 
%% ===== \begin{flushleft}
%% ===== y
%% ===== \end{flushleft}
%% ===== 
%% ===== 
%% ===== 
%% ===== 
%% ===== 
%% ===== \begin{flushleft}
%% ===== -X4-RG
%% ===== \end{flushleft}
%% ===== 
%% ===== 
%% ===== 
%% ===== 
%% ===== 
%% ===== \begin{flushleft}
%% ===== X4 not in use
%% ===== \end{flushleft}
%% ===== 
%% ===== 
%% ===== 
%% ===== 
%% ===== 
%% ===== \begin{flushleft}
%% ===== z
%% ===== \end{flushleft}
%% ===== 
%% ===== 
%% ===== 
%% ===== 
%% ===== 
%% ===== \begin{flushleft}
%% ===== -X5-RG
%% ===== \end{flushleft}
%% ===== 
%% ===== 
%% ===== 
%% ===== 
%% ===== 
%% ===== \begin{flushleft}
%% ===== X5 not in use
%% ===== \end{flushleft}
%% ===== 
%% ===== 
%% ===== 
%% ===== 
%% ===== 
%% ===== \begin{flushleft}
%% ===== A
%% ===== \end{flushleft}
%% ===== 
%% ===== 
%% ===== 
%% ===== 
%% ===== 
%% ===== \begin{flushleft}
%% ===== -X6-RG
%% ===== \end{flushleft}
%% ===== 
%% ===== 
%% ===== 
%% ===== 
%% ===== 
%% ===== \begin{flushleft}
%% ===== X6 not in use
%% ===== \end{flushleft}
%% ===== 
%% ===== 
%% ===== 
%% ===== 
%% ===== 
%% ===== \begin{flushleft}
%% ===== B
%% ===== \end{flushleft}
%% ===== 
%% ===== 
%% ===== 
%% ===== 
%% ===== 
%% ===== \begin{flushleft}
%% ===== -X7-RG
%% ===== \end{flushleft}
%% ===== 
%% ===== 
%% ===== 
%% ===== 
%% ===== 
%% ===== \begin{flushleft}
%% ===== X7 not in use
%% ===== \end{flushleft}
%% ===== 
%% ===== 
%% ===== 
%% ===== 
%% ===== 
%% ===== \begin{flushleft}
%% ===== ICT
%% ===== \end{flushleft}
%% ===== 
%% ===== 
%% ===== \begin{flushleft}
%% ===== TRACKER
%% ===== \end{flushleft}
%% ===== 
%% ===== 
%% ===== 
%% ===== 
%% ===== 
%% ===== \begin{flushleft}
%% ===== The current value of the instruction counter (PPR.IC). Since the Control
%% ===== \end{flushleft}
%% ===== 
%% ===== 
%% ===== \begin{flushleft}
%% ===== Unit and Operations Unit run asynchronously and overlap is usually
%% ===== \end{flushleft}
%% ===== 
%% ===== 
%% ===== \begin{flushleft}
%% ===== enabled, the value of ICT TRACKER may not be the address of the
%% ===== \end{flushleft}
%% ===== 
%% ===== 
%% ===== \begin{flushleft}
%% ===== operations unit instruction currently being executed.
%% ===== \end{flushleft}
%% ===== 
%% ===== 
%% ===== 
%% ===== 
%% ===== 
%% ===== \begin{flushleft}

\subsection{DECIMAL UNIT (DU) HISTORY REGISTERS - DPS AND L68}

%% ===== \end{flushleft}
%% ===== 
%% ===== 
%% ===== \begin{flushleft}
%% ===== Format: - 72 bits each
%% ===== \end{flushleft}
%% ===== 
%% ===== 
%% ===== \begin{flushleft}
%% ===== Decimal Unit History Register data is stored with the Store Central Processor Register
%% ===== \end{flushleft}
%% ===== 
%% ===== 
%% ===== \begin{flushleft}
%% ===== (scpr), TAG = 10, instruction. There is no format diagram because the data is defined as
%% ===== \end{flushleft}
%% ===== 
%% ===== 
%% ===== \begin{flushleft}
%% ===== individual bits.
%% ===== \end{flushleft}
%% ===== 
%% ===== 
%% ===== \begin{flushleft}
%% ===== Description:
%% ===== \end{flushleft}
%% ===== 
%% ===== 
%% ===== \begin{flushleft}
%% ===== A combination of 16 flags from the decimal unit. The 16 registers are handled as a rotating
%% ===== \end{flushleft}
%% ===== 
%% ===== 
%% ===== \begin{flushleft}
%% ===== queue controlled by the decimal unit history register counter. The counter is always set to
%% ===== \end{flushleft}
%% ===== 
%% ===== 
%% ===== \begin{flushleft}
%% ===== the number of the oldest entry and advances by one for each history register reference
%% ===== \end{flushleft}
%% ===== 
%% ===== 
%% ===== \begin{flushleft}
%% ===== (data entry or Store Central Processor Register (scpr) instruction).
%% ===== \end{flushleft}
%% ===== 
%% ===== 
%% ===== \begin{flushleft}
%% ===== The decimal unit and the control unit run synchronously. There is a control unit history
%% ===== \end{flushleft}
%% ===== 
%% ===== 
%% ===== \begin{flushleft}
%% ===== register entry for every decimal unit history register entry and vice versa (except for
%% ===== \end{flushleft}
%% ===== 
%% ===== 
%% ===== \begin{flushleft}
%% ===== instruction fetch and EIS descriptor fetch cycles). If the processor is not executing a
%% ===== \end{flushleft}
%% ===== 
%% ===== 
%% ===== \begin{flushleft}
%% ===== decimal instruction, the decimal unit history register entry shows an idle condition.
%% ===== \end{flushleft}
%% ===== 
%% ===== 
%% ===== \begin{flushleft}
%% ===== Function:
%% ===== \end{flushleft}
%% ===== 
%% ===== 
%% ===== \begin{flushleft}
%% ===== A decimal unit history register entry shows the conditions in the decimal unit at the end of
%% ===== \end{flushleft}
%% ===== 
%% ===== 
%% ===== \begin{flushleft}
%% ===== the control unit cycle to which it applies. The 16 registers hold the conditions for the last
%% ===== \end{flushleft}
%% ===== 
%% ===== 
%% ===== \begin{flushleft}
%% ===== 16 control unit cycles. Entries are made according to controls set in the Mode Register.
%% ===== \end{flushleft}
%% ===== 
%% ===== 
%% ===== \begin{flushleft}
%% ===== (See Mode Register earlier in this section.)
%% ===== \end{flushleft}
%% ===== 
%% ===== 
%% ===== \begin{flushleft}
%% ===== A minus (-) sign preceding the flag name indicates that the complement of the flag is shown.
%% ===== \end{flushleft}
%% ===== 
%% ===== 
%% ===== \begin{flushleft}
%% ===== Unused bits are set ON.
%% ===== \end{flushleft}
%% ===== 
%% ===== 
%% ===== \begin{flushleft}
%% ===== The meanings of the constituent flags are:
%% ===== \end{flushleft}
%% ===== 
%% ===== 
%% ===== 
%% ===== 
%% ===== 
%% ===== \begin{flushleft}
%% ===== bit Flag Name
%% ===== \end{flushleft}
%% ===== 
%% ===== 
%% ===== 
%% ===== 
%% ===== 
%% ===== \begin{flushleft}
%% ===== Meaning
%% ===== \end{flushleft}
%% ===== 
%% ===== 
%% ===== 
%% ===== 
%% ===== 
%% ===== 0
%% ===== 
%% ===== 
%% ===== 
%% ===== 
%% ===== 
%% ===== \begin{flushleft}
%% ===== -FPOL
%% ===== \end{flushleft}
%% ===== 
%% ===== 
%% ===== 
%% ===== 
%% ===== 
%% ===== \begin{flushleft}
%% ===== Prepare operand length
%% ===== \end{flushleft}
%% ===== 
%% ===== 
%% ===== 
%% ===== 
%% ===== 
%% ===== \begin{flushleft}
%% ===== l
%% ===== \end{flushleft}
%% ===== 
%% ===== 
%% ===== 
%% ===== 
%% ===== 
%% ===== \begin{flushleft}
%% ===== -FPOP
%% ===== \end{flushleft}
%% ===== 
%% ===== 
%% ===== 
%% ===== 
%% ===== 
%% ===== \begin{flushleft}
%% ===== Prepare operand pointer
%% ===== \end{flushleft}
%% ===== 
%% ===== 
%% ===== 
%% ===== 
%% ===== 
%% ===== 2
%% ===== 
%% ===== 
%% ===== 
%% ===== 
%% ===== 
%% ===== \begin{flushleft}
%% ===== -NEED-DESC
%% ===== \end{flushleft}
%% ===== 
%% ===== 
%% ===== 
%% ===== 
%% ===== 
%% ===== \begin{flushleft}
%% ===== Need descriptor
%% ===== \end{flushleft}
%% ===== 
%% ===== 
%% ===== 
%% ===== 
%% ===== 
%% ===== 3
%% ===== 
%% ===== 
%% ===== 
%% ===== 
%% ===== 
%% ===== \begin{flushleft}
%% ===== -SEL-ADR
%% ===== \end{flushleft}
%% ===== 
%% ===== 
%% ===== 
%% ===== 
%% ===== 
%% ===== \begin{flushleft}
%% ===== Select address register
%% ===== \end{flushleft}
%% ===== 
%% ===== 
%% ===== 
%% ===== 
%% ===== 
%% ===== 4
%% ===== 
%% ===== 
%% ===== 
%% ===== 
%% ===== 
%% ===== \begin{flushleft}
%% ===== -DLEN=DIRECT
%% ===== \end{flushleft}
%% ===== 
%% ===== 
%% ===== 
%% ===== 
%% ===== 
%% ===== \begin{flushleft}
%% ===== Length equals direct
%% ===== \end{flushleft}
%% ===== 
%% ===== 
%% ===== 
%% ===== 
%% ===== 
%% ===== 5
%% ===== 
%% ===== 
%% ===== 
%% ===== 
%% ===== 
%% ===== \begin{flushleft}
%% ===== -DFRST
%% ===== \end{flushleft}
%% ===== 
%% ===== 
%% ===== 
%% ===== 
%% ===== 
%% ===== \begin{flushleft}
%% ===== Descriptor processed for first time
%% ===== \end{flushleft}
%% ===== 
%% ===== 
%% ===== 
%% ===== 
%% ===== 
%% ===== 6
%% ===== 
%% ===== 
%% ===== 
%% ===== 
%% ===== 
%% ===== \begin{flushleft}
%% ===== -FEXR
%% ===== \end{flushleft}
%% ===== 
%% ===== 
%% ===== 
%% ===== 
%% ===== 
%% ===== \begin{flushleft}
%% ===== Extended register modification
%% ===== \end{flushleft}
%% ===== 
%% ===== 
%% ===== 
%% ===== 
%% ===== 
%% ===== 7
%% ===== 
%% ===== 
%% ===== 
%% ===== 
%% ===== 
%% ===== \begin{flushleft}
%% ===== -DLAST-FRST
%% ===== \end{flushleft}
%% ===== 
%% ===== 
%% ===== 
%% ===== 
%% ===== 
%% ===== \begin{flushleft}
%% ===== Last cycle of DFRST
%% ===== \end{flushleft}
%% ===== 
%% ===== 
%% ===== 
%% ===== 
%% ===== 
%% ===== \begin{flushleft}
%% ===== \newpage
%% ===== bit Flag Name
%% ===== \end{flushleft}
%% ===== 
%% ===== 
%% ===== 
%% ===== 
%% ===== 
%% ===== \begin{flushleft}
%% ===== Meaning
%% ===== \end{flushleft}
%% ===== 
%% ===== 
%% ===== 
%% ===== 
%% ===== 
%% ===== 8
%% ===== 
%% ===== 
%% ===== 
%% ===== 
%% ===== 
%% ===== \begin{flushleft}
%% ===== -DDU-LDEA
%% ===== \end{flushleft}
%% ===== 
%% ===== 
%% ===== 
%% ===== 
%% ===== 
%% ===== \begin{flushleft}
%% ===== Decimal unit load
%% ===== \end{flushleft}
%% ===== 
%% ===== 
%% ===== 
%% ===== 
%% ===== 
%% ===== 9
%% ===== 
%% ===== 
%% ===== 
%% ===== 
%% ===== 
%% ===== \begin{flushleft}
%% ===== -DDU-STAE
%% ===== \end{flushleft}
%% ===== 
%% ===== 
%% ===== 
%% ===== 
%% ===== 
%% ===== \begin{flushleft}
%% ===== Decimal unit store
%% ===== \end{flushleft}
%% ===== 
%% ===== 
%% ===== 
%% ===== 
%% ===== 
%% ===== 10
%% ===== 
%% ===== 
%% ===== 
%% ===== 
%% ===== 
%% ===== \begin{flushleft}
%% ===== -DREDO
%% ===== \end{flushleft}
%% ===== 
%% ===== 
%% ===== 
%% ===== 
%% ===== 
%% ===== \begin{flushleft}
%% ===== Redo operation without pointer and length update
%% ===== \end{flushleft}
%% ===== 
%% ===== 
%% ===== 
%% ===== 
%% ===== 
%% ===== 11
%% ===== 
%% ===== 
%% ===== 
%% ===== 
%% ===== 
%% ===== \begin{flushleft}
%% ===== -DLVL$<$WD-SZ
%% ===== \end{flushleft}
%% ===== 
%% ===== 
%% ===== 
%% ===== 
%% ===== 
%% ===== \begin{flushleft}
%% ===== Load with count less than word size
%% ===== \end{flushleft}
%% ===== 
%% ===== 
%% ===== 
%% ===== 
%% ===== 
%% ===== 12
%% ===== 
%% ===== 
%% ===== 
%% ===== 
%% ===== 
%% ===== \begin{flushleft}
%% ===== -EXH
%% ===== \end{flushleft}
%% ===== 
%% ===== 
%% ===== 
%% ===== 
%% ===== 
%% ===== \begin{flushleft}
%% ===== Exhaust
%% ===== \end{flushleft}
%% ===== 
%% ===== 
%% ===== 
%% ===== 
%% ===== 
%% ===== 13
%% ===== 
%% ===== 
%% ===== 
%% ===== 
%% ===== 
%% ===== \begin{flushleft}
%% ===== DEND-SEQ
%% ===== \end{flushleft}
%% ===== 
%% ===== 
%% ===== 
%% ===== 
%% ===== 
%% ===== \begin{flushleft}
%% ===== End of sequence
%% ===== \end{flushleft}
%% ===== 
%% ===== 
%% ===== 
%% ===== 
%% ===== 
%% ===== 14
%% ===== 
%% ===== 
%% ===== 
%% ===== 
%% ===== 
%% ===== \begin{flushleft}
%% ===== -DEND
%% ===== \end{flushleft}
%% ===== 
%% ===== 
%% ===== 
%% ===== 
%% ===== 
%% ===== \begin{flushleft}
%% ===== End of instruction
%% ===== \end{flushleft}
%% ===== 
%% ===== 
%% ===== 
%% ===== 
%% ===== 
%% ===== 15
%% ===== 
%% ===== 
%% ===== 
%% ===== 
%% ===== 
%% ===== \begin{flushleft}
%% ===== -DU=RD+WRT
%% ===== \end{flushleft}
%% ===== 
%% ===== 
%% ===== 
%% ===== 
%% ===== 
%% ===== \begin{flushleft}
%% ===== Decimal unit write-back
%% ===== \end{flushleft}
%% ===== 
%% ===== 
%% ===== 
%% ===== 
%% ===== 
%% ===== 16
%% ===== 
%% ===== 
%% ===== 
%% ===== 
%% ===== 
%% ===== \begin{flushleft}
%% ===== -PTRA00
%% ===== \end{flushleft}
%% ===== 
%% ===== 
%% ===== 
%% ===== 
%% ===== 
%% ===== \begin{flushleft}
%% ===== PR address bit 0
%% ===== \end{flushleft}
%% ===== 
%% ===== 
%% ===== 
%% ===== 
%% ===== 
%% ===== 17
%% ===== 
%% ===== 
%% ===== 
%% ===== 
%% ===== 
%% ===== \begin{flushleft}
%% ===== -PTRA01
%% ===== \end{flushleft}
%% ===== 
%% ===== 
%% ===== 
%% ===== 
%% ===== 
%% ===== \begin{flushleft}
%% ===== PR address bit l
%% ===== \end{flushleft}
%% ===== 
%% ===== 
%% ===== 
%% ===== 
%% ===== 
%% ===== 18
%% ===== 
%% ===== 
%% ===== 
%% ===== 
%% ===== 
%% ===== \begin{flushleft}
%% ===== FA/Il
%% ===== \end{flushleft}
%% ===== 
%% ===== 
%% ===== 
%% ===== 
%% ===== 
%% ===== \begin{flushleft}
%% ===== Descriptor l active
%% ===== \end{flushleft}
%% ===== 
%% ===== 
%% ===== 
%% ===== 
%% ===== 
%% ===== 19
%% ===== 
%% ===== 
%% ===== 
%% ===== 
%% ===== 
%% ===== \begin{flushleft}
%% ===== FA/I2
%% ===== \end{flushleft}
%% ===== 
%% ===== 
%% ===== 
%% ===== 
%% ===== 
%% ===== \begin{flushleft}
%% ===== Descriptor 2 active
%% ===== \end{flushleft}
%% ===== 
%% ===== 
%% ===== 
%% ===== 
%% ===== 
%% ===== 20
%% ===== 
%% ===== 
%% ===== 
%% ===== 
%% ===== 
%% ===== \begin{flushleft}
%% ===== FA/I3
%% ===== \end{flushleft}
%% ===== 
%% ===== 
%% ===== 
%% ===== 
%% ===== 
%% ===== \begin{flushleft}
%% ===== Descriptor 3 active
%% ===== \end{flushleft}
%% ===== 
%% ===== 
%% ===== 
%% ===== 
%% ===== 
%% ===== 21
%% ===== 
%% ===== 
%% ===== 
%% ===== 
%% ===== 
%% ===== \begin{flushleft}
%% ===== -WRD
%% ===== \end{flushleft}
%% ===== 
%% ===== 
%% ===== 
%% ===== 
%% ===== 
%% ===== \begin{flushleft}
%% ===== Word operation
%% ===== \end{flushleft}
%% ===== 
%% ===== 
%% ===== 
%% ===== 
%% ===== 
%% ===== 22
%% ===== 
%% ===== 
%% ===== 
%% ===== 
%% ===== 
%% ===== \begin{flushleft}
%% ===== -NINE
%% ===== \end{flushleft}
%% ===== 
%% ===== 
%% ===== 
%% ===== 
%% ===== 
%% ===== \begin{flushleft}
%% ===== 9-bit character operation
%% ===== \end{flushleft}
%% ===== 
%% ===== 
%% ===== 
%% ===== 
%% ===== 
%% ===== 23
%% ===== 
%% ===== 
%% ===== 
%% ===== 
%% ===== 
%% ===== \begin{flushleft}
%% ===== -SIX
%% ===== \end{flushleft}
%% ===== 
%% ===== 
%% ===== 
%% ===== 
%% ===== 
%% ===== \begin{flushleft}
%% ===== 6-bit character operation
%% ===== \end{flushleft}
%% ===== 
%% ===== 
%% ===== 
%% ===== 
%% ===== 
%% ===== 24
%% ===== 
%% ===== 
%% ===== 
%% ===== 
%% ===== 
%% ===== \begin{flushleft}
%% ===== -FOUR
%% ===== \end{flushleft}
%% ===== 
%% ===== 
%% ===== 
%% ===== 
%% ===== 
%% ===== \begin{flushleft}
%% ===== 4-bit character operation
%% ===== \end{flushleft}
%% ===== 
%% ===== 
%% ===== 
%% ===== 
%% ===== 
%% ===== 25
%% ===== 
%% ===== 
%% ===== 
%% ===== 
%% ===== 
%% ===== \begin{flushleft}
%% ===== -BIT
%% ===== \end{flushleft}
%% ===== 
%% ===== 
%% ===== 
%% ===== 
%% ===== 
%% ===== \begin{flushleft}
%% ===== Bit operation
%% ===== \end{flushleft}
%% ===== 
%% ===== 
%% ===== 
%% ===== 
%% ===== 
%% ===== 26
%% ===== 
%% ===== 
%% ===== 
%% ===== 
%% ===== 
%% ===== \begin{flushleft}
%% ===== Unused
%% ===== \end{flushleft}
%% ===== 
%% ===== 
%% ===== 
%% ===== 
%% ===== 
%% ===== 27
%% ===== 
%% ===== 
%% ===== 
%% ===== 
%% ===== 
%% ===== \begin{flushleft}
%% ===== Unused
%% ===== \end{flushleft}
%% ===== 
%% ===== 
%% ===== 
%% ===== 
%% ===== 
%% ===== 28
%% ===== 
%% ===== 
%% ===== 
%% ===== 
%% ===== 
%% ===== \begin{flushleft}
%% ===== Unused
%% ===== \end{flushleft}
%% ===== 
%% ===== 
%% ===== 
%% ===== 
%% ===== 
%% ===== 29
%% ===== 
%% ===== 
%% ===== 
%% ===== 
%% ===== 
%% ===== \begin{flushleft}
%% ===== Unused
%% ===== \end{flushleft}
%% ===== 
%% ===== 
%% ===== 
%% ===== 
%% ===== 
%% ===== 30
%% ===== 
%% ===== 
%% ===== 
%% ===== 
%% ===== 
%% ===== \begin{flushleft}
%% ===== FSAMPL
%% ===== \end{flushleft}
%% ===== 
%% ===== 
%% ===== 
%% ===== 
%% ===== 
%% ===== \begin{flushleft}
%% ===== Sample for mid-instruction interrupt
%% ===== \end{flushleft}
%% ===== 
%% ===== 
%% ===== 
%% ===== 
%% ===== 
%% ===== 31
%% ===== 
%% ===== 
%% ===== 
%% ===== 
%% ===== 
%% ===== \begin{flushleft}
%% ===== -DFRST-CT
%% ===== \end{flushleft}
%% ===== 
%% ===== 
%% ===== 
%% ===== 
%% ===== 
%% ===== \begin{flushleft}
%% ===== Specified first count of a sequence
%% ===== \end{flushleft}
%% ===== 
%% ===== 
%% ===== 
%% ===== 
%% ===== 
%% ===== 32
%% ===== 
%% ===== 
%% ===== 
%% ===== 
%% ===== 
%% ===== \begin{flushleft}
%% ===== -ADJ-LENGTH
%% ===== \end{flushleft}
%% ===== 
%% ===== 
%% ===== 
%% ===== 
%% ===== 
%% ===== \begin{flushleft}
%% ===== Adjust length
%% ===== \end{flushleft}
%% ===== 
%% ===== 
%% ===== 
%% ===== 
%% ===== 
%% ===== 33
%% ===== 
%% ===== 
%% ===== 
%% ===== 
%% ===== 
%% ===== \begin{flushleft}
%% ===== -INTRPTD
%% ===== \end{flushleft}
%% ===== 
%% ===== 
%% ===== 
%% ===== 
%% ===== 
%% ===== \begin{flushleft}
%% ===== Mid-instruction interrupt
%% ===== \end{flushleft}
%% ===== 
%% ===== 
%% ===== 
%% ===== 
%% ===== 
%% ===== 34
%% ===== 
%% ===== 
%% ===== 
%% ===== 
%% ===== 
%% ===== \begin{flushleft}
%% ===== -INHIB
%% ===== \end{flushleft}
%% ===== 
%% ===== 
%% ===== 
%% ===== 
%% ===== 
%% ===== \begin{flushleft}
%% ===== Inhibit STC1 (force {``}STC0'')
%% ===== \end{flushleft}
%% ===== 
%% ===== 
%% ===== 
%% ===== 
%% ===== 
%% ===== 35
%% ===== 
%% ===== 
%% ===== 
%% ===== 
%% ===== 
%% ===== \begin{flushleft}
%% ===== Unused
%% ===== \end{flushleft}
%% ===== 
%% ===== 
%% ===== 
%% ===== 
%% ===== 
%% ===== 36
%% ===== 
%% ===== 
%% ===== 
%% ===== 
%% ===== 
%% ===== \begin{flushleft}
%% ===== DUD
%% ===== \end{flushleft}
%% ===== 
%% ===== 
%% ===== 
%% ===== 
%% ===== 
%% ===== \begin{flushleft}
%% ===== Decimal unit idle
%% ===== \end{flushleft}
%% ===== 
%% ===== 
%% ===== 
%% ===== 
%% ===== 
%% ===== 37
%% ===== 
%% ===== 
%% ===== 
%% ===== 
%% ===== 
%% ===== \begin{flushleft}
%% ===== -GDLDA
%% ===== \end{flushleft}
%% ===== 
%% ===== 
%% ===== 
%% ===== 
%% ===== 
%% ===== \begin{flushleft}
%% ===== Descriptor load gate A
%% ===== \end{flushleft}
%% ===== 
%% ===== 
%% ===== 
%% ===== 
%% ===== 
%% ===== 38
%% ===== 
%% ===== 
%% ===== 
%% ===== 
%% ===== 
%% ===== \begin{flushleft}
%% ===== -GDLDB
%% ===== \end{flushleft}
%% ===== 
%% ===== 
%% ===== 
%% ===== 
%% ===== 
%% ===== \begin{flushleft}
%% ===== Descriptor load gate B
%% ===== \end{flushleft}
%% ===== 
%% ===== 
%% ===== 
%% ===== 
%% ===== 
%% ===== 39
%% ===== 
%% ===== 
%% ===== 
%% ===== 
%% ===== 
%% ===== \begin{flushleft}
%% ===== -GDLDC
%% ===== \end{flushleft}
%% ===== 
%% ===== 
%% ===== 
%% ===== 
%% ===== 
%% ===== \begin{flushleft}
%% ===== Descriptor load gate C
%% ===== \end{flushleft}
%% ===== 
%% ===== 
%% ===== 
%% ===== 
%% ===== 
%% ===== 40
%% ===== 
%% ===== 
%% ===== 
%% ===== 
%% ===== 
%% ===== \begin{flushleft}
%% ===== NLD1
%% ===== \end{flushleft}
%% ===== 
%% ===== 
%% ===== 
%% ===== 
%% ===== 
%% ===== \begin{flushleft}
%% ===== Prepare alignment count for first numeric operand load
%% ===== \end{flushleft}
%% ===== 
%% ===== 
%% ===== 
%% ===== 
%% ===== 
%% ===== 41
%% ===== 
%% ===== 
%% ===== 
%% ===== 
%% ===== 
%% ===== \begin{flushleft}
%% ===== GLDP1
%% ===== \end{flushleft}
%% ===== 
%% ===== 
%% ===== 
%% ===== 
%% ===== 
%% ===== \begin{flushleft}
%% ===== Numeric operand one load gate
%% ===== \end{flushleft}
%% ===== 
%% ===== 
%% ===== 
%% ===== 
%% ===== 
%% ===== 42
%% ===== 
%% ===== 
%% ===== 
%% ===== 
%% ===== 
%% ===== \begin{flushleft}
%% ===== NLD2
%% ===== \end{flushleft}
%% ===== 
%% ===== 
%% ===== 
%% ===== 
%% ===== 
%% ===== \begin{flushleft}
%% ===== Prepare alignment count for second numeric operand load
%% ===== \end{flushleft}
%% ===== 
%% ===== 
%% ===== 
%% ===== 
%% ===== 
%% ===== 43
%% ===== 
%% ===== 
%% ===== 
%% ===== 
%% ===== 
%% ===== \begin{flushleft}
%% ===== GLDP2
%% ===== \end{flushleft}
%% ===== 
%% ===== 
%% ===== 
%% ===== 
%% ===== 
%% ===== \begin{flushleft}
%% ===== Numeric operand two load gate
%% ===== \end{flushleft}
%% ===== 
%% ===== 
%% ===== 
%% ===== 
%% ===== 
%% ===== 44
%% ===== 
%% ===== 
%% ===== 
%% ===== 
%% ===== 
%% ===== \begin{flushleft}
%% ===== ANLD1
%% ===== \end{flushleft}
%% ===== 
%% ===== 
%% ===== 
%% ===== 
%% ===== 
%% ===== \begin{flushleft}
%% ===== Alphanumeric operand one load gate
%% ===== \end{flushleft}
%% ===== 
%% ===== 
%% ===== 
%% ===== 
%% ===== 
%% ===== 45
%% ===== 
%% ===== 
%% ===== 
%% ===== 
%% ===== 
%% ===== \begin{flushleft}
%% ===== ANLD2
%% ===== \end{flushleft}
%% ===== 
%% ===== 
%% ===== 
%% ===== 
%% ===== 
%% ===== \begin{flushleft}
%% ===== Alphanumeric operand two load gate
%% ===== \end{flushleft}
%% ===== 
%% ===== 
%% ===== 
%% ===== 
%% ===== 
%% ===== 46
%% ===== 
%% ===== 
%% ===== 
%% ===== 
%% ===== 
%% ===== \begin{flushleft}
%% ===== LDWRT1
%% ===== \end{flushleft}
%% ===== 
%% ===== 
%% ===== 
%% ===== 
%% ===== 
%% ===== \begin{flushleft}
%% ===== Load rewrite register one gate
%% ===== \end{flushleft}
%% ===== 
%% ===== 
%% ===== 
%% ===== 
%% ===== 
%% ===== 47
%% ===== 
%% ===== 
%% ===== 
%% ===== 
%% ===== 
%% ===== \begin{flushleft}
%% ===== LDWRT2
%% ===== \end{flushleft}
%% ===== 
%% ===== 
%% ===== 
%% ===== 
%% ===== 
%% ===== \begin{flushleft}
%% ===== Load rewrite register two gate
%% ===== \end{flushleft}
%% ===== 
%% ===== 
%% ===== 
%% ===== 
%% ===== 
%% ===== \begin{flushleft}
%% ===== \newpage
%% ===== bit Flag Name
%% ===== \end{flushleft}
%% ===== 
%% ===== 
%% ===== 
%% ===== 
%% ===== 
%% ===== \begin{flushleft}
%% ===== Meaning
%% ===== \end{flushleft}
%% ===== 
%% ===== 
%% ===== 
%% ===== 
%% ===== 
%% ===== 48
%% ===== 
%% ===== 
%% ===== 
%% ===== 
%% ===== 
%% ===== \begin{flushleft}
%% ===== -DATA-AVLDU
%% ===== \end{flushleft}
%% ===== 
%% ===== 
%% ===== 
%% ===== 
%% ===== 
%% ===== \begin{flushleft}
%% ===== Decimal unit data available
%% ===== \end{flushleft}
%% ===== 
%% ===== 
%% ===== 
%% ===== 
%% ===== 
%% ===== 49
%% ===== 
%% ===== 
%% ===== 
%% ===== 
%% ===== 
%% ===== \begin{flushleft}
%% ===== WRT1
%% ===== \end{flushleft}
%% ===== 
%% ===== 
%% ===== 
%% ===== 
%% ===== 
%% ===== \begin{flushleft}
%% ===== Rewrite register one loaded
%% ===== \end{flushleft}
%% ===== 
%% ===== 
%% ===== 
%% ===== 
%% ===== 
%% ===== 50
%% ===== 
%% ===== 
%% ===== 
%% ===== 
%% ===== 
%% ===== \begin{flushleft}
%% ===== GSTR
%% ===== \end{flushleft}
%% ===== 
%% ===== 
%% ===== 
%% ===== 
%% ===== 
%% ===== \begin{flushleft}
%% ===== Numeric store gate
%% ===== \end{flushleft}
%% ===== 
%% ===== 
%% ===== 
%% ===== 
%% ===== 
%% ===== 51
%% ===== 
%% ===== 
%% ===== 
%% ===== 
%% ===== 
%% ===== \begin{flushleft}
%% ===== ANSTR
%% ===== \end{flushleft}
%% ===== 
%% ===== 
%% ===== 
%% ===== 
%% ===== 
%% ===== \begin{flushleft}
%% ===== Alphanumeric store gate
%% ===== \end{flushleft}
%% ===== 
%% ===== 
%% ===== 
%% ===== 
%% ===== 
%% ===== 52
%% ===== 
%% ===== 
%% ===== 
%% ===== 
%% ===== 
%% ===== \begin{flushleft}
%% ===== FSTR-OP-AV
%% ===== \end{flushleft}
%% ===== 
%% ===== 
%% ===== 
%% ===== 
%% ===== 
%% ===== \begin{flushleft}
%% ===== Operand available to be stored
%% ===== \end{flushleft}
%% ===== 
%% ===== 
%% ===== 
%% ===== 
%% ===== 
%% ===== 53
%% ===== 
%% ===== 
%% ===== 
%% ===== 
%% ===== 
%% ===== \begin{flushleft}
%% ===== -FEND-SEQ
%% ===== \end{flushleft}
%% ===== 
%% ===== 
%% ===== 
%% ===== 
%% ===== 
%% ===== \begin{flushleft}
%% ===== End sequence flag
%% ===== \end{flushleft}
%% ===== 
%% ===== 
%% ===== 
%% ===== 
%% ===== 
%% ===== 54
%% ===== 
%% ===== 
%% ===== 
%% ===== 
%% ===== 
%% ===== \begin{flushleft}
%% ===== -FLEN$<$128
%% ===== \end{flushleft}
%% ===== 
%% ===== 
%% ===== 
%% ===== 
%% ===== 
%% ===== \begin{flushleft}
%% ===== Length less than 128
%% ===== \end{flushleft}
%% ===== 
%% ===== 
%% ===== 
%% ===== 
%% ===== 
%% ===== 55
%% ===== 
%% ===== 
%% ===== 
%% ===== 
%% ===== 
%% ===== \begin{flushleft}
%% ===== FGCH
%% ===== \end{flushleft}
%% ===== 
%% ===== 
%% ===== 
%% ===== 
%% ===== 
%% ===== \begin{flushleft}
%% ===== Character operation gate
%% ===== \end{flushleft}
%% ===== 
%% ===== 
%% ===== 
%% ===== 
%% ===== 
%% ===== 56
%% ===== 
%% ===== 
%% ===== 
%% ===== 
%% ===== 
%% ===== \begin{flushleft}
%% ===== FANPK
%% ===== \end{flushleft}
%% ===== 
%% ===== 
%% ===== 
%% ===== 
%% ===== 
%% ===== \begin{flushleft}
%% ===== Alphanumeric packing cycle gate
%% ===== \end{flushleft}
%% ===== 
%% ===== 
%% ===== 
%% ===== 
%% ===== 
%% ===== 57
%% ===== 
%% ===== 
%% ===== 
%% ===== 
%% ===== 
%% ===== \begin{flushleft}
%% ===== FEXMOP
%% ===== \end{flushleft}
%% ===== 
%% ===== 
%% ===== 
%% ===== 
%% ===== 
%% ===== \begin{flushleft}
%% ===== Execute MOP gate
%% ===== \end{flushleft}
%% ===== 
%% ===== 
%% ===== 
%% ===== 
%% ===== 
%% ===== 58
%% ===== 
%% ===== 
%% ===== 
%% ===== 
%% ===== 
%% ===== \begin{flushleft}
%% ===== FBLNK
%% ===== \end{flushleft}
%% ===== 
%% ===== 
%% ===== 
%% ===== 
%% ===== 
%% ===== \begin{flushleft}
%% ===== Blanking gate
%% ===== \end{flushleft}
%% ===== 
%% ===== 
%% ===== 
%% ===== 
%% ===== 
%% ===== 59
%% ===== 
%% ===== 
%% ===== 
%% ===== 
%% ===== 
%% ===== \begin{flushleft}
%% ===== Unused
%% ===== \end{flushleft}
%% ===== 
%% ===== 
%% ===== 
%% ===== 
%% ===== 
%% ===== 60
%% ===== 
%% ===== 
%% ===== 
%% ===== 
%% ===== 
%% ===== \begin{flushleft}
%% ===== DGBD
%% ===== \end{flushleft}
%% ===== 
%% ===== 
%% ===== 
%% ===== 
%% ===== 
%% ===== \begin{flushleft}
%% ===== Binary to decimal execution gate
%% ===== \end{flushleft}
%% ===== 
%% ===== 
%% ===== 
%% ===== 
%% ===== 
%% ===== 61
%% ===== 
%% ===== 
%% ===== 
%% ===== 
%% ===== 
%% ===== \begin{flushleft}
%% ===== DGDB
%% ===== \end{flushleft}
%% ===== 
%% ===== 
%% ===== 
%% ===== 
%% ===== 
%% ===== \begin{flushleft}
%% ===== Decimal to binary execution gate
%% ===== \end{flushleft}
%% ===== 
%% ===== 
%% ===== 
%% ===== 
%% ===== 
%% ===== 62
%% ===== 
%% ===== 
%% ===== 
%% ===== 
%% ===== 
%% ===== \begin{flushleft}
%% ===== DGSP
%% ===== \end{flushleft}
%% ===== 
%% ===== 
%% ===== 
%% ===== 
%% ===== 
%% ===== \begin{flushleft}
%% ===== Shift procedure gate
%% ===== \end{flushleft}
%% ===== 
%% ===== 
%% ===== 
%% ===== 
%% ===== 
%% ===== 63
%% ===== 
%% ===== 
%% ===== 
%% ===== 
%% ===== 
%% ===== \begin{flushleft}
%% ===== FFLTG
%% ===== \end{flushleft}
%% ===== 
%% ===== 
%% ===== 
%% ===== 
%% ===== 
%% ===== \begin{flushleft}
%% ===== Floating result flag
%% ===== \end{flushleft}
%% ===== 
%% ===== 
%% ===== 
%% ===== 
%% ===== 
%% ===== 64
%% ===== 
%% ===== 
%% ===== 
%% ===== 
%% ===== 
%% ===== \begin{flushleft}
%% ===== FRND
%% ===== \end{flushleft}
%% ===== 
%% ===== 
%% ===== 
%% ===== 
%% ===== 
%% ===== \begin{flushleft}
%% ===== Rounding flag
%% ===== \end{flushleft}
%% ===== 
%% ===== 
%% ===== 
%% ===== 
%% ===== 
%% ===== 65
%% ===== 
%% ===== 
%% ===== 
%% ===== 
%% ===== 
%% ===== \begin{flushleft}
%% ===== DADD-GATE
%% ===== \end{flushleft}
%% ===== 
%% ===== 
%% ===== 
%% ===== 
%% ===== 
%% ===== \begin{flushleft}
%% ===== Add/subtract execute gate
%% ===== \end{flushleft}
%% ===== 
%% ===== 
%% ===== 
%% ===== 
%% ===== 
%% ===== 66
%% ===== 
%% ===== 
%% ===== 
%% ===== 
%% ===== 
%% ===== \begin{flushleft}
%% ===== DMP+DV-GATE
%% ===== \end{flushleft}
%% ===== 
%% ===== 
%% ===== 
%% ===== 
%% ===== 
%% ===== \begin{flushleft}
%% ===== Multiply/divide execution gate
%% ===== \end{flushleft}
%% ===== 
%% ===== 
%% ===== 
%% ===== 
%% ===== 
%% ===== 67
%% ===== 
%% ===== 
%% ===== 
%% ===== 
%% ===== 
%% ===== \begin{flushleft}
%% ===== DXPN-GATE
%% ===== \end{flushleft}
%% ===== 
%% ===== 
%% ===== 
%% ===== 
%% ===== 
%% ===== \begin{flushleft}
%% ===== Exponent network execution gate
%% ===== \end{flushleft}
%% ===== 
%% ===== 
%% ===== 
%% ===== 
%% ===== 
%% ===== 68
%% ===== 
%% ===== 
%% ===== 
%% ===== 
%% ===== 
%% ===== \begin{flushleft}
%% ===== Unused
%% ===== \end{flushleft}
%% ===== 
%% ===== 
%% ===== 
%% ===== 
%% ===== 
%% ===== 69
%% ===== 
%% ===== 
%% ===== 
%% ===== 
%% ===== 
%% ===== \begin{flushleft}
%% ===== Unused
%% ===== \end{flushleft}
%% ===== 
%% ===== 
%% ===== 
%% ===== 
%% ===== 
%% ===== 70
%% ===== 
%% ===== 
%% ===== 
%% ===== 
%% ===== 
%% ===== \begin{flushleft}
%% ===== Unused
%% ===== \end{flushleft}
%% ===== 
%% ===== 
%% ===== 
%% ===== 
%% ===== 
%% ===== 71
%% ===== 
%% ===== 
%% ===== 
%% ===== 
%% ===== 
%% ===== \begin{flushleft}
%% ===== Unused
%% ===== \end{flushleft}
%% ===== 
%% ===== 
%% ===== 
%% ===== 
%% ===== 
%% ===== \begin{flushleft}

\subsection{DECIMAL/OPERATIONS UNIT (DU/OU) HISTORY REGISTERS DPS 8M}

%% ===== \end{flushleft}
%% ===== 
%% ===== 
%% ===== \begin{flushleft}
%% ===== Format: - 72 bits each
%% ===== \end{flushleft}
%% ===== 
%% ===== 
%% ===== \begin{flushleft}
%% ===== Even word of Y-pair as stored by Store Central Processor Register (scpr), TAG = 40
%% ===== \end{flushleft}
%% ===== 
%% ===== 
%% ===== 0 0 0 0 0 0 0 0 0 0 1 1 1 1 1 1 1 1 1 1 2 2 2 2 2 2 2 2 2 2 3 3 3 3 3 3
%% ===== 
%% ===== 
%% ===== 0 1 2 3 4 5 6 7 8 9 0 1 2 3 4 5 6 7 8 9 0 1 2 3 4 5 6 7 8 9 0 1 2 3 4 5
%% ===== 
%% ===== 
%% ===== \begin{flushleft}
%% ===== a b c d e f g h i
%% ===== \end{flushleft}
%% ===== 
%% ===== 
%% ===== 
%% ===== 
%% ===== 
%% ===== \begin{flushleft}
%% ===== j k l m n o p q r s t u v w x y z A B C D E F G H I 0
%% ===== \end{flushleft}
%% ===== 
%% ===== 
%% ===== 35 1
%% ===== 
%% ===== 
%% ===== 
%% ===== 
%% ===== 
%% ===== \begin{flushleft}
%% ===== \newpage
%% ===== Odd word of Y-pair as stored by Store Central Processor Register (scpr), TAG = 40
%% ===== \end{flushleft}
%% ===== 
%% ===== 
%% ===== 3
%% ===== 
%% ===== 
%% ===== 6
%% ===== 
%% ===== 
%% ===== 
%% ===== 
%% ===== 
%% ===== 5 5
%% ===== 
%% ===== 
%% ===== 3 4
%% ===== 
%% ===== 
%% ===== \begin{flushleft}
%% ===== ICT
%% ===== \end{flushleft}
%% ===== 
%% ===== 
%% ===== 
%% ===== 
%% ===== 
%% ===== 6 6 6 6 6 6 6 6 7 7
%% ===== 
%% ===== 
%% ===== 2 3 4 5 6 7 8 9 0 1
%% ===== 
%% ===== 
%% ===== \begin{flushleft}
%% ===== RS REG
%% ===== \end{flushleft}
%% ===== 
%% ===== 
%% ===== 
%% ===== 
%% ===== 
%% ===== 18
%% ===== 
%% ===== 
%% ===== 
%% ===== 
%% ===== 
%% ===== \begin{flushleft}
%% ===== J K L M N O P Q R
%% ===== \end{flushleft}
%% ===== 
%% ===== 
%% ===== 9 1 1 1 1 1 1 1 1 1
%% ===== 
%% ===== 
%% ===== 
%% ===== 
%% ===== 
%% ===== \begin{flushleft}
%% ===== Figure 3-27. Decimal/Operations (DU/OU) History Register Format - DPS 8M
%% ===== \end{flushleft}
%% ===== 
%% ===== 
%% ===== \begin{flushleft}
%% ===== Description:
%% ===== \end{flushleft}
%% ===== 
%% ===== 
%% ===== \begin{flushleft}
%% ===== A combination of 16 flags and registers from the operation unit and decimal unit. The 16
%% ===== \end{flushleft}
%% ===== 
%% ===== 
%% ===== \begin{flushleft}
%% ===== registers are handled as a rotating queue controlled by the operations unit history register
%% ===== \end{flushleft}
%% ===== 
%% ===== 
%% ===== \begin{flushleft}
%% ===== counter. The counter is always set to the number of the oldest entry and advances by one
%% ===== \end{flushleft}
%% ===== 
%% ===== 
%% ===== \begin{flushleft}
%% ===== for each history register reference (data entry or Store Central Processor Register (scpr)
%% ===== \end{flushleft}
%% ===== 
%% ===== 
%% ===== \begin{flushleft}
%% ===== instruction).
%% ===== \end{flushleft}
%% ===== 
%% ===== 
%% ===== \begin{flushleft}
%% ===== The decimal unit and the control unit run synchronously. There is a control unit history
%% ===== \end{flushleft}
%% ===== 
%% ===== 
%% ===== \begin{flushleft}
%% ===== register entry for every decimal unit history register entry and vice versa (except for
%% ===== \end{flushleft}
%% ===== 
%% ===== 
%% ===== \begin{flushleft}
%% ===== instruction fetch and EIS descriptor fetch cycles). If the processor is not executing a
%% ===== \end{flushleft}
%% ===== 
%% ===== 
%% ===== \begin{flushleft}
%% ===== decimal instruction, the decimal unit history register entry shows an idle condition.
%% ===== \end{flushleft}
%% ===== 
%% ===== 
%% ===== \begin{flushleft}
%% ===== Function:
%% ===== \end{flushleft}
%% ===== 
%% ===== 
%% ===== \begin{flushleft}
%% ===== An operations unit history register entry shows the conditions at the end of the operations
%% ===== \end{flushleft}
%% ===== 
%% ===== 
%% ===== \begin{flushleft}
%% ===== unit cycle to which it applies. The 16 registers hold the conditions for the last 16 operations
%% ===== \end{flushleft}
%% ===== 
%% ===== 
%% ===== \begin{flushleft}
%% ===== unit cycles. As the operations unit performs various cycles in the execution of an
%% ===== \end{flushleft}
%% ===== 
%% ===== 
%% ===== \begin{flushleft}
%% ===== instruction, it does not advance the counter for each such cycle. The counter is advanced
%% ===== \end{flushleft}
%% ===== 
%% ===== 
%% ===== \begin{flushleft}
%% ===== only at successful completion of the instruction or if the instruction is aborted for a fault
%% ===== \end{flushleft}
%% ===== 
%% ===== 
%% ===== \begin{flushleft}
%% ===== condition. Entries are made according to controls set in the Mode Register. (See Mode
%% ===== \end{flushleft}
%% ===== 
%% ===== 
%% ===== \begin{flushleft}
%% ===== Register earlier in this section.)
%% ===== \end{flushleft}
%% ===== 
%% ===== 
%% ===== \begin{flushleft}
%% ===== The meanings of the constituent flags and registers are:
%% ===== \end{flushleft}
%% ===== 
%% ===== 
%% ===== 
%% ===== 
%% ===== 
%% ===== \begin{flushleft}
%% ===== key Flag Name
%% ===== \end{flushleft}
%% ===== 
%% ===== 
%% ===== 
%% ===== 
%% ===== 
%% ===== \begin{flushleft}
%% ===== Meaning
%% ===== \end{flushleft}
%% ===== 
%% ===== 
%% ===== 
%% ===== 
%% ===== 
%% ===== \begin{flushleft}
%% ===== a
%% ===== \end{flushleft}
%% ===== 
%% ===== 
%% ===== 
%% ===== 
%% ===== 
%% ===== \begin{flushleft}
%% ===== FANLD1
%% ===== \end{flushleft}
%% ===== 
%% ===== 
%% ===== 
%% ===== 
%% ===== 
%% ===== \begin{flushleft}
%% ===== Alpha-num load desc l (complemented)
%% ===== \end{flushleft}
%% ===== 
%% ===== 
%% ===== 
%% ===== 
%% ===== 
%% ===== \begin{flushleft}
%% ===== b
%% ===== \end{flushleft}
%% ===== 
%% ===== 
%% ===== 
%% ===== 
%% ===== 
%% ===== \begin{flushleft}
%% ===== FANLD2
%% ===== \end{flushleft}
%% ===== 
%% ===== 
%% ===== 
%% ===== 
%% ===== 
%% ===== \begin{flushleft}
%% ===== Alpha-num load desc 2 (complemented)
%% ===== \end{flushleft}
%% ===== 
%% ===== 
%% ===== 
%% ===== 
%% ===== 
%% ===== \begin{flushleft}
%% ===== c
%% ===== \end{flushleft}
%% ===== 
%% ===== 
%% ===== 
%% ===== 
%% ===== 
%% ===== \begin{flushleft}
%% ===== FANSTR
%% ===== \end{flushleft}
%% ===== 
%% ===== 
%% ===== 
%% ===== 
%% ===== 
%% ===== \begin{flushleft}
%% ===== Alpha-num store (complemented)
%% ===== \end{flushleft}
%% ===== 
%% ===== 
%% ===== 
%% ===== 
%% ===== 
%% ===== \begin{flushleft}
%% ===== d
%% ===== \end{flushleft}
%% ===== 
%% ===== 
%% ===== 
%% ===== 
%% ===== 
%% ===== \begin{flushleft}
%% ===== FLDWRT1
%% ===== \end{flushleft}
%% ===== 
%% ===== 
%% ===== 
%% ===== 
%% ===== 
%% ===== \begin{flushleft}
%% ===== Load re-write reg l (complemented)
%% ===== \end{flushleft}
%% ===== 
%% ===== 
%% ===== 
%% ===== 
%% ===== 
%% ===== \begin{flushleft}
%% ===== e
%% ===== \end{flushleft}
%% ===== 
%% ===== 
%% ===== 
%% ===== 
%% ===== 
%% ===== \begin{flushleft}
%% ===== FLDWRT2
%% ===== \end{flushleft}
%% ===== 
%% ===== 
%% ===== 
%% ===== 
%% ===== 
%% ===== \begin{flushleft}
%% ===== Load re-write reg 2 (complemented)
%% ===== \end{flushleft}
%% ===== 
%% ===== 
%% ===== 
%% ===== 
%% ===== 
%% ===== \begin{flushleft}
%% ===== f
%% ===== \end{flushleft}
%% ===== 
%% ===== 
%% ===== 
%% ===== 
%% ===== 
%% ===== \begin{flushleft}
%% ===== FNLD1
%% ===== \end{flushleft}
%% ===== 
%% ===== 
%% ===== 
%% ===== 
%% ===== 
%% ===== \begin{flushleft}
%% ===== Numeric load desc l (complemented)
%% ===== \end{flushleft}
%% ===== 
%% ===== 
%% ===== 
%% ===== 
%% ===== 
%% ===== \begin{flushleft}
%% ===== g
%% ===== \end{flushleft}
%% ===== 
%% ===== 
%% ===== 
%% ===== 
%% ===== 
%% ===== \begin{flushleft}
%% ===== FNLD2
%% ===== \end{flushleft}
%% ===== 
%% ===== 
%% ===== 
%% ===== 
%% ===== 
%% ===== \begin{flushleft}
%% ===== Numeric load desc 2 (complemented)
%% ===== \end{flushleft}
%% ===== 
%% ===== 
%% ===== 
%% ===== 
%% ===== 
%% ===== \begin{flushleft}
%% ===== h
%% ===== \end{flushleft}
%% ===== 
%% ===== 
%% ===== 
%% ===== 
%% ===== 
%% ===== \begin{flushleft}
%% ===== NOSEQF
%% ===== \end{flushleft}
%% ===== 
%% ===== 
%% ===== 
%% ===== 
%% ===== 
%% ===== \begin{flushleft}
%% ===== End sequence flag
%% ===== \end{flushleft}
%% ===== 
%% ===== 
%% ===== 
%% ===== 
%% ===== 
%% ===== \begin{flushleft}
%% ===== i
%% ===== \end{flushleft}
%% ===== 
%% ===== 
%% ===== 
%% ===== 
%% ===== 
%% ===== \begin{flushleft}
%% ===== FDUD
%% ===== \end{flushleft}
%% ===== 
%% ===== 
%% ===== 
%% ===== 
%% ===== 
%% ===== \begin{flushleft}
%% ===== Decimal unit idle (complemented)
%% ===== \end{flushleft}
%% ===== 
%% ===== 
%% ===== 
%% ===== 
%% ===== 
%% ===== \begin{flushleft}
%% ===== j
%% ===== \end{flushleft}
%% ===== 
%% ===== 
%% ===== 
%% ===== 
%% ===== 
%% ===== \begin{flushleft}
%% ===== FGSTR
%% ===== \end{flushleft}
%% ===== 
%% ===== 
%% ===== 
%% ===== 
%% ===== 
%% ===== \begin{flushleft}
%% ===== General store flag (complemented)
%% ===== \end{flushleft}
%% ===== 
%% ===== 
%% ===== 
%% ===== 
%% ===== 
%% ===== \begin{flushleft}
%% ===== k
%% ===== \end{flushleft}
%% ===== 
%% ===== 
%% ===== 
%% ===== 
%% ===== 
%% ===== \begin{flushleft}
%% ===== NOSEQ
%% ===== \end{flushleft}
%% ===== 
%% ===== 
%% ===== 
%% ===== 
%% ===== 
%% ===== \begin{flushleft}
%% ===== End of sequence (complemented)
%% ===== \end{flushleft}
%% ===== 
%% ===== 
%% ===== 
%% ===== 
%% ===== 
%% ===== \begin{flushleft}
%% ===== l
%% ===== \end{flushleft}
%% ===== 
%% ===== 
%% ===== 
%% ===== 
%% ===== 
%% ===== \begin{flushleft}
%% ===== NINE
%% ===== \end{flushleft}
%% ===== 
%% ===== 
%% ===== 
%% ===== 
%% ===== 
%% ===== \begin{flushleft}
%% ===== 9-bit character operation
%% ===== \end{flushleft}
%% ===== 
%% ===== 
%% ===== 
%% ===== 
%% ===== 
%% ===== \begin{flushleft}
%% ===== m
%% ===== \end{flushleft}
%% ===== 
%% ===== 
%% ===== 
%% ===== 
%% ===== 
%% ===== \begin{flushleft}
%% ===== SIX
%% ===== \end{flushleft}
%% ===== 
%% ===== 
%% ===== 
%% ===== 
%% ===== 
%% ===== \begin{flushleft}
%% ===== 6-bit character operation
%% ===== \end{flushleft}
%% ===== 
%% ===== 
%% ===== 
%% ===== 
%% ===== 
%% ===== \begin{flushleft}
%% ===== n
%% ===== \end{flushleft}
%% ===== 
%% ===== 
%% ===== 
%% ===== 
%% ===== 
%% ===== \begin{flushleft}
%% ===== FOUR
%% ===== \end{flushleft}
%% ===== 
%% ===== 
%% ===== 
%% ===== 
%% ===== 
%% ===== \begin{flushleft}
%% ===== 4-bit character operation
%% ===== \end{flushleft}
%% ===== 
%% ===== 
%% ===== 
%% ===== 
%% ===== 
%% ===== \begin{flushleft}
%% ===== o
%% ===== \end{flushleft}
%% ===== 
%% ===== 
%% ===== 
%% ===== 
%% ===== 
%% ===== \begin{flushleft}
%% ===== DUBIT
%% ===== \end{flushleft}
%% ===== 
%% ===== 
%% ===== 
%% ===== 
%% ===== 
%% ===== \begin{flushleft}
%% ===== Bit operation
%% ===== \end{flushleft}
%% ===== 
%% ===== 
%% ===== 
%% ===== 
%% ===== 
%% ===== \begin{flushleft}
%% ===== \newpage
%% ===== key Flag Name
%% ===== \end{flushleft}
%% ===== 
%% ===== 
%% ===== 
%% ===== 
%% ===== 
%% ===== \begin{flushleft}
%% ===== NOTE:
%% ===== \end{flushleft}
%% ===== 
%% ===== 
%% ===== 
%% ===== 
%% ===== 
%% ===== \begin{flushleft}
%% ===== Meaning
%% ===== \end{flushleft}
%% ===== 
%% ===== 
%% ===== 
%% ===== 
%% ===== 
%% ===== \begin{flushleft}
%% ===== p
%% ===== \end{flushleft}
%% ===== 
%% ===== 
%% ===== 
%% ===== 
%% ===== 
%% ===== \begin{flushleft}
%% ===== DUWORD
%% ===== \end{flushleft}
%% ===== 
%% ===== 
%% ===== 
%% ===== 
%% ===== 
%% ===== \begin{flushleft}
%% ===== Word operation
%% ===== \end{flushleft}
%% ===== 
%% ===== 
%% ===== 
%% ===== 
%% ===== 
%% ===== \begin{flushleft}
%% ===== q
%% ===== \end{flushleft}
%% ===== 
%% ===== 
%% ===== 
%% ===== 
%% ===== 
%% ===== \begin{flushleft}
%% ===== PTR1
%% ===== \end{flushleft}
%% ===== 
%% ===== 
%% ===== 
%% ===== 
%% ===== 
%% ===== \begin{flushleft}
%% ===== Select ptr l
%% ===== \end{flushleft}
%% ===== 
%% ===== 
%% ===== 
%% ===== 
%% ===== 
%% ===== \begin{flushleft}
%% ===== r
%% ===== \end{flushleft}
%% ===== 
%% ===== 
%% ===== 
%% ===== 
%% ===== 
%% ===== \begin{flushleft}
%% ===== PTR2
%% ===== \end{flushleft}
%% ===== 
%% ===== 
%% ===== 
%% ===== 
%% ===== 
%% ===== \begin{flushleft}
%% ===== Select ptr 2
%% ===== \end{flushleft}
%% ===== 
%% ===== 
%% ===== 
%% ===== 
%% ===== 
%% ===== \begin{flushleft}
%% ===== s
%% ===== \end{flushleft}
%% ===== 
%% ===== 
%% ===== 
%% ===== 
%% ===== 
%% ===== \begin{flushleft}
%% ===== PRT3
%% ===== \end{flushleft}
%% ===== 
%% ===== 
%% ===== 
%% ===== 
%% ===== 
%% ===== \begin{flushleft}
%% ===== Select ptr 3
%% ===== \end{flushleft}
%% ===== 
%% ===== 
%% ===== 
%% ===== 
%% ===== 
%% ===== \begin{flushleft}
%% ===== t
%% ===== \end{flushleft}
%% ===== 
%% ===== 
%% ===== 
%% ===== 
%% ===== 
%% ===== \begin{flushleft}
%% ===== FPOP
%% ===== \end{flushleft}
%% ===== 
%% ===== 
%% ===== 
%% ===== 
%% ===== 
%% ===== \begin{flushleft}
%% ===== Prepare operand pointer
%% ===== \end{flushleft}
%% ===== 
%% ===== 
%% ===== 
%% ===== 
%% ===== 
%% ===== \begin{flushleft}
%% ===== u
%% ===== \end{flushleft}
%% ===== 
%% ===== 
%% ===== 
%% ===== 
%% ===== 
%% ===== \begin{flushleft}
%% ===== GEAM
%% ===== \end{flushleft}
%% ===== 
%% ===== 
%% ===== 
%% ===== 
%% ===== 
%% ===== \begin{flushleft}
%% ===== Add timing gates (complemented)
%% ===== \end{flushleft}
%% ===== 
%% ===== 
%% ===== 
%% ===== 
%% ===== 
%% ===== \begin{flushleft}
%% ===== v
%% ===== \end{flushleft}
%% ===== 
%% ===== 
%% ===== 
%% ===== 
%% ===== 
%% ===== \begin{flushleft}
%% ===== LPD12
%% ===== \end{flushleft}
%% ===== 
%% ===== 
%% ===== 
%% ===== 
%% ===== 
%% ===== \begin{flushleft}
%% ===== Load pointer l or 2 (complemented)
%% ===== \end{flushleft}
%% ===== 
%% ===== 
%% ===== 
%% ===== 
%% ===== 
%% ===== \begin{flushleft}
%% ===== w
%% ===== \end{flushleft}
%% ===== 
%% ===== 
%% ===== 
%% ===== 
%% ===== 
%% ===== \begin{flushleft}
%% ===== GEMAE
%% ===== \end{flushleft}
%% ===== 
%% ===== 
%% ===== 
%% ===== 
%% ===== 
%% ===== \begin{flushleft}
%% ===== Multiply gates A E (complemented)
%% ===== \end{flushleft}
%% ===== 
%% ===== 
%% ===== 
%% ===== 
%% ===== 
%% ===== \begin{flushleft}
%% ===== x
%% ===== \end{flushleft}
%% ===== 
%% ===== 
%% ===== 
%% ===== 
%% ===== 
%% ===== \begin{flushleft}
%% ===== BTDS
%% ===== \end{flushleft}
%% ===== 
%% ===== 
%% ===== 
%% ===== 
%% ===== 
%% ===== \begin{flushleft}
%% ===== Binary to decimal gates (complemented)
%% ===== \end{flushleft}
%% ===== 
%% ===== 
%% ===== 
%% ===== 
%% ===== 
%% ===== \begin{flushleft}
%% ===== y
%% ===== \end{flushleft}
%% ===== 
%% ===== 
%% ===== 
%% ===== 
%% ===== 
%% ===== \begin{flushleft}
%% ===== SP15
%% ===== \end{flushleft}
%% ===== 
%% ===== 
%% ===== 
%% ===== 
%% ===== 
%% ===== \begin{flushleft}
%% ===== Align cycles (complemented)
%% ===== \end{flushleft}
%% ===== 
%% ===== 
%% ===== 
%% ===== 
%% ===== 
%% ===== \begin{flushleft}
%% ===== z
%% ===== \end{flushleft}
%% ===== 
%% ===== 
%% ===== 
%% ===== 
%% ===== 
%% ===== \begin{flushleft}
%% ===== FSWEQ
%% ===== \end{flushleft}
%% ===== 
%% ===== 
%% ===== 
%% ===== 
%% ===== 
%% ===== \begin{flushleft}
%% ===== Single word sequence flag (complemented)
%% ===== \end{flushleft}
%% ===== 
%% ===== 
%% ===== 
%% ===== 
%% ===== 
%% ===== \begin{flushleft}
%% ===== A
%% ===== \end{flushleft}
%% ===== 
%% ===== 
%% ===== 
%% ===== 
%% ===== 
%% ===== \begin{flushleft}
%% ===== FGCH
%% ===== \end{flushleft}
%% ===== 
%% ===== 
%% ===== 
%% ===== 
%% ===== 
%% ===== \begin{flushleft}
%% ===== Character cycle (complemented)
%% ===== \end{flushleft}
%% ===== 
%% ===== 
%% ===== 
%% ===== 
%% ===== 
%% ===== \begin{flushleft}
%% ===== B
%% ===== \end{flushleft}
%% ===== 
%% ===== 
%% ===== 
%% ===== 
%% ===== 
%% ===== \begin{flushleft}
%% ===== DFRST
%% ===== \end{flushleft}
%% ===== 
%% ===== 
%% ===== 
%% ===== 
%% ===== 
%% ===== \begin{flushleft}
%% ===== Processing descriptor for first time
%% ===== \end{flushleft}
%% ===== 
%% ===== 
%% ===== 
%% ===== 
%% ===== 
%% ===== \begin{flushleft}
%% ===== C
%% ===== \end{flushleft}
%% ===== 
%% ===== 
%% ===== 
%% ===== 
%% ===== 
%% ===== \begin{flushleft}
%% ===== EXH
%% ===== \end{flushleft}
%% ===== 
%% ===== 
%% ===== 
%% ===== 
%% ===== 
%% ===== \begin{flushleft}
%% ===== Exhaust
%% ===== \end{flushleft}
%% ===== 
%% ===== 
%% ===== 
%% ===== 
%% ===== 
%% ===== \begin{flushleft}
%% ===== D
%% ===== \end{flushleft}
%% ===== 
%% ===== 
%% ===== 
%% ===== 
%% ===== 
%% ===== \begin{flushleft}
%% ===== FGADO
%% ===== \end{flushleft}
%% ===== 
%% ===== 
%% ===== 
%% ===== 
%% ===== 
%% ===== \begin{flushleft}
%% ===== Add cycle (complemented)
%% ===== \end{flushleft}
%% ===== 
%% ===== 
%% ===== 
%% ===== 
%% ===== 
%% ===== \begin{flushleft}
%% ===== E
%% ===== \end{flushleft}
%% ===== 
%% ===== 
%% ===== 
%% ===== 
%% ===== 
%% ===== \begin{flushleft}
%% ===== INTRPTD
%% ===== \end{flushleft}
%% ===== 
%% ===== 
%% ===== 
%% ===== 
%% ===== 
%% ===== \begin{flushleft}
%% ===== Interrupted
%% ===== \end{flushleft}
%% ===== 
%% ===== 
%% ===== 
%% ===== 
%% ===== 
%% ===== \begin{flushleft}
%% ===== F
%% ===== \end{flushleft}
%% ===== 
%% ===== 
%% ===== 
%% ===== 
%% ===== 
%% ===== \begin{flushleft}
%% ===== GLDP2
%% ===== \end{flushleft}
%% ===== 
%% ===== 
%% ===== 
%% ===== 
%% ===== 
%% ===== \begin{flushleft}
%% ===== Load DP2
%% ===== \end{flushleft}
%% ===== 
%% ===== 
%% ===== 
%% ===== 
%% ===== 
%% ===== \begin{flushleft}
%% ===== G
%% ===== \end{flushleft}
%% ===== 
%% ===== 
%% ===== 
%% ===== 
%% ===== 
%% ===== \begin{flushleft}
%% ===== GEMC
%% ===== \end{flushleft}
%% ===== 
%% ===== 
%% ===== 
%% ===== 
%% ===== 
%% ===== \begin{flushleft}
%% ===== Multiply gate C
%% ===== \end{flushleft}
%% ===== 
%% ===== 
%% ===== 
%% ===== 
%% ===== 
%% ===== \begin{flushleft}
%% ===== H
%% ===== \end{flushleft}
%% ===== 
%% ===== 
%% ===== 
%% ===== 
%% ===== 
%% ===== \begin{flushleft}
%% ===== GBDA
%% ===== \end{flushleft}
%% ===== 
%% ===== 
%% ===== 
%% ===== 
%% ===== 
%% ===== \begin{flushleft}
%% ===== Binary to decimal gate A
%% ===== \end{flushleft}
%% ===== 
%% ===== 
%% ===== 
%% ===== 
%% ===== 
%% ===== \begin{flushleft}
%% ===== I
%% ===== \end{flushleft}
%% ===== 
%% ===== 
%% ===== 
%% ===== 
%% ===== 
%% ===== \begin{flushleft}
%% ===== GSP5
%% ===== \end{flushleft}
%% ===== 
%% ===== 
%% ===== 
%% ===== 
%% ===== 
%% ===== \begin{flushleft}
%% ===== Final align cycle
%% ===== \end{flushleft}
%% ===== 
%% ===== 
%% ===== 
%% ===== 
%% ===== 
%% ===== \begin{flushleft}
%% ===== ICT
%% ===== \end{flushleft}
%% ===== 
%% ===== 
%% ===== 
%% ===== 
%% ===== 
%% ===== \begin{flushleft}
%% ===== Instruction counter (See NOTE below.)
%% ===== \end{flushleft}
%% ===== 
%% ===== 
%% ===== 
%% ===== 
%% ===== 
%% ===== \begin{flushleft}
%% ===== RS
%% ===== \end{flushleft}
%% ===== 
%% ===== 
%% ===== 
%% ===== 
%% ===== 
%% ===== \begin{flushleft}
%% ===== OU op-code register (RS0-8)
%% ===== \end{flushleft}
%% ===== 
%% ===== 
%% ===== 
%% ===== 
%% ===== 
%% ===== \begin{flushleft}
%% ===== IR
%% ===== \end{flushleft}
%% ===== 
%% ===== 
%% ===== 
%% ===== 
%% ===== 
%% ===== \begin{flushleft}
%% ===== Indicator register (IR):
%% ===== \end{flushleft}
%% ===== 
%% ===== 
%% ===== 
%% ===== 
%% ===== 
%% ===== \begin{flushleft}
%% ===== J
%% ===== \end{flushleft}
%% ===== 
%% ===== 
%% ===== 
%% ===== 
%% ===== 
%% ===== \begin{flushleft}
%% ===== ZERO
%% ===== \end{flushleft}
%% ===== 
%% ===== 
%% ===== 
%% ===== 
%% ===== 
%% ===== \begin{flushleft}
%% ===== Zero indicator
%% ===== \end{flushleft}
%% ===== 
%% ===== 
%% ===== 
%% ===== 
%% ===== 
%% ===== \begin{flushleft}
%% ===== K
%% ===== \end{flushleft}
%% ===== 
%% ===== 
%% ===== 
%% ===== 
%% ===== 
%% ===== \begin{flushleft}
%% ===== NEG
%% ===== \end{flushleft}
%% ===== 
%% ===== 
%% ===== 
%% ===== 
%% ===== 
%% ===== \begin{flushleft}
%% ===== Negative indicator
%% ===== \end{flushleft}
%% ===== 
%% ===== 
%% ===== 
%% ===== 
%% ===== 
%% ===== \begin{flushleft}
%% ===== L
%% ===== \end{flushleft}
%% ===== 
%% ===== 
%% ===== 
%% ===== 
%% ===== 
%% ===== \begin{flushleft}
%% ===== CARRY
%% ===== \end{flushleft}
%% ===== 
%% ===== 
%% ===== 
%% ===== 
%% ===== 
%% ===== \begin{flushleft}
%% ===== Carry indicator
%% ===== \end{flushleft}
%% ===== 
%% ===== 
%% ===== 
%% ===== 
%% ===== 
%% ===== \begin{flushleft}
%% ===== M
%% ===== \end{flushleft}
%% ===== 
%% ===== 
%% ===== 
%% ===== 
%% ===== 
%% ===== \begin{flushleft}
%% ===== OVFL
%% ===== \end{flushleft}
%% ===== 
%% ===== 
%% ===== 
%% ===== 
%% ===== 
%% ===== \begin{flushleft}
%% ===== Overflow indicator
%% ===== \end{flushleft}
%% ===== 
%% ===== 
%% ===== 
%% ===== 
%% ===== 
%% ===== \begin{flushleft}
%% ===== N
%% ===== \end{flushleft}
%% ===== 
%% ===== 
%% ===== 
%% ===== 
%% ===== 
%% ===== \begin{flushleft}
%% ===== EOVFL
%% ===== \end{flushleft}
%% ===== 
%% ===== 
%% ===== 
%% ===== 
%% ===== 
%% ===== \begin{flushleft}
%% ===== Exponent overflow indicator
%% ===== \end{flushleft}
%% ===== 
%% ===== 
%% ===== 
%% ===== 
%% ===== 
%% ===== \begin{flushleft}
%% ===== O
%% ===== \end{flushleft}
%% ===== 
%% ===== 
%% ===== 
%% ===== 
%% ===== 
%% ===== \begin{flushleft}
%% ===== EUFL
%% ===== \end{flushleft}
%% ===== 
%% ===== 
%% ===== 
%% ===== 
%% ===== 
%% ===== \begin{flushleft}
%% ===== Exponent underflow indicator
%% ===== \end{flushleft}
%% ===== 
%% ===== 
%% ===== 
%% ===== 
%% ===== 
%% ===== \begin{flushleft}
%% ===== P
%% ===== \end{flushleft}
%% ===== 
%% ===== 
%% ===== 
%% ===== 
%% ===== 
%% ===== \begin{flushleft}
%% ===== OFLM
%% ===== \end{flushleft}
%% ===== 
%% ===== 
%% ===== 
%% ===== 
%% ===== 
%% ===== \begin{flushleft}
%% ===== Overflow mask indicator
%% ===== \end{flushleft}
%% ===== 
%% ===== 
%% ===== 
%% ===== 
%% ===== 
%% ===== \begin{flushleft}
%% ===== Q
%% ===== \end{flushleft}
%% ===== 
%% ===== 
%% ===== 
%% ===== 
%% ===== 
%% ===== \begin{flushleft}
%% ===== HEX
%% ===== \end{flushleft}
%% ===== 
%% ===== 
%% ===== 
%% ===== 
%% ===== 
%% ===== \begin{flushleft}
%% ===== Hex mode indicator
%% ===== \end{flushleft}
%% ===== 
%% ===== 
%% ===== 
%% ===== 
%% ===== 
%% ===== \begin{flushleft}
%% ===== R
%% ===== \end{flushleft}
%% ===== 
%% ===== 
%% ===== 
%% ===== 
%% ===== 
%% ===== \begin{flushleft}
%% ===== DTRGO
%% ===== \end{flushleft}
%% ===== 
%% ===== 
%% ===== 
%% ===== 
%% ===== 
%% ===== \begin{flushleft}
%% ===== Transfer go
%% ===== \end{flushleft}
%% ===== 
%% ===== 
%% ===== 
%% ===== 
%% ===== 
%% ===== \begin{flushleft}
%% ===== The current value of the instruction counter (PPR.IC). Since the control unit and
%% ===== \end{flushleft}
%% ===== 
%% ===== 
%% ===== \begin{flushleft}
%% ===== operations unit run asynchronously and overlap is usually enabled, the value of ICT
%% ===== \end{flushleft}
%% ===== 
%% ===== 
%% ===== \begin{flushleft}
%% ===== TRACKER may not be the address of the operations unit instruction currently being
%% ===== \end{flushleft}
%% ===== 
%% ===== 
%% ===== \begin{flushleft}
%% ===== executed.
%% ===== \end{flushleft}
%% ===== 
%% ===== 
%% ===== 
%% ===== 
%% ===== 
%% ===== \begin{flushleft}
%% ===== \newpage

\subsection{APPENDING UNIT (APU) HISTORY REGISTERS - DPS AND L68}

%% ===== \end{flushleft}
%% ===== 
%% ===== 
%% ===== \begin{flushleft}
%% ===== Format: - 72 bits each
%% ===== \end{flushleft}
%% ===== 
%% ===== 
%% ===== \begin{flushleft}
%% ===== Even word of Y-pair as stored by Store Central Processor Register (scpr), TAG = 00
%% ===== \end{flushleft}
%% ===== 
%% ===== 
%% ===== 0
%% ===== 
%% ===== 
%% ===== 0
%% ===== 
%% ===== 
%% ===== 
%% ===== 
%% ===== 
%% ===== 1 1 1 1 1 1 2 2 2 2 2 2 2
%% ===== 
%% ===== 
%% ===== 4 5 6 7 8 9 0 1 2 3 4 5 6
%% ===== 
%% ===== 
%% ===== \begin{flushleft}
%% ===== ESN
%% ===== \end{flushleft}
%% ===== 
%% ===== 
%% ===== 
%% ===== 
%% ===== 
%% ===== \begin{flushleft}
%% ===== a
%% ===== \end{flushleft}
%% ===== 
%% ===== 
%% ===== 15
%% ===== 
%% ===== 
%% ===== 
%% ===== 
%% ===== 
%% ===== \begin{flushleft}
%% ===== b c d e f g h i
%% ===== \end{flushleft}
%% ===== 
%% ===== 
%% ===== 
%% ===== 
%% ===== 
%% ===== 2 3 3
%% ===== 
%% ===== 
%% ===== 9 0 1
%% ===== 
%% ===== 
%% ===== 
%% ===== 
%% ===== 
%% ===== 3 3
%% ===== 
%% ===== 
%% ===== 4 5
%% ===== 
%% ===== 
%% ===== 
%% ===== 
%% ===== 
%% ===== \begin{flushleft}
%% ===== j SDWAMR k PTWAMR l
%% ===== \end{flushleft}
%% ===== 
%% ===== 
%% ===== 
%% ===== 
%% ===== 
%% ===== 2 1 1 1 1 1 1 1 1 1
%% ===== 
%% ===== 
%% ===== 
%% ===== 
%% ===== 
%% ===== 4 1
%% ===== 
%% ===== 
%% ===== 
%% ===== 
%% ===== 
%% ===== 4 1
%% ===== 
%% ===== 
%% ===== 
%% ===== 
%% ===== 
%% ===== \begin{flushleft}
%% ===== Odd word of Y-pair as stored by Store Central Processor Register (scpr), TAG = 00
%% ===== \end{flushleft}
%% ===== 
%% ===== 
%% ===== 3
%% ===== 
%% ===== 
%% ===== 6
%% ===== 
%% ===== 
%% ===== 
%% ===== 
%% ===== 
%% ===== 5 6
%% ===== 
%% ===== 
%% ===== 9 0
%% ===== 
%% ===== 
%% ===== \begin{flushleft}
%% ===== ADD
%% ===== \end{flushleft}
%% ===== 
%% ===== 
%% ===== 
%% ===== 
%% ===== 
%% ===== 6 6
%% ===== 
%% ===== 
%% ===== 2 3
%% ===== 
%% ===== 
%% ===== 
%% ===== 
%% ===== 
%% ===== \begin{flushleft}
%% ===== TRR
%% ===== \end{flushleft}
%% ===== 
%% ===== 
%% ===== 24
%% ===== 
%% ===== 
%% ===== 
%% ===== 
%% ===== 
%% ===== 6 6 6
%% ===== 
%% ===== 
%% ===== 5 6 7
%% ===== 
%% ===== 
%% ===== 
%% ===== 
%% ===== 
%% ===== 6 7 7
%% ===== 
%% ===== 
%% ===== 9 0 1
%% ===== 
%% ===== 
%% ===== 
%% ===== 
%% ===== 
%% ===== \begin{flushleft}
%% ===== 0 0 0 m 0 0 0 n 0
%% ===== \end{flushleft}
%% ===== 
%% ===== 
%% ===== 3
%% ===== 
%% ===== 
%% ===== 
%% ===== 
%% ===== 
%% ===== 3 1
%% ===== 
%% ===== 
%% ===== 
%% ===== 
%% ===== 
%% ===== 3 1 1
%% ===== 
%% ===== 
%% ===== 
%% ===== 
%% ===== 
%% ===== \begin{flushleft}
%% ===== Figure 3-28. Appending Unit (APU) History Register Format - DPS and L68
%% ===== \end{flushleft}
%% ===== 
%% ===== 
%% ===== \begin{flushleft}
%% ===== Description:
%% ===== \end{flushleft}
%% ===== 
%% ===== 
%% ===== \begin{flushleft}
%% ===== A combination of 16 flags and registers from the appending unit. The 16 registers are
%% ===== \end{flushleft}
%% ===== 
%% ===== 
%% ===== \begin{flushleft}
%% ===== handled as a rotating queue controlled by the appending unit history register counter. The
%% ===== \end{flushleft}
%% ===== 
%% ===== 
%% ===== \begin{flushleft}
%% ===== counter is always set to the number of the oldest entry and advances by one for each history
%% ===== \end{flushleft}
%% ===== 
%% ===== 
%% ===== \begin{flushleft}
%% ===== register reference (data entry or Store Central Processor Register (scpr) instruction).
%% ===== \end{flushleft}
%% ===== 
%% ===== 
%% ===== \begin{flushleft}
%% ===== Function:
%% ===== \end{flushleft}
%% ===== 
%% ===== 
%% ===== \begin{flushleft}
%% ===== An appending unit history register entry shows the conditions in the appending unit at the
%% ===== \end{flushleft}
%% ===== 
%% ===== 
%% ===== \begin{flushleft}
%% ===== end of an address preparation cycle in appending mode. The 16 registers hold the
%% ===== \end{flushleft}
%% ===== 
%% ===== 
%% ===== \begin{flushleft}
%% ===== conditions for the last 16 such address preparation cycles. Entries are made according to
%% ===== \end{flushleft}
%% ===== 
%% ===== 
%% ===== \begin{flushleft}
%% ===== controls set in the Mode Register. (See Mode Register earlier in this section.)
%% ===== \end{flushleft}
%% ===== 
%% ===== 
%% ===== \begin{flushleft}
%% ===== The meanings of the constituent flags and registers are:
%% ===== \end{flushleft}
%% ===== 
%% ===== 
%% ===== 
%% ===== 
%% ===== 
%% ===== \begin{flushleft}
%% ===== key Flag Name
%% ===== \end{flushleft}
%% ===== 
%% ===== 
%% ===== 
%% ===== 
%% ===== 
%% ===== \begin{flushleft}
%% ===== Meaning
%% ===== \end{flushleft}
%% ===== 
%% ===== 
%% ===== 
%% ===== 
%% ===== 
%% ===== \begin{flushleft}
%% ===== ESN
%% ===== \end{flushleft}
%% ===== 
%% ===== 
%% ===== 
%% ===== 
%% ===== 
%% ===== \begin{flushleft}
%% ===== Effective segment number (TPR.TSR)
%% ===== \end{flushleft}
%% ===== 
%% ===== 
%% ===== 
%% ===== 
%% ===== 
%% ===== \begin{flushleft}
%% ===== a
%% ===== \end{flushleft}
%% ===== 
%% ===== 
%% ===== 
%% ===== 
%% ===== 
%% ===== \begin{flushleft}
%% ===== BSY
%% ===== \end{flushleft}
%% ===== 
%% ===== 
%% ===== 
%% ===== 
%% ===== 
%% ===== \begin{flushleft}
%% ===== Data source for ESN
%% ===== \end{flushleft}
%% ===== 
%% ===== 
%% ===== \begin{flushleft}
%% ===== 00 = from PPR.PSR
%% ===== \end{flushleft}
%% ===== 
%% ===== 
%% ===== \begin{flushleft}
%% ===== 01 = from PRn.SNR
%% ===== \end{flushleft}
%% ===== 
%% ===== 
%% ===== \begin{flushleft}
%% ===== 10 = from TPR.TSR
%% ===== \end{flushleft}
%% ===== 
%% ===== 
%% ===== \begin{flushleft}
%% ===== 11 = not used
%% ===== \end{flushleft}
%% ===== 
%% ===== 
%% ===== 
%% ===== 
%% ===== 
%% ===== \begin{flushleft}
%% ===== b
%% ===== \end{flushleft}
%% ===== 
%% ===== 
%% ===== 
%% ===== 
%% ===== 
%% ===== \begin{flushleft}
%% ===== FDSPTW
%% ===== \end{flushleft}
%% ===== 
%% ===== 
%% ===== 
%% ===== 
%% ===== 
%% ===== \begin{flushleft}
%% ===== Descriptor segment PTW fetch
%% ===== \end{flushleft}
%% ===== 
%% ===== 
%% ===== 
%% ===== 
%% ===== 
%% ===== \begin{flushleft}
%% ===== c
%% ===== \end{flushleft}
%% ===== 
%% ===== 
%% ===== 
%% ===== 
%% ===== 
%% ===== \begin{flushleft}
%% ===== MDSPTW
%% ===== \end{flushleft}
%% ===== 
%% ===== 
%% ===== 
%% ===== 
%% ===== 
%% ===== \begin{flushleft}
%% ===== Descriptor segment PTW modification
%% ===== \end{flushleft}
%% ===== 
%% ===== 
%% ===== 
%% ===== 
%% ===== 
%% ===== \begin{flushleft}
%% ===== d
%% ===== \end{flushleft}
%% ===== 
%% ===== 
%% ===== 
%% ===== 
%% ===== 
%% ===== \begin{flushleft}
%% ===== FSDWP
%% ===== \end{flushleft}
%% ===== 
%% ===== 
%% ===== 
%% ===== 
%% ===== 
%% ===== \begin{flushleft}
%% ===== SDW fetch from paged descriptor segment
%% ===== \end{flushleft}
%% ===== 
%% ===== 
%% ===== 
%% ===== 
%% ===== 
%% ===== \begin{flushleft}
%% ===== e
%% ===== \end{flushleft}
%% ===== 
%% ===== 
%% ===== 
%% ===== 
%% ===== 
%% ===== \begin{flushleft}
%% ===== FPTW
%% ===== \end{flushleft}
%% ===== 
%% ===== 
%% ===== 
%% ===== 
%% ===== 
%% ===== \begin{flushleft}
%% ===== PTW fetch
%% ===== \end{flushleft}
%% ===== 
%% ===== 
%% ===== 
%% ===== 
%% ===== 
%% ===== \begin{flushleft}
%% ===== f
%% ===== \end{flushleft}
%% ===== 
%% ===== 
%% ===== 
%% ===== 
%% ===== 
%% ===== \begin{flushleft}
%% ===== FPTW2
%% ===== \end{flushleft}
%% ===== 
%% ===== 
%% ===== 
%% ===== 
%% ===== 
%% ===== \begin{flushleft}
%% ===== PTW+1 fetch (prepaging for certain EIS instructions)
%% ===== \end{flushleft}
%% ===== 
%% ===== 
%% ===== 
%% ===== 
%% ===== 
%% ===== \begin{flushleft}
%% ===== g
%% ===== \end{flushleft}
%% ===== 
%% ===== 
%% ===== 
%% ===== 
%% ===== 
%% ===== \begin{flushleft}
%% ===== MPTW
%% ===== \end{flushleft}
%% ===== 
%% ===== 
%% ===== 
%% ===== 
%% ===== 
%% ===== \begin{flushleft}
%% ===== PTW modification
%% ===== \end{flushleft}
%% ===== 
%% ===== 
%% ===== 
%% ===== 
%% ===== 
%% ===== \begin{flushleft}
%% ===== \newpage
%% ===== key Flag Name
%% ===== \end{flushleft}
%% ===== 
%% ===== 
%% ===== 
%% ===== 
%% ===== 
%% ===== \begin{flushleft}
%% ===== Meaning
%% ===== \end{flushleft}
%% ===== 
%% ===== 
%% ===== 
%% ===== 
%% ===== 
%% ===== \begin{flushleft}
%% ===== h
%% ===== \end{flushleft}
%% ===== 
%% ===== 
%% ===== 
%% ===== 
%% ===== 
%% ===== \begin{flushleft}
%% ===== FANP
%% ===== \end{flushleft}
%% ===== 
%% ===== 
%% ===== 
%% ===== 
%% ===== 
%% ===== \begin{flushleft}
%% ===== Final address fetch from nonpaged segment
%% ===== \end{flushleft}
%% ===== 
%% ===== 
%% ===== 
%% ===== 
%% ===== 
%% ===== \begin{flushleft}
%% ===== i
%% ===== \end{flushleft}
%% ===== 
%% ===== 
%% ===== 
%% ===== 
%% ===== 
%% ===== \begin{flushleft}
%% ===== FAP
%% ===== \end{flushleft}
%% ===== 
%% ===== 
%% ===== 
%% ===== 
%% ===== 
%% ===== \begin{flushleft}
%% ===== Final address fetch from paged segment
%% ===== \end{flushleft}
%% ===== 
%% ===== 
%% ===== 
%% ===== 
%% ===== 
%% ===== \begin{flushleft}
%% ===== j
%% ===== \end{flushleft}
%% ===== 
%% ===== 
%% ===== 
%% ===== 
%% ===== 
%% ===== \begin{flushleft}
%% ===== SDWAMM
%% ===== \end{flushleft}
%% ===== 
%% ===== 
%% ===== 
%% ===== 
%% ===== 
%% ===== \begin{flushleft}
%% ===== SDWAM match occurred
%% ===== \end{flushleft}
%% ===== 
%% ===== 
%% ===== 
%% ===== 
%% ===== 
%% ===== \begin{flushleft}
%% ===== SDWAMR
%% ===== \end{flushleft}
%% ===== 
%% ===== 
%% ===== 
%% ===== 
%% ===== 
%% ===== \begin{flushleft}
%% ===== SDWAM register number if SDWAMM=1
%% ===== \end{flushleft}
%% ===== 
%% ===== 
%% ===== 
%% ===== 
%% ===== 
%% ===== \begin{flushleft}
%% ===== PTWAMM
%% ===== \end{flushleft}
%% ===== 
%% ===== 
%% ===== 
%% ===== 
%% ===== 
%% ===== \begin{flushleft}
%% ===== PTWAM match occurred
%% ===== \end{flushleft}
%% ===== 
%% ===== 
%% ===== 
%% ===== 
%% ===== 
%% ===== \begin{flushleft}
%% ===== PTWAMR
%% ===== \end{flushleft}
%% ===== 
%% ===== 
%% ===== 
%% ===== 
%% ===== 
%% ===== \begin{flushleft}
%% ===== PTWAM register number if PTWAMM=1
%% ===== \end{flushleft}
%% ===== 
%% ===== 
%% ===== 
%% ===== 
%% ===== 
%% ===== \begin{flushleft}
%% ===== FLT
%% ===== \end{flushleft}
%% ===== 
%% ===== 
%% ===== 
%% ===== 
%% ===== 
%% ===== \begin{flushleft}
%% ===== Access violation or directed fault on this cycle
%% ===== \end{flushleft}
%% ===== 
%% ===== 
%% ===== 
%% ===== 
%% ===== 
%% ===== \begin{flushleft}
%% ===== ADD
%% ===== \end{flushleft}
%% ===== 
%% ===== 
%% ===== 
%% ===== 
%% ===== 
%% ===== \begin{flushleft}
%% ===== 24-bit absolute main memory address from this cycle
%% ===== \end{flushleft}
%% ===== 
%% ===== 
%% ===== 
%% ===== 
%% ===== 
%% ===== \begin{flushleft}
%% ===== TRR
%% ===== \end{flushleft}
%% ===== 
%% ===== 
%% ===== 
%% ===== 
%% ===== 
%% ===== \begin{flushleft}
%% ===== Ring number from this cycle (TPR.TRR)
%% ===== \end{flushleft}
%% ===== 
%% ===== 
%% ===== 
%% ===== 
%% ===== 
%% ===== \begin{flushleft}
%% ===== k
%% ===== \end{flushleft}
%% ===== 
%% ===== 
%% ===== \begin{flushleft}
%% ===== l
%% ===== \end{flushleft}
%% ===== 
%% ===== 
%% ===== 
%% ===== 
%% ===== 
%% ===== \begin{flushleft}
%% ===== m
%% ===== \end{flushleft}
%% ===== 
%% ===== 
%% ===== \begin{flushleft}
%% ===== n
%% ===== \end{flushleft}
%% ===== 
%% ===== 
%% ===== 
%% ===== 
%% ===== 
%% ===== \begin{flushleft}
%% ===== Multiple match error in SDWAM
%% ===== \end{flushleft}
%% ===== 
%% ===== 
%% ===== \begin{flushleft}
%% ===== CA
%% ===== \end{flushleft}
%% ===== 
%% ===== 
%% ===== 
%% ===== 
%% ===== 
%% ===== \begin{flushleft}
%% ===== Segment is encacheable
%% ===== \end{flushleft}
%% ===== 
%% ===== 
%% ===== 
%% ===== 
%% ===== 
%% ===== \begin{flushleft}
%% ===== p
%% ===== \end{flushleft}
%% ===== 
%% ===== 
%% ===== \begin{flushleft}
%% ===== r
%% ===== \end{flushleft}
%% ===== 
%% ===== 
%% ===== 
%% ===== 
%% ===== 
%% ===== \begin{flushleft}
%% ===== Multiple match error in PTWAM
%% ===== \end{flushleft}
%% ===== 
%% ===== 
%% ===== \begin{flushleft}
%% ===== FHLD
%% ===== \end{flushleft}
%% ===== 
%% ===== 
%% ===== 
%% ===== 
%% ===== 
%% ===== \begin{flushleft}
%% ===== An access violation or directed fault is waiting
%% ===== \end{flushleft}
%% ===== 
%% ===== 
%% ===== 
%% ===== 
%% ===== 
%% ===== \begin{flushleft}

\subsection{APPENDING UNIT (APU) HISTORY REGISTERS -- DPS 8M}

%% ===== \end{flushleft}
%% ===== 
%% ===== 
%% ===== \begin{flushleft}
%% ===== Format: - 72 bits each
%% ===== \end{flushleft}
%% ===== 
%% ===== 
%% ===== \begin{flushleft}
%% ===== Even word of Y-pair as stored by Store Central Processor Register (scpr), TAG = 00
%% ===== \end{flushleft}
%% ===== 
%% ===== 
%% ===== 0
%% ===== 
%% ===== 
%% ===== 0
%% ===== 
%% ===== 
%% ===== 
%% ===== 
%% ===== 
%% ===== 1 1 1 1 1 1 2 2 2 2 2 2 2 2 2 2 3 3 3
%% ===== 
%% ===== 
%% ===== 4 5 6 7 8 9 0 1 2 3 4 5 6 7 8 9 0 1 2
%% ===== 
%% ===== 
%% ===== \begin{flushleft}
%% ===== ESN
%% ===== \end{flushleft}
%% ===== 
%% ===== 
%% ===== 
%% ===== 
%% ===== 
%% ===== \begin{flushleft}
%% ===== a b c d e f g h i
%% ===== \end{flushleft}
%% ===== 
%% ===== 
%% ===== 
%% ===== 
%% ===== 
%% ===== \begin{flushleft}
%% ===== j k l BSY m m1
%% ===== \end{flushleft}
%% ===== 
%% ===== 
%% ===== 
%% ===== 
%% ===== 
%% ===== 15 1 1 1 1 1 1 1 1 1 1 1 1
%% ===== 
%% ===== 
%% ===== 
%% ===== 
%% ===== 
%% ===== 2 1
%% ===== 
%% ===== 
%% ===== 
%% ===== 
%% ===== 
%% ===== 3 3
%% ===== 
%% ===== 
%% ===== 4 5
%% ===== 
%% ===== 
%% ===== \begin{flushleft}
%% ===== n
%% ===== \end{flushleft}
%% ===== 
%% ===== 
%% ===== 
%% ===== 
%% ===== 
%% ===== 2
%% ===== 
%% ===== 
%% ===== 
%% ===== 
%% ===== 
%% ===== \begin{flushleft}
%% ===== o
%% ===== \end{flushleft}
%% ===== 
%% ===== 
%% ===== 3 1
%% ===== 
%% ===== 
%% ===== 
%% ===== 
%% ===== 
%% ===== \begin{flushleft}
%% ===== Odd word of Y-pair as stored by Store Central Processor Register (scpr), TAG = 00
%% ===== \end{flushleft}
%% ===== 
%% ===== 
%% ===== 3
%% ===== 
%% ===== 
%% ===== 6
%% ===== 
%% ===== 
%% ===== 
%% ===== 
%% ===== 
%% ===== 5 6
%% ===== 
%% ===== 
%% ===== 9 0
%% ===== 
%% ===== 
%% ===== \begin{flushleft}
%% ===== RMA
%% ===== \end{flushleft}
%% ===== 
%% ===== 
%% ===== 
%% ===== 
%% ===== 
%% ===== 6 6 6 6 6 6 6 6 7 7
%% ===== 
%% ===== 
%% ===== 2 3 4 5 6 7 8 9 0 1
%% ===== 
%% ===== 
%% ===== \begin{flushleft}
%% ===== p
%% ===== \end{flushleft}
%% ===== 
%% ===== 
%% ===== 
%% ===== 
%% ===== 
%% ===== 24
%% ===== 
%% ===== 
%% ===== 
%% ===== 
%% ===== 
%% ===== \begin{flushleft}
%% ===== q
%% ===== \end{flushleft}
%% ===== 
%% ===== 
%% ===== 3 1
%% ===== 
%% ===== 
%% ===== 
%% ===== 
%% ===== 
%% ===== \begin{flushleft}
%% ===== r
%% ===== \end{flushleft}
%% ===== 
%% ===== 
%% ===== 
%% ===== 
%% ===== 
%% ===== \begin{flushleft}
%% ===== s t
%% ===== \end{flushleft}
%% ===== 
%% ===== 
%% ===== 2 1 1
%% ===== 
%% ===== 
%% ===== 
%% ===== 
%% ===== 
%% ===== \begin{flushleft}
%% ===== u
%% ===== \end{flushleft}
%% ===== 
%% ===== 
%% ===== 
%% ===== 
%% ===== 
%% ===== \begin{flushleft}
%% ===== v w
%% ===== \end{flushleft}
%% ===== 
%% ===== 
%% ===== 2 1 1
%% ===== 
%% ===== 
%% ===== 
%% ===== 
%% ===== 
%% ===== \begin{flushleft}
%% ===== \newpage
%% ===== Extended APU History Register:
%% ===== \end{flushleft}
%% ===== 
%% ===== 
%% ===== \begin{flushleft}
%% ===== Even word of Y-pair as stored by Store Central Processor Register (scpr), TAG = 10
%% ===== \end{flushleft}
%% ===== 
%% ===== 
%% ===== 0
%% ===== 
%% ===== 
%% ===== 0
%% ===== 
%% ===== 
%% ===== 
%% ===== 
%% ===== 
%% ===== 1 1
%% ===== 
%% ===== 
%% ===== 7 8
%% ===== 
%% ===== 
%% ===== \begin{flushleft}
%% ===== ZCA
%% ===== \end{flushleft}
%% ===== 
%% ===== 
%% ===== 
%% ===== 
%% ===== 
%% ===== 2 2 2
%% ===== 
%% ===== 
%% ===== 7 8 9
%% ===== 
%% ===== 
%% ===== \begin{flushleft}
%% ===== Instr
%% ===== \end{flushleft}
%% ===== 
%% ===== 
%% ===== 
%% ===== 
%% ===== 
%% ===== 18
%% ===== 
%% ===== 
%% ===== 
%% ===== 
%% ===== 
%% ===== 3
%% ===== 
%% ===== 
%% ===== 6
%% ===== 
%% ===== 
%% ===== 
%% ===== 
%% ===== 
%% ===== \begin{flushleft}
%% ===== I
%% ===== \end{flushleft}
%% ===== 
%% ===== 
%% ===== 10 1
%% ===== 
%% ===== 
%% ===== 
%% ===== 
%% ===== 
%% ===== 3
%% ===== 
%% ===== 
%% ===== 5
%% ===== 
%% ===== 
%% ===== \begin{flushleft}
%% ===== MOD
%% ===== \end{flushleft}
%% ===== 
%% ===== 
%% ===== 7
%% ===== 
%% ===== 
%% ===== 7
%% ===== 
%% ===== 
%% ===== 1
%% ===== 
%% ===== 
%% ===== 
%% ===== 
%% ===== 
%% ===== \begin{flushleft}
%% ===== NOT USED
%% ===== \end{flushleft}
%% ===== 
%% ===== 
%% ===== 36
%% ===== 
%% ===== 
%% ===== 
%% ===== 
%% ===== 
%% ===== \begin{flushleft}
%% ===== Figure 3-29. Appending Unit (APU) History Register Format - DPS 8M
%% ===== \end{flushleft}
%% ===== 
%% ===== 
%% ===== \begin{flushleft}
%% ===== Description:
%% ===== \end{flushleft}
%% ===== 
%% ===== 
%% ===== \begin{flushleft}
%% ===== A combination of 64 flags and registers from the appending unit. The 64 registers are
%% ===== \end{flushleft}
%% ===== 
%% ===== 
%% ===== \begin{flushleft}
%% ===== handled as a rotating queue controlled by the appending unit history register counter. The
%% ===== \end{flushleft}
%% ===== 
%% ===== 
%% ===== \begin{flushleft}
%% ===== counter is always set to the number of the oldest entry and advances by one for each history
%% ===== \end{flushleft}
%% ===== 
%% ===== 
%% ===== \begin{flushleft}
%% ===== register reference (data entry or Store Central Processor Register (scpr) instruction).
%% ===== \end{flushleft}
%% ===== 
%% ===== 
%% ===== \begin{flushleft}
%% ===== Function:
%% ===== \end{flushleft}
%% ===== 
%% ===== 
%% ===== \begin{flushleft}
%% ===== An appending unit history register entry shows the conditions in the appending unit at the
%% ===== \end{flushleft}
%% ===== 
%% ===== 
%% ===== \begin{flushleft}
%% ===== end of an address preparation cycle in appending mode. The 64 registers hold the
%% ===== \end{flushleft}
%% ===== 
%% ===== 
%% ===== \begin{flushleft}
%% ===== conditions for the last 64 such address preparation cycles. Entries are made according to
%% ===== \end{flushleft}
%% ===== 
%% ===== 
%% ===== \begin{flushleft}
%% ===== controls set in the Mode Register. (See Mode Register earlier in this section.)
%% ===== \end{flushleft}
%% ===== 
%% ===== 
%% ===== \begin{flushleft}
%% ===== The meanings of the constituent flags and registers are:
%% ===== \end{flushleft}
%% ===== 
%% ===== 
%% ===== 
%% ===== 
%% ===== 
%% ===== \begin{flushleft}
%% ===== key Flag Name
%% ===== \end{flushleft}
%% ===== 
%% ===== 
%% ===== \begin{flushleft}
%% ===== ESN
%% ===== \end{flushleft}
%% ===== 
%% ===== 
%% ===== 
%% ===== 
%% ===== 
%% ===== \begin{flushleft}
%% ===== Meaning
%% ===== \end{flushleft}
%% ===== 
%% ===== 
%% ===== \begin{flushleft}
%% ===== Effective segment number
%% ===== \end{flushleft}
%% ===== 
%% ===== 
%% ===== 
%% ===== 
%% ===== 
%% ===== \begin{flushleft}
%% ===== a
%% ===== \end{flushleft}
%% ===== 
%% ===== 
%% ===== 
%% ===== 
%% ===== 
%% ===== \begin{flushleft}
%% ===== PIA Page overflow
%% ===== \end{flushleft}
%% ===== 
%% ===== 
%% ===== 
%% ===== 
%% ===== 
%% ===== \begin{flushleft}
%% ===== b
%% ===== \end{flushleft}
%% ===== 
%% ===== 
%% ===== 
%% ===== 
%% ===== 
%% ===== \begin{flushleft}
%% ===== PIA out of segment bounds
%% ===== \end{flushleft}
%% ===== 
%% ===== 
%% ===== 
%% ===== 
%% ===== 
%% ===== \begin{flushleft}
%% ===== c
%% ===== \end{flushleft}
%% ===== 
%% ===== 
%% ===== 
%% ===== 
%% ===== 
%% ===== \begin{flushleft}
%% ===== FDSPTW
%% ===== \end{flushleft}
%% ===== 
%% ===== 
%% ===== 
%% ===== 
%% ===== 
%% ===== \begin{flushleft}
%% ===== Fetch descriptor segment PTW
%% ===== \end{flushleft}
%% ===== 
%% ===== 
%% ===== 
%% ===== 
%% ===== 
%% ===== \begin{flushleft}
%% ===== d
%% ===== \end{flushleft}
%% ===== 
%% ===== 
%% ===== 
%% ===== 
%% ===== 
%% ===== \begin{flushleft}
%% ===== MDSPTW
%% ===== \end{flushleft}
%% ===== 
%% ===== 
%% ===== 
%% ===== 
%% ===== 
%% ===== \begin{flushleft}
%% ===== Descriptor segment PTW is modified
%% ===== \end{flushleft}
%% ===== 
%% ===== 
%% ===== 
%% ===== 
%% ===== 
%% ===== \begin{flushleft}
%% ===== e
%% ===== \end{flushleft}
%% ===== 
%% ===== 
%% ===== 
%% ===== 
%% ===== 
%% ===== \begin{flushleft}
%% ===== FSDW
%% ===== \end{flushleft}
%% ===== 
%% ===== 
%% ===== 
%% ===== 
%% ===== 
%% ===== \begin{flushleft}
%% ===== Fetch SDW
%% ===== \end{flushleft}
%% ===== 
%% ===== 
%% ===== 
%% ===== 
%% ===== 
%% ===== \begin{flushleft}
%% ===== f
%% ===== \end{flushleft}
%% ===== 
%% ===== 
%% ===== 
%% ===== 
%% ===== 
%% ===== \begin{flushleft}
%% ===== FPTW
%% ===== \end{flushleft}
%% ===== 
%% ===== 
%% ===== 
%% ===== 
%% ===== 
%% ===== \begin{flushleft}
%% ===== Fetch PTW
%% ===== \end{flushleft}
%% ===== 
%% ===== 
%% ===== 
%% ===== 
%% ===== 
%% ===== \begin{flushleft}
%% ===== g
%% ===== \end{flushleft}
%% ===== 
%% ===== 
%% ===== 
%% ===== 
%% ===== 
%% ===== \begin{flushleft}
%% ===== FPTW2
%% ===== \end{flushleft}
%% ===== 
%% ===== 
%% ===== 
%% ===== 
%% ===== 
%% ===== \begin{flushleft}
%% ===== Fetch pre-page PTW
%% ===== \end{flushleft}
%% ===== 
%% ===== 
%% ===== 
%% ===== 
%% ===== 
%% ===== \begin{flushleft}
%% ===== h
%% ===== \end{flushleft}
%% ===== 
%% ===== 
%% ===== 
%% ===== 
%% ===== 
%% ===== \begin{flushleft}
%% ===== MPTW
%% ===== \end{flushleft}
%% ===== 
%% ===== 
%% ===== 
%% ===== 
%% ===== 
%% ===== \begin{flushleft}
%% ===== PTW modified
%% ===== \end{flushleft}
%% ===== 
%% ===== 
%% ===== 
%% ===== 
%% ===== 
%% ===== \begin{flushleft}
%% ===== i
%% ===== \end{flushleft}
%% ===== 
%% ===== 
%% ===== 
%% ===== 
%% ===== 
%% ===== \begin{flushleft}
%% ===== FANP
%% ===== \end{flushleft}
%% ===== 
%% ===== 
%% ===== 
%% ===== 
%% ===== 
%% ===== \begin{flushleft}
%% ===== Final address nonpaged
%% ===== \end{flushleft}
%% ===== 
%% ===== 
%% ===== 
%% ===== 
%% ===== 
%% ===== \begin{flushleft}
%% ===== j
%% ===== \end{flushleft}
%% ===== 
%% ===== 
%% ===== 
%% ===== 
%% ===== 
%% ===== \begin{flushleft}
%% ===== FAP
%% ===== \end{flushleft}
%% ===== 
%% ===== 
%% ===== 
%% ===== 
%% ===== 
%% ===== \begin{flushleft}
%% ===== Final address paged
%% ===== \end{flushleft}
%% ===== 
%% ===== 
%% ===== 
%% ===== 
%% ===== 
%% ===== \begin{flushleft}
%% ===== k
%% ===== \end{flushleft}
%% ===== 
%% ===== 
%% ===== 
%% ===== 
%% ===== 
%% ===== \begin{flushleft}
%% ===== MTCHSDW
%% ===== \end{flushleft}
%% ===== 
%% ===== 
%% ===== 
%% ===== 
%% ===== 
%% ===== \begin{flushleft}
%% ===== SDW match found
%% ===== \end{flushleft}
%% ===== 
%% ===== 
%% ===== 
%% ===== 
%% ===== 
%% ===== \begin{flushleft}
%% ===== l
%% ===== \end{flushleft}
%% ===== 
%% ===== 
%% ===== 
%% ===== 
%% ===== 
%% ===== \begin{flushleft}
%% ===== SDWMF
%% ===== \end{flushleft}
%% ===== 
%% ===== 
%% ===== 
%% ===== 
%% ===== 
%% ===== \begin{flushleft}
%% ===== SDW match found and used
%% ===== \end{flushleft}
%% ===== 
%% ===== 
%% ===== 
%% ===== 
%% ===== 
%% ===== \begin{flushleft}
%% ===== \newpage
%% ===== key Flag Name
%% ===== \end{flushleft}
%% ===== 
%% ===== 
%% ===== 
%% ===== 
%% ===== 
%% ===== \begin{flushleft}
%% ===== m
%% ===== \end{flushleft}
%% ===== 
%% ===== 
%% ===== 
%% ===== 
%% ===== 
%% ===== \begin{flushleft}
%% ===== Meaning
%% ===== \end{flushleft}
%% ===== 
%% ===== 
%% ===== 
%% ===== 
%% ===== 
%% ===== \begin{flushleft}
%% ===== BSY
%% ===== \end{flushleft}
%% ===== 
%% ===== 
%% ===== 
%% ===== 
%% ===== 
%% ===== \begin{flushleft}
%% ===== Data source for ESN
%% ===== \end{flushleft}
%% ===== 
%% ===== 
%% ===== \begin{flushleft}
%% ===== 00 = from ppr.ic
%% ===== \end{flushleft}
%% ===== 
%% ===== 
%% ===== \begin{flushleft}
%% ===== 01 = from prn.tsr
%% ===== \end{flushleft}
%% ===== 
%% ===== 
%% ===== \begin{flushleft}
%% ===== 10 = from tpr.swr
%% ===== \end{flushleft}
%% ===== 
%% ===== 
%% ===== \begin{flushleft}
%% ===== 11 = from tpr.ca
%% ===== \end{flushleft}
%% ===== 
%% ===== 
%% ===== 
%% ===== 
%% ===== 
%% ===== \begin{flushleft}
%% ===== MTCHPTW
%% ===== \end{flushleft}
%% ===== 
%% ===== 
%% ===== 
%% ===== 
%% ===== 
%% ===== \begin{flushleft}
%% ===== PTW match found (AM)
%% ===== \end{flushleft}
%% ===== 
%% ===== 
%% ===== 
%% ===== 
%% ===== 
%% ===== \begin{flushleft}
%% ===== m1 PTWMF
%% ===== \end{flushleft}
%% ===== 
%% ===== 
%% ===== 
%% ===== 
%% ===== 
%% ===== \begin{flushleft}
%% ===== PTW match found (AM) and used
%% ===== \end{flushleft}
%% ===== 
%% ===== 
%% ===== 
%% ===== 
%% ===== 
%% ===== \begin{flushleft}
%% ===== n
%% ===== \end{flushleft}
%% ===== 
%% ===== 
%% ===== 
%% ===== 
%% ===== 
%% ===== \begin{flushleft}
%% ===== PTWAM
%% ===== \end{flushleft}
%% ===== 
%% ===== 
%% ===== 
%% ===== 
%% ===== 
%% ===== \begin{flushleft}
%% ===== PTW AM direct address (ZCA bits 4-7)
%% ===== \end{flushleft}
%% ===== 
%% ===== 
%% ===== 
%% ===== 
%% ===== 
%% ===== \begin{flushleft}
%% ===== o
%% ===== \end{flushleft}
%% ===== 
%% ===== 
%% ===== 
%% ===== 
%% ===== 
%% ===== \begin{flushleft}
%% ===== SDWMF
%% ===== \end{flushleft}
%% ===== 
%% ===== 
%% ===== 
%% ===== 
%% ===== 
%% ===== \begin{flushleft}
%% ===== SDW match found
%% ===== \end{flushleft}
%% ===== 
%% ===== 
%% ===== 
%% ===== 
%% ===== 
%% ===== \begin{flushleft}
%% ===== RMA
%% ===== \end{flushleft}
%% ===== 
%% ===== 
%% ===== 
%% ===== 
%% ===== 
%% ===== \begin{flushleft}
%% ===== Read 24 bit memory address
%% ===== \end{flushleft}
%% ===== 
%% ===== 
%% ===== 
%% ===== 
%% ===== 
%% ===== \begin{flushleft}
%% ===== p
%% ===== \end{flushleft}
%% ===== 
%% ===== 
%% ===== 
%% ===== 
%% ===== 
%% ===== \begin{flushleft}
%% ===== RTRR
%% ===== \end{flushleft}
%% ===== 
%% ===== 
%% ===== 
%% ===== 
%% ===== 
%% ===== \begin{flushleft}
%% ===== Temporary ring register
%% ===== \end{flushleft}
%% ===== 
%% ===== 
%% ===== 
%% ===== 
%% ===== 
%% ===== \begin{flushleft}
%% ===== q
%% ===== \end{flushleft}
%% ===== 
%% ===== 
%% ===== 
%% ===== 
%% ===== 
%% ===== \begin{flushleft}
%% ===== SDWME
%% ===== \end{flushleft}
%% ===== 
%% ===== 
%% ===== 
%% ===== 
%% ===== 
%% ===== \begin{flushleft}
%% ===== SDW match error
%% ===== \end{flushleft}
%% ===== 
%% ===== 
%% ===== 
%% ===== 
%% ===== 
%% ===== \begin{flushleft}
%% ===== r
%% ===== \end{flushleft}
%% ===== 
%% ===== 
%% ===== 
%% ===== 
%% ===== 
%% ===== \begin{flushleft}
%% ===== SDWLVL
%% ===== \end{flushleft}
%% ===== 
%% ===== 
%% ===== 
%% ===== 
%% ===== 
%% ===== \begin{flushleft}
%% ===== SDW match level count (0 = Level A)
%% ===== \end{flushleft}
%% ===== 
%% ===== 
%% ===== 
%% ===== 
%% ===== 
%% ===== \begin{flushleft}
%% ===== s
%% ===== \end{flushleft}
%% ===== 
%% ===== 
%% ===== 
%% ===== 
%% ===== 
%% ===== \begin{flushleft}
%% ===== CACHE
%% ===== \end{flushleft}
%% ===== 
%% ===== 
%% ===== 
%% ===== 
%% ===== 
%% ===== \begin{flushleft}
%% ===== Cache used this cycle
%% ===== \end{flushleft}
%% ===== 
%% ===== 
%% ===== 
%% ===== 
%% ===== 
%% ===== \begin{flushleft}
%% ===== t
%% ===== \end{flushleft}
%% ===== 
%% ===== 
%% ===== 
%% ===== 
%% ===== 
%% ===== \begin{flushleft}
%% ===== PTW match error
%% ===== \end{flushleft}
%% ===== 
%% ===== 
%% ===== 
%% ===== 
%% ===== 
%% ===== \begin{flushleft}
%% ===== u
%% ===== \end{flushleft}
%% ===== 
%% ===== 
%% ===== 
%% ===== 
%% ===== 
%% ===== \begin{flushleft}
%% ===== PTWLVL
%% ===== \end{flushleft}
%% ===== 
%% ===== 
%% ===== 
%% ===== 
%% ===== 
%% ===== \begin{flushleft}
%% ===== PTW match level count (0 = level A)
%% ===== \end{flushleft}
%% ===== 
%% ===== 
%% ===== 
%% ===== 
%% ===== 
%% ===== \begin{flushleft}
%% ===== v
%% ===== \end{flushleft}
%% ===== 
%% ===== 
%% ===== 
%% ===== 
%% ===== 
%% ===== \begin{flushleft}
%% ===== FLTHLD
%% ===== \end{flushleft}
%% ===== 
%% ===== 
%% ===== 
%% ===== 
%% ===== 
%% ===== \begin{flushleft}
%% ===== A directed fault or access violation fault is waiting
%% ===== \end{flushleft}
%% ===== 
%% ===== 
%% ===== 
%% ===== 
%% ===== 
%% ===== \begin{flushleft}
%% ===== ZCA
%% ===== \end{flushleft}
%% ===== 
%% ===== 
%% ===== 
%% ===== 
%% ===== 
%% ===== \begin{flushleft}
%% ===== Computed address
%% ===== \end{flushleft}
%% ===== 
%% ===== 
%% ===== 
%% ===== 
%% ===== 
%% ===== \begin{flushleft}
%% ===== INSTR
%% ===== \end{flushleft}
%% ===== 
%% ===== 
%% ===== 
%% ===== 
%% ===== 
%% ===== \begin{flushleft}
%% ===== Instruction executed
%% ===== \end{flushleft}
%% ===== 
%% ===== 
%% ===== 
%% ===== 
%% ===== 
%% ===== \begin{flushleft}
%% ===== I
%% ===== \end{flushleft}
%% ===== 
%% ===== 
%% ===== 
%% ===== 
%% ===== 
%% ===== \begin{flushleft}
%% ===== Inhibit bit
%% ===== \end{flushleft}
%% ===== 
%% ===== 
%% ===== 
%% ===== 
%% ===== 
%% ===== \begin{flushleft}
%% ===== MOD
%% ===== \end{flushleft}
%% ===== 
%% ===== 
%% ===== 
%% ===== 
%% ===== 
%% ===== \begin{flushleft}
%% ===== Instruction modifier
%% ===== \end{flushleft}
%% ===== 
%% ===== 
%% ===== 
%% ===== 
%% ===== 
%% ===== \begin{flushleft}
%% ===== \newpage

\subsection{CONFIGURATION SWITCH DATA - DPS AND L68}

%% ===== \end{flushleft}
%% ===== 
%% ===== 
%% ===== \begin{flushleft}
%% ===== Format: - 36 bits each
%% ===== \end{flushleft}
%% ===== 
%% ===== 
%% ===== \begin{flushleft}
%% ===== Data read by Read Switches (rsw), y = xxxxx0
%% ===== \end{flushleft}
%% ===== 
%% ===== 
%% ===== 0
%% ===== 
%% ===== 
%% ===== 0
%% ===== 
%% ===== 
%% ===== 
%% ===== 
%% ===== 
%% ===== 3
%% ===== 
%% ===== 
%% ===== 5
%% ===== 
%% ===== 
%% ===== \begin{flushleft}
%% ===== Maintenance panel data switches
%% ===== \end{flushleft}
%% ===== 
%% ===== 
%% ===== 36
%% ===== 
%% ===== 
%% ===== 
%% ===== 
%% ===== 
%% ===== \begin{flushleft}
%% ===== Data read by Read Switches (rsw), y = xxxxx2
%% ===== \end{flushleft}
%% ===== 
%% ===== 
%% ===== 0
%% ===== 
%% ===== 
%% ===== 0
%% ===== 
%% ===== 
%% ===== 
%% ===== 
%% ===== 
%% ===== 0 0 0 0
%% ===== 
%% ===== 
%% ===== 3 4 5 6
%% ===== 
%% ===== 
%% ===== 
%% ===== 
%% ===== 
%% ===== 0 0 0 0
%% ===== 
%% ===== 
%% ===== 4
%% ===== 
%% ===== 
%% ===== 
%% ===== 
%% ===== 
%% ===== \begin{flushleft}
%% ===== a
%% ===== \end{flushleft}
%% ===== 
%% ===== 
%% ===== 
%% ===== 
%% ===== 
%% ===== 1 1
%% ===== 
%% ===== 
%% ===== 2 3
%% ===== 
%% ===== 
%% ===== \begin{flushleft}
%% ===== FAULT BASE
%% ===== \end{flushleft}
%% ===== 
%% ===== 
%% ===== 
%% ===== 
%% ===== 
%% ===== 2
%% ===== 
%% ===== 
%% ===== 
%% ===== 
%% ===== 
%% ===== 1 1 2
%% ===== 
%% ===== 
%% ===== 8 9 0
%% ===== 
%% ===== 
%% ===== 
%% ===== 
%% ===== 
%% ===== 2 2 2 2
%% ===== 
%% ===== 
%% ===== 6 7 8 9
%% ===== 
%% ===== 
%% ===== 
%% ===== 
%% ===== 
%% ===== \begin{flushleft}
%% ===== 0 0 0 0 0 0 b 0 0 0 0 0 0 0 c d
%% ===== \end{flushleft}
%% ===== 
%% ===== 
%% ===== 7
%% ===== 
%% ===== 
%% ===== 
%% ===== 
%% ===== 
%% ===== 6 1
%% ===== 
%% ===== 
%% ===== 
%% ===== 
%% ===== 
%% ===== 3 3
%% ===== 
%% ===== 
%% ===== 2 3
%% ===== 
%% ===== 
%% ===== 
%% ===== 
%% ===== 
%% ===== \begin{flushleft}
%% ===== CPU ID
%% ===== \end{flushleft}
%% ===== 
%% ===== 
%% ===== 
%% ===== 
%% ===== 
%% ===== 7 1 1
%% ===== 
%% ===== 
%% ===== 
%% ===== 
%% ===== 
%% ===== 3
%% ===== 
%% ===== 
%% ===== 5
%% ===== 
%% ===== 
%% ===== 
%% ===== 
%% ===== 
%% ===== \begin{flushleft}
%% ===== CPU
%% ===== \end{flushleft}
%% ===== 
%% ===== 
%% ===== 
%% ===== 
%% ===== 
%% ===== 4
%% ===== 
%% ===== 
%% ===== 
%% ===== 
%% ===== 
%% ===== 3
%% ===== 
%% ===== 
%% ===== 
%% ===== 
%% ===== 
%% ===== \begin{flushleft}
%% ===== Data read by Read Switches (rsw), y = xxxxx1 (port A-D) or xxxxx3 (port E-H)
%% ===== \end{flushleft}
%% ===== 
%% ===== 
%% ===== 0
%% ===== 
%% ===== 
%% ===== 0
%% ===== 
%% ===== 
%% ===== 
%% ===== 
%% ===== 
%% ===== 0 0 0 0 0
%% ===== 
%% ===== 
%% ===== 2 3 4 5 6
%% ===== 
%% ===== 
%% ===== 
%% ===== 
%% ===== 
%% ===== 0 0
%% ===== 
%% ===== 
%% ===== 8 9
%% ===== 
%% ===== 
%% ===== 
%% ===== 
%% ===== 
%% ===== \begin{flushleft}
%% ===== PORT A or E
%% ===== \end{flushleft}
%% ===== 
%% ===== 
%% ===== \begin{flushleft}
%% ===== ADR c d e
%% ===== \end{flushleft}
%% ===== 
%% ===== 
%% ===== 3 1 1 1
%% ===== 
%% ===== 
%% ===== 
%% ===== 
%% ===== 
%% ===== \begin{flushleft}
%% ===== MEM
%% ===== \end{flushleft}
%% ===== 
%% ===== 
%% ===== 3
%% ===== 
%% ===== 
%% ===== 
%% ===== 
%% ===== 
%% ===== 1 1 1 1 1
%% ===== 
%% ===== 
%% ===== 1 2 3 4 5
%% ===== 
%% ===== 
%% ===== 
%% ===== 
%% ===== 
%% ===== 1 1
%% ===== 
%% ===== 
%% ===== 7 8
%% ===== 
%% ===== 
%% ===== 
%% ===== 
%% ===== 
%% ===== \begin{flushleft}
%% ===== PORT B or F
%% ===== \end{flushleft}
%% ===== 
%% ===== 
%% ===== \begin{flushleft}
%% ===== ADR c d e
%% ===== \end{flushleft}
%% ===== 
%% ===== 
%% ===== 3 1 1 1
%% ===== 
%% ===== 
%% ===== 
%% ===== 
%% ===== 
%% ===== 2 2 2 2 2
%% ===== 
%% ===== 
%% ===== 0 1 2 3 4
%% ===== 
%% ===== 
%% ===== 
%% ===== 
%% ===== 
%% ===== 2 2
%% ===== 
%% ===== 
%% ===== 6 7
%% ===== 
%% ===== 
%% ===== 
%% ===== 
%% ===== 
%% ===== \begin{flushleft}
%% ===== PORT C or G
%% ===== \end{flushleft}
%% ===== 
%% ===== 
%% ===== 
%% ===== 
%% ===== 
%% ===== \begin{flushleft}
%% ===== MEM
%% ===== \end{flushleft}
%% ===== 
%% ===== 
%% ===== 3
%% ===== 
%% ===== 
%% ===== 
%% ===== 
%% ===== 
%% ===== \begin{flushleft}
%% ===== ADR c d e
%% ===== \end{flushleft}
%% ===== 
%% ===== 
%% ===== 3 1 1 1
%% ===== 
%% ===== 
%% ===== 
%% ===== 
%% ===== 
%% ===== 2 3 3 3 3
%% ===== 
%% ===== 
%% ===== 9 0 1 2 3
%% ===== 
%% ===== 
%% ===== 
%% ===== 
%% ===== 
%% ===== 3
%% ===== 
%% ===== 
%% ===== 5
%% ===== 
%% ===== 
%% ===== 
%% ===== 
%% ===== 
%% ===== \begin{flushleft}
%% ===== PORT D or H
%% ===== \end{flushleft}
%% ===== 
%% ===== 
%% ===== 
%% ===== 
%% ===== 
%% ===== \begin{flushleft}
%% ===== MEM
%% ===== \end{flushleft}
%% ===== 
%% ===== 
%% ===== 3
%% ===== 
%% ===== 
%% ===== 
%% ===== 
%% ===== 
%% ===== \begin{flushleft}
%% ===== ADR c d e
%% ===== \end{flushleft}
%% ===== 
%% ===== 
%% ===== 3 1 1 1
%% ===== 
%% ===== 
%% ===== 
%% ===== 
%% ===== 
%% ===== \begin{flushleft}
%% ===== MEM
%% ===== \end{flushleft}
%% ===== 
%% ===== 
%% ===== 3
%% ===== 
%% ===== 
%% ===== 
%% ===== 
%% ===== 
%% ===== \begin{flushleft}
%% ===== Data read by Read Switches (rsw), y = xxxxx4
%% ===== \end{flushleft}
%% ===== 
%% ===== 
%% ===== 0
%% ===== 
%% ===== 
%% ===== 0
%% ===== 
%% ===== 
%% ===== 
%% ===== 
%% ===== 
%% ===== 1 1 1 1 1 1 1 1 2 2 2 2 2 2 2 2 2 2
%% ===== 
%% ===== 
%% ===== 2 3 4 5 6 7 8 9 0 1 2 3 4 5 6 7 8 9
%% ===== 
%% ===== 
%% ===== 
%% ===== 
%% ===== 
%% ===== 0 0 0 0 0 0 0 0 0 0 0 0 0
%% ===== 
%% ===== 
%% ===== 
%% ===== 
%% ===== 
%% ===== \begin{flushleft}
%% ===== A
%% ===== \end{flushleft}
%% ===== 
%% ===== 
%% ===== 
%% ===== 
%% ===== 
%% ===== \begin{flushleft}
%% ===== B
%% ===== \end{flushleft}
%% ===== 
%% ===== 
%% ===== 
%% ===== 
%% ===== 
%% ===== \begin{flushleft}
%% ===== C
%% ===== \end{flushleft}
%% ===== 
%% ===== 
%% ===== 
%% ===== 
%% ===== 
%% ===== \begin{flushleft}
%% ===== D
%% ===== \end{flushleft}
%% ===== 
%% ===== 
%% ===== 
%% ===== 
%% ===== 
%% ===== \begin{flushleft}
%% ===== E
%% ===== \end{flushleft}
%% ===== 
%% ===== 
%% ===== 
%% ===== 
%% ===== 
%% ===== \begin{flushleft}
%% ===== F
%% ===== \end{flushleft}
%% ===== 
%% ===== 
%% ===== 
%% ===== 
%% ===== 
%% ===== \begin{flushleft}
%% ===== G
%% ===== \end{flushleft}
%% ===== 
%% ===== 
%% ===== 
%% ===== 
%% ===== 
%% ===== \begin{flushleft}
%% ===== H
%% ===== \end{flushleft}
%% ===== 
%% ===== 
%% ===== 
%% ===== 
%% ===== 
%% ===== \begin{flushleft}
%% ===== f g f g f g f g f g f g f g f g
%% ===== \end{flushleft}
%% ===== 
%% ===== 
%% ===== 13 1 1 1 1 1 1 1 1 1 1 1 1 1 1 1 1
%% ===== 
%% ===== 
%% ===== 
%% ===== 
%% ===== 
%% ===== 3
%% ===== 
%% ===== 
%% ===== 5
%% ===== 
%% ===== 
%% ===== 
%% ===== 
%% ===== 
%% ===== 0 0 0 0 0 0 0
%% ===== 
%% ===== 
%% ===== 7
%% ===== 
%% ===== 
%% ===== 
%% ===== 
%% ===== 
%% ===== \begin{flushleft}
%% ===== Figure 3-30. Configuration Switch Data Formats - DPS and L68
%% ===== \end{flushleft}
%% ===== 
%% ===== 
%% ===== \begin{flushleft}
%% ===== Description:
%% ===== \end{flushleft}
%% ===== 
%% ===== 
%% ===== \begin{flushleft}
%% ===== The Read Switches (rsw) instruction provides the ability to interrogate various switches and
%% ===== \end{flushleft}
%% ===== 
%% ===== 
%% ===== \begin{flushleft}
%% ===== options on the processor maintenance and configuration panels. The 3 low-order bits of the
%% ===== \end{flushleft}
%% ===== 
%% ===== 
%% ===== \begin{flushleft}
%% ===== computed address (TPR.CA) select the switches to be read. High-order address bits are
%% ===== \end{flushleft}
%% ===== 
%% ===== 
%% ===== \begin{flushleft}
%% ===== ignored. Data are placed in the A Register.
%% ===== \end{flushleft}
%% ===== 
%% ===== 
%% ===== \begin{flushleft}
%% ===== Read Switches (rsw), y = xxxxx1 reads data for ports A, B, C, and D. Read Switches (rsw), y
%% ===== \end{flushleft}
%% ===== 
%% ===== 
%% ===== \begin{flushleft}
%% ===== = xxxxx3 reads data for ports E, F, G, and H.
%% ===== \end{flushleft}
%% ===== 
%% ===== 
%% ===== \begin{flushleft}
%% ===== Function:
%% ===== \end{flushleft}
%% ===== 
%% ===== 
%% ===== \begin{flushleft}
%% ===== The meanings of the constituent fields are:
%% ===== \end{flushleft}
%% ===== 
%% ===== 
%% ===== 
%% ===== 
%% ===== 
%% ===== \begin{flushleft}
%% ===== \newpage
%% ===== key Field Name
%% ===== \end{flushleft}
%% ===== 
%% ===== 
%% ===== \begin{flushleft}
%% ===== a
%% ===== \end{flushleft}
%% ===== 
%% ===== 
%% ===== 
%% ===== 
%% ===== 
%% ===== \begin{flushleft}
%% ===== Meaning
%% ===== \end{flushleft}
%% ===== 
%% ===== 
%% ===== 
%% ===== 
%% ===== 
%% ===== \begin{flushleft}
%% ===== CPU-Type
%% ===== \end{flushleft}
%% ===== 
%% ===== 
%% ===== 
%% ===== 
%% ===== 
%% ===== \begin{flushleft}
%% ===== Equals {``}00'' for a L68 or a DPS processor.
%% ===== \end{flushleft}
%% ===== 
%% ===== 
%% ===== 
%% ===== 
%% ===== 
%% ===== \begin{flushleft}
%% ===== FLT BASE
%% ===== \end{flushleft}
%% ===== 
%% ===== 
%% ===== 
%% ===== 
%% ===== 
%% ===== \begin{flushleft}
%% ===== The seven MSB of the 12-bit fault base address
%% ===== \end{flushleft}
%% ===== 
%% ===== 
%% ===== 
%% ===== 
%% ===== 
%% ===== \begin{flushleft}
%% ===== b
%% ===== \end{flushleft}
%% ===== 
%% ===== 
%% ===== 
%% ===== 
%% ===== 
%% ===== \begin{flushleft}
%% ===== dps\_option
%% ===== \end{flushleft}
%% ===== 
%% ===== 
%% ===== 
%% ===== 
%% ===== 
%% ===== \begin{flushleft}
%% ===== Processor option
%% ===== \end{flushleft}
%% ===== 
%% ===== 
%% ===== \begin{flushleft}
%% ===== 0 = L68 processor
%% ===== \end{flushleft}
%% ===== 
%% ===== 
%% ===== \begin{flushleft}
%% ===== 1 = DPS processor
%% ===== \end{flushleft}
%% ===== 
%% ===== 
%% ===== 
%% ===== 
%% ===== 
%% ===== \begin{flushleft}
%% ===== c
%% ===== \end{flushleft}
%% ===== 
%% ===== 
%% ===== 
%% ===== 
%% ===== 
%% ===== \begin{flushleft}
%% ===== cache
%% ===== \end{flushleft}
%% ===== 
%% ===== 
%% ===== 
%% ===== 
%% ===== 
%% ===== \begin{flushleft}
%% ===== 2K cache option
%% ===== \end{flushleft}
%% ===== 
%% ===== 
%% ===== \begin{flushleft}
%% ===== 0 = disabled
%% ===== \end{flushleft}
%% ===== 
%% ===== 
%% ===== \begin{flushleft}
%% ===== 1 = enabled
%% ===== \end{flushleft}
%% ===== 
%% ===== 
%% ===== 
%% ===== 
%% ===== 
%% ===== \begin{flushleft}
%% ===== d
%% ===== \end{flushleft}
%% ===== 
%% ===== 
%% ===== 
%% ===== 
%% ===== 
%% ===== \begin{flushleft}
%% ===== ext\_gcos
%% ===== \end{flushleft}
%% ===== 
%% ===== 
%% ===== 
%% ===== 
%% ===== 
%% ===== \begin{flushleft}
%% ===== GCOS mode extended memory option
%% ===== \end{flushleft}
%% ===== 
%% ===== 
%% ===== \begin{flushleft}
%% ===== 0 = disabled
%% ===== \end{flushleft}
%% ===== 
%% ===== 
%% ===== \begin{flushleft}
%% ===== 1 = enabled
%% ===== \end{flushleft}
%% ===== 
%% ===== 
%% ===== 
%% ===== 
%% ===== 
%% ===== \begin{flushleft}
%% ===== CPU\_ID
%% ===== \end{flushleft}
%% ===== 
%% ===== 
%% ===== 
%% ===== 
%% ===== 
%% ===== \begin{flushleft}
%% ===== These bit positions have a configuration of {``}1110'' for a L68 or a
%% ===== \end{flushleft}
%% ===== 
%% ===== 
%% ===== \begin{flushleft}
%% ===== DPS CPU.
%% ===== \end{flushleft}
%% ===== 
%% ===== 
%% ===== 
%% ===== 
%% ===== 
%% ===== \begin{flushleft}
%% ===== CPU
%% ===== \end{flushleft}
%% ===== 
%% ===== 
%% ===== 
%% ===== 
%% ===== 
%% ===== \begin{flushleft}
%% ===== Processor number from processor configuration panel number
%% ===== \end{flushleft}
%% ===== 
%% ===== 
%% ===== \begin{flushleft}
%% ===== switches.
%% ===== \end{flushleft}
%% ===== 
%% ===== 
%% ===== 
%% ===== 
%% ===== 
%% ===== \begin{flushleft}
%% ===== PORT A or E, etc.
%% ===== \end{flushleft}
%% ===== 
%% ===== 
%% ===== 
%% ===== 
%% ===== 
%% ===== \begin{flushleft}
%% ===== Port data fields further substructured as:
%% ===== \end{flushleft}
%% ===== 
%% ===== 
%% ===== 
%% ===== 
%% ===== 
%% ===== \begin{flushleft}
%% ===== ADR
%% ===== \end{flushleft}
%% ===== 
%% ===== 
%% ===== 
%% ===== 
%% ===== 
%% ===== \begin{flushleft}
%% ===== Address assignment switch setting for port
%% ===== \end{flushleft}
%% ===== 
%% ===== 
%% ===== 
%% ===== 
%% ===== 
%% ===== \begin{flushleft}
%% ===== c
%% ===== \end{flushleft}
%% ===== 
%% ===== 
%% ===== 
%% ===== 
%% ===== 
%% ===== \begin{flushleft}
%% ===== Port enabled flag
%% ===== \end{flushleft}
%% ===== 
%% ===== 
%% ===== 
%% ===== 
%% ===== 
%% ===== \begin{flushleft}
%% ===== d
%% ===== \end{flushleft}
%% ===== 
%% ===== 
%% ===== 
%% ===== 
%% ===== 
%% ===== \begin{flushleft}
%% ===== System initialize enabled flag
%% ===== \end{flushleft}
%% ===== 
%% ===== 
%% ===== 
%% ===== 
%% ===== 
%% ===== \begin{flushleft}
%% ===== e
%% ===== \end{flushleft}
%% ===== 
%% ===== 
%% ===== 
%% ===== 
%% ===== 
%% ===== \begin{flushleft}
%% ===== Interlace enabled flag
%% ===== \end{flushleft}
%% ===== 
%% ===== 
%% ===== \begin{flushleft}
%% ===== MEM
%% ===== \end{flushleft}
%% ===== 
%% ===== 
%% ===== 
%% ===== 
%% ===== 
%% ===== \begin{flushleft}
%% ===== Coded memory size . . .
%% ===== \end{flushleft}
%% ===== 
%% ===== 
%% ===== 000
%% ===== 
%% ===== 
%% ===== 001
%% ===== 
%% ===== 
%% ===== 010
%% ===== 
%% ===== 
%% ===== 011
%% ===== 
%% ===== 
%% ===== 100
%% ===== 
%% ===== 
%% ===== 101
%% ===== 
%% ===== 
%% ===== 110
%% ===== 
%% ===== 
%% ===== 111
%% ===== 
%% ===== 
%% ===== 
%% ===== 
%% ===== 
%% ===== \begin{flushleft}
%% ===== A, B, etc.
%% ===== \end{flushleft}
%% ===== 
%% ===== 
%% ===== 
%% ===== 
%% ===== 
%% ===== \begin{flushleft}
%% ===== 32K
%% ===== \end{flushleft}
%% ===== 
%% ===== 
%% ===== \begin{flushleft}
%% ===== 64K
%% ===== \end{flushleft}
%% ===== 
%% ===== 
%% ===== \begin{flushleft}
%% ===== 128K
%% ===== \end{flushleft}
%% ===== 
%% ===== 
%% ===== \begin{flushleft}
%% ===== 256K
%% ===== \end{flushleft}
%% ===== 
%% ===== 
%% ===== \begin{flushleft}
%% ===== 512K
%% ===== \end{flushleft}
%% ===== 
%% ===== 
%% ===== \begin{flushleft}
%% ===== 1024K
%% ===== \end{flushleft}
%% ===== 
%% ===== 
%% ===== \begin{flushleft}
%% ===== 2048K
%% ===== \end{flushleft}
%% ===== 
%% ===== 
%% ===== \begin{flushleft}
%% ===== 4096K
%% ===== \end{flushleft}
%% ===== 
%% ===== 
%% ===== 
%% ===== 
%% ===== 
%% ===== \begin{flushleft}
%% ===== Port data fields further substructured as:
%% ===== \end{flushleft}
%% ===== 
%% ===== 
%% ===== 
%% ===== 
%% ===== 
%% ===== \begin{flushleft}
%% ===== f
%% ===== \end{flushleft}
%% ===== 
%% ===== 
%% ===== 
%% ===== 
%% ===== 
%% ===== \begin{flushleft}
%% ===== Interlace mode
%% ===== \end{flushleft}
%% ===== 
%% ===== 
%% ===== \begin{flushleft}
%% ===== 0 = 4 word if interlace enabled for port
%% ===== \end{flushleft}
%% ===== 
%% ===== 
%% ===== \begin{flushleft}
%% ===== 1 = 2 word if interlace enabled for port
%% ===== \end{flushleft}
%% ===== 
%% ===== 
%% ===== 
%% ===== 
%% ===== 
%% ===== \begin{flushleft}
%% ===== g
%% ===== \end{flushleft}
%% ===== 
%% ===== 
%% ===== 
%% ===== 
%% ===== 
%% ===== \begin{flushleft}
%% ===== Main memory size
%% ===== \end{flushleft}
%% ===== 
%% ===== 
%% ===== \begin{flushleft}
%% ===== 0 = full, all of MEM is configured
%% ===== \end{flushleft}
%% ===== 
%% ===== 
%% ===== \begin{flushleft}
%% ===== 1 = half, half of MEM is configured
%% ===== \end{flushleft}
%% ===== 
%% ===== 
%% ===== 
%% ===== 
%% ===== 
%% ===== \begin{flushleft}

\subsection{CONFIGURATION SWITCH DATA - DPS 8M}

%% ===== \end{flushleft}
%% ===== 
%% ===== 
%% ===== \begin{flushleft}
%% ===== The following changes apply to the DPS 8M processor.
%% ===== \end{flushleft}
%% ===== 
%% ===== 
%% ===== 
%% ===== 
%% ===== 
%% ===== \begin{flushleft}
%% ===== \newpage
%% ===== Format: - 36 bits each
%% ===== \end{flushleft}
%% ===== 
%% ===== 
%% ===== \begin{flushleft}
%% ===== Data read by Read Switches (rsw), y = xxxxx2
%% ===== \end{flushleft}
%% ===== 
%% ===== 
%% ===== 0
%% ===== 
%% ===== 
%% ===== 0
%% ===== 
%% ===== 
%% ===== 
%% ===== 
%% ===== 
%% ===== 0 0 0 0
%% ===== 
%% ===== 
%% ===== 3 4 5 6
%% ===== 
%% ===== 
%% ===== 
%% ===== 
%% ===== 
%% ===== \begin{flushleft}
%% ===== A B C D
%% ===== \end{flushleft}
%% ===== 
%% ===== 
%% ===== \begin{flushleft}
%% ===== a a a a
%% ===== \end{flushleft}
%% ===== 
%% ===== 
%% ===== 4
%% ===== 
%% ===== 
%% ===== 
%% ===== 
%% ===== 
%% ===== \begin{flushleft}
%% ===== b
%% ===== \end{flushleft}
%% ===== 
%% ===== 
%% ===== 
%% ===== 
%% ===== 
%% ===== 1 1 1
%% ===== 
%% ===== 
%% ===== 2 3 4
%% ===== 
%% ===== 
%% ===== \begin{flushleft}
%% ===== FLT BASE
%% ===== \end{flushleft}
%% ===== 
%% ===== 
%% ===== 
%% ===== 
%% ===== 
%% ===== 2
%% ===== 
%% ===== 
%% ===== 
%% ===== 
%% ===== 
%% ===== 1 1 1 2 2 2 2 2 2 2
%% ===== 
%% ===== 
%% ===== 7 8 9 0 1 2 3 4 5 6
%% ===== 
%% ===== 
%% ===== 
%% ===== 
%% ===== 
%% ===== 2 2
%% ===== 
%% ===== 
%% ===== 8 9
%% ===== 
%% ===== 
%% ===== 
%% ===== 
%% ===== 
%% ===== \begin{flushleft}
%% ===== c 0 0 0 0 d e f 0 0 g h i 0 0 0
%% ===== \end{flushleft}
%% ===== 
%% ===== 
%% ===== 7 1
%% ===== 
%% ===== 
%% ===== 
%% ===== 
%% ===== 
%% ===== 4 1 1 1
%% ===== 
%% ===== 
%% ===== 
%% ===== 
%% ===== 
%% ===== 2 1 1 1
%% ===== 
%% ===== 
%% ===== 
%% ===== 
%% ===== 
%% ===== 3
%% ===== 
%% ===== 
%% ===== 
%% ===== 
%% ===== 
%% ===== 3 3
%% ===== 
%% ===== 
%% ===== 2 3
%% ===== 
%% ===== 
%% ===== 
%% ===== 
%% ===== 
%% ===== \begin{flushleft}
%% ===== SPEED
%% ===== \end{flushleft}
%% ===== 
%% ===== 
%% ===== 4
%% ===== 
%% ===== 
%% ===== 
%% ===== 
%% ===== 
%% ===== 3
%% ===== 
%% ===== 
%% ===== 5
%% ===== 
%% ===== 
%% ===== 
%% ===== 
%% ===== 
%% ===== \begin{flushleft}
%% ===== CPU
%% ===== \end{flushleft}
%% ===== 
%% ===== 
%% ===== 3
%% ===== 
%% ===== 
%% ===== 
%% ===== 
%% ===== 
%% ===== \begin{flushleft}
%% ===== Data read by Read Switches (rsw), y = xxxxx1 (port A-D)
%% ===== \end{flushleft}
%% ===== 
%% ===== 
%% ===== 0
%% ===== 
%% ===== 
%% ===== 0
%% ===== 
%% ===== 
%% ===== 
%% ===== 
%% ===== 
%% ===== 0 0
%% ===== 
%% ===== 
%% ===== 8 9
%% ===== 
%% ===== 
%% ===== \begin{flushleft}
%% ===== PORT A
%% ===== \end{flushleft}
%% ===== 
%% ===== 
%% ===== 
%% ===== 
%% ===== 
%% ===== \begin{flushleft}
%% ===== ADR j k l
%% ===== \end{flushleft}
%% ===== 
%% ===== 
%% ===== 3 1 1 1
%% ===== 
%% ===== 
%% ===== 
%% ===== 
%% ===== 
%% ===== 1 1
%% ===== 
%% ===== 
%% ===== 7 8
%% ===== 
%% ===== 
%% ===== \begin{flushleft}
%% ===== PORT B
%% ===== \end{flushleft}
%% ===== 
%% ===== 
%% ===== 
%% ===== 
%% ===== 
%% ===== \begin{flushleft}
%% ===== MEM
%% ===== \end{flushleft}
%% ===== 
%% ===== 
%% ===== 3
%% ===== 
%% ===== 
%% ===== 
%% ===== 
%% ===== 
%% ===== \begin{flushleft}
%% ===== ADR j k l
%% ===== \end{flushleft}
%% ===== 
%% ===== 
%% ===== 3 1 1 1
%% ===== 
%% ===== 
%% ===== 
%% ===== 
%% ===== 
%% ===== 2 2
%% ===== 
%% ===== 
%% ===== 6 7
%% ===== 
%% ===== 
%% ===== \begin{flushleft}
%% ===== PORT C
%% ===== \end{flushleft}
%% ===== 
%% ===== 
%% ===== 
%% ===== 
%% ===== 
%% ===== \begin{flushleft}
%% ===== MEM
%% ===== \end{flushleft}
%% ===== 
%% ===== 
%% ===== 3
%% ===== 
%% ===== 
%% ===== 
%% ===== 
%% ===== 
%% ===== \begin{flushleft}
%% ===== ADR j k l
%% ===== \end{flushleft}
%% ===== 
%% ===== 
%% ===== 3 1 1 1
%% ===== 
%% ===== 
%% ===== 
%% ===== 
%% ===== 
%% ===== 3
%% ===== 
%% ===== 
%% ===== 5
%% ===== 
%% ===== 
%% ===== \begin{flushleft}
%% ===== PORT D
%% ===== \end{flushleft}
%% ===== 
%% ===== 
%% ===== 
%% ===== 
%% ===== 
%% ===== \begin{flushleft}
%% ===== MEM
%% ===== \end{flushleft}
%% ===== 
%% ===== 
%% ===== 3
%% ===== 
%% ===== 
%% ===== 
%% ===== 
%% ===== 
%% ===== \begin{flushleft}
%% ===== ADR j k l
%% ===== \end{flushleft}
%% ===== 
%% ===== 
%% ===== 3 1 1 1
%% ===== 
%% ===== 
%% ===== 
%% ===== 
%% ===== 
%% ===== \begin{flushleft}
%% ===== MEM
%% ===== \end{flushleft}
%% ===== 
%% ===== 
%% ===== 3
%% ===== 
%% ===== 
%% ===== 
%% ===== 
%% ===== 
%% ===== \begin{flushleft}
%% ===== Figure 3-31. Configuration Switch Data Formats - DPS 8M
%% ===== \end{flushleft}
%% ===== 
%% ===== 
%% ===== \begin{flushleft}
%% ===== Description:
%% ===== \end{flushleft}
%% ===== 
%% ===== 
%% ===== \begin{flushleft}
%% ===== The Read Switches (rsw) instruction provides the ability to interrogate various switches and
%% ===== \end{flushleft}
%% ===== 
%% ===== 
%% ===== \begin{flushleft}
%% ===== options on the processor maintenance and configuration panels. The two low-order bits of
%% ===== \end{flushleft}
%% ===== 
%% ===== 
%% ===== \begin{flushleft}
%% ===== the computed address (TPR.CA) select the switches to be read. High-order address bits are
%% ===== \end{flushleft}
%% ===== 
%% ===== 
%% ===== \begin{flushleft}
%% ===== ignored. Data are placed in the A Register.
%% ===== \end{flushleft}
%% ===== 
%% ===== 
%% ===== \begin{flushleft}
%% ===== Read Switches (rsw), y = xxxxx1 reads data for ports A, B, C, and D.
%% ===== \end{flushleft}
%% ===== 
%% ===== 
%% ===== \begin{flushleft}
%% ===== Function:
%% ===== \end{flushleft}
%% ===== 
%% ===== 
%% ===== \begin{flushleft}
%% ===== The meanings of the constituent fields are:
%% ===== \end{flushleft}
%% ===== 
%% ===== 
%% ===== 
%% ===== 
%% ===== 
%% ===== \begin{flushleft}
%% ===== key Field Name
%% ===== \end{flushleft}
%% ===== 
%% ===== 
%% ===== 
%% ===== 
%% ===== 
%% ===== \begin{flushleft}
%% ===== Meaning
%% ===== \end{flushleft}
%% ===== 
%% ===== 
%% ===== 
%% ===== 
%% ===== 
%% ===== \begin{flushleft}
%% ===== a
%% ===== \end{flushleft}
%% ===== 
%% ===== 
%% ===== 
%% ===== 
%% ===== 
%% ===== \begin{flushleft}
%% ===== If the corresponding rsw 1 interface enabled flag, bit (e) is ON, then
%% ===== \end{flushleft}
%% ===== 
%% ===== 
%% ===== \begin{flushleft}
%% ===== 0 = 4 word interfaces
%% ===== \end{flushleft}
%% ===== 
%% ===== 
%% ===== \begin{flushleft}
%% ===== 1 = 2 word interfaces
%% ===== \end{flushleft}
%% ===== 
%% ===== 
%% ===== \begin{flushleft}
%% ===== For ports A - D
%% ===== \end{flushleft}
%% ===== 
%% ===== 
%% ===== 
%% ===== 
%% ===== 
%% ===== \begin{flushleft}
%% ===== b
%% ===== \end{flushleft}
%% ===== 
%% ===== 
%% ===== 
%% ===== 
%% ===== 
%% ===== \begin{flushleft}
%% ===== Indicates processor type
%% ===== \end{flushleft}
%% ===== 
%% ===== 
%% ===== \begin{flushleft}
%% ===== 00 = L68 or DPS Processor
%% ===== \end{flushleft}
%% ===== 
%% ===== 
%% ===== \begin{flushleft}
%% ===== 01 = DPS 8M Processor
%% ===== \end{flushleft}
%% ===== 
%% ===== 
%% ===== \begin{flushleft}
%% ===== 10 = reserved for future use
%% ===== \end{flushleft}
%% ===== 
%% ===== 
%% ===== \begin{flushleft}
%% ===== 11 = reserved for future use
%% ===== \end{flushleft}
%% ===== 
%% ===== 
%% ===== \begin{flushleft}
%% ===== FLTBASE
%% ===== \end{flushleft}
%% ===== 
%% ===== 
%% ===== 
%% ===== 
%% ===== 
%% ===== \begin{flushleft}
%% ===== The seven MSB of the 12-bit fault base address
%% ===== \end{flushleft}
%% ===== 
%% ===== 
%% ===== 
%% ===== 
%% ===== 
%% ===== \begin{flushleft}
%% ===== c
%% ===== \end{flushleft}
%% ===== 
%% ===== 
%% ===== 
%% ===== 
%% ===== 
%% ===== \begin{flushleft}
%% ===== ID prom
%% ===== \end{flushleft}
%% ===== 
%% ===== 
%% ===== \begin{flushleft}
%% ===== 0 = id prom not installed
%% ===== \end{flushleft}
%% ===== 
%% ===== 
%% ===== \begin{flushleft}
%% ===== 1 = id prom installed
%% ===== \end{flushleft}
%% ===== 
%% ===== 
%% ===== 
%% ===== 
%% ===== 
%% ===== \begin{flushleft}
%% ===== d
%% ===== \end{flushleft}
%% ===== 
%% ===== 
%% ===== 
%% ===== 
%% ===== 
%% ===== \begin{flushleft}
%% ===== BCD option (Marketing designation)
%% ===== \end{flushleft}
%% ===== 
%% ===== 
%% ===== \begin{flushleft}
%% ===== 1 = BCD option installed
%% ===== \end{flushleft}
%% ===== 
%% ===== 
%% ===== 
%% ===== 
%% ===== 
%% ===== \begin{flushleft}
%% ===== e
%% ===== \end{flushleft}
%% ===== 
%% ===== 
%% ===== 
%% ===== 
%% ===== 
%% ===== \begin{flushleft}
%% ===== DPS option (Marketing designation)
%% ===== \end{flushleft}
%% ===== 
%% ===== 
%% ===== \begin{flushleft}
%% ===== 1 = DPS option
%% ===== \end{flushleft}
%% ===== 
%% ===== 
%% ===== 
%% ===== 
%% ===== 
%% ===== \begin{flushleft}
%% ===== \newpage
%% ===== key Field Name
%% ===== \end{flushleft}
%% ===== 
%% ===== 
%% ===== 
%% ===== 
%% ===== 
%% ===== \begin{flushleft}
%% ===== Meaning
%% ===== \end{flushleft}
%% ===== 
%% ===== 
%% ===== 
%% ===== 
%% ===== 
%% ===== \begin{flushleft}
%% ===== f
%% ===== \end{flushleft}
%% ===== 
%% ===== 
%% ===== 
%% ===== 
%% ===== 
%% ===== \begin{flushleft}
%% ===== 8K cache
%% ===== \end{flushleft}
%% ===== 
%% ===== 
%% ===== \begin{flushleft}
%% ===== 1 = 8K cache installed
%% ===== \end{flushleft}
%% ===== 
%% ===== 
%% ===== 
%% ===== 
%% ===== 
%% ===== \begin{flushleft}
%% ===== g
%% ===== \end{flushleft}
%% ===== 
%% ===== 
%% ===== 
%% ===== 
%% ===== 
%% ===== \begin{flushleft}
%% ===== DPS 8M Processor type designation
%% ===== \end{flushleft}
%% ===== 
%% ===== 
%% ===== \begin{flushleft}
%% ===== 1 = DPS 8/xxM
%% ===== \end{flushleft}
%% ===== 
%% ===== 
%% ===== \begin{flushleft}
%% ===== 0 = DPS 8/xx
%% ===== \end{flushleft}
%% ===== 
%% ===== 
%% ===== 
%% ===== 
%% ===== 
%% ===== \begin{flushleft}
%% ===== h
%% ===== \end{flushleft}
%% ===== 
%% ===== 
%% ===== 
%% ===== 
%% ===== 
%% ===== \begin{flushleft}
%% ===== GCOS/VMS switch position
%% ===== \end{flushleft}
%% ===== 
%% ===== 
%% ===== \begin{flushleft}
%% ===== 1 = Virtual Mode
%% ===== \end{flushleft}
%% ===== 
%% ===== 
%% ===== \begin{flushleft}
%% ===== 0 = GCOS Mode
%% ===== \end{flushleft}
%% ===== 
%% ===== 
%% ===== 
%% ===== 
%% ===== 
%% ===== \begin{flushleft}
%% ===== i
%% ===== \end{flushleft}
%% ===== 
%% ===== 
%% ===== 
%% ===== 
%% ===== 
%% ===== \begin{flushleft}
%% ===== Current or new product line peripheral type
%% ===== \end{flushleft}
%% ===== 
%% ===== 
%% ===== \begin{flushleft}
%% ===== 1 = NPL
%% ===== \end{flushleft}
%% ===== 
%% ===== 
%% ===== \begin{flushleft}
%% ===== 0 = CPL
%% ===== \end{flushleft}
%% ===== 
%% ===== 
%% ===== \begin{flushleft}
%% ===== SPEED
%% ===== \end{flushleft}
%% ===== 
%% ===== 
%% ===== 
%% ===== 
%% ===== 
%% ===== \begin{flushleft}
%% ===== Processor speed options
%% ===== \end{flushleft}
%% ===== 
%% ===== 
%% ===== 0000 = 8/70
%% ===== 
%% ===== 
%% ===== 0100 = 8/52
%% ===== 
%% ===== 
%% ===== 
%% ===== 
%% ===== 
%% ===== \begin{flushleft}
%% ===== CPU
%% ===== \end{flushleft}
%% ===== 
%% ===== 
%% ===== 
%% ===== 
%% ===== 
%% ===== \begin{flushleft}
%% ===== Processor number
%% ===== \end{flushleft}
%% ===== 
%% ===== 
%% ===== 
%% ===== 
%% ===== 
%% ===== \begin{flushleft}
%% ===== ADR
%% ===== \end{flushleft}
%% ===== 
%% ===== 
%% ===== 
%% ===== 
%% ===== 
%% ===== \begin{flushleft}
%% ===== Address assignment switch setting for port
%% ===== \end{flushleft}
%% ===== 
%% ===== 
%% ===== 
%% ===== 
%% ===== 
%% ===== \begin{flushleft}
%% ===== j
%% ===== \end{flushleft}
%% ===== 
%% ===== 
%% ===== 
%% ===== 
%% ===== 
%% ===== \begin{flushleft}
%% ===== Port enabled flag
%% ===== \end{flushleft}
%% ===== 
%% ===== 
%% ===== 
%% ===== 
%% ===== 
%% ===== \begin{flushleft}
%% ===== k
%% ===== \end{flushleft}
%% ===== 
%% ===== 
%% ===== 
%% ===== 
%% ===== 
%% ===== \begin{flushleft}
%% ===== System initialize enabled flag
%% ===== \end{flushleft}
%% ===== 
%% ===== 
%% ===== 
%% ===== 
%% ===== 
%% ===== \begin{flushleft}
%% ===== l
%% ===== \end{flushleft}
%% ===== 
%% ===== 
%% ===== 
%% ===== 
%% ===== 
%% ===== \begin{flushleft}
%% ===== Interface enabled flag
%% ===== \end{flushleft}
%% ===== 
%% ===== 
%% ===== \begin{flushleft}
%% ===== MEM
%% ===== \end{flushleft}
%% ===== 
%% ===== 
%% ===== 
%% ===== 
%% ===== 
%% ===== \begin{flushleft}
%% ===== Coded memory size:
%% ===== \end{flushleft}
%% ===== 
%% ===== 
%% ===== 000
%% ===== 
%% ===== 
%% ===== 001
%% ===== 
%% ===== 
%% ===== 010
%% ===== 
%% ===== 
%% ===== 011
%% ===== 
%% ===== 
%% ===== 100
%% ===== 
%% ===== 
%% ===== 101
%% ===== 
%% ===== 
%% ===== 110
%% ===== 
%% ===== 
%% ===== 111
%% ===== 
%% ===== 
%% ===== 
%% ===== 
%% ===== 
%% ===== \begin{flushleft}
%% ===== 32K
%% ===== \end{flushleft}
%% ===== 
%% ===== 
%% ===== \begin{flushleft}
%% ===== 64K
%% ===== \end{flushleft}
%% ===== 
%% ===== 
%% ===== \begin{flushleft}
%% ===== 128K
%% ===== \end{flushleft}
%% ===== 
%% ===== 
%% ===== \begin{flushleft}
%% ===== 256K
%% ===== \end{flushleft}
%% ===== 
%% ===== 
%% ===== \begin{flushleft}
%% ===== 512K
%% ===== \end{flushleft}
%% ===== 
%% ===== 
%% ===== \begin{flushleft}
%% ===== 1024K
%% ===== \end{flushleft}
%% ===== 
%% ===== 
%% ===== \begin{flushleft}
%% ===== 2048K
%% ===== \end{flushleft}
%% ===== 
%% ===== 
%% ===== \begin{flushleft}
%% ===== 4096K
%% ===== \end{flushleft}
%% ===== 
%% ===== 
%% ===== 
%% ===== 
%% ===== 
%% ===== \begin{flushleft}

\subsection{CONTROL UNIT DATA}

%% ===== \end{flushleft}
%% ===== 
%% ===== 
%% ===== \begin{flushleft}
%% ===== Format: - 288 bits, 8 machine words
%% ===== \end{flushleft}
%% ===== 
%% ===== 
%% ===== \begin{flushleft}
%% ===== Data as stored by Store Control Unit (scu) instruction
%% ===== \end{flushleft}
%% ===== 
%% ===== 
%% ===== \begin{flushleft}
%% ===== Word
%% ===== \end{flushleft}
%% ===== 
%% ===== 
%% ===== 0
%% ===== 
%% ===== 
%% ===== 0
%% ===== 
%% ===== 
%% ===== 0
%% ===== 
%% ===== 
%% ===== 
%% ===== 
%% ===== 
%% ===== 0 0
%% ===== 
%% ===== 
%% ===== 2 3
%% ===== 
%% ===== 
%% ===== 
%% ===== 
%% ===== 
%% ===== \begin{flushleft}
%% ===== PRR
%% ===== \end{flushleft}
%% ===== 
%% ===== 
%% ===== 3
%% ===== 
%% ===== 
%% ===== 
%% ===== 
%% ===== 
%% ===== 1 1 1 2 2 2 2 2 2 2 2 2 2 3 3 3 3
%% ===== 
%% ===== 
%% ===== 7 8 9 0 1 2 3 4 5 6 7 8 9 0 1 2 3
%% ===== 
%% ===== 
%% ===== \begin{flushleft}
%% ===== PSR
%% ===== \end{flushleft}
%% ===== 
%% ===== 
%% ===== 
%% ===== 
%% ===== 
%% ===== \begin{flushleft}
%% ===== a b c d e f g h i
%% ===== \end{flushleft}
%% ===== 
%% ===== 
%% ===== 
%% ===== 
%% ===== 
%% ===== \begin{flushleft}
%% ===== a b c d e f g h i
%% ===== \end{flushleft}
%% ===== 
%% ===== 
%% ===== 
%% ===== 
%% ===== 
%% ===== \begin{flushleft}
%% ===== FCT
%% ===== \end{flushleft}
%% ===== 
%% ===== 
%% ===== 
%% ===== 
%% ===== 
%% ===== 15 1 1 1 1 1 1 1 1 1 1 1 1 1 1 1
%% ===== 
%% ===== 
%% ===== 
%% ===== 
%% ===== 
%% ===== 0 0 0 0 0 0 0 0 0 0 1 1 1 1 1 1 1 1 1 1 2
%% ===== 
%% ===== 
%% ===== 0 1 2 3 4 5 6 7 8 9 0 1 2 3 4 5 6 7 8 9 0
%% ===== 
%% ===== 
%% ===== 1
%% ===== 
%% ===== 
%% ===== 
%% ===== 
%% ===== 
%% ===== \begin{flushleft}
%% ===== j k l m n o
%% ===== \end{flushleft}
%% ===== 
%% ===== 
%% ===== 
%% ===== 
%% ===== 
%% ===== 3
%% ===== 
%% ===== 
%% ===== 5
%% ===== 
%% ===== 
%% ===== 
%% ===== 
%% ===== 
%% ===== \begin{flushleft}
%% ===== j k l m n o p q r s t
%% ===== \end{flushleft}
%% ===== 
%% ===== 
%% ===== 
%% ===== 
%% ===== 
%% ===== 1 1 1 1 1 1 1 1 1 1 1 1 1 1 1 1 1 1 1 1
%% ===== 
%% ===== 
%% ===== 
%% ===== 
%% ===== 
%% ===== 2 2
%% ===== 
%% ===== 
%% ===== 3 4
%% ===== 
%% ===== 
%% ===== \begin{flushleft}
%% ===== IA
%% ===== \end{flushleft}
%% ===== 
%% ===== 
%% ===== 
%% ===== 
%% ===== 
%% ===== 2 2
%% ===== 
%% ===== 
%% ===== 6 7
%% ===== 
%% ===== 
%% ===== 
%% ===== 
%% ===== 
%% ===== 2 3
%% ===== 
%% ===== 
%% ===== 9 0
%% ===== 
%% ===== 
%% ===== 
%% ===== 
%% ===== 
%% ===== \begin{flushleft}
%% ===== IACHN CNCHN
%% ===== \end{flushleft}
%% ===== 
%% ===== 
%% ===== 4
%% ===== 
%% ===== 
%% ===== 
%% ===== 
%% ===== 
%% ===== 3
%% ===== 
%% ===== 
%% ===== 
%% ===== 
%% ===== 
%% ===== 3
%% ===== 
%% ===== 
%% ===== 
%% ===== 
%% ===== 
%% ===== 3
%% ===== 
%% ===== 
%% ===== 3 3
%% ===== 
%% ===== 
%% ===== 4 5
%% ===== 
%% ===== 
%% ===== 
%% ===== 
%% ===== 
%% ===== \begin{flushleft}
%% ===== F/I ADDR
%% ===== \end{flushleft}
%% ===== 
%% ===== 
%% ===== 
%% ===== 
%% ===== 
%% ===== \begin{flushleft}
%% ===== u
%% ===== \end{flushleft}
%% ===== 
%% ===== 
%% ===== 5 1
%% ===== 
%% ===== 
%% ===== 
%% ===== 
%% ===== 
%% ===== \newpage
%% ===== 0
%% ===== 
%% ===== 
%% ===== 0
%% ===== 
%% ===== 
%% ===== 2
%% ===== 
%% ===== 
%% ===== 
%% ===== 
%% ===== 
%% ===== 0 0
%% ===== 
%% ===== 
%% ===== 2 3
%% ===== 
%% ===== 
%% ===== 
%% ===== 
%% ===== 
%% ===== 1 1 1 2 2 2 2 2 2 2 2
%% ===== 
%% ===== 
%% ===== 7 8 9 0 1 2 3 4 5 6 7
%% ===== 
%% ===== 
%% ===== 
%% ===== 
%% ===== 
%% ===== \begin{flushleft}
%% ===== TRR
%% ===== \end{flushleft}
%% ===== 
%% ===== 
%% ===== 
%% ===== 
%% ===== 
%% ===== \begin{flushleft}
%% ===== TSR
%% ===== \end{flushleft}
%% ===== 
%% ===== 
%% ===== 
%% ===== 
%% ===== 
%% ===== 3
%% ===== 
%% ===== 
%% ===== 0
%% ===== 
%% ===== 
%% ===== 0
%% ===== 
%% ===== 
%% ===== 3
%% ===== 
%% ===== 
%% ===== 
%% ===== 
%% ===== 
%% ===== \begin{flushleft}
%% ===== PTW
%% ===== \end{flushleft}
%% ===== 
%% ===== 
%% ===== 
%% ===== 
%% ===== 
%% ===== 0
%% ===== 
%% ===== 
%% ===== \begin{flushleft}
%% ===== a b c d e f g h
%% ===== \end{flushleft}
%% ===== 
%% ===== 
%% ===== 15
%% ===== 
%% ===== 
%% ===== 4
%% ===== 
%% ===== 
%% ===== 4 1
%% ===== 
%% ===== 
%% ===== 1 1
%% ===== 
%% ===== 
%% ===== 7 8
%% ===== 
%% ===== 
%% ===== \begin{flushleft}
%% ===== a
%% ===== \end{flushleft}
%% ===== 
%% ===== 
%% ===== 
%% ===== 
%% ===== 
%% ===== 18
%% ===== 
%% ===== 
%% ===== 
%% ===== 
%% ===== 
%% ===== 4
%% ===== 
%% ===== 
%% ===== 
%% ===== 
%% ===== 
%% ===== 2 2 2
%% ===== 
%% ===== 
%% ===== 0 1 2
%% ===== 
%% ===== 
%% ===== \begin{flushleft}
%% ===== TSNA
%% ===== \end{flushleft}
%% ===== 
%% ===== 
%% ===== 
%% ===== 
%% ===== 
%% ===== 0 0 0 0 0 0 0 0 0 0 0 0 0 0 0 0 0 0
%% ===== 
%% ===== 
%% ===== 
%% ===== 
%% ===== 
%% ===== 0
%% ===== 
%% ===== 
%% ===== 0
%% ===== 
%% ===== 
%% ===== 
%% ===== 
%% ===== 
%% ===== \begin{flushleft}
%% ===== SDW
%% ===== \end{flushleft}
%% ===== 
%% ===== 
%% ===== 
%% ===== 
%% ===== 
%% ===== \begin{flushleft}
%% ===== b
%% ===== \end{flushleft}
%% ===== 
%% ===== 
%% ===== 3 1
%% ===== 
%% ===== 
%% ===== 
%% ===== 
%% ===== 
%% ===== \begin{flushleft}
%% ===== CPU
%% ===== \end{flushleft}
%% ===== 
%% ===== 
%% ===== 
%% ===== 
%% ===== 
%% ===== 2 2 2
%% ===== 
%% ===== 
%% ===== 4 5 6
%% ===== 
%% ===== 
%% ===== \begin{flushleft}
%% ===== TSNB
%% ===== \end{flushleft}
%% ===== 
%% ===== 
%% ===== \begin{flushleft}
%% ===== a
%% ===== \end{flushleft}
%% ===== 
%% ===== 
%% ===== 
%% ===== 
%% ===== 
%% ===== \begin{flushleft}
%% ===== b
%% ===== \end{flushleft}
%% ===== 
%% ===== 
%% ===== 3 1
%% ===== 
%% ===== 
%% ===== 
%% ===== 
%% ===== 
%% ===== 2 3
%% ===== 
%% ===== 
%% ===== 9 0
%% ===== 
%% ===== 
%% ===== \begin{flushleft}
%% ===== DELTA
%% ===== \end{flushleft}
%% ===== 
%% ===== 
%% ===== 
%% ===== 
%% ===== 
%% ===== 3
%% ===== 
%% ===== 
%% ===== 
%% ===== 
%% ===== 
%% ===== 6
%% ===== 
%% ===== 
%% ===== 
%% ===== 
%% ===== 
%% ===== 2 2 3
%% ===== 
%% ===== 
%% ===== 8 9 0
%% ===== 
%% ===== 
%% ===== 
%% ===== 
%% ===== 
%% ===== 3
%% ===== 
%% ===== 
%% ===== 5
%% ===== 
%% ===== 
%% ===== 
%% ===== 
%% ===== 
%% ===== \begin{flushleft}
%% ===== TSNC
%% ===== \end{flushleft}
%% ===== 
%% ===== 
%% ===== \begin{flushleft}
%% ===== a
%% ===== \end{flushleft}
%% ===== 
%% ===== 
%% ===== 
%% ===== 
%% ===== 
%% ===== 3
%% ===== 
%% ===== 
%% ===== 5
%% ===== 
%% ===== 
%% ===== 
%% ===== 
%% ===== 
%% ===== \begin{flushleft}
%% ===== b
%% ===== \end{flushleft}
%% ===== 
%% ===== 
%% ===== 3 1
%% ===== 
%% ===== 
%% ===== 
%% ===== 
%% ===== 
%% ===== \begin{flushleft}
%% ===== TEMP BIT
%% ===== \end{flushleft}
%% ===== 
%% ===== 
%% ===== 6
%% ===== 
%% ===== 
%% ===== 
%% ===== 
%% ===== 
%% ===== 1 1 1 2 2 2 2 2 2 2 2 2 2 3 3 3
%% ===== 
%% ===== 
%% ===== 7 8 9 0 1 2 3 4 5 6 7 8 9 0 1 2
%% ===== 
%% ===== 
%% ===== \begin{flushleft}
%% ===== IC
%% ===== \end{flushleft}
%% ===== 
%% ===== 
%% ===== 
%% ===== 
%% ===== 
%% ===== \begin{flushleft}
%% ===== a b c d e f g h i
%% ===== \end{flushleft}
%% ===== 
%% ===== 
%% ===== 
%% ===== 
%% ===== 
%% ===== \begin{flushleft}
%% ===== j k l m n 0 0 0 0
%% ===== \end{flushleft}
%% ===== 
%% ===== 
%% ===== 
%% ===== 
%% ===== 
%% ===== 18 1 1 1 1 1 1 1 1 1 1 1 1 1 1
%% ===== 
%% ===== 
%% ===== 0
%% ===== 
%% ===== 
%% ===== 0
%% ===== 
%% ===== 
%% ===== 5
%% ===== 
%% ===== 
%% ===== 
%% ===== 
%% ===== 
%% ===== 4
%% ===== 
%% ===== 
%% ===== 
%% ===== 
%% ===== 
%% ===== 1 1 1 2 2 2 2 2 2 2 2 2 2 3
%% ===== 
%% ===== 
%% ===== 7 8 9 0 1 2 3 4 5 6 7 8 9 0
%% ===== 
%% ===== 
%% ===== \begin{flushleft}
%% ===== COMPUTED ADDRESS
%% ===== \end{flushleft}
%% ===== 
%% ===== 
%% ===== 
%% ===== 
%% ===== 
%% ===== \begin{flushleft}
%% ===== a b c d e f g h i
%% ===== \end{flushleft}
%% ===== 
%% ===== 
%% ===== 
%% ===== 
%% ===== 
%% ===== \begin{flushleft}
%% ===== j k l
%% ===== \end{flushleft}
%% ===== 
%% ===== 
%% ===== 
%% ===== 
%% ===== 
%% ===== 3
%% ===== 
%% ===== 
%% ===== 5
%% ===== 
%% ===== 
%% ===== \begin{flushleft}
%% ===== CT HOLD
%% ===== \end{flushleft}
%% ===== 
%% ===== 
%% ===== 
%% ===== 
%% ===== 
%% ===== 18 1 1 1 1 1 1 1 1 1 1 1 1
%% ===== 
%% ===== 
%% ===== 0
%% ===== 
%% ===== 
%% ===== 0
%% ===== 
%% ===== 
%% ===== 6
%% ===== 
%% ===== 
%% ===== 
%% ===== 
%% ===== 
%% ===== 1 1
%% ===== 
%% ===== 
%% ===== 7 8
%% ===== 
%% ===== 
%% ===== \begin{flushleft}
%% ===== ADDRESS
%% ===== \end{flushleft}
%% ===== 
%% ===== 
%% ===== 18
%% ===== 
%% ===== 
%% ===== 
%% ===== 
%% ===== 
%% ===== 0
%% ===== 
%% ===== 
%% ===== 0
%% ===== 
%% ===== 
%% ===== 7
%% ===== 
%% ===== 
%% ===== 
%% ===== 
%% ===== 
%% ===== \begin{flushleft}
%% ===== I P
%% ===== \end{flushleft}
%% ===== 
%% ===== 
%% ===== 
%% ===== 
%% ===== 
%% ===== 3
%% ===== 
%% ===== 
%% ===== 5
%% ===== 
%% ===== 
%% ===== \begin{flushleft}
%% ===== TAG
%% ===== \end{flushleft}
%% ===== 
%% ===== 
%% ===== 
%% ===== 
%% ===== 
%% ===== 10 1 1
%% ===== 
%% ===== 
%% ===== 
%% ===== 
%% ===== 
%% ===== 1 1
%% ===== 
%% ===== 
%% ===== 7 8
%% ===== 
%% ===== 
%% ===== \begin{flushleft}
%% ===== ADDRESS
%% ===== \end{flushleft}
%% ===== 
%% ===== 
%% ===== 
%% ===== 
%% ===== 
%% ===== 6
%% ===== 
%% ===== 
%% ===== 
%% ===== 
%% ===== 
%% ===== 2 2 2 3
%% ===== 
%% ===== 
%% ===== 7 8 9 0
%% ===== 
%% ===== 
%% ===== \begin{flushleft}
%% ===== OPCODE
%% ===== \end{flushleft}
%% ===== 
%% ===== 
%% ===== 
%% ===== 
%% ===== 
%% ===== 6
%% ===== 
%% ===== 
%% ===== 
%% ===== 
%% ===== 
%% ===== 2 2 2 3
%% ===== 
%% ===== 
%% ===== 7 8 9 0
%% ===== 
%% ===== 
%% ===== \begin{flushleft}
%% ===== OPCODE
%% ===== \end{flushleft}
%% ===== 
%% ===== 
%% ===== 
%% ===== 
%% ===== 
%% ===== 18
%% ===== 
%% ===== 
%% ===== 
%% ===== 
%% ===== 
%% ===== 3
%% ===== 
%% ===== 
%% ===== 5
%% ===== 
%% ===== 
%% ===== 
%% ===== 
%% ===== 
%% ===== \begin{flushleft}
%% ===== I P
%% ===== \end{flushleft}
%% ===== 
%% ===== 
%% ===== 10 1 1
%% ===== 
%% ===== 
%% ===== 
%% ===== 
%% ===== 
%% ===== 3
%% ===== 
%% ===== 
%% ===== 5
%% ===== 
%% ===== 
%% ===== \begin{flushleft}
%% ===== TAG
%% ===== \end{flushleft}
%% ===== 
%% ===== 
%% ===== 6
%% ===== 
%% ===== 
%% ===== 
%% ===== 
%% ===== 
%% ===== \begin{flushleft}
%% ===== Figure 3-32. Control Unit Data Format
%% ===== \end{flushleft}
%% ===== 
%% ===== 
%% ===== \begin{flushleft}
%% ===== Description:
%% ===== \end{flushleft}
%% ===== 
%% ===== 
%% ===== \begin{flushleft}
%% ===== A collection of flags and registers from the appending unit and the control unit. In general,
%% ===== \end{flushleft}
%% ===== 
%% ===== 
%% ===== \begin{flushleft}
%% ===== the data has valid meaning only when stored with the Store Control Unit (scu) instruction
%% ===== \end{flushleft}
%% ===== 
%% ===== 
%% ===== \begin{flushleft}
%% ===== as the first instruction of a fault or interrupt trap pair.
%% ===== \end{flushleft}
%% ===== 
%% ===== 
%% ===== \begin{flushleft}
%% ===== Function:
%% ===== \end{flushleft}
%% ===== 
%% ===== 
%% ===== \begin{flushleft}
%% ===== The control unit data allows the processor to restart an instruction at the point of
%% ===== \end{flushleft}
%% ===== 
%% ===== 
%% ===== \begin{flushleft}
%% ===== interruption when it is interrupted by an access violation fault, a directed fault, or (for
%% ===== \end{flushleft}
%% ===== 
%% ===== 
%% ===== \begin{flushleft}
%% ===== certain EIS instructions) an interrupt. Directed faults are intentional, and most access
%% ===== \end{flushleft}
%% ===== 
%% ===== 
%% ===== \begin{flushleft}
%% ===== violation faults and interrupts are recoverable. If the interruption is not recoverable, the
%% ===== \end{flushleft}
%% ===== 
%% ===== 
%% ===== \begin{flushleft}
%% ===== control unit data provides enough information to determine the exact nature of the error.
%% ===== \end{flushleft}
%% ===== 
%% ===== 
%% ===== \begin{flushleft}
%% ===== Instruction execution restarts immediately upon execution of a Restore Control Unit (rcu)
%% ===== \end{flushleft}
%% ===== 
%% ===== 
%% ===== \begin{flushleft}
%% ===== instruction referencing the Y-block8 area into which the control unit data was stored.
%% ===== \end{flushleft}
%% ===== 
%% ===== 
%% ===== 
%% ===== 
%% ===== 
%% ===== \begin{flushleft}
%% ===== \newpage
%% ===== Fields having an {``}x'' in the column headed L are not restored by the Restore Control Unit
%% ===== \end{flushleft}
%% ===== 
%% ===== 
%% ===== \begin{flushleft}
%% ===== (rcu) instruction.
%% ===== \end{flushleft}
%% ===== 
%% ===== 
%% ===== \begin{flushleft}
%% ===== The meanings of the constituent fields are:
%% ===== \end{flushleft}
%% ===== 
%% ===== 
%% ===== 
%% ===== 
%% ===== 
%% ===== \begin{flushleft}
%% ===== Word
%% ===== \end{flushleft}
%% ===== 
%% ===== 
%% ===== 
%% ===== 
%% ===== 
%% ===== \begin{flushleft}
%% ===== key L Field Name
%% ===== \end{flushleft}
%% ===== 
%% ===== 
%% ===== 
%% ===== 
%% ===== 
%% ===== \begin{flushleft}
%% ===== Meaning
%% ===== \end{flushleft}
%% ===== 
%% ===== 
%% ===== 
%% ===== 
%% ===== 
%% ===== 0
%% ===== 
%% ===== 
%% ===== 
%% ===== 
%% ===== 
%% ===== \begin{flushleft}
%% ===== PRR
%% ===== \end{flushleft}
%% ===== 
%% ===== 
%% ===== 
%% ===== 
%% ===== 
%% ===== \begin{flushleft}
%% ===== Procedure ring register (PPR.PRR)
%% ===== \end{flushleft}
%% ===== 
%% ===== 
%% ===== 
%% ===== 
%% ===== 
%% ===== 0
%% ===== 
%% ===== 
%% ===== 
%% ===== 
%% ===== 
%% ===== \begin{flushleft}
%% ===== PSR
%% ===== \end{flushleft}
%% ===== 
%% ===== 
%% ===== 
%% ===== 
%% ===== 
%% ===== \begin{flushleft}
%% ===== Procedure segment register (PPR.PSR)
%% ===== \end{flushleft}
%% ===== 
%% ===== 
%% ===== 
%% ===== 
%% ===== 
%% ===== 0
%% ===== 
%% ===== 
%% ===== 
%% ===== 
%% ===== 
%% ===== \begin{flushleft}
%% ===== a
%% ===== \end{flushleft}
%% ===== 
%% ===== 
%% ===== 
%% ===== 
%% ===== 
%% ===== \begin{flushleft}
%% ===== P
%% ===== \end{flushleft}
%% ===== 
%% ===== 
%% ===== 
%% ===== 
%% ===== 
%% ===== \begin{flushleft}
%% ===== Privileged bit (PPR.P)
%% ===== \end{flushleft}
%% ===== 
%% ===== 
%% ===== 
%% ===== 
%% ===== 
%% ===== 0
%% ===== 
%% ===== 
%% ===== 
%% ===== 
%% ===== 
%% ===== \begin{flushleft}
%% ===== b
%% ===== \end{flushleft}
%% ===== 
%% ===== 
%% ===== 
%% ===== 
%% ===== 
%% ===== \begin{flushleft}
%% ===== XSF
%% ===== \end{flushleft}
%% ===== 
%% ===== 
%% ===== 
%% ===== 
%% ===== 
%% ===== \begin{flushleft}
%% ===== External segment flag
%% ===== \end{flushleft}
%% ===== 
%% ===== 
%% ===== 
%% ===== 
%% ===== 
%% ===== 0
%% ===== 
%% ===== 
%% ===== 
%% ===== 
%% ===== 
%% ===== \begin{flushleft}
%% ===== c
%% ===== \end{flushleft}
%% ===== 
%% ===== 
%% ===== 
%% ===== 
%% ===== 
%% ===== \begin{flushleft}
%% ===== x SDWAMM
%% ===== \end{flushleft}
%% ===== 
%% ===== 
%% ===== 
%% ===== 
%% ===== 
%% ===== \begin{flushleft}
%% ===== Match on SDWAM
%% ===== \end{flushleft}
%% ===== 
%% ===== 
%% ===== 
%% ===== 
%% ===== 
%% ===== 0
%% ===== 
%% ===== 
%% ===== 
%% ===== 
%% ===== 
%% ===== \begin{flushleft}
%% ===== d
%% ===== \end{flushleft}
%% ===== 
%% ===== 
%% ===== 
%% ===== 
%% ===== 
%% ===== \begin{flushleft}
%% ===== x SD-ON
%% ===== \end{flushleft}
%% ===== 
%% ===== 
%% ===== 
%% ===== 
%% ===== 
%% ===== \begin{flushleft}
%% ===== SDWAM enabled
%% ===== \end{flushleft}
%% ===== 
%% ===== 
%% ===== 
%% ===== 
%% ===== 
%% ===== 0
%% ===== 
%% ===== 
%% ===== 
%% ===== 
%% ===== 
%% ===== \begin{flushleft}
%% ===== e
%% ===== \end{flushleft}
%% ===== 
%% ===== 
%% ===== 
%% ===== 
%% ===== 
%% ===== \begin{flushleft}
%% ===== x PTWAMM
%% ===== \end{flushleft}
%% ===== 
%% ===== 
%% ===== 
%% ===== 
%% ===== 
%% ===== \begin{flushleft}
%% ===== Match on PTWAM
%% ===== \end{flushleft}
%% ===== 
%% ===== 
%% ===== 
%% ===== 
%% ===== 
%% ===== 0
%% ===== 
%% ===== 
%% ===== 
%% ===== 
%% ===== 
%% ===== \begin{flushleft}
%% ===== f
%% ===== \end{flushleft}
%% ===== 
%% ===== 
%% ===== 
%% ===== 
%% ===== 
%% ===== \begin{flushleft}
%% ===== x PT-ON
%% ===== \end{flushleft}
%% ===== 
%% ===== 
%% ===== 
%% ===== 
%% ===== 
%% ===== \begin{flushleft}
%% ===== PTWAM enabled
%% ===== \end{flushleft}
%% ===== 
%% ===== 
%% ===== 
%% ===== 
%% ===== 
%% ===== 0
%% ===== 
%% ===== 
%% ===== 
%% ===== 
%% ===== 
%% ===== \begin{flushleft}
%% ===== g
%% ===== \end{flushleft}
%% ===== 
%% ===== 
%% ===== 
%% ===== 
%% ===== 
%% ===== \begin{flushleft}
%% ===== x PI-AP
%% ===== \end{flushleft}
%% ===== 
%% ===== 
%% ===== 
%% ===== 
%% ===== 
%% ===== \begin{flushleft}
%% ===== Instruction fetch append cycle
%% ===== \end{flushleft}
%% ===== 
%% ===== 
%% ===== 
%% ===== 
%% ===== 
%% ===== 0
%% ===== 
%% ===== 
%% ===== 
%% ===== 
%% ===== 
%% ===== \begin{flushleft}
%% ===== h
%% ===== \end{flushleft}
%% ===== 
%% ===== 
%% ===== 
%% ===== 
%% ===== 
%% ===== \begin{flushleft}
%% ===== x DSPTW
%% ===== \end{flushleft}
%% ===== 
%% ===== 
%% ===== 
%% ===== 
%% ===== 
%% ===== \begin{flushleft}
%% ===== Fetch descriptor segment PTW
%% ===== \end{flushleft}
%% ===== 
%% ===== 
%% ===== 
%% ===== 
%% ===== 
%% ===== 0
%% ===== 
%% ===== 
%% ===== 
%% ===== 
%% ===== 
%% ===== \begin{flushleft}
%% ===== i
%% ===== \end{flushleft}
%% ===== 
%% ===== 
%% ===== 
%% ===== 
%% ===== 
%% ===== \begin{flushleft}
%% ===== x SDWNP
%% ===== \end{flushleft}
%% ===== 
%% ===== 
%% ===== 
%% ===== 
%% ===== 
%% ===== \begin{flushleft}
%% ===== Fetch SDW - nonpaged
%% ===== \end{flushleft}
%% ===== 
%% ===== 
%% ===== 
%% ===== 
%% ===== 
%% ===== 0
%% ===== 
%% ===== 
%% ===== 
%% ===== 
%% ===== 
%% ===== \begin{flushleft}
%% ===== j
%% ===== \end{flushleft}
%% ===== 
%% ===== 
%% ===== 
%% ===== 
%% ===== 
%% ===== \begin{flushleft}
%% ===== x SDWP
%% ===== \end{flushleft}
%% ===== 
%% ===== 
%% ===== 
%% ===== 
%% ===== 
%% ===== \begin{flushleft}
%% ===== Fetch SDW - paged
%% ===== \end{flushleft}
%% ===== 
%% ===== 
%% ===== 
%% ===== 
%% ===== 
%% ===== 0
%% ===== 
%% ===== 
%% ===== 
%% ===== 
%% ===== 
%% ===== \begin{flushleft}
%% ===== k
%% ===== \end{flushleft}
%% ===== 
%% ===== 
%% ===== 
%% ===== 
%% ===== 
%% ===== \begin{flushleft}
%% ===== x PTW
%% ===== \end{flushleft}
%% ===== 
%% ===== 
%% ===== 
%% ===== 
%% ===== 
%% ===== \begin{flushleft}
%% ===== Fetch PTW
%% ===== \end{flushleft}
%% ===== 
%% ===== 
%% ===== 
%% ===== 
%% ===== 
%% ===== 0
%% ===== 
%% ===== 
%% ===== 
%% ===== 
%% ===== 
%% ===== \begin{flushleft}
%% ===== l
%% ===== \end{flushleft}
%% ===== 
%% ===== 
%% ===== 
%% ===== 
%% ===== 
%% ===== \begin{flushleft}
%% ===== x PTW2
%% ===== \end{flushleft}
%% ===== 
%% ===== 
%% ===== 
%% ===== 
%% ===== 
%% ===== \begin{flushleft}
%% ===== Fetch prepage PTW
%% ===== \end{flushleft}
%% ===== 
%% ===== 
%% ===== 
%% ===== 
%% ===== 
%% ===== 0
%% ===== 
%% ===== 
%% ===== 
%% ===== 
%% ===== 
%% ===== \begin{flushleft}
%% ===== m
%% ===== \end{flushleft}
%% ===== 
%% ===== 
%% ===== 
%% ===== 
%% ===== 
%% ===== \begin{flushleft}
%% ===== x FAP
%% ===== \end{flushleft}
%% ===== 
%% ===== 
%% ===== 
%% ===== 
%% ===== 
%% ===== \begin{flushleft}
%% ===== Fetch final address - paged
%% ===== \end{flushleft}
%% ===== 
%% ===== 
%% ===== 
%% ===== 
%% ===== 
%% ===== 0
%% ===== 
%% ===== 
%% ===== 
%% ===== 
%% ===== 
%% ===== \begin{flushleft}
%% ===== n
%% ===== \end{flushleft}
%% ===== 
%% ===== 
%% ===== 
%% ===== 
%% ===== 
%% ===== \begin{flushleft}
%% ===== x FANP
%% ===== \end{flushleft}
%% ===== 
%% ===== 
%% ===== 
%% ===== 
%% ===== 
%% ===== \begin{flushleft}
%% ===== Fetch final address - nonpaged
%% ===== \end{flushleft}
%% ===== 
%% ===== 
%% ===== 
%% ===== 
%% ===== 
%% ===== 0
%% ===== 
%% ===== 
%% ===== 
%% ===== 
%% ===== 
%% ===== \begin{flushleft}
%% ===== o
%% ===== \end{flushleft}
%% ===== 
%% ===== 
%% ===== 
%% ===== 
%% ===== 
%% ===== \begin{flushleft}
%% ===== x FABS
%% ===== \end{flushleft}
%% ===== 
%% ===== 
%% ===== 
%% ===== 
%% ===== 
%% ===== \begin{flushleft}
%% ===== Fetch final address - absolute
%% ===== \end{flushleft}
%% ===== 
%% ===== 
%% ===== 
%% ===== 
%% ===== 
%% ===== 0
%% ===== 
%% ===== 
%% ===== 
%% ===== 
%% ===== 
%% ===== \begin{flushleft}
%% ===== FCT
%% ===== \end{flushleft}
%% ===== 
%% ===== 
%% ===== 
%% ===== 
%% ===== 
%% ===== \begin{flushleft}
%% ===== Fault counter - counts retries
%% ===== \end{flushleft}
%% ===== 
%% ===== 
%% ===== 
%% ===== 
%% ===== 
%% ===== 1
%% ===== 
%% ===== 
%% ===== 
%% ===== 
%% ===== 
%% ===== \begin{flushleft}
%% ===== a
%% ===== \end{flushleft}
%% ===== 
%% ===== 
%% ===== 
%% ===== 
%% ===== 
%% ===== \begin{flushleft}
%% ===== x IRO
%% ===== \end{flushleft}
%% ===== 
%% ===== 
%% ===== \begin{flushleft}
%% ===== x ISN
%% ===== \end{flushleft}
%% ===== 
%% ===== 
%% ===== 
%% ===== 
%% ===== 
%% ===== \begin{flushleft}
%% ===== For access violation fault - illegal ring order
%% ===== \end{flushleft}
%% ===== 
%% ===== 
%% ===== \begin{flushleft}
%% ===== For store fault - illegal segment number
%% ===== \end{flushleft}
%% ===== 
%% ===== 
%% ===== 
%% ===== 
%% ===== 
%% ===== 1
%% ===== 
%% ===== 
%% ===== 
%% ===== 
%% ===== 
%% ===== \begin{flushleft}
%% ===== b
%% ===== \end{flushleft}
%% ===== 
%% ===== 
%% ===== 
%% ===== 
%% ===== 
%% ===== \begin{flushleft}
%% ===== x ORB
%% ===== \end{flushleft}
%% ===== 
%% ===== 
%% ===== \begin{flushleft}
%% ===== x IOC
%% ===== \end{flushleft}
%% ===== 
%% ===== 
%% ===== 
%% ===== 
%% ===== 
%% ===== \begin{flushleft}
%% ===== For access violation fault - out of execute bracket
%% ===== \end{flushleft}
%% ===== 
%% ===== 
%% ===== \begin{flushleft}
%% ===== For illegal procedure fault - illegal op code
%% ===== \end{flushleft}
%% ===== 
%% ===== 
%% ===== 
%% ===== 
%% ===== 
%% ===== 1
%% ===== 
%% ===== 
%% ===== 
%% ===== 
%% ===== 
%% ===== \begin{flushleft}
%% ===== c
%% ===== \end{flushleft}
%% ===== 
%% ===== 
%% ===== 
%% ===== 
%% ===== 
%% ===== \begin{flushleft}
%% ===== x E-OFF
%% ===== \end{flushleft}
%% ===== 
%% ===== 
%% ===== \begin{flushleft}
%% ===== x IA+IM
%% ===== \end{flushleft}
%% ===== 
%% ===== 
%% ===== 
%% ===== 
%% ===== 
%% ===== \begin{flushleft}
%% ===== For access violation fault - execute bit is OFF
%% ===== \end{flushleft}
%% ===== 
%% ===== 
%% ===== \begin{flushleft}
%% ===== For illegal procedure fault - illegal address or modifier
%% ===== \end{flushleft}
%% ===== 
%% ===== 
%% ===== 
%% ===== 
%% ===== 
%% ===== 1
%% ===== 
%% ===== 
%% ===== 
%% ===== 
%% ===== 
%% ===== \begin{flushleft}
%% ===== d
%% ===== \end{flushleft}
%% ===== 
%% ===== 
%% ===== 
%% ===== 
%% ===== 
%% ===== \begin{flushleft}
%% ===== x ORB
%% ===== \end{flushleft}
%% ===== 
%% ===== 
%% ===== \begin{flushleft}
%% ===== x ISP
%% ===== \end{flushleft}
%% ===== 
%% ===== 
%% ===== 
%% ===== 
%% ===== 
%% ===== \begin{flushleft}
%% ===== For access violation fault - out of read bracket
%% ===== \end{flushleft}
%% ===== 
%% ===== 
%% ===== \begin{flushleft}
%% ===== For illegal procedure fault - illegal slave procedure
%% ===== \end{flushleft}
%% ===== 
%% ===== 
%% ===== 
%% ===== 
%% ===== 
%% ===== 1
%% ===== 
%% ===== 
%% ===== 
%% ===== 
%% ===== 
%% ===== \begin{flushleft}
%% ===== e
%% ===== \end{flushleft}
%% ===== 
%% ===== 
%% ===== 
%% ===== 
%% ===== 
%% ===== \begin{flushleft}
%% ===== x R-OFF
%% ===== \end{flushleft}
%% ===== 
%% ===== 
%% ===== \begin{flushleft}
%% ===== x IPR
%% ===== \end{flushleft}
%% ===== 
%% ===== 
%% ===== 
%% ===== 
%% ===== 
%% ===== \begin{flushleft}
%% ===== For access violation fault - read bit is OFF
%% ===== \end{flushleft}
%% ===== 
%% ===== 
%% ===== \begin{flushleft}
%% ===== For illegal procedure fault - illegal EIS digit
%% ===== \end{flushleft}
%% ===== 
%% ===== 
%% ===== 
%% ===== 
%% ===== 
%% ===== 1
%% ===== 
%% ===== 
%% ===== 
%% ===== 
%% ===== 
%% ===== \begin{flushleft}
%% ===== f
%% ===== \end{flushleft}
%% ===== 
%% ===== 
%% ===== 
%% ===== 
%% ===== 
%% ===== \begin{flushleft}
%% ===== x OWB
%% ===== \end{flushleft}
%% ===== 
%% ===== 
%% ===== \begin{flushleft}
%% ===== x NEA
%% ===== \end{flushleft}
%% ===== 
%% ===== 
%% ===== 
%% ===== 
%% ===== 
%% ===== \begin{flushleft}
%% ===== For access violation fault - out of write bracket
%% ===== \end{flushleft}
%% ===== 
%% ===== 
%% ===== \begin{flushleft}
%% ===== For store fault - nonexistent address
%% ===== \end{flushleft}
%% ===== 
%% ===== 
%% ===== 
%% ===== 
%% ===== 
%% ===== 1
%% ===== 
%% ===== 
%% ===== 
%% ===== 
%% ===== 
%% ===== \begin{flushleft}
%% ===== g
%% ===== \end{flushleft}
%% ===== 
%% ===== 
%% ===== 
%% ===== 
%% ===== 
%% ===== \begin{flushleft}
%% ===== x W-OFF
%% ===== \end{flushleft}
%% ===== 
%% ===== 
%% ===== \begin{flushleft}
%% ===== x OOB
%% ===== \end{flushleft}
%% ===== 
%% ===== 
%% ===== 
%% ===== 
%% ===== 
%% ===== \begin{flushleft}
%% ===== For access violation fault - write bit is OFF
%% ===== \end{flushleft}
%% ===== 
%% ===== 
%% ===== \begin{flushleft}
%% ===== For store fault - out of bounds (BAR mode)
%% ===== \end{flushleft}
%% ===== 
%% ===== 
%% ===== 
%% ===== 
%% ===== 
%% ===== 1
%% ===== 
%% ===== 
%% ===== 
%% ===== 
%% ===== 
%% ===== \begin{flushleft}
%% ===== h
%% ===== \end{flushleft}
%% ===== 
%% ===== 
%% ===== 
%% ===== 
%% ===== 
%% ===== \begin{flushleft}
%% ===== x NO GA
%% ===== \end{flushleft}
%% ===== 
%% ===== 
%% ===== 
%% ===== 
%% ===== 
%% ===== \begin{flushleft}
%% ===== For access violation fault - not a gate
%% ===== \end{flushleft}
%% ===== 
%% ===== 
%% ===== 
%% ===== 
%% ===== 
%% ===== 1
%% ===== 
%% ===== 
%% ===== 
%% ===== 
%% ===== 
%% ===== \begin{flushleft}
%% ===== i
%% ===== \end{flushleft}
%% ===== 
%% ===== 
%% ===== 
%% ===== 
%% ===== 
%% ===== \begin{flushleft}
%% ===== x OCB
%% ===== \end{flushleft}
%% ===== 
%% ===== 
%% ===== 
%% ===== 
%% ===== 
%% ===== \begin{flushleft}
%% ===== For access violation fault - out of call bracket
%% ===== \end{flushleft}
%% ===== 
%% ===== 
%% ===== 
%% ===== 
%% ===== 
%% ===== 1
%% ===== 
%% ===== 
%% ===== 
%% ===== 
%% ===== 
%% ===== \begin{flushleft}
%% ===== j
%% ===== \end{flushleft}
%% ===== 
%% ===== 
%% ===== 
%% ===== 
%% ===== 
%% ===== \begin{flushleft}
%% ===== x OCALL
%% ===== \end{flushleft}
%% ===== 
%% ===== 
%% ===== 
%% ===== 
%% ===== 
%% ===== \begin{flushleft}
%% ===== For access violation fault - outward call
%% ===== \end{flushleft}
%% ===== 
%% ===== 
%% ===== 
%% ===== 
%% ===== 
%% ===== 1
%% ===== 
%% ===== 
%% ===== 
%% ===== 
%% ===== 
%% ===== \begin{flushleft}
%% ===== k
%% ===== \end{flushleft}
%% ===== 
%% ===== 
%% ===== 
%% ===== 
%% ===== 
%% ===== \begin{flushleft}
%% ===== x BOC
%% ===== \end{flushleft}
%% ===== 
%% ===== 
%% ===== 
%% ===== 
%% ===== 
%% ===== \begin{flushleft}
%% ===== For access violation fault - bad outward call
%% ===== \end{flushleft}
%% ===== 
%% ===== 
%% ===== 
%% ===== 
%% ===== 
%% ===== 1
%% ===== 
%% ===== 
%% ===== 
%% ===== 
%% ===== 
%% ===== \begin{flushleft}
%% ===== l
%% ===== \end{flushleft}
%% ===== 
%% ===== 
%% ===== 
%% ===== 
%% ===== 
%% ===== \begin{flushleft}
%% ===== x PTWAM\_ER
%% ===== \end{flushleft}
%% ===== 
%% ===== 
%% ===== 
%% ===== 
%% ===== 
%% ===== \begin{flushleft}
%% ===== For access violation fault - on DPS 8M processors, a PTW
%% ===== \end{flushleft}
%% ===== 
%% ===== 
%% ===== \begin{flushleft}
%% ===== associative memory error. Not used on DPS/L68
%% ===== \end{flushleft}
%% ===== 
%% ===== 
%% ===== \begin{flushleft}
%% ===== processors.
%% ===== \end{flushleft}
%% ===== 
%% ===== 
%% ===== 
%% ===== 
%% ===== 
%% ===== \begin{flushleft}
%% ===== \newpage
%% ===== Word
%% ===== \end{flushleft}
%% ===== 
%% ===== 
%% ===== 
%% ===== 
%% ===== 
%% ===== \begin{flushleft}
%% ===== key L Field Name
%% ===== \end{flushleft}
%% ===== 
%% ===== 
%% ===== 
%% ===== 
%% ===== 
%% ===== \begin{flushleft}
%% ===== Meaning
%% ===== \end{flushleft}
%% ===== 
%% ===== 
%% ===== 
%% ===== 
%% ===== 
%% ===== 1
%% ===== 
%% ===== 
%% ===== 
%% ===== 
%% ===== 
%% ===== \begin{flushleft}
%% ===== m
%% ===== \end{flushleft}
%% ===== 
%% ===== 
%% ===== 
%% ===== 
%% ===== 
%% ===== \begin{flushleft}
%% ===== x CRT
%% ===== \end{flushleft}
%% ===== 
%% ===== 
%% ===== 
%% ===== 
%% ===== 
%% ===== \begin{flushleft}
%% ===== For access violation fault - cross ring transfer
%% ===== \end{flushleft}
%% ===== 
%% ===== 
%% ===== 
%% ===== 
%% ===== 
%% ===== 1
%% ===== 
%% ===== 
%% ===== 
%% ===== 
%% ===== 
%% ===== \begin{flushleft}
%% ===== n
%% ===== \end{flushleft}
%% ===== 
%% ===== 
%% ===== 
%% ===== 
%% ===== 
%% ===== \begin{flushleft}
%% ===== x RALR
%% ===== \end{flushleft}
%% ===== 
%% ===== 
%% ===== 
%% ===== 
%% ===== 
%% ===== \begin{flushleft}
%% ===== For access violation fault - ring alarm
%% ===== \end{flushleft}
%% ===== 
%% ===== 
%% ===== 
%% ===== 
%% ===== 
%% ===== 1
%% ===== 
%% ===== 
%% ===== 
%% ===== 
%% ===== 
%% ===== \begin{flushleft}
%% ===== o
%% ===== \end{flushleft}
%% ===== 
%% ===== 
%% ===== 
%% ===== 
%% ===== 
%% ===== \begin{flushleft}
%% ===== x SDWAM\_ER
%% ===== \end{flushleft}
%% ===== 
%% ===== 
%% ===== 
%% ===== 
%% ===== 
%% ===== \begin{flushleft}
%% ===== For access violation fault - on DPS 8M an SDW associative
%% ===== \end{flushleft}
%% ===== 
%% ===== 
%% ===== \begin{flushleft}
%% ===== memory error. An associative memory error on DPS/L68.
%% ===== \end{flushleft}
%% ===== 
%% ===== 
%% ===== 
%% ===== 
%% ===== 
%% ===== 1
%% ===== 
%% ===== 
%% ===== 
%% ===== 
%% ===== 
%% ===== \begin{flushleft}
%% ===== p
%% ===== \end{flushleft}
%% ===== 
%% ===== 
%% ===== 
%% ===== 
%% ===== 
%% ===== \begin{flushleft}
%% ===== x OOSB
%% ===== \end{flushleft}
%% ===== 
%% ===== 
%% ===== 
%% ===== 
%% ===== 
%% ===== \begin{flushleft}
%% ===== For access violation fault - out of segment bounds
%% ===== \end{flushleft}
%% ===== 
%% ===== 
%% ===== 
%% ===== 
%% ===== 
%% ===== 1
%% ===== 
%% ===== 
%% ===== 
%% ===== 
%% ===== 
%% ===== \begin{flushleft}
%% ===== q
%% ===== \end{flushleft}
%% ===== 
%% ===== 
%% ===== 
%% ===== 
%% ===== 
%% ===== \begin{flushleft}
%% ===== x PARU
%% ===== \end{flushleft}
%% ===== 
%% ===== 
%% ===== 
%% ===== 
%% ===== 
%% ===== \begin{flushleft}
%% ===== For parity fault - processor parity upper
%% ===== \end{flushleft}
%% ===== 
%% ===== 
%% ===== 
%% ===== 
%% ===== 
%% ===== 1
%% ===== 
%% ===== 
%% ===== 
%% ===== 
%% ===== 
%% ===== \begin{flushleft}
%% ===== r
%% ===== \end{flushleft}
%% ===== 
%% ===== 
%% ===== 
%% ===== 
%% ===== 
%% ===== \begin{flushleft}
%% ===== x PARL
%% ===== \end{flushleft}
%% ===== 
%% ===== 
%% ===== 
%% ===== 
%% ===== 
%% ===== \begin{flushleft}
%% ===== For parity fault - processor parity lower
%% ===== \end{flushleft}
%% ===== 
%% ===== 
%% ===== 
%% ===== 
%% ===== 
%% ===== 1
%% ===== 
%% ===== 
%% ===== 
%% ===== 
%% ===== 
%% ===== \begin{flushleft}
%% ===== s
%% ===== \end{flushleft}
%% ===== 
%% ===== 
%% ===== 
%% ===== 
%% ===== 
%% ===== \begin{flushleft}
%% ===== x ONC1
%% ===== \end{flushleft}
%% ===== 
%% ===== 
%% ===== 
%% ===== 
%% ===== 
%% ===== \begin{flushleft}
%% ===== For operation not complete fault -- processor/system
%% ===== \end{flushleft}
%% ===== 
%% ===== 
%% ===== \begin{flushleft}
%% ===== controller sequence error \#1
%% ===== \end{flushleft}
%% ===== 
%% ===== 
%% ===== 
%% ===== 
%% ===== 
%% ===== 1
%% ===== 
%% ===== 
%% ===== 
%% ===== 
%% ===== 
%% ===== \begin{flushleft}
%% ===== t
%% ===== \end{flushleft}
%% ===== 
%% ===== 
%% ===== 
%% ===== 
%% ===== 
%% ===== \begin{flushleft}
%% ===== x ONC2
%% ===== \end{flushleft}
%% ===== 
%% ===== 
%% ===== 
%% ===== 
%% ===== 
%% ===== \begin{flushleft}
%% ===== For operation not complete fault -- processor/system
%% ===== \end{flushleft}
%% ===== 
%% ===== 
%% ===== \begin{flushleft}
%% ===== controller sequence error \#2
%% ===== \end{flushleft}
%% ===== 
%% ===== 
%% ===== 
%% ===== 
%% ===== 
%% ===== 1
%% ===== 
%% ===== 
%% ===== 
%% ===== 
%% ===== 
%% ===== \begin{flushleft}
%% ===== x IA
%% ===== \end{flushleft}
%% ===== 
%% ===== 
%% ===== 
%% ===== 
%% ===== 
%% ===== \begin{flushleft}
%% ===== System controller illegal action lines (see Table 3-2)
%% ===== \end{flushleft}
%% ===== 
%% ===== 
%% ===== 
%% ===== 
%% ===== 
%% ===== 1
%% ===== 
%% ===== 
%% ===== 
%% ===== 
%% ===== 
%% ===== \begin{flushleft}
%% ===== x IACHN
%% ===== \end{flushleft}
%% ===== 
%% ===== 
%% ===== 
%% ===== 
%% ===== 
%% ===== \begin{flushleft}
%% ===== Illegal action processor port
%% ===== \end{flushleft}
%% ===== 
%% ===== 
%% ===== 
%% ===== 
%% ===== 
%% ===== 1
%% ===== 
%% ===== 
%% ===== 
%% ===== 
%% ===== 
%% ===== \begin{flushleft}
%% ===== x CNCHN
%% ===== \end{flushleft}
%% ===== 
%% ===== 
%% ===== 
%% ===== 
%% ===== 
%% ===== \begin{flushleft}
%% ===== For connect fault - connect processor port
%% ===== \end{flushleft}
%% ===== 
%% ===== 
%% ===== 
%% ===== 
%% ===== 
%% ===== 1
%% ===== 
%% ===== 
%% ===== 
%% ===== 
%% ===== 
%% ===== \begin{flushleft}
%% ===== x F/I ADDR
%% ===== \end{flushleft}
%% ===== 
%% ===== 
%% ===== 
%% ===== 
%% ===== 
%% ===== \begin{flushleft}
%% ===== Modulo 2 fault/interrupt vector address
%% ===== \end{flushleft}
%% ===== 
%% ===== 
%% ===== 
%% ===== 
%% ===== 
%% ===== \begin{flushleft}
%% ===== x F/I
%% ===== \end{flushleft}
%% ===== 
%% ===== 
%% ===== 
%% ===== 
%% ===== 
%% ===== \begin{flushleft}
%% ===== Fault/interrupt flag
%% ===== \end{flushleft}
%% ===== 
%% ===== 
%% ===== \begin{flushleft}
%% ===== 0 = interrupt
%% ===== \end{flushleft}
%% ===== 
%% ===== 
%% ===== \begin{flushleft}
%% ===== 1 = fault
%% ===== \end{flushleft}
%% ===== 
%% ===== 
%% ===== 
%% ===== 
%% ===== 
%% ===== 1
%% ===== 
%% ===== 
%% ===== 
%% ===== 
%% ===== 
%% ===== \begin{flushleft}
%% ===== u
%% ===== \end{flushleft}
%% ===== 
%% ===== 
%% ===== 
%% ===== 
%% ===== 
%% ===== 2
%% ===== 
%% ===== 
%% ===== 
%% ===== 
%% ===== 
%% ===== \begin{flushleft}
%% ===== TRR
%% ===== \end{flushleft}
%% ===== 
%% ===== 
%% ===== 
%% ===== 
%% ===== 
%% ===== \begin{flushleft}
%% ===== Temporary ring register (TPR.TRR)
%% ===== \end{flushleft}
%% ===== 
%% ===== 
%% ===== 
%% ===== 
%% ===== 
%% ===== 2
%% ===== 
%% ===== 
%% ===== 
%% ===== 
%% ===== 
%% ===== \begin{flushleft}
%% ===== TSR
%% ===== \end{flushleft}
%% ===== 
%% ===== 
%% ===== 
%% ===== 
%% ===== 
%% ===== \begin{flushleft}
%% ===== Temporary segment register (TPR.TSR)
%% ===== \end{flushleft}
%% ===== 
%% ===== 
%% ===== 
%% ===== 
%% ===== 
%% ===== \begin{flushleft}
%% ===== PTW
%% ===== \end{flushleft}
%% ===== 
%% ===== 
%% ===== 
%% ===== 
%% ===== 
%% ===== \begin{flushleft}
%% ===== DPS 8M processors only; this field mbz on DPS/L68
%% ===== \end{flushleft}
%% ===== 
%% ===== 
%% ===== \begin{flushleft}
%% ===== processors:
%% ===== \end{flushleft}
%% ===== 
%% ===== 
%% ===== 
%% ===== 
%% ===== 
%% ===== 2
%% ===== 
%% ===== 
%% ===== 2
%% ===== 
%% ===== 
%% ===== 2
%% ===== 
%% ===== 
%% ===== 2
%% ===== 
%% ===== 
%% ===== 
%% ===== 
%% ===== 
%% ===== \begin{flushleft}
%% ===== a
%% ===== \end{flushleft}
%% ===== 
%% ===== 
%% ===== \begin{flushleft}
%% ===== b
%% ===== \end{flushleft}
%% ===== 
%% ===== 
%% ===== \begin{flushleft}
%% ===== c
%% ===== \end{flushleft}
%% ===== 
%% ===== 
%% ===== \begin{flushleft}
%% ===== d
%% ===== \end{flushleft}
%% ===== 
%% ===== 
%% ===== 
%% ===== 
%% ===== 
%% ===== \begin{flushleft}
%% ===== x
%% ===== \end{flushleft}
%% ===== 
%% ===== 
%% ===== \begin{flushleft}
%% ===== x
%% ===== \end{flushleft}
%% ===== 
%% ===== 
%% ===== \begin{flushleft}
%% ===== x
%% ===== \end{flushleft}
%% ===== 
%% ===== 
%% ===== \begin{flushleft}
%% ===== x
%% ===== \end{flushleft}
%% ===== 
%% ===== 
%% ===== 
%% ===== 
%% ===== 
%% ===== \begin{flushleft}
%% ===== PTWAM
%% ===== \end{flushleft}
%% ===== 
%% ===== 
%% ===== \begin{flushleft}
%% ===== PTWAM
%% ===== \end{flushleft}
%% ===== 
%% ===== 
%% ===== \begin{flushleft}
%% ===== PTWAM
%% ===== \end{flushleft}
%% ===== 
%% ===== 
%% ===== \begin{flushleft}
%% ===== PTWAM
%% ===== \end{flushleft}
%% ===== 
%% ===== 
%% ===== \begin{flushleft}
%% ===== SDW
%% ===== \end{flushleft}
%% ===== 
%% ===== 
%% ===== 
%% ===== 
%% ===== 
%% ===== 2
%% ===== 
%% ===== 
%% ===== 2
%% ===== 
%% ===== 
%% ===== 2
%% ===== 
%% ===== 
%% ===== 2
%% ===== 
%% ===== 
%% ===== 
%% ===== 
%% ===== 
%% ===== \begin{flushleft}
%% ===== e
%% ===== \end{flushleft}
%% ===== 
%% ===== 
%% ===== \begin{flushleft}
%% ===== f
%% ===== \end{flushleft}
%% ===== 
%% ===== 
%% ===== \begin{flushleft}
%% ===== g
%% ===== \end{flushleft}
%% ===== 
%% ===== 
%% ===== \begin{flushleft}
%% ===== h
%% ===== \end{flushleft}
%% ===== 
%% ===== 
%% ===== 
%% ===== 
%% ===== 
%% ===== \begin{flushleft}
%% ===== x
%% ===== \end{flushleft}
%% ===== 
%% ===== 
%% ===== \begin{flushleft}
%% ===== x
%% ===== \end{flushleft}
%% ===== 
%% ===== 
%% ===== \begin{flushleft}
%% ===== x
%% ===== \end{flushleft}
%% ===== 
%% ===== 
%% ===== \begin{flushleft}
%% ===== x
%% ===== \end{flushleft}
%% ===== 
%% ===== 
%% ===== 
%% ===== 
%% ===== 
%% ===== \begin{flushleft}
%% ===== levels
%% ===== \end{flushleft}
%% ===== 
%% ===== 
%% ===== \begin{flushleft}
%% ===== levels
%% ===== \end{flushleft}
%% ===== 
%% ===== 
%% ===== \begin{flushleft}
%% ===== levels
%% ===== \end{flushleft}
%% ===== 
%% ===== 
%% ===== \begin{flushleft}
%% ===== levels
%% ===== \end{flushleft}
%% ===== 
%% ===== 
%% ===== 
%% ===== 
%% ===== 
%% ===== \begin{flushleft}
%% ===== A, B enabled (enabled = 1)
%% ===== \end{flushleft}
%% ===== 
%% ===== 
%% ===== \begin{flushleft}
%% ===== C, D enabled
%% ===== \end{flushleft}
%% ===== 
%% ===== 
%% ===== \begin{flushleft}
%% ===== A, B match (match = 1)
%% ===== \end{flushleft}
%% ===== 
%% ===== 
%% ===== \begin{flushleft}
%% ===== C, D match
%% ===== \end{flushleft}
%% ===== 
%% ===== 
%% ===== 
%% ===== 
%% ===== 
%% ===== \begin{flushleft}
%% ===== DPS 8M processors only; this field mbz on DPS/L68
%% ===== \end{flushleft}
%% ===== 
%% ===== 
%% ===== \begin{flushleft}
%% ===== processors:
%% ===== \end{flushleft}
%% ===== 
%% ===== 
%% ===== \begin{flushleft}
%% ===== SDWAM
%% ===== \end{flushleft}
%% ===== 
%% ===== 
%% ===== \begin{flushleft}
%% ===== SDWAM
%% ===== \end{flushleft}
%% ===== 
%% ===== 
%% ===== \begin{flushleft}
%% ===== SDWAM
%% ===== \end{flushleft}
%% ===== 
%% ===== 
%% ===== \begin{flushleft}
%% ===== SDWAM
%% ===== \end{flushleft}
%% ===== 
%% ===== 
%% ===== 
%% ===== 
%% ===== 
%% ===== \begin{flushleft}
%% ===== levels
%% ===== \end{flushleft}
%% ===== 
%% ===== 
%% ===== \begin{flushleft}
%% ===== levels
%% ===== \end{flushleft}
%% ===== 
%% ===== 
%% ===== \begin{flushleft}
%% ===== levels
%% ===== \end{flushleft}
%% ===== 
%% ===== 
%% ===== \begin{flushleft}
%% ===== levels
%% ===== \end{flushleft}
%% ===== 
%% ===== 
%% ===== 
%% ===== 
%% ===== 
%% ===== \begin{flushleft}
%% ===== A, B enabled
%% ===== \end{flushleft}
%% ===== 
%% ===== 
%% ===== \begin{flushleft}
%% ===== C, D enabled
%% ===== \end{flushleft}
%% ===== 
%% ===== 
%% ===== \begin{flushleft}
%% ===== A, B match
%% ===== \end{flushleft}
%% ===== 
%% ===== 
%% ===== \begin{flushleft}
%% ===== C, D match
%% ===== \end{flushleft}
%% ===== 
%% ===== 
%% ===== 
%% ===== 
%% ===== 
%% ===== 2
%% ===== 
%% ===== 
%% ===== 
%% ===== 
%% ===== 
%% ===== \begin{flushleft}
%% ===== CPU
%% ===== \end{flushleft}
%% ===== 
%% ===== 
%% ===== 
%% ===== 
%% ===== 
%% ===== \begin{flushleft}
%% ===== CPU number
%% ===== \end{flushleft}
%% ===== 
%% ===== 
%% ===== 
%% ===== 
%% ===== 
%% ===== 2
%% ===== 
%% ===== 
%% ===== 
%% ===== 
%% ===== 
%% ===== \begin{flushleft}
%% ===== DELTA
%% ===== \end{flushleft}
%% ===== 
%% ===== 
%% ===== 
%% ===== 
%% ===== 
%% ===== \begin{flushleft}
%% ===== Address increment for repeats
%% ===== \end{flushleft}
%% ===== 
%% ===== 
%% ===== 
%% ===== 
%% ===== 
%% ===== 3
%% ===== 
%% ===== 
%% ===== 
%% ===== 
%% ===== 
%% ===== \begin{flushleft}
%% ===== TSNA
%% ===== \end{flushleft}
%% ===== 
%% ===== 
%% ===== 
%% ===== 
%% ===== 
%% ===== \begin{flushleft}
%% ===== Pointer register number for non-EIS operands or for EIS
%% ===== \end{flushleft}
%% ===== 
%% ===== 
%% ===== \begin{flushleft}
%% ===== operand \#1 further substructured as:
%% ===== \end{flushleft}
%% ===== 
%% ===== 
%% ===== 
%% ===== 
%% ===== 
%% ===== 3
%% ===== 
%% ===== 
%% ===== 
%% ===== 
%% ===== 
%% ===== \begin{flushleft}
%% ===== a
%% ===== \end{flushleft}
%% ===== 
%% ===== 
%% ===== 
%% ===== 
%% ===== 
%% ===== \begin{flushleft}
%% ===== PRNO
%% ===== \end{flushleft}
%% ===== 
%% ===== 
%% ===== 
%% ===== 
%% ===== 
%% ===== \begin{flushleft}
%% ===== Pointer register number
%% ===== \end{flushleft}
%% ===== 
%% ===== 
%% ===== 
%% ===== 
%% ===== 
%% ===== 3
%% ===== 
%% ===== 
%% ===== 
%% ===== 
%% ===== 
%% ===== \begin{flushleft}
%% ===== b
%% ===== \end{flushleft}
%% ===== 
%% ===== 
%% ===== 
%% ===== 
%% ===== 
%% ===== ----
%% ===== 
%% ===== 
%% ===== 
%% ===== 
%% ===== 
%% ===== \begin{flushleft}
%% ===== 1 = PRNO is valid
%% ===== \end{flushleft}
%% ===== 
%% ===== 
%% ===== 
%% ===== 
%% ===== 
%% ===== 3
%% ===== 
%% ===== 
%% ===== 
%% ===== 
%% ===== 
%% ===== \begin{flushleft}
%% ===== TSNB
%% ===== \end{flushleft}
%% ===== 
%% ===== 
%% ===== 
%% ===== 
%% ===== 
%% ===== \begin{flushleft}
%% ===== Pointer register number for EIS operand \#2 further
%% ===== \end{flushleft}
%% ===== 
%% ===== 
%% ===== \begin{flushleft}
%% ===== substructured as for TSNA above
%% ===== \end{flushleft}
%% ===== 
%% ===== 
%% ===== 
%% ===== 
%% ===== 
%% ===== 3
%% ===== 
%% ===== 
%% ===== 
%% ===== 
%% ===== 
%% ===== \begin{flushleft}
%% ===== TSNC
%% ===== \end{flushleft}
%% ===== 
%% ===== 
%% ===== 
%% ===== 
%% ===== 
%% ===== \begin{flushleft}
%% ===== Pointer register number for EIS operand \#3 further
%% ===== \end{flushleft}
%% ===== 
%% ===== 
%% ===== \begin{flushleft}
%% ===== substructured as for TSNA above
%% ===== \end{flushleft}
%% ===== 
%% ===== 
%% ===== 
%% ===== 
%% ===== 
%% ===== 3
%% ===== 
%% ===== 
%% ===== 
%% ===== 
%% ===== 
%% ===== \begin{flushleft}
%% ===== TEMP BIT
%% ===== \end{flushleft}
%% ===== 
%% ===== 
%% ===== 
%% ===== 
%% ===== 
%% ===== \begin{flushleft}
%% ===== Current bit offset (TPR.TBR)
%% ===== \end{flushleft}
%% ===== 
%% ===== 
%% ===== 
%% ===== 
%% ===== 
%% ===== 4
%% ===== 
%% ===== 
%% ===== 
%% ===== 
%% ===== 
%% ===== \begin{flushleft}
%% ===== IC
%% ===== \end{flushleft}
%% ===== 
%% ===== 
%% ===== 
%% ===== 
%% ===== 
%% ===== \begin{flushleft}
%% ===== Instruction counter (PPR.IC)
%% ===== \end{flushleft}
%% ===== 
%% ===== 
%% ===== 
%% ===== 
%% ===== 
%% ===== \begin{flushleft}
%% ===== ZERO
%% ===== \end{flushleft}
%% ===== 
%% ===== 
%% ===== 
%% ===== 
%% ===== 
%% ===== \begin{flushleft}
%% ===== Zero indicator
%% ===== \end{flushleft}
%% ===== 
%% ===== 
%% ===== 
%% ===== 
%% ===== 
%% ===== 4
%% ===== 
%% ===== 
%% ===== 
%% ===== 
%% ===== 
%% ===== \begin{flushleft}
%% ===== a
%% ===== \end{flushleft}
%% ===== 
%% ===== 
%% ===== 
%% ===== 
%% ===== 
%% ===== \begin{flushleft}
%% ===== \newpage
%% ===== Word
%% ===== \end{flushleft}
%% ===== 
%% ===== 
%% ===== 
%% ===== 
%% ===== 
%% ===== \begin{flushleft}
%% ===== key L Field Name
%% ===== \end{flushleft}
%% ===== 
%% ===== 
%% ===== 
%% ===== 
%% ===== 
%% ===== \begin{flushleft}
%% ===== Meaning
%% ===== \end{flushleft}
%% ===== 
%% ===== 
%% ===== 
%% ===== 
%% ===== 
%% ===== 4
%% ===== 
%% ===== 
%% ===== 
%% ===== 
%% ===== 
%% ===== \begin{flushleft}
%% ===== b
%% ===== \end{flushleft}
%% ===== 
%% ===== 
%% ===== 
%% ===== 
%% ===== 
%% ===== \begin{flushleft}
%% ===== NEG
%% ===== \end{flushleft}
%% ===== 
%% ===== 
%% ===== 
%% ===== 
%% ===== 
%% ===== \begin{flushleft}
%% ===== Negative indicator
%% ===== \end{flushleft}
%% ===== 
%% ===== 
%% ===== 
%% ===== 
%% ===== 
%% ===== 4
%% ===== 
%% ===== 
%% ===== 
%% ===== 
%% ===== 
%% ===== \begin{flushleft}
%% ===== c
%% ===== \end{flushleft}
%% ===== 
%% ===== 
%% ===== 
%% ===== 
%% ===== 
%% ===== \begin{flushleft}
%% ===== CARY
%% ===== \end{flushleft}
%% ===== 
%% ===== 
%% ===== 
%% ===== 
%% ===== 
%% ===== \begin{flushleft}
%% ===== Carry indicator
%% ===== \end{flushleft}
%% ===== 
%% ===== 
%% ===== 
%% ===== 
%% ===== 
%% ===== 4
%% ===== 
%% ===== 
%% ===== 
%% ===== 
%% ===== 
%% ===== \begin{flushleft}
%% ===== d
%% ===== \end{flushleft}
%% ===== 
%% ===== 
%% ===== 
%% ===== 
%% ===== 
%% ===== \begin{flushleft}
%% ===== OVFL
%% ===== \end{flushleft}
%% ===== 
%% ===== 
%% ===== 
%% ===== 
%% ===== 
%% ===== \begin{flushleft}
%% ===== Overflow indicator
%% ===== \end{flushleft}
%% ===== 
%% ===== 
%% ===== 
%% ===== 
%% ===== 
%% ===== 4
%% ===== 
%% ===== 
%% ===== 
%% ===== 
%% ===== 
%% ===== \begin{flushleft}
%% ===== e
%% ===== \end{flushleft}
%% ===== 
%% ===== 
%% ===== 
%% ===== 
%% ===== 
%% ===== \begin{flushleft}
%% ===== EOVF
%% ===== \end{flushleft}
%% ===== 
%% ===== 
%% ===== 
%% ===== 
%% ===== 
%% ===== \begin{flushleft}
%% ===== Exponent overflow indicator
%% ===== \end{flushleft}
%% ===== 
%% ===== 
%% ===== 
%% ===== 
%% ===== 
%% ===== 4
%% ===== 
%% ===== 
%% ===== 
%% ===== 
%% ===== 
%% ===== \begin{flushleft}
%% ===== f
%% ===== \end{flushleft}
%% ===== 
%% ===== 
%% ===== 
%% ===== 
%% ===== 
%% ===== \begin{flushleft}
%% ===== EUFL
%% ===== \end{flushleft}
%% ===== 
%% ===== 
%% ===== 
%% ===== 
%% ===== 
%% ===== \begin{flushleft}
%% ===== Exponent underflow indicator
%% ===== \end{flushleft}
%% ===== 
%% ===== 
%% ===== 
%% ===== 
%% ===== 
%% ===== 4
%% ===== 
%% ===== 
%% ===== 
%% ===== 
%% ===== 
%% ===== \begin{flushleft}
%% ===== g
%% ===== \end{flushleft}
%% ===== 
%% ===== 
%% ===== 
%% ===== 
%% ===== 
%% ===== \begin{flushleft}
%% ===== OFLM
%% ===== \end{flushleft}
%% ===== 
%% ===== 
%% ===== 
%% ===== 
%% ===== 
%% ===== \begin{flushleft}
%% ===== Overflow mask indicator
%% ===== \end{flushleft}
%% ===== 
%% ===== 
%% ===== 
%% ===== 
%% ===== 
%% ===== 4
%% ===== 
%% ===== 
%% ===== 
%% ===== 
%% ===== 
%% ===== \begin{flushleft}
%% ===== h
%% ===== \end{flushleft}
%% ===== 
%% ===== 
%% ===== 
%% ===== 
%% ===== 
%% ===== \begin{flushleft}
%% ===== TRO
%% ===== \end{flushleft}
%% ===== 
%% ===== 
%% ===== 
%% ===== 
%% ===== 
%% ===== \begin{flushleft}
%% ===== Tally runout indicator
%% ===== \end{flushleft}
%% ===== 
%% ===== 
%% ===== 
%% ===== 
%% ===== 
%% ===== 4
%% ===== 
%% ===== 
%% ===== 
%% ===== 
%% ===== 
%% ===== \begin{flushleft}
%% ===== i
%% ===== \end{flushleft}
%% ===== 
%% ===== 
%% ===== 
%% ===== 
%% ===== 
%% ===== \begin{flushleft}
%% ===== PAR
%% ===== \end{flushleft}
%% ===== 
%% ===== 
%% ===== 
%% ===== 
%% ===== 
%% ===== \begin{flushleft}
%% ===== Parity error indicator
%% ===== \end{flushleft}
%% ===== 
%% ===== 
%% ===== 
%% ===== 
%% ===== 
%% ===== 4
%% ===== 
%% ===== 
%% ===== 
%% ===== 
%% ===== 
%% ===== \begin{flushleft}
%% ===== j
%% ===== \end{flushleft}
%% ===== 
%% ===== 
%% ===== 
%% ===== 
%% ===== 
%% ===== \begin{flushleft}
%% ===== PARM
%% ===== \end{flushleft}
%% ===== 
%% ===== 
%% ===== 
%% ===== 
%% ===== 
%% ===== \begin{flushleft}
%% ===== Parity mask indicator
%% ===== \end{flushleft}
%% ===== 
%% ===== 
%% ===== 
%% ===== 
%% ===== 
%% ===== 4
%% ===== 
%% ===== 
%% ===== 
%% ===== 
%% ===== 
%% ===== \begin{flushleft}
%% ===== k
%% ===== \end{flushleft}
%% ===== 
%% ===== 
%% ===== 
%% ===== 
%% ===== 
%% ===== \begin{flushleft}
%% ===== -BM
%% ===== \end{flushleft}
%% ===== 
%% ===== 
%% ===== 
%% ===== 
%% ===== 
%% ===== \begin{flushleft}
%% ===== Not BAR mode indicator
%% ===== \end{flushleft}
%% ===== 
%% ===== 
%% ===== 
%% ===== 
%% ===== 
%% ===== 4
%% ===== 
%% ===== 
%% ===== 
%% ===== 
%% ===== 
%% ===== \begin{flushleft}
%% ===== l
%% ===== \end{flushleft}
%% ===== 
%% ===== 
%% ===== 
%% ===== 
%% ===== 
%% ===== \begin{flushleft}
%% ===== TRU
%% ===== \end{flushleft}
%% ===== 
%% ===== 
%% ===== 
%% ===== 
%% ===== 
%% ===== \begin{flushleft}
%% ===== EIS truncation indicator
%% ===== \end{flushleft}
%% ===== 
%% ===== 
%% ===== 
%% ===== 
%% ===== 
%% ===== 4
%% ===== 
%% ===== 
%% ===== 
%% ===== 
%% ===== 
%% ===== \begin{flushleft}
%% ===== m
%% ===== \end{flushleft}
%% ===== 
%% ===== 
%% ===== 
%% ===== 
%% ===== 
%% ===== \begin{flushleft}
%% ===== MIF
%% ===== \end{flushleft}
%% ===== 
%% ===== 
%% ===== 
%% ===== 
%% ===== 
%% ===== \begin{flushleft}
%% ===== Mid-instruction interrupt indicator
%% ===== \end{flushleft}
%% ===== 
%% ===== 
%% ===== 
%% ===== 
%% ===== 
%% ===== 4
%% ===== 
%% ===== 
%% ===== 
%% ===== 
%% ===== 
%% ===== \begin{flushleft}
%% ===== n
%% ===== \end{flushleft}
%% ===== 
%% ===== 
%% ===== 
%% ===== 
%% ===== 
%% ===== \begin{flushleft}
%% ===== ABS
%% ===== \end{flushleft}
%% ===== 
%% ===== 
%% ===== 
%% ===== 
%% ===== 
%% ===== \begin{flushleft}
%% ===== Absolute mode indicator
%% ===== \end{flushleft}
%% ===== 
%% ===== 
%% ===== 
%% ===== 
%% ===== 
%% ===== 4
%% ===== 
%% ===== 
%% ===== 
%% ===== 
%% ===== 
%% ===== \begin{flushleft}
%% ===== o
%% ===== \end{flushleft}
%% ===== 
%% ===== 
%% ===== 
%% ===== 
%% ===== 
%% ===== \begin{flushleft}
%% ===== HEX
%% ===== \end{flushleft}
%% ===== 
%% ===== 
%% ===== 
%% ===== 
%% ===== 
%% ===== \begin{flushleft}
%% ===== Hex mode indicator (DPS 8M processors only)
%% ===== \end{flushleft}
%% ===== 
%% ===== 
%% ===== 
%% ===== 
%% ===== 
%% ===== 5
%% ===== 
%% ===== 
%% ===== 
%% ===== 
%% ===== 
%% ===== \begin{flushleft}
%% ===== x CA
%% ===== \end{flushleft}
%% ===== 
%% ===== 
%% ===== 
%% ===== 
%% ===== 
%% ===== \begin{flushleft}
%% ===== Current computed address (TPR.CA)
%% ===== \end{flushleft}
%% ===== 
%% ===== 
%% ===== 
%% ===== 
%% ===== 
%% ===== 5
%% ===== 
%% ===== 
%% ===== 
%% ===== 
%% ===== 
%% ===== \begin{flushleft}
%% ===== a
%% ===== \end{flushleft}
%% ===== 
%% ===== 
%% ===== 
%% ===== 
%% ===== 
%% ===== \begin{flushleft}
%% ===== RF
%% ===== \end{flushleft}
%% ===== 
%% ===== 
%% ===== 
%% ===== 
%% ===== 
%% ===== \begin{flushleft}
%% ===== First cycle of all repeat instructions
%% ===== \end{flushleft}
%% ===== 
%% ===== 
%% ===== 
%% ===== 
%% ===== 
%% ===== 5
%% ===== 
%% ===== 
%% ===== 
%% ===== 
%% ===== 
%% ===== \begin{flushleft}
%% ===== b
%% ===== \end{flushleft}
%% ===== 
%% ===== 
%% ===== 
%% ===== 
%% ===== 
%% ===== \begin{flushleft}
%% ===== RPT
%% ===== \end{flushleft}
%% ===== 
%% ===== 
%% ===== 
%% ===== 
%% ===== 
%% ===== \begin{flushleft}
%% ===== Execute a Repeat (rpt) instruction
%% ===== \end{flushleft}
%% ===== 
%% ===== 
%% ===== 
%% ===== 
%% ===== 
%% ===== 5
%% ===== 
%% ===== 
%% ===== 
%% ===== 
%% ===== 
%% ===== \begin{flushleft}
%% ===== c
%% ===== \end{flushleft}
%% ===== 
%% ===== 
%% ===== 
%% ===== 
%% ===== 
%% ===== \begin{flushleft}
%% ===== RD
%% ===== \end{flushleft}
%% ===== 
%% ===== 
%% ===== 
%% ===== 
%% ===== 
%% ===== \begin{flushleft}
%% ===== Execute a Repeat Double (rpd) instruction
%% ===== \end{flushleft}
%% ===== 
%% ===== 
%% ===== 
%% ===== 
%% ===== 
%% ===== 5
%% ===== 
%% ===== 
%% ===== 
%% ===== 
%% ===== 
%% ===== \begin{flushleft}
%% ===== d
%% ===== \end{flushleft}
%% ===== 
%% ===== 
%% ===== 
%% ===== 
%% ===== 
%% ===== \begin{flushleft}
%% ===== RL
%% ===== \end{flushleft}
%% ===== 
%% ===== 
%% ===== 
%% ===== 
%% ===== 
%% ===== \begin{flushleft}
%% ===== Execute a Repeat Link (rpl) instruction
%% ===== \end{flushleft}
%% ===== 
%% ===== 
%% ===== 
%% ===== 
%% ===== 
%% ===== 5
%% ===== 
%% ===== 
%% ===== 
%% ===== 
%% ===== 
%% ===== \begin{flushleft}
%% ===== e
%% ===== \end{flushleft}
%% ===== 
%% ===== 
%% ===== 
%% ===== 
%% ===== 
%% ===== \begin{flushleft}
%% ===== POT
%% ===== \end{flushleft}
%% ===== 
%% ===== 
%% ===== 
%% ===== 
%% ===== 
%% ===== \begin{flushleft}
%% ===== Prepare operand tally.
%% ===== \end{flushleft}
%% ===== 
%% ===== 
%% ===== \begin{flushleft}
%% ===== word of an indirect
%% ===== \end{flushleft}
%% ===== 
%% ===== 
%% ===== \begin{flushleft}
%% ===== successfully fetched.
%% ===== \end{flushleft}
%% ===== 
%% ===== 
%% ===== 
%% ===== 
%% ===== 
%% ===== 5
%% ===== 
%% ===== 
%% ===== 
%% ===== 
%% ===== 
%% ===== \begin{flushleft}
%% ===== f
%% ===== \end{flushleft}
%% ===== 
%% ===== 
%% ===== 
%% ===== 
%% ===== 
%% ===== \begin{flushleft}
%% ===== PON
%% ===== \end{flushleft}
%% ===== 
%% ===== 
%% ===== 
%% ===== 
%% ===== 
%% ===== \begin{flushleft}
%% ===== Prepare operand no tally. This flag is up until the indirect
%% ===== \end{flushleft}
%% ===== 
%% ===== 
%% ===== \begin{flushleft}
%% ===== word of a return type transfer instruction is successfully
%% ===== \end{flushleft}
%% ===== 
%% ===== 
%% ===== \begin{flushleft}
%% ===== fetched. It indicates that there is no indirect chain even
%% ===== \end{flushleft}
%% ===== 
%% ===== 
%% ===== \begin{flushleft}
%% ===== though an indirect fetch is being performed.
%% ===== \end{flushleft}
%% ===== 
%% ===== 
%% ===== 
%% ===== 
%% ===== 
%% ===== 5
%% ===== 
%% ===== 
%% ===== 
%% ===== 
%% ===== 
%% ===== \begin{flushleft}
%% ===== g
%% ===== \end{flushleft}
%% ===== 
%% ===== 
%% ===== 
%% ===== 
%% ===== 
%% ===== \begin{flushleft}
%% ===== XDE
%% ===== \end{flushleft}
%% ===== 
%% ===== 
%% ===== 
%% ===== 
%% ===== 
%% ===== \begin{flushleft}
%% ===== Execute instruction from Execute Double even pair
%% ===== \end{flushleft}
%% ===== 
%% ===== 
%% ===== 
%% ===== 
%% ===== 
%% ===== 5
%% ===== 
%% ===== 
%% ===== 
%% ===== 
%% ===== 
%% ===== \begin{flushleft}
%% ===== h
%% ===== \end{flushleft}
%% ===== 
%% ===== 
%% ===== 
%% ===== 
%% ===== 
%% ===== \begin{flushleft}
%% ===== XDO
%% ===== \end{flushleft}
%% ===== 
%% ===== 
%% ===== 
%% ===== 
%% ===== 
%% ===== \begin{flushleft}
%% ===== Execute instruction from Execute Double odd pair
%% ===== \end{flushleft}
%% ===== 
%% ===== 
%% ===== 
%% ===== 
%% ===== 
%% ===== 5
%% ===== 
%% ===== 
%% ===== 
%% ===== 
%% ===== 
%% ===== \begin{flushleft}
%% ===== i
%% ===== \end{flushleft}
%% ===== 
%% ===== 
%% ===== 
%% ===== 
%% ===== 
%% ===== \begin{flushleft}
%% ===== ITP
%% ===== \end{flushleft}
%% ===== 
%% ===== 
%% ===== 
%% ===== 
%% ===== 
%% ===== \begin{flushleft}
%% ===== Execute ITP indirect cycle
%% ===== \end{flushleft}
%% ===== 
%% ===== 
%% ===== 
%% ===== 
%% ===== 
%% ===== 5
%% ===== 
%% ===== 
%% ===== 
%% ===== 
%% ===== 
%% ===== \begin{flushleft}
%% ===== j
%% ===== \end{flushleft}
%% ===== 
%% ===== 
%% ===== 
%% ===== 
%% ===== 
%% ===== \begin{flushleft}
%% ===== RFI
%% ===== \end{flushleft}
%% ===== 
%% ===== 
%% ===== 
%% ===== 
%% ===== 
%% ===== \begin{flushleft}
%% ===== Restart this instruction
%% ===== \end{flushleft}
%% ===== 
%% ===== 
%% ===== 
%% ===== 
%% ===== 
%% ===== 5
%% ===== 
%% ===== 
%% ===== 
%% ===== 
%% ===== 
%% ===== \begin{flushleft}
%% ===== k
%% ===== \end{flushleft}
%% ===== 
%% ===== 
%% ===== 
%% ===== 
%% ===== 
%% ===== \begin{flushleft}
%% ===== ITS
%% ===== \end{flushleft}
%% ===== 
%% ===== 
%% ===== 
%% ===== 
%% ===== 
%% ===== \begin{flushleft}
%% ===== Execute ITS indirect cycle
%% ===== \end{flushleft}
%% ===== 
%% ===== 
%% ===== 
%% ===== 
%% ===== 
%% ===== 5
%% ===== 
%% ===== 
%% ===== 
%% ===== 
%% ===== 
%% ===== \begin{flushleft}
%% ===== l
%% ===== \end{flushleft}
%% ===== 
%% ===== 
%% ===== 
%% ===== 
%% ===== 
%% ===== \begin{flushleft}
%% ===== FIF
%% ===== \end{flushleft}
%% ===== 
%% ===== 
%% ===== 
%% ===== 
%% ===== 
%% ===== \begin{flushleft}
%% ===== Fault occurred during instruction fetch
%% ===== \end{flushleft}
%% ===== 
%% ===== 
%% ===== 
%% ===== 
%% ===== 
%% ===== \begin{flushleft}
%% ===== CT HOLD
%% ===== \end{flushleft}
%% ===== 
%% ===== 
%% ===== 
%% ===== 
%% ===== 
%% ===== \begin{flushleft}
%% ===== Contents of the modifier holding register
%% ===== \end{flushleft}
%% ===== 
%% ===== 
%% ===== 
%% ===== 
%% ===== 
%% ===== 5
%% ===== 
%% ===== 
%% ===== 6
%% ===== 
%% ===== 
%% ===== 
%% ===== 
%% ===== 
%% ===== \begin{flushleft}
%% ===== This flag is up until the indirect
%% ===== \end{flushleft}
%% ===== 
%% ===== 
%% ===== \begin{flushleft}
%% ===== then tally address modifier is
%% ===== \end{flushleft}
%% ===== 
%% ===== 
%% ===== 
%% ===== 
%% ===== 
%% ===== \begin{flushleft}
%% ===== Word 6 is the contents of the working instruction register
%% ===== \end{flushleft}
%% ===== 
%% ===== 
%% ===== \begin{flushleft}
%% ===== and reflects conditions at the exact point of address
%% ===== \end{flushleft}
%% ===== 
%% ===== 
%% ===== \begin{flushleft}
%% ===== preparation when the fault or interrupt occurred. The
%% ===== \end{flushleft}
%% ===== 
%% ===== 
%% ===== \begin{flushleft}
%% ===== ADDRESS and TAG fields are replaced with data from
%% ===== \end{flushleft}
%% ===== 
%% ===== 
%% ===== \begin{flushleft}
%% ===== pointer registers, indirect pointers, and/or indirect words
%% ===== \end{flushleft}
%% ===== 
%% ===== 
%% ===== \begin{flushleft}
%% ===== during each indirect cycle. Each instruction of the current
%% ===== \end{flushleft}
%% ===== 
%% ===== 
%% ===== \begin{flushleft}
%% ===== pair is moved to this register before actual address
%% ===== \end{flushleft}
%% ===== 
%% ===== 
%% ===== \begin{flushleft}
%% ===== preparation begins.
%% ===== \end{flushleft}
%% ===== 
%% ===== 
%% ===== 
%% ===== 
%% ===== 
%% ===== \begin{flushleft}
%% ===== \newpage
%% ===== Word
%% ===== \end{flushleft}
%% ===== 
%% ===== 
%% ===== 
%% ===== 
%% ===== 
%% ===== \begin{flushleft}
%% ===== key L Field Name
%% ===== \end{flushleft}
%% ===== 
%% ===== 
%% ===== 
%% ===== 
%% ===== 
%% ===== 7
%% ===== 
%% ===== 
%% ===== 
%% ===== 
%% ===== 
%% ===== \begin{flushleft}
%% ===== Meaning
%% ===== \end{flushleft}
%% ===== 
%% ===== 
%% ===== \begin{flushleft}
%% ===== Word 7 is the contents of the instruction holding register. It
%% ===== \end{flushleft}
%% ===== 
%% ===== 
%% ===== \begin{flushleft}
%% ===== contains the odd word of the last instruction pair fetched
%% ===== \end{flushleft}
%% ===== 
%% ===== 
%% ===== \begin{flushleft}
%% ===== from main memory.
%% ===== \end{flushleft}
%% ===== 
%% ===== 
%% ===== \begin{flushleft}
%% ===== Note that, primarily because of
%% ===== \end{flushleft}
%% ===== 
%% ===== 
%% ===== \begin{flushleft}
%% ===== overlap, this instruction is not necessarily paired with the
%% ===== \end{flushleft}
%% ===== 
%% ===== 
%% ===== \begin{flushleft}
%% ===== instruction in word 6.
%% ===== \end{flushleft}
%% ===== 
%% ===== 
%% ===== 
%% ===== 
%% ===== 
%% ===== \begin{flushleft}

\subsection{DECIMAL UNIT DATA}

%% ===== \end{flushleft}
%% ===== 
%% ===== 
%% ===== \begin{flushleft}
%% ===== Format: - 288 bits, 8 machine words
%% ===== \end{flushleft}
%% ===== 
%% ===== 
%% ===== \begin{flushleft}
%% ===== Data as stored by Store Pointers and Lengths (spl) instruction
%% ===== \end{flushleft}
%% ===== 
%% ===== 
%% ===== \begin{flushleft}
%% ===== Word
%% ===== \end{flushleft}
%% ===== 
%% ===== 
%% ===== 0
%% ===== 
%% ===== 
%% ===== 0
%% ===== 
%% ===== 
%% ===== 0
%% ===== 
%% ===== 
%% ===== 
%% ===== 
%% ===== 
%% ===== 0 0 1 1 1
%% ===== 
%% ===== 
%% ===== 8 9 0 1 2
%% ===== 
%% ===== 
%% ===== 
%% ===== 
%% ===== 
%% ===== \begin{flushleft}
%% ===== 0 0 0 0 0 0 0 0 0 Z {\O} 0
%% ===== \end{flushleft}
%% ===== 
%% ===== 
%% ===== 
%% ===== 
%% ===== 
%% ===== 3
%% ===== 
%% ===== 
%% ===== 5
%% ===== 
%% ===== 
%% ===== \begin{flushleft}
%% ===== CH TALLY
%% ===== \end{flushleft}
%% ===== 
%% ===== 
%% ===== 
%% ===== 
%% ===== 
%% ===== 9 1 1 1
%% ===== 
%% ===== 
%% ===== 
%% ===== 
%% ===== 
%% ===== 24
%% ===== 
%% ===== 
%% ===== 
%% ===== 
%% ===== 
%% ===== 0
%% ===== 
%% ===== 
%% ===== 0
%% ===== 
%% ===== 
%% ===== 1
%% ===== 
%% ===== 
%% ===== 
%% ===== 
%% ===== 
%% ===== 3
%% ===== 
%% ===== 
%% ===== 5
%% ===== 
%% ===== 
%% ===== 
%% ===== 
%% ===== 
%% ===== 0 0 0 0 0 0 0 0 0 0 0 0 0 0 0 0 0 0 0 0 0 0 0 0 0 0 0 0 0 0 0 0 0 0 0 0
%% ===== 
%% ===== 
%% ===== 36
%% ===== 
%% ===== 
%% ===== 0
%% ===== 
%% ===== 
%% ===== 0
%% ===== 
%% ===== 
%% ===== 
%% ===== 
%% ===== 
%% ===== 2 2 2 2 2
%% ===== 
%% ===== 
%% ===== 3 4 5 6 7
%% ===== 
%% ===== 
%% ===== 
%% ===== 
%% ===== 
%% ===== 2
%% ===== 
%% ===== 
%% ===== 
%% ===== 
%% ===== 
%% ===== \begin{flushleft}
%% ===== D1 PTR
%% ===== \end{flushleft}
%% ===== 
%% ===== 
%% ===== 
%% ===== 
%% ===== 
%% ===== 0
%% ===== 
%% ===== 
%% ===== 24 1
%% ===== 
%% ===== 
%% ===== 
%% ===== 
%% ===== 
%% ===== 0
%% ===== 
%% ===== 
%% ===== 0
%% ===== 
%% ===== 
%% ===== 3
%% ===== 
%% ===== 
%% ===== 
%% ===== 
%% ===== 
%% ===== \begin{flushleft}
%% ===== LEVEL 1
%% ===== \end{flushleft}
%% ===== 
%% ===== 
%% ===== 
%% ===== 
%% ===== 
%% ===== 0 0
%% ===== 
%% ===== 
%% ===== 
%% ===== 
%% ===== 
%% ===== 2
%% ===== 
%% ===== 
%% ===== 
%% ===== 
%% ===== 
%% ===== 3 1 1 1
%% ===== 
%% ===== 
%% ===== 
%% ===== 
%% ===== 
%% ===== \begin{flushleft}
%% ===== D1 RES
%% ===== \end{flushleft}
%% ===== 
%% ===== 
%% ===== 24
%% ===== 
%% ===== 
%% ===== 2 2 2 2 2
%% ===== 
%% ===== 
%% ===== 3 4 5 6 7
%% ===== 
%% ===== 
%% ===== 
%% ===== 
%% ===== 
%% ===== 4
%% ===== 
%% ===== 
%% ===== 
%% ===== 
%% ===== 
%% ===== \begin{flushleft}
%% ===== D2 PTR
%% ===== \end{flushleft}
%% ===== 
%% ===== 
%% ===== 
%% ===== 
%% ===== 
%% ===== 0
%% ===== 
%% ===== 
%% ===== 24 1
%% ===== 
%% ===== 
%% ===== 
%% ===== 
%% ===== 
%% ===== 0
%% ===== 
%% ===== 
%% ===== 0
%% ===== 
%% ===== 
%% ===== 
%% ===== 
%% ===== 
%% ===== 0 1 1 1
%% ===== 
%% ===== 
%% ===== 9 0 1 2
%% ===== 
%% ===== 
%% ===== \begin{flushleft}
%% ===== LEVEL 2
%% ===== \end{flushleft}
%% ===== 
%% ===== 
%% ===== 
%% ===== 
%% ===== 
%% ===== 0 0
%% ===== 
%% ===== 
%% ===== 10
%% ===== 
%% ===== 
%% ===== 
%% ===== 
%% ===== 
%% ===== 2
%% ===== 
%% ===== 
%% ===== 
%% ===== 
%% ===== 
%% ===== 3
%% ===== 
%% ===== 
%% ===== 3
%% ===== 
%% ===== 
%% ===== 5
%% ===== 
%% ===== 
%% ===== 
%% ===== 
%% ===== 
%% ===== 2
%% ===== 
%% ===== 
%% ===== 
%% ===== 
%% ===== 
%% ===== 0
%% ===== 
%% ===== 
%% ===== 0
%% ===== 
%% ===== 
%% ===== 
%% ===== 
%% ===== 
%% ===== 3
%% ===== 
%% ===== 
%% ===== 5
%% ===== 
%% ===== 
%% ===== 
%% ===== 
%% ===== 
%% ===== \begin{flushleft}
%% ===== 0 0 0 I F A 0 0 0
%% ===== \end{flushleft}
%% ===== 
%% ===== 
%% ===== 
%% ===== 
%% ===== 
%% ===== 0 1 1 1
%% ===== 
%% ===== 
%% ===== 9 0 1 2
%% ===== 
%% ===== 
%% ===== 
%% ===== 
%% ===== 
%% ===== 10
%% ===== 
%% ===== 
%% ===== 
%% ===== 
%% ===== 
%% ===== 5
%% ===== 
%% ===== 
%% ===== 
%% ===== 
%% ===== 
%% ===== \begin{flushleft}
%% ===== TA
%% ===== \end{flushleft}
%% ===== 
%% ===== 
%% ===== 
%% ===== 
%% ===== 
%% ===== 2 3 3 3 3
%% ===== 
%% ===== 
%% ===== 9 0 1 2 3
%% ===== 
%% ===== 
%% ===== 
%% ===== 
%% ===== 
%% ===== \begin{flushleft}
%% ===== TA
%% ===== \end{flushleft}
%% ===== 
%% ===== 
%% ===== 2
%% ===== 
%% ===== 
%% ===== 
%% ===== 
%% ===== 
%% ===== 2 3 3 3 3
%% ===== 
%% ===== 
%% ===== 9 0 1 2 3
%% ===== 
%% ===== 
%% ===== 
%% ===== 
%% ===== 
%% ===== 3
%% ===== 
%% ===== 
%% ===== 5
%% ===== 
%% ===== 
%% ===== 
%% ===== 
%% ===== 
%% ===== \begin{flushleft}
%% ===== 0 0 0 R F A 0 0 0
%% ===== \end{flushleft}
%% ===== 
%% ===== 
%% ===== 3 1 1 1
%% ===== 
%% ===== 
%% ===== 
%% ===== 
%% ===== 
%% ===== 3
%% ===== 
%% ===== 
%% ===== 3
%% ===== 
%% ===== 
%% ===== 5
%% ===== 
%% ===== 
%% ===== 
%% ===== 
%% ===== 
%% ===== \begin{flushleft}
%% ===== D2 RES
%% ===== \end{flushleft}
%% ===== 
%% ===== 
%% ===== 24
%% ===== 
%% ===== 
%% ===== 
%% ===== 
%% ===== 
%% ===== \newpage
%% ===== 0
%% ===== 
%% ===== 
%% ===== 0
%% ===== 
%% ===== 
%% ===== 
%% ===== 
%% ===== 
%% ===== 2 2 2 2 2
%% ===== 
%% ===== 
%% ===== 3 4 5 6 7
%% ===== 
%% ===== 
%% ===== 
%% ===== 
%% ===== 
%% ===== 6
%% ===== 
%% ===== 
%% ===== 
%% ===== 
%% ===== 
%% ===== \begin{flushleft}
%% ===== D3 PTR
%% ===== \end{flushleft}
%% ===== 
%% ===== 
%% ===== 
%% ===== 
%% ===== 
%% ===== 0
%% ===== 
%% ===== 
%% ===== 
%% ===== 
%% ===== 
%% ===== \begin{flushleft}
%% ===== TA
%% ===== \end{flushleft}
%% ===== 
%% ===== 
%% ===== 
%% ===== 
%% ===== 
%% ===== 24 1
%% ===== 
%% ===== 
%% ===== 0
%% ===== 
%% ===== 
%% ===== 0
%% ===== 
%% ===== 
%% ===== 7
%% ===== 
%% ===== 
%% ===== 
%% ===== 
%% ===== 
%% ===== 2 3 3 3 3
%% ===== 
%% ===== 
%% ===== 9 0 1 2 3
%% ===== 
%% ===== 
%% ===== 
%% ===== 
%% ===== 
%% ===== \begin{flushleft}
%% ===== 0 0 0 R F A
%% ===== \end{flushleft}
%% ===== 
%% ===== 
%% ===== 
%% ===== 
%% ===== 
%% ===== 2
%% ===== 
%% ===== 
%% ===== 
%% ===== 
%% ===== 
%% ===== 3
%% ===== 
%% ===== 
%% ===== 5
%% ===== 
%% ===== 
%% ===== \begin{flushleft}
%% ===== JMP
%% ===== \end{flushleft}
%% ===== 
%% ===== 
%% ===== 
%% ===== 
%% ===== 
%% ===== 3 1 1 1
%% ===== 
%% ===== 
%% ===== 
%% ===== 
%% ===== 
%% ===== 1 1
%% ===== 
%% ===== 
%% ===== 1 2
%% ===== 
%% ===== 
%% ===== 
%% ===== 
%% ===== 
%% ===== 0 0 0 0 0 0 0 0 0 0 0 0
%% ===== 
%% ===== 
%% ===== 
%% ===== 
%% ===== 
%% ===== 3
%% ===== 
%% ===== 
%% ===== 3
%% ===== 
%% ===== 
%% ===== 5
%% ===== 
%% ===== 
%% ===== 
%% ===== 
%% ===== 
%% ===== \begin{flushleft}
%% ===== D3 RES
%% ===== \end{flushleft}
%% ===== 
%% ===== 
%% ===== 
%% ===== 
%% ===== 
%% ===== 12
%% ===== 
%% ===== 
%% ===== 
%% ===== 
%% ===== 
%% ===== 24
%% ===== 
%% ===== 
%% ===== 
%% ===== 
%% ===== 
%% ===== \begin{flushleft}
%% ===== Figure 3-33. Decimal Unit Data Format
%% ===== \end{flushleft}
%% ===== 
%% ===== 
%% ===== \begin{flushleft}
%% ===== Description:
%% ===== \end{flushleft}
%% ===== 
%% ===== 
%% ===== \begin{flushleft}
%% ===== A collection of flags and registers from the decimal unit.
%% ===== \end{flushleft}
%% ===== 
%% ===== 
%% ===== \begin{flushleft}
%% ===== Function:
%% ===== \end{flushleft}
%% ===== 
%% ===== 
%% ===== \begin{flushleft}
%% ===== The decimal unit data allows the processor to restart an EIS instruction at the point of
%% ===== \end{flushleft}
%% ===== 
%% ===== 
%% ===== \begin{flushleft}
%% ===== interruption when it is interrupted by an access violation fault, a directed fault, or (for
%% ===== \end{flushleft}
%% ===== 
%% ===== 
%% ===== \begin{flushleft}
%% ===== certain EIS instructions) an interrupt. Directed faults are intentional, and most access
%% ===== \end{flushleft}
%% ===== 
%% ===== 
%% ===== \begin{flushleft}
%% ===== violation faults and interrupts are recoverable.
%% ===== \end{flushleft}
%% ===== 
%% ===== 
%% ===== \begin{flushleft}
%% ===== The data are restored with the Load Pointers and Lengths (lpl) instruction. Fields having
%% ===== \end{flushleft}
%% ===== 
%% ===== 
%% ===== \begin{flushleft}
%% ===== an {``}x'' in the column headed L are not restored. When starting (or restarting) execution of
%% ===== \end{flushleft}
%% ===== 
%% ===== 
%% ===== \begin{flushleft}
%% ===== an EIS instruction, the decimal unit registers and flags are not initialized from the operand
%% ===== \end{flushleft}
%% ===== 
%% ===== 
%% ===== \begin{flushleft}
%% ===== descriptors if the mid-instruction interrupt fault (MIF) indicator is set ON.
%% ===== \end{flushleft}
%% ===== 
%% ===== 
%% ===== \begin{flushleft}
%% ===== The meanings of the constituent flags and registers are:
%% ===== \end{flushleft}
%% ===== 
%% ===== 
%% ===== 
%% ===== 
%% ===== 
%% ===== \begin{flushleft}
%% ===== Word L Field Name
%% ===== \end{flushleft}
%% ===== 
%% ===== 
%% ===== 
%% ===== 
%% ===== 
%% ===== \begin{flushleft}
%% ===== Meaning
%% ===== \end{flushleft}
%% ===== 
%% ===== 
%% ===== 
%% ===== 
%% ===== 
%% ===== 0
%% ===== 
%% ===== 
%% ===== 
%% ===== 
%% ===== 
%% ===== \begin{flushleft}
%% ===== Z
%% ===== \end{flushleft}
%% ===== 
%% ===== 
%% ===== 
%% ===== 
%% ===== 
%% ===== \begin{flushleft}
%% ===== All bit-string instruction results are zero
%% ===== \end{flushleft}
%% ===== 
%% ===== 
%% ===== 
%% ===== 
%% ===== 
%% ===== 0
%% ===== 
%% ===== 
%% ===== 
%% ===== 
%% ===== 
%% ===== \begin{flushleft}
%% ===== {\O}
%% ===== \end{flushleft}
%% ===== 
%% ===== 
%% ===== 
%% ===== 
%% ===== 
%% ===== \begin{flushleft}
%% ===== Negative overpunch found in 6-4 expanded move
%% ===== \end{flushleft}
%% ===== 
%% ===== 
%% ===== 
%% ===== 
%% ===== 
%% ===== 0
%% ===== 
%% ===== 
%% ===== 
%% ===== 
%% ===== 
%% ===== \begin{flushleft}
%% ===== CHTALLY
%% ===== \end{flushleft}
%% ===== 
%% ===== 
%% ===== 
%% ===== 
%% ===== 
%% ===== \begin{flushleft}
%% ===== The number of characters examined by the scm, scmr, scd,
%% ===== \end{flushleft}
%% ===== 
%% ===== 
%% ===== \begin{flushleft}
%% ===== scdr, tct, or tctr instructions (up to the interrupt or match)
%% ===== \end{flushleft}
%% ===== 
%% ===== 
%% ===== 
%% ===== 
%% ===== 
%% ===== 2
%% ===== 
%% ===== 
%% ===== 
%% ===== 
%% ===== 
%% ===== \begin{flushleft}
%% ===== D1 PTR
%% ===== \end{flushleft}
%% ===== 
%% ===== 
%% ===== 
%% ===== 
%% ===== 
%% ===== \begin{flushleft}
%% ===== Address of the last double-word accessed by operand descriptor 1;
%% ===== \end{flushleft}
%% ===== 
%% ===== 
%% ===== \begin{flushleft}
%% ===== bits 17-23 (bit-address) valid only for initial access
%% ===== \end{flushleft}
%% ===== 
%% ===== 
%% ===== 
%% ===== 
%% ===== 
%% ===== \begin{flushleft}
%% ===== TA
%% ===== \end{flushleft}
%% ===== 
%% ===== 
%% ===== 
%% ===== 
%% ===== 
%% ===== \begin{flushleft}
%% ===== Alphanumeric type of operand descriptor 1,2,3
%% ===== \end{flushleft}
%% ===== 
%% ===== 
%% ===== 
%% ===== 
%% ===== 
%% ===== 2,4,6
%% ===== 
%% ===== 
%% ===== 2
%% ===== 
%% ===== 
%% ===== 
%% ===== 
%% ===== 
%% ===== \begin{flushleft}
%% ===== x I
%% ===== \end{flushleft}
%% ===== 
%% ===== 
%% ===== 
%% ===== 
%% ===== 
%% ===== \begin{flushleft}
%% ===== Decimal unit interrupted flag; a copy of the mid-instruction
%% ===== \end{flushleft}
%% ===== 
%% ===== 
%% ===== \begin{flushleft}
%% ===== interrupt fault indicator
%% ===== \end{flushleft}
%% ===== 
%% ===== 
%% ===== 
%% ===== 
%% ===== 
%% ===== 2,4,6
%% ===== 
%% ===== 
%% ===== 
%% ===== 
%% ===== 
%% ===== \begin{flushleft}
%% ===== F
%% ===== \end{flushleft}
%% ===== 
%% ===== 
%% ===== 
%% ===== 
%% ===== 
%% ===== \begin{flushleft}
%% ===== First time; data in operand descriptor 1,2,3 is valid
%% ===== \end{flushleft}
%% ===== 
%% ===== 
%% ===== 
%% ===== 
%% ===== 
%% ===== 2,4,6
%% ===== 
%% ===== 
%% ===== 
%% ===== 
%% ===== 
%% ===== \begin{flushleft}
%% ===== A
%% ===== \end{flushleft}
%% ===== 
%% ===== 
%% ===== 
%% ===== 
%% ===== 
%% ===== \begin{flushleft}
%% ===== Operand descriptor 1,2,3 is active
%% ===== \end{flushleft}
%% ===== 
%% ===== 
%% ===== 
%% ===== 
%% ===== 
%% ===== 3
%% ===== 
%% ===== 
%% ===== 
%% ===== 
%% ===== 
%% ===== \begin{flushleft}
%% ===== LEVEL l
%% ===== \end{flushleft}
%% ===== 
%% ===== 
%% ===== 
%% ===== 
%% ===== 
%% ===== \begin{flushleft}
%% ===== Difference in the count of characters loaded into the processor
%% ===== \end{flushleft}
%% ===== 
%% ===== 
%% ===== \begin{flushleft}
%% ===== and characters not acted upon
%% ===== \end{flushleft}
%% ===== 
%% ===== 
%% ===== 
%% ===== 
%% ===== 
%% ===== 3
%% ===== 
%% ===== 
%% ===== 
%% ===== 
%% ===== 
%% ===== \begin{flushleft}
%% ===== D1 RES
%% ===== \end{flushleft}
%% ===== 
%% ===== 
%% ===== 
%% ===== 
%% ===== 
%% ===== \begin{flushleft}
%% ===== Count of characters remaining in operand descriptor l
%% ===== \end{flushleft}
%% ===== 
%% ===== 
%% ===== 
%% ===== 
%% ===== 
%% ===== 4
%% ===== 
%% ===== 
%% ===== 
%% ===== 
%% ===== 
%% ===== \begin{flushleft}
%% ===== D2 PTR
%% ===== \end{flushleft}
%% ===== 
%% ===== 
%% ===== 
%% ===== 
%% ===== 
%% ===== \begin{flushleft}
%% ===== Address of the last double-word accessed by operand descriptor 2;
%% ===== \end{flushleft}
%% ===== 
%% ===== 
%% ===== \begin{flushleft}
%% ===== bits 17-23 (bit-address) valid only for initial access
%% ===== \end{flushleft}
%% ===== 
%% ===== 
%% ===== 
%% ===== 
%% ===== 
%% ===== 4,6
%% ===== 
%% ===== 
%% ===== 5
%% ===== 
%% ===== 
%% ===== 
%% ===== 
%% ===== 
%% ===== \begin{flushleft}
%% ===== x R
%% ===== \end{flushleft}
%% ===== 
%% ===== 
%% ===== \begin{flushleft}
%% ===== LEVEL 2
%% ===== \end{flushleft}
%% ===== 
%% ===== 
%% ===== 
%% ===== 
%% ===== 
%% ===== \begin{flushleft}
%% ===== Last cycle performed must be repeated
%% ===== \end{flushleft}
%% ===== 
%% ===== 
%% ===== \begin{flushleft}
%% ===== Same as LEVEL 1, but used mainly for OP 2 information
%% ===== \end{flushleft}
%% ===== 
%% ===== 
%% ===== 
%% ===== 
%% ===== 
%% ===== \begin{flushleft}
%% ===== \newpage
%% ===== Word L Field Name
%% ===== \end{flushleft}
%% ===== 
%% ===== 
%% ===== 
%% ===== 
%% ===== 
%% ===== \begin{flushleft}
%% ===== Meaning
%% ===== \end{flushleft}
%% ===== 
%% ===== 
%% ===== 
%% ===== 
%% ===== 
%% ===== 5
%% ===== 
%% ===== 
%% ===== 
%% ===== 
%% ===== 
%% ===== \begin{flushleft}
%% ===== D2 RES
%% ===== \end{flushleft}
%% ===== 
%% ===== 
%% ===== 
%% ===== 
%% ===== 
%% ===== \begin{flushleft}
%% ===== Count of characters remaining in operand descriptor 2
%% ===== \end{flushleft}
%% ===== 
%% ===== 
%% ===== 
%% ===== 
%% ===== 
%% ===== 6
%% ===== 
%% ===== 
%% ===== 
%% ===== 
%% ===== 
%% ===== \begin{flushleft}
%% ===== D3 PTR
%% ===== \end{flushleft}
%% ===== 
%% ===== 
%% ===== 
%% ===== 
%% ===== 
%% ===== \begin{flushleft}
%% ===== Address of the last double-word accessed by operand descriptor 3;
%% ===== \end{flushleft}
%% ===== 
%% ===== 
%% ===== \begin{flushleft}
%% ===== bits 17-23 (bit-address) valid only for initial access
%% ===== \end{flushleft}
%% ===== 
%% ===== 
%% ===== 
%% ===== 
%% ===== 
%% ===== 6
%% ===== 
%% ===== 
%% ===== 
%% ===== 
%% ===== 
%% ===== \begin{flushleft}
%% ===== JMP
%% ===== \end{flushleft}
%% ===== 
%% ===== 
%% ===== 
%% ===== 
%% ===== 
%% ===== \begin{flushleft}
%% ===== Descriptor count; number of words to skip to find the next
%% ===== \end{flushleft}
%% ===== 
%% ===== 
%% ===== \begin{flushleft}
%% ===== instruction following this multiword instruction
%% ===== \end{flushleft}
%% ===== 
%% ===== 
%% ===== 
%% ===== 
%% ===== 
%% ===== 7
%% ===== 
%% ===== 
%% ===== 
%% ===== 
%% ===== 
%% ===== \begin{flushleft}
%% ===== D3 RES
%% ===== \end{flushleft}
%% ===== 
%% ===== 
%% ===== 
%% ===== 
%% ===== 
%% ===== \begin{flushleft}
%% ===== Count of characters remaining in operand descriptor 3
%% ===== \end{flushleft}
%% ===== 
%% ===== 
%% ===== 
%% ===== 
%% ===== 
%% ===== \begin{flushleft}
%% ===== \newpage
