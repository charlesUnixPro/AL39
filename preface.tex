\addcontentsline{toc}{section}{PREFACE}
\begin{center}
\Large\bfseries PREFACE
\end{center}

This manual describes the processors used in the Multics system. These are the
DPS/L68, which refers to the DPS, L68 or older model processors (excluding the
GE-645) and DPS 8M, which refers to the DPS 8 family of Multics processors,
i.e. DPS 8/70M, DPS 8/62M and DPS 8/52M.  The reader should be familiar with
the overall modular organization of the Multics system and with the philosophy
of asynchronous operation. In addition, this manual presents a discussion of
virtual memory addressing concepts including segmentation and paging.


The manual is intended for use by systems programmers responsible for writing
software to interface with the virtual memory hardware and with the fault and
interrupt portions of the hardware. It should also prove valuable to
programmers who must use machine instructions (particularly language translator
implementors) and to those persons responsible for analyzing crash conditions
in system dumps.

This manual includes the processor capabilities, modes of operation, functions,
and detailed descriptions of machine instructions. Data representation,
program-addressable registers, addressing by means of segmentation and paging,
faults and interrupts, hardware ring implementation, and cache operation are
also covered.  

\vfill

\small
The information and specifications in this document are subject to change
without notice. Consult your Honeywell Marketing Representative for product or
service availability.

\copyright Honeywell Information Systems Inc., 1985 \hspace{1cm} File No.: 1L03, 1L53
\hfill 11/85

\hfill AL39-01C

\normalsize
