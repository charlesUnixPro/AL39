
\section{VIRTUAL ADDRESS FORMATION}
\label{s6}

\subsection{DEFINITION OF VIRTUAL ADDRESS}

The virtual address in the Multics processor is the user's specification of the
location of a data item in the Multics virtual memory. Each reference to the
virtual memory for operands, indirect words, indirect pointers, operand
descriptors, or instructions must provide a virtual address. The hardware and
the operating system translate the virtual address into the true location of
the data item and assure that the data item is in main memory for the
reference.

The virtual address consists of two parts, an effective segment number and an
offset or computed address. The value of each part is the result of the
evaluation of a hardware algorithm (expression) of one or more terms. The
selection of the algorithm is made by the use of control bits in the
instruction word; for example, bit 29 for modification by pointer register and
bits 30-35 (the TAG field) for modification by index register or indirect word.
For certain modifications by indirect word, the TAG field of the indirect word
is also treated as an address modifier, thus establishing a continuing
{``}indirect chain''. Bit 29 of an indirect word has no meaning in the context
of virtual address formation.


The results of evaluation of the virtual address formation algorithms are
stored in temporary registers used as working registers by the processor. The
effective segment number is stored in the temporary segment register, TPR.TSR.
The offset is stored in the computed address register, TPR.CA. When each
virtual address computation has been completed, C(TPR.TSR) and C(TPR.CA) are
presented to the appending unit for translation to a 24-bit absolute main
memory address (see Section 5).

\subsection{TYPES OF VIRTUAL ADDRESS FORMATION}

There are two types of virtual address formation. The first type does not make
explicit use of segment numbers. The algorithms produce values for the computed
address, C(TPR.CA), only.  The effective segment number in C(TPR.TSR) does not
change from the value used to fetch the current instruction. In this case, all
references are said to be {``}local'' to the procedure segment pointed to by
the procedure pointer register (PPR).


The second type makes use of a segment number in an indirect word-pair in main
memory or in a pointer register (PRn). The algorithms produce values for both
the effective segment number, C(TPR.TSR), and the computed address, C(TPR.CA).
The effective segment number in C(TPR.TSR) may change and, if it changes,
references are said to be {``}external'' to the procedure segment.


Both types of virtual address formation for the operand of a basic or EIS
single-word instruction begin with a preliminary step of loading TPR.CA with
the ADDRESS field of the instruction word. This preliminary step takes place
during instruction decode.  

The two types of virtual address formation can be intermixed. In cases where
virtual address calculations are chained together through pointer registers or
indirect words, each virtual address is translated to a 24-bit absolute main
memory address to fetch the next item in the chain.  

This description of virtual address formation is divided into two parts
corresponding to the two types. The first part describes the type that involves
only the computed address, C(TPR.CA).  The effective segment number is
constant. In append mode its value is equal to C(PPR.PSR) (a local reference)
and in absolute mode its value is undefined.


The second part describes the type that involves both the effective segment
number, C(TPR.TSR), and the computed address, C(TPR.CA).

\subsection{SYMBOLOGY (ALM)}

In many instances in the discussions that follow, references to the features of
the ALM assembly program are unavoidable. Such references are explained briefly
here. The reader is advised to consult the appropriate software documentation
for further details and for possible changes in the various features.

\subsubsection{Symbolic Fields}

A symbolic field is an expression consisting of variables, constants, literals,
and operators that is evaluated by ALM to produce a value for the corresponding
field of a machine word. The values of the variables and constants are either
known or assignable and the operators are defined for the mode of the
evaluation (algebraic, logical, etc.). The necessary fields for a machine
instruction or ALM pseudo-instruction are given as a comma-separated string of
expressions.

\subsubsection{ALM Pseudo-Instructions}

The following ALM pseudo-instructions are used in this section:


aci string


This pseudo-instruction generates a sequence of 9-bit byte fields each of which
contains the ASCII octal value for the corresponding graphic character in
string. The last machine word generated is low-order filled with binary 0s to
the next word boundary.  

arg address,tag


This pseudo-instruction generates a machine word with the same format as the
basic and EIS single-word instructions but having binary 0s in the operation
code field.  

bci string

This pseudo-instruction generates a sequence of 6-bit character fields each of
which contains the binary coded decimal (BCD) octal value for the corresponding
graphic character in string. The last machine word generated is low-order
filled with binary 0s to the next word boundary.

vfd field1,field2, ... ,fieldn

This pseudo-instruction generates a machine word (or word-pair) containing an
arbitrary number of fields of arbitrary length up to a total bit count of 72.
The data generated is leftjustified in the machine word (or word-pair) and zero
filled to the next word boundary as necessary.

Each fieldi is given as:

md/expr

where: m is the data conversion mode and may be:

null for arithmetic operators and decimal literals,


o for Boolean operators and octal literals,

h for 6-bit character binary coded decimal (BCD) character strings, or

a for 9-bit byte ASCII character strings.

d is a literal giving the field width in bits and may have any value from 1 to
72.

expr is the expression to be evaluated or converted. Conversion is done with
full 36-bit precision and the field value is the conversion result modulo the
field width.

\subsection{COMPUTED ADDRESS FORMATION}

The address formation algorithms described here produce values only for the
computed address. The effective segment number is constant and equal to
C(PPR.PSR) if the processor is in append mode or is undefined if the processor
is in absolute mode.

\subsubsection{The Address Modifier (TAG) Field}

Bits 30-35 of an instruction word or indirect word constitute the address modifier or TAG field. The format of the TAG field is:

Figure 6-1. Address Modifier (TAG) Field Format


Field Name Function

Tm modifier field, specifies one of four general types of computed address
modification

Td designator field, selects among several variations available for the general
type given with Tm

\subsubsection{General Types of Computed Address Modification}

There are four general types of computed address modification: register,
register then indirect, indirect then register, and indirect then tally. The
general types are described in Table 6-1. The value loaded into TPR.CA is
symbolized by {``}y'' in the descriptions following.  

Table 6-1. General Computed Address Modification Types


Tm value Type Description

0 Register (r) The contents of the register specified in C(Td) are added to the
current computed address, C(TPR.CA), to form the modified computed address.
Addition is twos complement, modulo 218, and overflow does not occur.

1 Register then indirect (ri) The contents of the register specified in C(Td)
are added to the current computed address, C(TPR.CA), to form the modified
computed address as for register modification. The modified C(TPR.CA) is then
used to fetch an indirect word. The TAG field of the indirect word specifies
the next step in computed address formation. The use of du or dl as the
designator in this modification type will cause an illegal procedure, illegal
modifier, fault.  

2 Indirect then tally (it) The indirect word at C(TPR.CA) is fetched and the
modification performed according to the variation specified in C(Td) of the
instruction word and the contents of the indirect word. This modification type
allows automatic incrementing and decrementing of addresses and tally counting.

3 Indirect then register (ir) The register designator, C(Td), is safe-stored in a special holding register, CTHOLD. The word at C(TPR.CA) is fetched and interpreted as an indirect word. The TAG field of the indirect word specifies the next step in computed address formation as follows:

Indirect TAG Next step

r or it Perform register modification using Td from CT-HOLD.(1)

ri Perform the register then indirect modification immediately and fetch the
next indirect word from the result of that modification.  

ir Replace the safe-stored Td value in CT-HOLD with the Td value from the
indirect word TAG field and use the ADDRESS field of the indirect word as a
computed address value to fetch the next indirect word.

(1)In this instance, the indirect then tally variations fault tag 1, fault tag
2, and fault tag 3 are treated differently. The fault tag 1 variation results
in the action described here but fault tag 2 and fault tag 3 result in the
generation of a fault. See the discussion of indirect then tally modification
later in this section.

\subsubsection{Computed Address Formation Flowcharts}

The flowcharts depicting the computed address formation process are scattered
throughout this section and are linked together by figure references. The
flowcharts start with Figure 6-2.  

Figure 6-2. Common Computed Address Formation Flowchart

\subsubsection{Register (r) Modification}

In register modification (Tm = 0) the value of Td designates a register whose
contents are to be added to C(TPR.CA) to form a modified C(TPR.CA). This
modified C(TPR.CA) becomes the computed address of the operand. See Figure 6-3,
Table 6-2, and the examples following.

Figure 6-3. Register Modification Flowchart

Table 6-2. Register Modification Decode


Td value Register Coding Symbol Computed Address

0 none n, null y

1 A0,17 au y + C(A)0,17

2 Q0,17 qu y + C(Q)0,17 

3 none du none; y becomes the upper 18 bits of the 36-bit zero filled operand

4 PPR.IC ic y +C(PPR.IC) 

5 A18,35 al y +C(A)18,35 

6 Q18,35 ql y +C(Q)18,35

7 none dl none; y becomes the lower 18 bits of the 36-bit zero filled operand

10 X0 0, x0 y +C(X0)

11 X1 1, x1 y +C(X1)

12 X2 2, x2 y +C(X2) 

13 X3 3, x3 y +C(X3)

14 X4 4, x4 y +C(X4) 

15 X5 5, x5 y +C(X5) 

16 X6 6, x6 y +C(X6)

17 X7 7, x7 y +C(X7) 

Examples:

Location Instruction Computed address

1.  a lda y y

2.  a sta y,n y 

3.  a ldaq y,au y + C(A)0,17 

4.  a tra 3,ic a+3 

5.  a ldq y,du none; operand has the form y || (00...0)18 

6.  a lxl4 y,dl none; operand has the form (00...0)18 || y 

7.  a mpy y,1 y + C(X1) 

8.  a stx4 y,7 y + C(X7)

\subsubsection{Register Then Indirect (ri) Modifications}

In register then indirect modification (Tm = 1) the value of Td designates a
register whose contents are to be added to C(TPR.CA) to form a modified
C(TPR.CA). This modified C(TPR.CA) is used as a computed address to fetch an
indirect word. The ADDRESS field of the indirect word is loaded into TPR.CA and
the TAG field of the indirect word is interpreted in the next step of an
indirect chain. The TALLY field of the indirect word is ignored.

The indirect chain continues until an indirect word TAG field specifies a
modification without indirection.

The coding symbol for register then indirect modification is r* where r is any
of the coding symbols for register modification as given in Table 6-1 above
except du and dl. The du and dl register codes are illegal and and their use
causes an illegal procedure, illegal modifier, fault. See Figure 6-4, Table
6-1, and the examples following.

Figure 6-4. Register Then Indirect Modification Flowchart

Examples:

Location Instruction Computed address 

1. a lda b,* (r = null)

b arg y y

2. a ldq b,1*

b+C(X1) arg y,au y + C(A)0,17

3. a tra 4,ic*

a+4 arg c,*

c arg y y

4. a lxl4 b,0*
b+C(X0) arg c,1*
c+C(X1) arg y,dl none; operand has the form (00...0)18 || y

\subsubsection{Indirect Then Register (ir) Modification}

In indirect then register modification (Tm = 3) the value of Td designates a
register whose contents are to be added to C(TPR.CA) to form the final modified
C(TPR.CA) during the last step in the indirect chain. The value of Td is held
in a special holding register, CT-HOLD. The initial C(TPR.CA) is used as
computed address to fetch an indirect word. The ADDRESS of the indirect word is
loaded into TPR.CA and the TAG field of the indirect word is interpreted in the
next step of an indirect chain. The TALLY field of the indirect word is
ignored.


If the indirect word TAG field specifies a register then indirect modification,
that modification is performed and the indirect chain continues.  

If the indirect word TAG field specifies indirect then register modification,
the Td value from that TAG field replaces the Td value in CT-HOLD and the
indirect chain continues.  

If the indirect word TAG specifies register or indirect then tally
modification, that modification is replaced with a register modification using
the Td value in CT-HOLD and the indirect chain ends.

The coding symbol for indirect then register modification is * r where r is any
of the coding symbols for register modification as given in Table 6-2 except
null. See Figure 6-5, Table 6-1, and the examples following.

Figure 6-5. Indirect Then Register Modification Flowchart

Examples:

Location Instruction Computed address

1. a lda b,*n (CT-HOLD = n)

b arg y,2

2. a lxl2 b,*dl (CT-HOLD = dl)

b sta y,au none; operand has the form (00...0)18 || y

3. a lda b,1* (CT-HOLD = x1)

b arg c,n*

c arg d,*4 (CT-HOLD = x4)

d arg y,ql y + C(X4)

4. a ldx0 b,1*

b+C(X1) arg c,*ic (CT-HOLD = ic)

c arg 5,dl a+5

\subsubsection{Indirect Then Tally (it) Modification}

In indirect then tally modification (Tm = 2) the value of Td specifies a
variation. The initial C(TPR.CA) is used an as computed address to fetch an
indirect word. The indirect word is interpreted and possibly altered as the
modification is performed. If the specified variation involves alteration of
the indirect word, the indirect word is fetched with a special main memory
cycle that prevents other processors from accessing it until the alteration is
complete.


The TALLY field of the indirect word is used to count references made to the
indirect word.  It has a maximum range of 4096. If the TALLY field has the
value 0 after a reference to the indirect word, the tally runout indicator will
be set ON, otherwise the tally runout indicator is set OFF. The value of the
TALLY field and the state of the tally runout indicator have no effect on
computed address formation.


If there is more than one indirect word in an indirect chain that is referenced
by a tally counting variation, only the state of the TALLY field of the last
such word is reflected in the tally runout indicator.


The variations of the indirect then tally modification are given in Table 6-3
and explained in detail in the paragraphs following. Those entries given as
{``}Undefined'' cause an illegal procedure, illegal modifier, fault. See Figure
6-6, Table 6-1, and the examples following.

Table 6-3. Variations of Indirect Then Tally Modification

Td value Coding symbol Computed address

0 f1 Fault tag 1

1 Undefined (see itp modification later in this section)

2 Undefined 

3 Undefined (see its modification later in this section)

4 sd Subtract delta

5 scr Sequence character reverse

6 f2 Fault tag 2

7 f3 Fault tag 3 

10 ci Character indirect 

11 i Indirect 

12 sc Sequence character

13 ad Add delta

14 di Decrement address, increment tally

15 dic Decrement address, increment tally, and continue 

16 id Increment address, decrement tally

17 idc Increment address, decrement tally, and continue 

Fault tag 1 (Td = 0)


If this variation appears in an indirect word and the TAG of the instruction
word or preceding indirect word is indirect then register (ir), then terminate
computed address formation with a register (r) modification using the register
held in CT-HOLD. If this variation appears in an instruction word or in an
indirect word and the TAG of the instruction word or preceding indirect word is
not indirect then register (ir), then generate a fault tag 1 fault.


C(TPR.CA) at the time of the fault contains the computed address of the word
containing the fault tag 1 variation. Thus, the ADDRESS and TALLY fields of
that word may contain information relative to recovery from the fault.

Subtract delta (Td = 4)

The TAG field of the indirect word is interpreted as a 6-bit, unsigned,
positive address increment value, delta. For each reference to the indirect
word, the ADDRESS field is reduced by delta and the TALLY field is increased by
1 before the computed address is formed. ADDRESS arithmetic is modulo 218.
TALLY arithmetic is modulo 4096. If the TALLY field overflows to 0, the tally
runout indicator is set ON, otherwise it is set OFF. The computed address is
the value of the decremented ADDRESS field of the indirect word.  

Example:

Location Instruction Reference count Computed address Tally value

a lda b,sd 1 c-d t+l

b vfd 18/c,12/t,6/d 2 c-2d t+2

3 c-3d t+3

...

n c-nd t+n

Sequence character reverse (Td = 5)

Bit 30 of the TAG field of the indirect word is interpreted as a character size flag, tb, with the value 0 indicating 6-bit characters and the value 1 indicating 9-bit bytes. Bits 33-35 of the TAG field are interpreted as a 3-bit character/byte position counter, cf. Bits 31-32 of the TAG field must be zero.


For each reference to the indirect word, the character counter, cf, is reduced
by 1 and the TALLY field is increased by 1 before the computed address is
formed. Character count arithmetic is modulo 6 for 6-bit characters and modulo
4 for 9-bit bytes. If the character count, cf, underflows to -1, it is reset to
5 for 6-bit characters or to 3 for 9-bit bytes and ADDRESS is reduced by 1.
ADDRESS arithmetic is modulo 218. TALLY arithmetic is modulo 4096. If the TALLY
field overflows to 0, the tally runout indicator is set ON, otherwise it is set
OFF. The computed address is the (possibly) decremented value of the ADDRESS
field of the indirect word.  The effective character/byte number is the
decremented value of the character position count, cf, field of the indirect
word.  

A 36-bit operand is formed by high-order zero filling the value of character
cf-l of C(computed address) with an appropriate number of bits .


Examples:





Location Instruction Reference count cf Computed address Tally value Operand

a lda b,scr 1 2 c+l t+l (00...0)30 || {``}I''

b vfd 18/c+1,12/t,1/0,5/3 2 1 c+l t+2 (00...0)30 || {``}H''

c bci {``}ABCDEFGHIJKL'' 3 0 c+l t+3 (00...0)30 || {``}G''

4 5 c t+4 (00...0)30 || {``}F''

5 4 c t+5 (00...0)30 || {``}E''

...

a lda b,scr 1 2 c+l t+l (00...0)27 || {``}g''

b vfd 18/c+1,12/t,1/1,5/3 2 1 c+l t+2 (00...0)27 || ''f''

c aci {``}abcdefgh'' 3 0 c+l t+3 (00...0)27 || {``}e''

4 3 c t+4 (00...0)27 || {``}d''

5 2 c t+5 (00...0)27 || {``}c''

...


Fault tag 2 (Td = 6)


Terminate computed address formation immediately and generate a fault tag 2
fault.


C(TPR.CA) at the time of the fault contains the computed address of the word
containing the fault tag 2 variation. Thus, the ADDRESS and TALLY fields of
that word may contain information relative to recovery from the fault.


Fault tag 3 (Td = 7)

Terminate computed address formation immediately and generate a fault tag 3
fault.  C(TPR.CA) at the time of the fault contains the computed address of the
word containing the fault tag 3 variation. Thus, the ADDRESS and TALLY fields
of that word may contain information relative to recovery from the fault.  

Character indirect (Td = 10)

Bit 30 of the TAG field of the indirect word is interpreted as a character size
flag, tb, with the value 0 indicating 6-bit characters and the value 1
indicating 9-bit bytes. Bits 33-35 of the TAG field are interpreted as a 3-bit
character/byte position value, cf. Bits 31-32 of the TAG field must be zero.

If the character position value is greater than 5 for 6-bit characters or
greater than 3 for 9bit bytes, an illegal procedure, illegal modifier, fault
will occur. The TALLY field is ignored.  The computed address is the value of
the ADDRESS field of the indirect word. The effective character/byte number is
the value of the character position count, cf, field of the indirect word.

A 36-bit operand is formed by high-order zero filling the value of character cf
of C(computed address) with an appropriate number of bits .


Examples:

Location Instruction Operand

a lda b,ci

b vfd 18/c+1,12/0,1/0,5/2 (00...0)30 || {``}I''

c bci {``}ABCDEFGHIJKL''

a lda

d,ci (00...0)30 || {``}B'''

d vfd 18/c,12/0,1/0,5/1

a lda e,ci

e vfd 18/f,12/0,1/1,5/3 (00...0)27 || {``}d''

f aci {``}abcdefgh''

a lda g,ci

g vfd 18/f+1,12/0,1/1,5/0 (00...0)27 || {``}e''

Indirect (Td = 11)

The computed address is the value of the ADDRESS field of the indirect word. The TALLY and TAG fields of the indirect word are ignored.


Sequence character (Td = 12)

Bit 30 of the TAG field of the indirect word is interpreted as a character size
flag, tb, with the value 0 indicating 6-bit characters and the value 1
indicating 9-bit bytes. Bits 33-35 of the TAG field are interpreted as a 3-bit
character position counter, cf. Bits 31-32 of the TAG field must be zero.


For each reference to the indirect word, the character counter, cf, is
increased by 1 and the TALLY field is reduced by 1 after the computed address
is formed. Character count arithmetic is modulo 6 for 6-bit characters and
modulo 4 for 9-bit bytes. If the character count, cf, overflows to 6 for 6-bit
characters or to 4 for 9-bit bytes, it is reset to 0 and ADDRESS is increased
by 1. ADDRESS arithmetic is modulo 218. TALLY arithmetic is modulo 4096. If the
TALLY field is reduced to 0, the tally runout indicator is set ON, otherwise it
is set OFF. The computed address is the unmodified value of the ADDRESS field.
The effective character/byte number is the unmodified value of the character
position counter, cf, field of the indirect word.


A 36-bit operand is formed by high-order zero filling the value of character of
of C(computed address) with an appropriate number of bits .


Examples:

Location Instruction Reference count cf Computed address Tally value Operand

a lda b,sc 1 4 c t-1 (00...0)30 || {``}E''

b vfd 18/c,12/t,1/0,5/4 2 5 c t-2 (00...0)30 || {``}F''

c bci {``}ABCDEFGHIJKL'' 3 0 c+l t-3 (00...0)30 || {``}G''

4 1 c+l t-4 (00...0)30 || {``}H''

5 2 c+l t-5 (00...0)30 || {``}I''

...

a lda b,sc 1 2 c t-1 (00...0)27 || {``}c''

b vfd 18/c,12/t,1/1,5/2 2 3 c t-2 (00...0)27 || {``}d''

c aci {``}abcdefgh'' 3 0 c+l t-3 (00...0)27 || {``}e''

4 1 c+l t-4 (00...0)27 || {``}f''

5 2 c+l t-5 (00...0)27 || {``}g''

...


Add delta (Td = 13)


The TAG field of the indirect word is interpreted as a 6-bit, unsigned,
positive address increment value, delta. For each reference to the indirect
word, the ADDRESS field is increased by delta and the TALLY field is reduced by
1 after the computed address is formed. ADDRESS arithmetic is modulo 218. TALLY
arithmetic is modulo 4096. If the TALLY field is reduced to 0, the tally runout
indicator is set ON, otherwise it is set OFF.  The computed address is the
value of the unmodified ADDRESS field of the indirect word.


Example:

Location Instruction Reference count Computed address Tally value

a lda b,ad 1 c t-1

b vfd 18/c,1/t,6/d 2 c+d t-2

3 c+2d t-3

...

n c+(n-l)d t-n

Decrement address, increment tally (Td = 14)


For each reference to the indirect word, the ADDRESS field is reduced by 1 and
the TALLY field is increased by 1 before the computed address is formed.
ADDRESS arithmetic is modulo 218. TALLY arithmetic is modulo 4096. If the TALLY
field overflows to 0, the tally runout indicator is set ON, otherwise it is set
OFF. The TAG field of the indirect word is ignored. The computed address is the
value of the decremented ADDRESS field.


Example:

Location Instruction Reference count Computed address Tally value

a lda b,di 1 c-1 t+l

b vfd 18/c,12/t 2 c-2 t+2

3 c-3 t+3

...

n c-n t+n

Decrement address, increment tally, and continue (Td = 15)

The action for this variation is identical to that for the decrement address, increment tally variation except that the TAG field of the indirect word is interpreted and continuation of the indirect chain is possible. If the TAG of the indirect word invokes a register, that is, specifies r, ri, or ir modification, the effective Td value for the register is forced to {``}null'' 
before the next computed address is formed .

Increment address, decrement tally (Td = 16)

For each reference to the indirect word, the ADDRESS field is increased by 1
and the TALLY field is reduced by 1 after the computed address is formed.
ADDRESS arithmetic is modulo 218. TALLY arithmetic is modulo 4096. If the TALLY
field is reduced to 0, the tally runout indicator is set ON, otherwise it is
set OFF. The TAG field of the indirect word is ignored. The computed address is
the value of the unmodified ADDRESS field.


Example:

Location Instruction Reference count Computed address Tally value

a lda b,id 1 c t-1

b vfd 18/c,1/t 2 c+1 t-2

3 c+2 t-3

...

n c+(n-1) t-n

Increment address, decrement tally, and continue (Td = 17)


The action for this variation is identical to that for the increment address,
decrement tally variation except that the TAG field of the indirect word is
interpreted and continuation of the indirect chain is possible. If the TAG of
the indirect word invokes a register, that is, specifies r, ri, or ir
modification, the effective Td value for the register is forced to {``}null''
before the next computed address is formed.


Figure 6-6. Indirect Then Tally Modification Flowchart


\subsection{VIRTUAL ADDRESS FORMATION INVOLVING BOTH SEGMENT NUMBER AND COMPUTED ADDRESS}

The second type of virtual address formation generates an effective segment number and a computed address simultaneously.

\subsubsection{The Use of Bit 29 in the Instruction Word}


The reader is reminded that there is a preliminary step of loading TPR.CA with
the ADDRESS field of the instruction word during instruction decode.


If bit 29 of the instruction word is set to 1, modification by pointer register
is invoked and the preliminary step is executed as follows:


1. The ADDRESS field of the instruction word is interpreted as shown in Figure
6-7 below.  

2. C(PRn.SNR) $\rightarrow$ C(TPR.TSR)

3. maximum of ( C(PRn.RNR), C(TPR.TRR), C(PPR.PRR) ) $\rightarrow$ C(TPR.TRR) 

4. C(PRn.WORDNO) + OFFSET $\rightarrow$ C(TPR.CA)


(NOTE: OFFSET is a signed binary number.)


5. C(PRn.BITNO) $\rightarrow$ TPR.BITNO


Figure 6-7. Format of Instruction Word ADDRESS When Bit 29 = 1


After this preliminary step is performed, virtual address formation proceeds as discussed above or as discussed for the special address modifiers below.

\subsubsection{Special Address Modifiers}

Whenever the processor is forming a virtual address two special address
modifiers may be specified and are effective under certain restrictive
conditions. The special address modifiers are shown in Table 6-4 and discussed
in the paragraphs below.


The conditions for which the special address modifiers are effective are as
follows: 

1. The instruction word (or preceding indirect word) must specify indirect then
register or register then indirect modification.


2. The computed address for the indirect word must be even.


If these conditions are satisfied, the processor examines the indirect word TAG
field for the special address modifiers.


If either condition is violated, the indirect word TAG field is interpreted as
a normal address modifier and the presence of a special address modifier will
cause an illegal procedure, illegal modifier, fault.

Table 6-4. Special Address Modifiers


TAG Value Coding Symbol Name 

41 itp Indirect to pointer

43 its Indirect to segment

\subsubsubsection{Indirect to Pointer (ITP) Modification}

If the value for indirect to pointer modification is found in the test for
special modifiers, the indirect word-pair is interpreted as an ITP pointer pair
(see Figure 6-8 for format) and the following actions take place:


For n = C(ITP.PRNUM):

C(PRn.SNR) $\rightarrow$ C(TPR.TSR)

maximum of ( C(PRn.RNR), C(SDW.R1), C(TPR.TRR) ) $\rightarrow$ C(TPR.TRR)
C(ITP.BITNO) $\rightarrow$ C(TPR.TBR)

C(PRn.WORDNO) + C(ITP.WORDNO) + C(r) $\rightarrow$ C(TPR.CA)

where:

1. r = C(CT-HOLD) if the instruction word or preceding indirect word specified 
indirect then register modification, or


2. r = C(ITP.MOD.Td) if the instruction word or preceding indirect word
specified register then indirect modification and ITP.MOD.Tm specifies either
register or register then indirect modification.

3. SDW.R1 is the upper limit of the read/write ring bracket for the segment
C(TPR.TSR) (see Section 8).


Even word

Odd word

Figure 6-8. ITP Pointer Pair Format


Field Name Meaning 

PRNUM The number of the pointer register through which to make the segment
reference

WORDNO A word offset value to be added to C(PRn.WORDNO)

BITNO A bit offset value for the data item 

MOD Any normal address modifier (not ITP or ITS)


\subsubsubsection{Indirect to Segment (ITS) Modification}


If the value for indirect to segment modification is found in the test for
special modifiers, the indirect word-pair is interpreted as an ITS pointer pair
(see Figure 6-9 for format) and the following actions take place:

C(ITS.SEGNO) $\rightarrow$ C(TPR.TSR)

maximum of ( C(ITS. RN), C(SDW.R1), C(TPR.TRR) ) $\rightarrow$ C(TPR.TRR)

C(ITS.BITNO) $\rightarrow$ C(TPR.TBR)

C(ITS.WORDNO) + C(r) $\rightarrow$ C(TPR.CA)

where:

1. r = C(CT-HOLD) if the instruction word or preceding indirect word specified
indirect then register modification, or 

2. r = C(ITS.MOD.Td) if the instruction word or preceding indirect word
specified register then indirect modification and ITS.MOD.Tm specifies either
register or register then indirect modification.

3. SDW.R1 is the upper limit of the read/write ring bracket for the segment
C(TPR.TSR) (see Section 8).

Even word

Odd word

Figure 6-9. ITS Pointer Pair Format

Field Name Meaning

SEGNO The number of the segment to be referenced

WORDNO Word offset to be used in the computed address formation 

BITNO The bit offset for the data item

MOD Any valid normal address modifier (not ITS or ITP)

\subsubsection{Effective Segment Number Generation}

A simplified flowchart for effective segment number generation is given in
Figure 6-10.  Although effective ring number generation and access checking are
an integral part of this process, their treatment is deferred to Section 8.


Figure 6-10. Effective Segment Generation Flowchart

Figure 6-10(cont). Effective Segment Number Generation Flowchart

\subsection{VIRTUAL ADDRESS FORMATION FOR EXTENDED INSTRUCTION SET}


The steps involved in virtual address formation for the operand of an EIS
instruction are shown in Figure 6-11. The flowchart depicts the virtual address
formation for operand k as described by its modification field, MFk. This
virtual address formation is performed for each operand as its operand
descriptor is decoded.

Figure 6-11. EIS Virtual Address Formation Flowchart


NOTE 1: The symbol {``}Y'' stands for the contents of the ADDRESS field of the
operand descriptor.  The symbols '{``}CN'' and {``}C'' stand for the contents
of the character number field. The symbol {``}B'' stands for the contents of
the bit number field.

NOTE 2: The algorithms used in the formation of the effective word/char/bit
address are described below.

\subsubsection{Character- and Bit-String Addressing}

The processor represents the effective address of a character- or bit-string
operand in three different forms as follows:

1. Pointer register form

This form consists of a word value (PRn.WORDNO) and a bit value (PRn.BITNO).
The word value is the word offset of the word containing the first character or
bit of the operand and the bit value is the bit position of that character or
bit within the word. This form is seen when C(PRn) are stored as an ITS pointer
pair or as a packed pointer (see discussion of ITS pointers earlier in this
section and the Store Pointer Register n Packed (sprpn) instruction in Section
4).

2. Address register form

This form consists of a word value (ARn.WORDNO), a byte number (ARn.CHAR), and
a bit value (ARn.BITNO). The word value is the word offset of the word
containing the first character or bit of the operand. The byte number is the
number of the 9-bit byte containing the first character or bit. The bit value
is the bit position within AR n.CHAR of the first character or bit. This form
is seen when C(AR n) are stored with the Store Address Register n (sarn)
instruction (see Section 4).

3. Operand Descriptor Form

This form is valid for character-string operands only. It consists of a word
value (ADDRESS) and a character number (CN). The word value is the word offset
of the word containing the first character of the operand and the character
number is the number of that character within the word. This form is seen when
C(ARn) are stored with the Address Register n to Alphanumeric Descriptor (aran)
or Address Register n to Numeric Descriptor (arnn) instructions. (The operand
descriptor form for bit-string operands is identical to the address register
form.)


The terms {``}pointer register'' and {``}address register'' both apply to the
same physical hardware. The distinction arises from the manner in which the
register is used and in the interpretation of the register contents.
{``}Pointer register'' refers to the register as used by the appending unit and
{``}address register'' refers to the register as used by the decimal unit.


The three forms are compatible and may be freely intermixed. For example, PR n
may be loaded in pointer register form with the Effective Pointer to Pointer
Register n (eppn) instruction, then modified in pointer register form with the
Effective Address to Word/Bit Number of Pointer Register n (eawpn) instruction,
then further modified in address register form (assuming character size k) with
the Add k-Bit Displacement to Address Register (akbd) instruction, and finally
invoked in operand descriptor form by the use of MF.AR in an EIS multiword
instruction .  

\subsubsection{Character- and Bit-String Address Arithmetic Algorithms}

The arithmetic algorithms for calculating character- and bit-string addresses
are presented below.  The symbols {``}ADDRESS'' and {``}CN'' represent the
ADDRESS and CN fields of the operand descriptor being decoded. {``}r'' and
{``}n'' are set according to the flowchart in Figure 6-11. If either has the
value {``}null'', the contents of all associated fields are identically zero.  

\subsubsubsection{9-bit Byte String Address Arithmetic}



Effective BITNO = 0000

Effective CHAR = (CN + C(ARn.CHAR) + C(r))[4]

Effective WORDNO = ADDRESS + C(ARn.WORDNO) + (CN + C(ARn.CHAR) + C(r)) / 4

\subsubsubsection{6-bit Character String Address Arithmetic}

Effective BITNO = (9*C(ARn.CHAR) + 6*C(r) + C(ARn.BITNO))[9]

Effective CHAR = ((9*C(ARn.CHAR) + 6*C(r) + C(ARn.BITNO))[36]) / 9

Effective WORDNO = ADDRESS + C(ARn.WORDNO) + (9*C(ARn.CHAR) + 6*C(r) + C(ARn.BITNO)) / 36

\subsubsubsection{4-bit Byte String Address Arithmetic}

Effective BITNO = 4 * (C(ARn.CHAR) + 2*C(r) + C(ARn.BITNO)/4)[2] + 1

Effective CHAR = ((9*C(ARn.CHAR) + 4*C(r) + C(ARn.BITNO))[36] / 9

Effective WORDNO = ADDRESS + C(ARn.WORDNO) + (9*C(ARn.CHAR) + 4*C(r) + C(ARn.BITNO)) / 36

\subsubsubsection{Bit String Address Arithmetic}

Effective BITNO = (9*C(ARn.CHAR) + 36*C(r) + C(ARn.BITNO))[9]

Effective CHAR = ((9*C(ARn.CHAR) + 36*C(r) + C(ARn.BITNO))[36]) / 9

Effective WORDNO = ADDRESS + C(ARn.WORDNO) + (9*C(ARn.CHAR) + 36*C(r) + C(ARn.BITNO)) / 36





