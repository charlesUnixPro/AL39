
\section{VIRTUAL ADDRESS FORMATION}
\label{s6}

\subsection{DEFINITION OF VIRTUAL ADDRESS}

%% ===== \end{flushleft}
%% ===== 
%% ===== 
%% ===== \begin{flushleft}
%% ===== The virtual address in the Multics processor is the user's specification of the location of a
%% ===== \end{flushleft}
%% ===== 
%% ===== 
%% ===== \begin{flushleft}
%% ===== data item in the Multics virtual memory. Each reference to the virtual memory for operands,
%% ===== \end{flushleft}
%% ===== 
%% ===== 
%% ===== \begin{flushleft}
%% ===== indirect words, indirect pointers, operand descriptors, or instructions must provide a virtual
%% ===== \end{flushleft}
%% ===== 
%% ===== 
%% ===== \begin{flushleft}
%% ===== address. The hardware and the operating system translate the virtual address into the true
%% ===== \end{flushleft}
%% ===== 
%% ===== 
%% ===== \begin{flushleft}
%% ===== location of the data item and assure that the data item is in main memory for the reference.
%% ===== \end{flushleft}
%% ===== 
%% ===== 
%% ===== \begin{flushleft}
%% ===== The virtual address consists of two parts, an effective segment number and an offset or
%% ===== \end{flushleft}
%% ===== 
%% ===== 
%% ===== \begin{flushleft}
%% ===== computed address. The value of each part is the result of the evaluation of a hardware algorithm
%% ===== \end{flushleft}
%% ===== 
%% ===== 
%% ===== \begin{flushleft}
%% ===== (expression) of one or more terms. The selection of the algorithm is made by the use of control
%% ===== \end{flushleft}
%% ===== 
%% ===== 
%% ===== \begin{flushleft}
%% ===== bits in the instruction word; for example, bit 29 for modification by pointer register and bits 30-35
%% ===== \end{flushleft}
%% ===== 
%% ===== 
%% ===== \begin{flushleft}
%% ===== (the TAG field) for modification by index register or indirect word. For certain modifications by
%% ===== \end{flushleft}
%% ===== 
%% ===== 
%% ===== \begin{flushleft}
%% ===== indirect word, the TAG field of the indirect word is also treated as an address modifier, thus
%% ===== \end{flushleft}
%% ===== 
%% ===== 
%% ===== \begin{flushleft}
%% ===== establishing a continuing {``}indirect chain''. Bit 29 of an indirect word has no meaning in the context
%% ===== \end{flushleft}
%% ===== 
%% ===== 
%% ===== \begin{flushleft}
%% ===== of virtual address formation.
%% ===== \end{flushleft}
%% ===== 
%% ===== 
%% ===== \begin{flushleft}
%% ===== The results of evaluation of the virtual address formation algorithms are stored in
%% ===== \end{flushleft}
%% ===== 
%% ===== 
%% ===== \begin{flushleft}
%% ===== temporary registers used as working registers by the processor. The effective segment number is
%% ===== \end{flushleft}
%% ===== 
%% ===== 
%% ===== \begin{flushleft}
%% ===== stored in the temporary segment register, TPR.TSR. The offset is stored in the computed address
%% ===== \end{flushleft}
%% ===== 
%% ===== 
%% ===== \begin{flushleft}
%% ===== register, TPR.CA. When each virtual address computation has been completed, C(TPR.TSR) and
%% ===== \end{flushleft}
%% ===== 
%% ===== 
%% ===== \begin{flushleft}
%% ===== C(TPR.CA) are presented to the appending unit for translation to a 24-bit absolute main memory
%% ===== \end{flushleft}
%% ===== 
%% ===== 
%% ===== \begin{flushleft}
%% ===== address (see Section 5).
%% ===== \end{flushleft}
%% ===== 
%% ===== 
%% ===== 
%% ===== 
%% ===== 
%% ===== \begin{flushleft}

\subsection{TYPES OF VIRTUAL ADDRESS FORMATION}

%% ===== \end{flushleft}
%% ===== 
%% ===== 
%% ===== \begin{flushleft}
%% ===== There are two types of virtual address formation. The first type does not make explicit use
%% ===== \end{flushleft}
%% ===== 
%% ===== 
%% ===== \begin{flushleft}
%% ===== of segment numbers. The algorithms produce values for the computed address, C(TPR.CA), only.
%% ===== \end{flushleft}
%% ===== 
%% ===== 
%% ===== \begin{flushleft}
%% ===== The effective segment number in C(TPR.TSR) does not change from the value used to fetch the
%% ===== \end{flushleft}
%% ===== 
%% ===== 
%% ===== \begin{flushleft}
%% ===== current instruction. In this case, all references are said to be {``}local'' to the procedure segment
%% ===== \end{flushleft}
%% ===== 
%% ===== 
%% ===== \begin{flushleft}
%% ===== pointed to by the procedure pointer register (PPR).
%% ===== \end{flushleft}
%% ===== 
%% ===== 
%% ===== \begin{flushleft}
%% ===== The second type makes use of a segment number in an indirect word-pair in main memory
%% ===== \end{flushleft}
%% ===== 
%% ===== 
%% ===== \begin{flushleft}
%% ===== or in a pointer register (PRn). The algorithms produce values for both the effective segment
%% ===== \end{flushleft}
%% ===== 
%% ===== 
%% ===== \begin{flushleft}
%% ===== number, C(TPR.TSR), and the computed address, C(TPR.CA). The effective segment number in
%% ===== \end{flushleft}
%% ===== 
%% ===== 
%% ===== \begin{flushleft}
%% ===== C(TPR.TSR) may change and, if it changes, references are said to be {``}external'' to the procedure
%% ===== \end{flushleft}
%% ===== 
%% ===== 
%% ===== \begin{flushleft}
%% ===== segment.
%% ===== \end{flushleft}
%% ===== 
%% ===== 
%% ===== \begin{flushleft}
%% ===== Both types of virtual address formation for the operand of a basic or EIS single-word
%% ===== \end{flushleft}
%% ===== 
%% ===== 
%% ===== \begin{flushleft}
%% ===== instruction begin with a preliminary step of loading TPR.CA with the ADDRESS field of the
%% ===== \end{flushleft}
%% ===== 
%% ===== 
%% ===== \begin{flushleft}
%% ===== instruction word. This preliminary step takes place during instruction decode.
%% ===== \end{flushleft}
%% ===== 
%% ===== 
%% ===== \begin{flushleft}
%% ===== The two types of virtual address formation can be intermixed. In cases where virtual
%% ===== \end{flushleft}
%% ===== 
%% ===== 
%% ===== \begin{flushleft}
%% ===== address calculations are chained together through pointer registers or indirect words, each virtual
%% ===== \end{flushleft}
%% ===== 
%% ===== 
%% ===== \begin{flushleft}
%% ===== address is translated to a 24-bit absolute main memory address to fetch the next item in the chain.
%% ===== \end{flushleft}
%% ===== 
%% ===== 
%% ===== \begin{flushleft}
%% ===== This description of virtual address formation is divided into two parts corresponding to the
%% ===== \end{flushleft}
%% ===== 
%% ===== 
%% ===== \begin{flushleft}
%% ===== two types. The first part describes the type that involves only the computed address, C(TPR.CA).
%% ===== \end{flushleft}
%% ===== 
%% ===== 
%% ===== \begin{flushleft}
%% ===== The effective segment number is constant. In append mode its value is equal to C(PPR.PSR) (a
%% ===== \end{flushleft}
%% ===== 
%% ===== 
%% ===== \begin{flushleft}
%% ===== local reference) and in absolute mode its value is undefined.
%% ===== \end{flushleft}
%% ===== 
%% ===== 
%% ===== \begin{flushleft}
%% ===== The second part describes the type that involves both the effective segment number,
%% ===== \end{flushleft}
%% ===== 
%% ===== 
%% ===== \begin{flushleft}
%% ===== C(TPR.TSR), and the computed address, C(TPR.CA).
%% ===== \end{flushleft}
%% ===== 
%% ===== 
%% ===== 
%% ===== 
%% ===== 
%% ===== \begin{flushleft}
%% ===== \newpage

\subsection{SYMBOLOGY (ALM)}

%% ===== \end{flushleft}
%% ===== 
%% ===== 
%% ===== \begin{flushleft}
%% ===== In many instances in the discussions that follow, references to the features of the ALM
%% ===== \end{flushleft}
%% ===== 
%% ===== 
%% ===== \begin{flushleft}
%% ===== assembly program are unavoidable. Such references are explained briefly here. The reader is
%% ===== \end{flushleft}
%% ===== 
%% ===== 
%% ===== \begin{flushleft}
%% ===== advised to consult the appropriate software documentation for further details and for possible
%% ===== \end{flushleft}
%% ===== 
%% ===== 
%% ===== \begin{flushleft}
%% ===== changes in the various features.
%% ===== \end{flushleft}
%% ===== 
%% ===== 
%% ===== 
%% ===== 
%% ===== 
%% ===== \begin{flushleft}

\subsubsection{Symbolic Fields}

%% ===== \end{flushleft}
%% ===== 
%% ===== 
%% ===== \begin{flushleft}
%% ===== A symbolic field is an expression consisting of variables, constants, literals, and operators
%% ===== \end{flushleft}
%% ===== 
%% ===== 
%% ===== \begin{flushleft}
%% ===== that is evaluated by ALM to produce a value for the corresponding field of a machine word. The
%% ===== \end{flushleft}
%% ===== 
%% ===== 
%% ===== \begin{flushleft}
%% ===== values of the variables and constants are either known or assignable and the operators are defined
%% ===== \end{flushleft}
%% ===== 
%% ===== 
%% ===== \begin{flushleft}
%% ===== for the mode of the evaluation (algebraic, logical, etc.). The necessary fields for a machine
%% ===== \end{flushleft}
%% ===== 
%% ===== 
%% ===== \begin{flushleft}

\subsubsection{instruction or ALM pseudo-instruction are given as a comma-separated string of expressions.}

%% ===== \end{flushleft}
%% ===== 
%% ===== 
%% ===== 
%% ===== 
%% ===== 
%% ===== \begin{flushleft}
%% ===== ALM Pseudo-Instructions
%% ===== \end{flushleft}
%% ===== 
%% ===== 
%% ===== \begin{flushleft}
%% ===== The following ALM pseudo-instructions are used in this section:
%% ===== \end{flushleft}
%% ===== 
%% ===== 
%% ===== \begin{flushleft}
%% ===== aci
%% ===== \end{flushleft}
%% ===== 
%% ===== 
%% ===== 
%% ===== 
%% ===== 
%% ===== \begin{flushleft}
%% ===== string
%% ===== \end{flushleft}
%% ===== 
%% ===== 
%% ===== \begin{flushleft}
%% ===== This pseudo-instruction generates a sequence of 9-bit byte fields each of which contains the
%% ===== \end{flushleft}
%% ===== 
%% ===== 
%% ===== \begin{flushleft}
%% ===== ASCII octal value for the corresponding graphic character in string. The last machine word
%% ===== \end{flushleft}
%% ===== 
%% ===== 
%% ===== \begin{flushleft}
%% ===== generated is low-order filled with binary 0s to the next word boundary.
%% ===== \end{flushleft}
%% ===== 
%% ===== 
%% ===== 
%% ===== 
%% ===== 
%% ===== \begin{flushleft}
%% ===== arg
%% ===== \end{flushleft}
%% ===== 
%% ===== 
%% ===== 
%% ===== 
%% ===== 
%% ===== \begin{flushleft}
%% ===== address,tag
%% ===== \end{flushleft}
%% ===== 
%% ===== 
%% ===== \begin{flushleft}
%% ===== This pseudo-instruction generates a machine word with the same format as the basic and
%% ===== \end{flushleft}
%% ===== 
%% ===== 
%% ===== \begin{flushleft}
%% ===== EIS single-word instructions but having binary 0s in the operation code field.
%% ===== \end{flushleft}
%% ===== 
%% ===== 
%% ===== 
%% ===== 
%% ===== 
%% ===== \begin{flushleft}
%% ===== bci
%% ===== \end{flushleft}
%% ===== 
%% ===== 
%% ===== 
%% ===== 
%% ===== 
%% ===== \begin{flushleft}
%% ===== string
%% ===== \end{flushleft}
%% ===== 
%% ===== 
%% ===== \begin{flushleft}
%% ===== This pseudo-instruction generates a sequence of 6-bit character fields each of which
%% ===== \end{flushleft}
%% ===== 
%% ===== 
%% ===== \begin{flushleft}
%% ===== contains the binary coded decimal (BCD) octal value for the corresponding graphic
%% ===== \end{flushleft}
%% ===== 
%% ===== 
%% ===== \begin{flushleft}
%% ===== character in string. The last machine word generated is low-order filled with binary 0s to
%% ===== \end{flushleft}
%% ===== 
%% ===== 
%% ===== \begin{flushleft}
%% ===== the next word boundary.
%% ===== \end{flushleft}
%% ===== 
%% ===== 
%% ===== 
%% ===== 
%% ===== 
%% ===== \begin{flushleft}
%% ===== vfd
%% ===== \end{flushleft}
%% ===== 
%% ===== 
%% ===== 
%% ===== 
%% ===== 
%% ===== \begin{flushleft}
%% ===== field1,field2, ... ,fieldn
%% ===== \end{flushleft}
%% ===== 
%% ===== 
%% ===== \begin{flushleft}
%% ===== This pseudo-instruction generates a machine word (or word-pair) containing an arbitrary
%% ===== \end{flushleft}
%% ===== 
%% ===== 
%% ===== \begin{flushleft}
%% ===== number of fields of arbitrary length up to a total bit count of 72. The data generated is leftjustified in the machine word (or word-pair) and zero filled to the next word boundary as
%% ===== \end{flushleft}
%% ===== 
%% ===== 
%% ===== \begin{flushleft}
%% ===== necessary.
%% ===== \end{flushleft}
%% ===== 
%% ===== 
%% ===== \begin{flushleft}
%% ===== Each fieldi is given as:
%% ===== \end{flushleft}
%% ===== 
%% ===== 
%% ===== 
%% ===== 
%% ===== 
%% ===== \begin{flushleft}
%% ===== md/expr
%% ===== \end{flushleft}
%% ===== 
%% ===== 
%% ===== \begin{flushleft}
%% ===== where:
%% ===== \end{flushleft}
%% ===== 
%% ===== 
%% ===== 
%% ===== 
%% ===== 
%% ===== \begin{flushleft}
%% ===== m is the data conversion mode and may be:
%% ===== \end{flushleft}
%% ===== 
%% ===== 
%% ===== \begin{flushleft}
%% ===== null
%% ===== \end{flushleft}
%% ===== 
%% ===== 
%% ===== 
%% ===== 
%% ===== 
%% ===== \begin{flushleft}
%% ===== for arithmetic operators and decimal literals,
%% ===== \end{flushleft}
%% ===== 
%% ===== 
%% ===== 
%% ===== 
%% ===== 
%% ===== \begin{flushleft}
%% ===== o
%% ===== \end{flushleft}
%% ===== 
%% ===== 
%% ===== 
%% ===== 
%% ===== 
%% ===== \begin{flushleft}
%% ===== for Boolean operators and octal literals,
%% ===== \end{flushleft}
%% ===== 
%% ===== 
%% ===== 
%% ===== 
%% ===== 
%% ===== \begin{flushleft}
%% ===== h
%% ===== \end{flushleft}
%% ===== 
%% ===== 
%% ===== 
%% ===== 
%% ===== 
%% ===== \begin{flushleft}
%% ===== for 6-bit character binary coded decimal (BCD) character strings, or
%% ===== \end{flushleft}
%% ===== 
%% ===== 
%% ===== 
%% ===== 
%% ===== 
%% ===== \begin{flushleft}
%% ===== a
%% ===== \end{flushleft}
%% ===== 
%% ===== 
%% ===== 
%% ===== 
%% ===== 
%% ===== \begin{flushleft}
%% ===== for 9-bit byte ASCII character strings.
%% ===== \end{flushleft}
%% ===== 
%% ===== 
%% ===== 
%% ===== 
%% ===== 
%% ===== \begin{flushleft}
%% ===== d is a literal giving the field width in bits and may have any value from 1 to 72.
%% ===== \end{flushleft}
%% ===== 
%% ===== 
%% ===== 
%% ===== 
%% ===== 
%% ===== \begin{flushleft}
%% ===== \newpage
%% ===== expr is the expression to be evaluated or converted. Conversion is done with
%% ===== \end{flushleft}
%% ===== 
%% ===== 
%% ===== \begin{flushleft}
%% ===== full 36-bit precision and the field value is the conversion result modulo the field
%% ===== \end{flushleft}
%% ===== 
%% ===== 
%% ===== \begin{flushleft}
%% ===== width.
%% ===== \end{flushleft}
%% ===== 
%% ===== 
%% ===== 
%% ===== 
%% ===== 
%% ===== \begin{flushleft}

\subsection{COMPUTED ADDRESS FORMATION}

%% ===== \end{flushleft}
%% ===== 
%% ===== 
%% ===== \begin{flushleft}
%% ===== The address formation algorithms described here produce values only for the computed
%% ===== \end{flushleft}
%% ===== 
%% ===== 
%% ===== \begin{flushleft}
%% ===== address. The effective segment number is constant and equal to C(PPR.PSR) if the processor is in
%% ===== \end{flushleft}
%% ===== 
%% ===== 
%% ===== \begin{flushleft}
%% ===== append mode or is undefined if the processor is in absolute mode.
%% ===== \end{flushleft}
%% ===== 
%% ===== 
%% ===== 
%% ===== 
%% ===== 
%% ===== \begin{flushleft}

\subsubsection{The Address Modifier (TAG) Field}

%% ===== \end{flushleft}
%% ===== 
%% ===== 
%% ===== \begin{flushleft}
%% ===== Bits 30-35 of an instruction word or indirect word constitute the address modifier or TAG
%% ===== \end{flushleft}
%% ===== 
%% ===== 
%% ===== \begin{flushleft}
%% ===== field. The format of the TAG field is:
%% ===== \end{flushleft}
%% ===== 
%% ===== 
%% ===== 3 3 3
%% ===== 
%% ===== 
%% ===== 0 1 2
%% ===== 
%% ===== 
%% ===== \begin{flushleft}
%% ===== Tm
%% ===== \end{flushleft}
%% ===== 
%% ===== 
%% ===== 2
%% ===== 
%% ===== 
%% ===== 
%% ===== 
%% ===== 
%% ===== 3
%% ===== 
%% ===== 
%% ===== 5
%% ===== 
%% ===== 
%% ===== \begin{flushleft}
%% ===== Td
%% ===== \end{flushleft}
%% ===== 
%% ===== 
%% ===== 4
%% ===== 
%% ===== 
%% ===== 
%% ===== 
%% ===== 
%% ===== \begin{flushleft}
%% ===== Figure 6-1. Address Modifier (TAG) Field Format
%% ===== \end{flushleft}
%% ===== 
%% ===== 
%% ===== \begin{flushleft}
%% ===== Field Name Function
%% ===== \end{flushleft}
%% ===== 
%% ===== 
%% ===== \begin{flushleft}
%% ===== Tm
%% ===== \end{flushleft}
%% ===== 
%% ===== 
%% ===== 
%% ===== 
%% ===== 
%% ===== \begin{flushleft}
%% ===== modifier field, specifies one of four general types of computed address
%% ===== \end{flushleft}
%% ===== 
%% ===== 
%% ===== \begin{flushleft}
%% ===== modification
%% ===== \end{flushleft}
%% ===== 
%% ===== 
%% ===== 
%% ===== 
%% ===== 
%% ===== \begin{flushleft}
%% ===== Td
%% ===== \end{flushleft}
%% ===== 
%% ===== 
%% ===== 
%% ===== 
%% ===== 
%% ===== \begin{flushleft}
%% ===== designator field, selects among several variations available for the general
%% ===== \end{flushleft}
%% ===== 
%% ===== 
%% ===== \begin{flushleft}
%% ===== type given with Tm
%% ===== \end{flushleft}
%% ===== 
%% ===== 
%% ===== 
%% ===== 
%% ===== 
%% ===== \begin{flushleft}

\subsubsection{General Types of Computed Address Modification}

%% ===== \end{flushleft}
%% ===== 
%% ===== 
%% ===== \begin{flushleft}
%% ===== There are four general types of computed address modification: register, register then
%% ===== \end{flushleft}
%% ===== 
%% ===== 
%% ===== \begin{flushleft}
%% ===== indirect, indirect then register, and indirect then tally. The general types are described in Table
%% ===== \end{flushleft}
%% ===== 
%% ===== 
%% ===== \begin{flushleft}
%% ===== 6-1. The value loaded into TPR.CA is symbolized by {``}y'' in the descriptions following.
%% ===== \end{flushleft}
%% ===== 
%% ===== 
%% ===== 
%% ===== 
%% ===== 
%% ===== \begin{flushleft}
%% ===== Table 6-1. General Computed Address Modification Types
%% ===== \end{flushleft}
%% ===== 
%% ===== 
%% ===== \begin{flushleft}
%% ===== Tm
%% ===== \end{flushleft}
%% ===== 
%% ===== 
%% ===== \begin{flushleft}
%% ===== value
%% ===== \end{flushleft}
%% ===== 
%% ===== 
%% ===== 
%% ===== 
%% ===== 
%% ===== \begin{flushleft}
%% ===== Type
%% ===== \end{flushleft}
%% ===== 
%% ===== 
%% ===== 
%% ===== 
%% ===== 
%% ===== \begin{flushleft}
%% ===== Description
%% ===== \end{flushleft}
%% ===== 
%% ===== 
%% ===== 
%% ===== 
%% ===== 
%% ===== 0
%% ===== 
%% ===== 
%% ===== 
%% ===== 
%% ===== 
%% ===== \begin{flushleft}
%% ===== Register
%% ===== \end{flushleft}
%% ===== 
%% ===== 
%% ===== \begin{flushleft}
%% ===== (r)
%% ===== \end{flushleft}
%% ===== 
%% ===== 
%% ===== 
%% ===== 
%% ===== 
%% ===== \begin{flushleft}
%% ===== The contents of the register specified in C(Td) are added to the current
%% ===== \end{flushleft}
%% ===== 
%% ===== 
%% ===== \begin{flushleft}
%% ===== computed address, C(TPR.CA), to form the modified computed address.
%% ===== \end{flushleft}
%% ===== 
%% ===== 
%% ===== \begin{flushleft}
%% ===== Addition is twos complement, modulo 218, and overflow does not occur.
%% ===== \end{flushleft}
%% ===== 
%% ===== 
%% ===== 
%% ===== 
%% ===== 
%% ===== 1
%% ===== 
%% ===== 
%% ===== 
%% ===== 
%% ===== 
%% ===== \begin{flushleft}
%% ===== Register
%% ===== \end{flushleft}
%% ===== 
%% ===== 
%% ===== \begin{flushleft}
%% ===== then
%% ===== \end{flushleft}
%% ===== 
%% ===== 
%% ===== \begin{flushleft}
%% ===== indirect
%% ===== \end{flushleft}
%% ===== 
%% ===== 
%% ===== \begin{flushleft}
%% ===== (ri)
%% ===== \end{flushleft}
%% ===== 
%% ===== 
%% ===== 
%% ===== 
%% ===== 
%% ===== \begin{flushleft}
%% ===== The contents of the register specified in C(Td) are added to the current
%% ===== \end{flushleft}
%% ===== 
%% ===== 
%% ===== \begin{flushleft}
%% ===== computed address, C(TPR.CA), to form the modified computed address as for
%% ===== \end{flushleft}
%% ===== 
%% ===== 
%% ===== \begin{flushleft}
%% ===== register modification. The modified C(TPR.CA) is then used to fetch an
%% ===== \end{flushleft}
%% ===== 
%% ===== 
%% ===== \begin{flushleft}
%% ===== indirect word. The TAG field of the indirect word specifies the next step in
%% ===== \end{flushleft}
%% ===== 
%% ===== 
%% ===== \begin{flushleft}
%% ===== computed address formation. The use of du or dl as the designator in this
%% ===== \end{flushleft}
%% ===== 
%% ===== 
%% ===== \begin{flushleft}
%% ===== modification type will cause an illegal procedure, illegal modifier, fault.
%% ===== \end{flushleft}
%% ===== 
%% ===== 
%% ===== 
%% ===== 
%% ===== 
%% ===== \begin{flushleft}
%% ===== \newpage
%% ===== Tm
%% ===== \end{flushleft}
%% ===== 
%% ===== 
%% ===== \begin{flushleft}
%% ===== value
%% ===== \end{flushleft}
%% ===== 
%% ===== 
%% ===== 
%% ===== 
%% ===== 
%% ===== \begin{flushleft}
%% ===== Type
%% ===== \end{flushleft}
%% ===== 
%% ===== 
%% ===== 
%% ===== 
%% ===== 
%% ===== \begin{flushleft}
%% ===== Description
%% ===== \end{flushleft}
%% ===== 
%% ===== 
%% ===== 
%% ===== 
%% ===== 
%% ===== 2
%% ===== 
%% ===== 
%% ===== 
%% ===== 
%% ===== 
%% ===== \begin{flushleft}
%% ===== Indirect
%% ===== \end{flushleft}
%% ===== 
%% ===== 
%% ===== \begin{flushleft}
%% ===== then
%% ===== \end{flushleft}
%% ===== 
%% ===== 
%% ===== \begin{flushleft}
%% ===== tally (it)
%% ===== \end{flushleft}
%% ===== 
%% ===== 
%% ===== 
%% ===== 
%% ===== 
%% ===== \begin{flushleft}
%% ===== The indirect word at C(TPR.CA) is fetched and the modification performed
%% ===== \end{flushleft}
%% ===== 
%% ===== 
%% ===== \begin{flushleft}
%% ===== according to the variation specified in C(Td) of the instruction word and the
%% ===== \end{flushleft}
%% ===== 
%% ===== 
%% ===== \begin{flushleft}
%% ===== contents of the indirect word. This modification type allows automatic
%% ===== \end{flushleft}
%% ===== 
%% ===== 
%% ===== \begin{flushleft}
%% ===== incrementing and decrementing of addresses and tally counting.
%% ===== \end{flushleft}
%% ===== 
%% ===== 
%% ===== 
%% ===== 
%% ===== 
%% ===== 3
%% ===== 
%% ===== 
%% ===== 
%% ===== 
%% ===== 
%% ===== \begin{flushleft}
%% ===== Indirect
%% ===== \end{flushleft}
%% ===== 
%% ===== 
%% ===== \begin{flushleft}
%% ===== then
%% ===== \end{flushleft}
%% ===== 
%% ===== 
%% ===== \begin{flushleft}
%% ===== register
%% ===== \end{flushleft}
%% ===== 
%% ===== 
%% ===== \begin{flushleft}
%% ===== (ir)
%% ===== \end{flushleft}
%% ===== 
%% ===== 
%% ===== 
%% ===== 
%% ===== 
%% ===== \begin{flushleft}
%% ===== The register designator, C(Td), is safe-stored in a special holding register, CTHOLD. The word at C(TPR.CA) is fetched and interpreted as an indirect
%% ===== \end{flushleft}
%% ===== 
%% ===== 
%% ===== \begin{flushleft}
%% ===== word. The TAG field of the indirect word specifies the next step in computed
%% ===== \end{flushleft}
%% ===== 
%% ===== 
%% ===== \begin{flushleft}
%% ===== address formation as follows:
%% ===== \end{flushleft}
%% ===== 
%% ===== 
%% ===== 
%% ===== 
%% ===== 
%% ===== \begin{flushleft}
%% ===== Indirect
%% ===== \end{flushleft}
%% ===== 
%% ===== 
%% ===== \begin{flushleft}
%% ===== TAG
%% ===== \end{flushleft}
%% ===== 
%% ===== 
%% ===== \begin{flushleft}
%% ===== r or it
%% ===== \end{flushleft}
%% ===== 
%% ===== 
%% ===== 
%% ===== 
%% ===== 
%% ===== \begin{flushleft}
%% ===== Next step
%% ===== \end{flushleft}
%% ===== 
%% ===== 
%% ===== \begin{flushleft}
%% ===== Perform register modification using Td from CT-HOLD.(1)
%% ===== \end{flushleft}
%% ===== 
%% ===== 
%% ===== 
%% ===== 
%% ===== 
%% ===== \begin{flushleft}
%% ===== ri
%% ===== \end{flushleft}
%% ===== 
%% ===== 
%% ===== 
%% ===== 
%% ===== 
%% ===== \begin{flushleft}
%% ===== Perform the register then indirect modification immediately and
%% ===== \end{flushleft}
%% ===== 
%% ===== 
%% ===== \begin{flushleft}
%% ===== fetch the next indirect word from the result of that modification.
%% ===== \end{flushleft}
%% ===== 
%% ===== 
%% ===== 
%% ===== 
%% ===== 
%% ===== \begin{flushleft}
%% ===== ir
%% ===== \end{flushleft}
%% ===== 
%% ===== 
%% ===== 
%% ===== 
%% ===== 
%% ===== \begin{flushleft}
%% ===== Replace the safe-stored Td value in CT-HOLD with the Td value
%% ===== \end{flushleft}
%% ===== 
%% ===== 
%% ===== \begin{flushleft}
%% ===== from the indirect word TAG field and use the ADDRESS field of
%% ===== \end{flushleft}
%% ===== 
%% ===== 
%% ===== \begin{flushleft}
%% ===== the indirect word as a computed address value to fetch the next
%% ===== \end{flushleft}
%% ===== 
%% ===== 
%% ===== \begin{flushleft}
%% ===== indirect word.
%% ===== \end{flushleft}
%% ===== 
%% ===== 
%% ===== 
%% ===== 
%% ===== 
%% ===== \begin{flushleft}
%% ===== (1)In this instance, the indirect then tally variations fault tag 1, fault tag 2, and fault tag 3 are
%% ===== \end{flushleft}
%% ===== 
%% ===== 
%% ===== \begin{flushleft}
%% ===== treated differently. The fault tag 1 variation results in the action described here but fault tag 2
%% ===== \end{flushleft}
%% ===== 
%% ===== 
%% ===== \begin{flushleft}
%% ===== and fault tag 3 result in the generation of a fault. See the discussion of indirect then tally
%% ===== \end{flushleft}
%% ===== 
%% ===== 
%% ===== \begin{flushleft}
%% ===== modification later in this section.
%% ===== \end{flushleft}
%% ===== 
%% ===== 
%% ===== 
%% ===== 
%% ===== 
%% ===== \begin{flushleft}

\subsubsection{Computed Address Formation Flowcharts}

%% ===== \end{flushleft}
%% ===== 
%% ===== 
%% ===== \begin{flushleft}
%% ===== The flowcharts depicting the computed address formation process are scattered throughout
%% ===== \end{flushleft}
%% ===== 
%% ===== 
%% ===== \begin{flushleft}
%% ===== this section and are linked together by figure references. The flowcharts start with Figure 6-2.
%% ===== \end{flushleft}
%% ===== 
%% ===== 
%% ===== \begin{flushleft}
%% ===== START CA
%% ===== \end{flushleft}
%% ===== 
%% ===== 
%% ===== 
%% ===== 
%% ===== 
%% ===== \begin{flushleft}
%% ===== Interpret
%% ===== \end{flushleft}
%% ===== 
%% ===== 
%% ===== \begin{flushleft}
%% ===== Tm
%% ===== \end{flushleft}
%% ===== 
%% ===== 
%% ===== 
%% ===== 
%% ===== 
%% ===== \begin{flushleft}
%% ===== Tm=r
%% ===== \end{flushleft}
%% ===== 
%% ===== 
%% ===== 
%% ===== 
%% ===== 
%% ===== \begin{flushleft}
%% ===== Tm=ri
%% ===== \end{flushleft}
%% ===== 
%% ===== 
%% ===== 
%% ===== 
%% ===== 
%% ===== `
%% ===== 
%% ===== 
%% ===== 
%% ===== 
%% ===== 
%% ===== \begin{flushleft}
%% ===== Tm=it
%% ===== \end{flushleft}
%% ===== 
%% ===== 
%% ===== 
%% ===== 
%% ===== 
%% ===== \begin{flushleft}
%% ===== Tm=ir
%% ===== \end{flushleft}
%% ===== 
%% ===== 
%% ===== 
%% ===== 
%% ===== 
%% ===== \begin{flushleft}
%% ===== R MOD
%% ===== \end{flushleft}
%% ===== 
%% ===== 
%% ===== 
%% ===== 
%% ===== 
%% ===== \begin{flushleft}
%% ===== RI MOD
%% ===== \end{flushleft}
%% ===== 
%% ===== 
%% ===== 
%% ===== 
%% ===== 
%% ===== \begin{flushleft}
%% ===== IT MOD
%% ===== \end{flushleft}
%% ===== 
%% ===== 
%% ===== 
%% ===== 
%% ===== 
%% ===== \begin{flushleft}
%% ===== IR MOD
%% ===== \end{flushleft}
%% ===== 
%% ===== 
%% ===== 
%% ===== 
%% ===== 
%% ===== \begin{flushleft}
%% ===== (Figure 6-3)
%% ===== \end{flushleft}
%% ===== 
%% ===== 
%% ===== 
%% ===== 
%% ===== 
%% ===== \begin{flushleft}
%% ===== (Figure 6-4)
%% ===== \end{flushleft}
%% ===== 
%% ===== 
%% ===== 
%% ===== 
%% ===== 
%% ===== \begin{flushleft}
%% ===== (Figure 6-5)
%% ===== \end{flushleft}
%% ===== 
%% ===== 
%% ===== 
%% ===== 
%% ===== 
%% ===== \begin{flushleft}
%% ===== (Figure 6-6)
%% ===== \end{flushleft}
%% ===== 
%% ===== 
%% ===== 
%% ===== 
%% ===== 
%% ===== \begin{flushleft}
%% ===== Figure 6-2. Common Computed Address Formation Flowchart
%% ===== \end{flushleft}
%% ===== 
%% ===== 
%% ===== 
%% ===== 
%% ===== 
%% ===== \begin{flushleft}

\subsubsection{Register (r) Modification}

%% ===== \end{flushleft}
%% ===== 
%% ===== 
%% ===== \begin{flushleft}
%% ===== In register modification (Tm = 0) the value of Td designates a register whose contents are to
%% ===== \end{flushleft}
%% ===== 
%% ===== 
%% ===== \begin{flushleft}
%% ===== be added to C(TPR.CA) to form a modified C(TPR.CA). This modified C(TPR.CA) becomes the
%% ===== \end{flushleft}
%% ===== 
%% ===== 
%% ===== \begin{flushleft}
%% ===== computed address of the operand. See Figure 6-3, Table 6-2, and the examples following.
%% ===== \end{flushleft}
%% ===== 
%% ===== 
%% ===== 
%% ===== 
%% ===== 
%% ===== \begin{flushleft}
%% ===== \newpage
%% ===== R MOD
%% ===== \end{flushleft}
%% ===== 
%% ===== 
%% ===== 
%% ===== 
%% ===== 
%% ===== \begin{flushleft}
%% ===== Yes
%% ===== \end{flushleft}
%% ===== 
%% ===== 
%% ===== 
%% ===== 
%% ===== 
%% ===== \begin{flushleft}
%% ===== Td = 0?
%% ===== \end{flushleft}
%% ===== 
%% ===== 
%% ===== \begin{flushleft}
%% ===== No
%% ===== \end{flushleft}
%% ===== 
%% ===== 
%% ===== \begin{flushleft}
%% ===== Td = 3
%% ===== \end{flushleft}
%% ===== 
%% ===== 
%% ===== \begin{flushleft}
%% ===== or 7?
%% ===== \end{flushleft}
%% ===== 
%% ===== 
%% ===== 
%% ===== 
%% ===== 
%% ===== \begin{flushleft}
%% ===== Yes
%% ===== \end{flushleft}
%% ===== 
%% ===== 
%% ===== 
%% ===== 
%% ===== 
%% ===== \begin{flushleft}
%% ===== No
%% ===== \end{flushleft}
%% ===== 
%% ===== 
%% ===== 
%% ===== 
%% ===== 
%% ===== \begin{flushleft}
%% ===== r = Td
%% ===== \end{flushleft}
%% ===== 
%% ===== 
%% ===== \begin{flushleft}
%% ===== C(TPR.CA) + C(r ) $\rightarrow$ C(TPR.CA)
%% ===== \end{flushleft}
%% ===== 
%% ===== 
%% ===== 
%% ===== 
%% ===== 
%% ===== \begin{flushleft}
%% ===== Set direct operand flag
%% ===== \end{flushleft}
%% ===== 
%% ===== 
%% ===== \begin{flushleft}
%% ===== Form operand
%% ===== \end{flushleft}
%% ===== 
%% ===== 
%% ===== 
%% ===== 
%% ===== 
%% ===== \begin{flushleft}
%% ===== END CA
%% ===== \end{flushleft}
%% ===== 
%% ===== 
%% ===== 
%% ===== 
%% ===== 
%% ===== \begin{flushleft}
%% ===== Figure 6-3. Register Modification Flowchart
%% ===== \end{flushleft}
%% ===== 
%% ===== 
%% ===== 
%% ===== 
%% ===== 
%% ===== \begin{flushleft}
%% ===== Table 6-2. Register Modification Decode
%% ===== \end{flushleft}
%% ===== 
%% ===== 
%% ===== \begin{flushleft}
%% ===== Td
%% ===== \end{flushleft}
%% ===== 
%% ===== 
%% ===== \begin{flushleft}
%% ===== value
%% ===== \end{flushleft}
%% ===== 
%% ===== 
%% ===== 
%% ===== 
%% ===== 
%% ===== \begin{flushleft}
%% ===== Register
%% ===== \end{flushleft}
%% ===== 
%% ===== 
%% ===== 
%% ===== 
%% ===== 
%% ===== \begin{flushleft}
%% ===== Coding
%% ===== \end{flushleft}
%% ===== 
%% ===== 
%% ===== \begin{flushleft}
%% ===== Symbol
%% ===== \end{flushleft}
%% ===== 
%% ===== 
%% ===== 
%% ===== 
%% ===== 
%% ===== 0
%% ===== 
%% ===== 
%% ===== 
%% ===== 
%% ===== 
%% ===== \begin{flushleft}
%% ===== none
%% ===== \end{flushleft}
%% ===== 
%% ===== 
%% ===== 
%% ===== 
%% ===== 
%% ===== \begin{flushleft}
%% ===== n, null
%% ===== \end{flushleft}
%% ===== 
%% ===== 
%% ===== 
%% ===== 
%% ===== 
%% ===== 1
%% ===== 
%% ===== 
%% ===== 
%% ===== 
%% ===== 
%% ===== \begin{flushleft}
%% ===== A0,17
%% ===== \end{flushleft}
%% ===== 
%% ===== 
%% ===== 
%% ===== 
%% ===== 
%% ===== \begin{flushleft}
%% ===== au
%% ===== \end{flushleft}
%% ===== 
%% ===== 
%% ===== 
%% ===== 
%% ===== 
%% ===== \begin{flushleft}
%% ===== y + C(A)0,17
%% ===== \end{flushleft}
%% ===== 
%% ===== 
%% ===== 
%% ===== 
%% ===== 
%% ===== 2
%% ===== 
%% ===== 
%% ===== 
%% ===== 
%% ===== 
%% ===== \begin{flushleft}
%% ===== Q0,17
%% ===== \end{flushleft}
%% ===== 
%% ===== 
%% ===== 
%% ===== 
%% ===== 
%% ===== \begin{flushleft}
%% ===== qu
%% ===== \end{flushleft}
%% ===== 
%% ===== 
%% ===== 
%% ===== 
%% ===== 
%% ===== \begin{flushleft}
%% ===== y + C(Q)0,17
%% ===== \end{flushleft}
%% ===== 
%% ===== 
%% ===== 
%% ===== 
%% ===== 
%% ===== 3
%% ===== 
%% ===== 
%% ===== 
%% ===== 
%% ===== 
%% ===== \begin{flushleft}
%% ===== none
%% ===== \end{flushleft}
%% ===== 
%% ===== 
%% ===== 
%% ===== 
%% ===== 
%% ===== \begin{flushleft}
%% ===== du
%% ===== \end{flushleft}
%% ===== 
%% ===== 
%% ===== 
%% ===== 
%% ===== 
%% ===== \begin{flushleft}
%% ===== none; y becomes the upper 18 bits of the 36-bit zero
%% ===== \end{flushleft}
%% ===== 
%% ===== 
%% ===== \begin{flushleft}
%% ===== filled operand
%% ===== \end{flushleft}
%% ===== 
%% ===== 
%% ===== 
%% ===== 
%% ===== 
%% ===== 4
%% ===== 
%% ===== 
%% ===== 
%% ===== 
%% ===== 
%% ===== \begin{flushleft}
%% ===== PPR.IC
%% ===== \end{flushleft}
%% ===== 
%% ===== 
%% ===== 
%% ===== 
%% ===== 
%% ===== \begin{flushleft}
%% ===== ic
%% ===== \end{flushleft}
%% ===== 
%% ===== 
%% ===== 
%% ===== 
%% ===== 
%% ===== \begin{flushleft}
%% ===== y +C(PPR.IC)
%% ===== \end{flushleft}
%% ===== 
%% ===== 
%% ===== 
%% ===== 
%% ===== 
%% ===== 5
%% ===== 
%% ===== 
%% ===== 
%% ===== 
%% ===== 
%% ===== \begin{flushleft}
%% ===== A18,35
%% ===== \end{flushleft}
%% ===== 
%% ===== 
%% ===== 
%% ===== 
%% ===== 
%% ===== \begin{flushleft}
%% ===== al
%% ===== \end{flushleft}
%% ===== 
%% ===== 
%% ===== 
%% ===== 
%% ===== 
%% ===== \begin{flushleft}
%% ===== y +C(A)18,35
%% ===== \end{flushleft}
%% ===== 
%% ===== 
%% ===== 
%% ===== 
%% ===== 
%% ===== 6
%% ===== 
%% ===== 
%% ===== 
%% ===== 
%% ===== 
%% ===== \begin{flushleft}
%% ===== Q18,35
%% ===== \end{flushleft}
%% ===== 
%% ===== 
%% ===== 
%% ===== 
%% ===== 
%% ===== \begin{flushleft}
%% ===== ql
%% ===== \end{flushleft}
%% ===== 
%% ===== 
%% ===== 
%% ===== 
%% ===== 
%% ===== \begin{flushleft}
%% ===== y +C(Q)18,35
%% ===== \end{flushleft}
%% ===== 
%% ===== 
%% ===== 
%% ===== 
%% ===== 
%% ===== 7
%% ===== 
%% ===== 
%% ===== 
%% ===== 
%% ===== 
%% ===== \begin{flushleft}
%% ===== none
%% ===== \end{flushleft}
%% ===== 
%% ===== 
%% ===== 
%% ===== 
%% ===== 
%% ===== \begin{flushleft}
%% ===== dl
%% ===== \end{flushleft}
%% ===== 
%% ===== 
%% ===== 
%% ===== 
%% ===== 
%% ===== \begin{flushleft}
%% ===== none; y becomes the lower 18 bits of the 36-bit zero
%% ===== \end{flushleft}
%% ===== 
%% ===== 
%% ===== \begin{flushleft}
%% ===== filled operand
%% ===== \end{flushleft}
%% ===== 
%% ===== 
%% ===== 
%% ===== 
%% ===== 
%% ===== 10
%% ===== 
%% ===== 
%% ===== 
%% ===== 
%% ===== 
%% ===== \begin{flushleft}
%% ===== X0
%% ===== \end{flushleft}
%% ===== 
%% ===== 
%% ===== 
%% ===== 
%% ===== 
%% ===== \begin{flushleft}
%% ===== 0, x0
%% ===== \end{flushleft}
%% ===== 
%% ===== 
%% ===== 
%% ===== 
%% ===== 
%% ===== \begin{flushleft}
%% ===== y +C(X0)
%% ===== \end{flushleft}
%% ===== 
%% ===== 
%% ===== 
%% ===== 
%% ===== 
%% ===== 11
%% ===== 
%% ===== 
%% ===== 
%% ===== 
%% ===== 
%% ===== \begin{flushleft}
%% ===== X1
%% ===== \end{flushleft}
%% ===== 
%% ===== 
%% ===== 
%% ===== 
%% ===== 
%% ===== \begin{flushleft}
%% ===== 1, x1
%% ===== \end{flushleft}
%% ===== 
%% ===== 
%% ===== 
%% ===== 
%% ===== 
%% ===== \begin{flushleft}
%% ===== y +C(X1)
%% ===== \end{flushleft}
%% ===== 
%% ===== 
%% ===== 
%% ===== 
%% ===== 
%% ===== 12
%% ===== 
%% ===== 
%% ===== 
%% ===== 
%% ===== 
%% ===== \begin{flushleft}
%% ===== X2
%% ===== \end{flushleft}
%% ===== 
%% ===== 
%% ===== 
%% ===== 
%% ===== 
%% ===== \begin{flushleft}
%% ===== 2, x2
%% ===== \end{flushleft}
%% ===== 
%% ===== 
%% ===== 
%% ===== 
%% ===== 
%% ===== \begin{flushleft}
%% ===== y +C(X2)
%% ===== \end{flushleft}
%% ===== 
%% ===== 
%% ===== 
%% ===== 
%% ===== 
%% ===== 13
%% ===== 
%% ===== 
%% ===== 
%% ===== 
%% ===== 
%% ===== \begin{flushleft}
%% ===== X3
%% ===== \end{flushleft}
%% ===== 
%% ===== 
%% ===== 
%% ===== 
%% ===== 
%% ===== \begin{flushleft}
%% ===== 3, x3
%% ===== \end{flushleft}
%% ===== 
%% ===== 
%% ===== 
%% ===== 
%% ===== 
%% ===== \begin{flushleft}
%% ===== y +C(X3)
%% ===== \end{flushleft}
%% ===== 
%% ===== 
%% ===== 
%% ===== 
%% ===== 
%% ===== 14
%% ===== 
%% ===== 
%% ===== 
%% ===== 
%% ===== 
%% ===== \begin{flushleft}
%% ===== X4
%% ===== \end{flushleft}
%% ===== 
%% ===== 
%% ===== 
%% ===== 
%% ===== 
%% ===== \begin{flushleft}
%% ===== 4, x4
%% ===== \end{flushleft}
%% ===== 
%% ===== 
%% ===== 
%% ===== 
%% ===== 
%% ===== \begin{flushleft}
%% ===== y +C(X4)
%% ===== \end{flushleft}
%% ===== 
%% ===== 
%% ===== 
%% ===== 
%% ===== 
%% ===== 15
%% ===== 
%% ===== 
%% ===== 
%% ===== 
%% ===== 
%% ===== \begin{flushleft}
%% ===== X5
%% ===== \end{flushleft}
%% ===== 
%% ===== 
%% ===== 
%% ===== 
%% ===== 
%% ===== \begin{flushleft}
%% ===== 5, x5
%% ===== \end{flushleft}
%% ===== 
%% ===== 
%% ===== 
%% ===== 
%% ===== 
%% ===== \begin{flushleft}
%% ===== y +C(X5)
%% ===== \end{flushleft}
%% ===== 
%% ===== 
%% ===== 
%% ===== 
%% ===== 
%% ===== 16
%% ===== 
%% ===== 
%% ===== 
%% ===== 
%% ===== 
%% ===== \begin{flushleft}
%% ===== X6
%% ===== \end{flushleft}
%% ===== 
%% ===== 
%% ===== 
%% ===== 
%% ===== 
%% ===== \begin{flushleft}
%% ===== 6, x6
%% ===== \end{flushleft}
%% ===== 
%% ===== 
%% ===== 
%% ===== 
%% ===== 
%% ===== \begin{flushleft}
%% ===== y +C(X6)
%% ===== \end{flushleft}
%% ===== 
%% ===== 
%% ===== 
%% ===== 
%% ===== 
%% ===== 17
%% ===== 
%% ===== 
%% ===== 
%% ===== 
%% ===== 
%% ===== \begin{flushleft}
%% ===== X7
%% ===== \end{flushleft}
%% ===== 
%% ===== 
%% ===== 
%% ===== 
%% ===== 
%% ===== \begin{flushleft}
%% ===== 7, x7
%% ===== \end{flushleft}
%% ===== 
%% ===== 
%% ===== 
%% ===== 
%% ===== 
%% ===== \begin{flushleft}
%% ===== y +C(X7)
%% ===== \end{flushleft}
%% ===== 
%% ===== 
%% ===== 
%% ===== 
%% ===== 
%% ===== \begin{flushleft}
%% ===== Computed Address
%% ===== \end{flushleft}
%% ===== 
%% ===== 
%% ===== \begin{flushleft}
%% ===== y
%% ===== \end{flushleft}
%% ===== 
%% ===== 
%% ===== 
%% ===== 
%% ===== 
%% ===== \begin{flushleft}
%% ===== \newpage
%% ===== Examples:
%% ===== \end{flushleft}
%% ===== 
%% ===== 
%% ===== 
%% ===== 
%% ===== 
%% ===== \begin{flushleft}
%% ===== Location
%% ===== \end{flushleft}
%% ===== 
%% ===== 
%% ===== 
%% ===== 
%% ===== 
%% ===== \begin{flushleft}
%% ===== Instruction
%% ===== \end{flushleft}
%% ===== 
%% ===== 
%% ===== 
%% ===== 
%% ===== 
%% ===== \begin{flushleft}
%% ===== Computed address
%% ===== \end{flushleft}
%% ===== 
%% ===== 
%% ===== 
%% ===== 
%% ===== 
%% ===== 1.
%% ===== 
%% ===== 
%% ===== 
%% ===== 
%% ===== 
%% ===== \begin{flushleft}
%% ===== a
%% ===== \end{flushleft}
%% ===== 
%% ===== 
%% ===== 
%% ===== 
%% ===== 
%% ===== \begin{flushleft}
%% ===== lda
%% ===== \end{flushleft}
%% ===== 
%% ===== 
%% ===== 
%% ===== 
%% ===== 
%% ===== \begin{flushleft}
%% ===== y
%% ===== \end{flushleft}
%% ===== 
%% ===== 
%% ===== 
%% ===== 
%% ===== 
%% ===== \begin{flushleft}
%% ===== y
%% ===== \end{flushleft}
%% ===== 
%% ===== 
%% ===== 
%% ===== 
%% ===== 
%% ===== 2.
%% ===== 
%% ===== 
%% ===== 
%% ===== 
%% ===== 
%% ===== \begin{flushleft}
%% ===== a
%% ===== \end{flushleft}
%% ===== 
%% ===== 
%% ===== 
%% ===== 
%% ===== 
%% ===== \begin{flushleft}
%% ===== sta
%% ===== \end{flushleft}
%% ===== 
%% ===== 
%% ===== 
%% ===== 
%% ===== 
%% ===== \begin{flushleft}
%% ===== y,n
%% ===== \end{flushleft}
%% ===== 
%% ===== 
%% ===== 
%% ===== 
%% ===== 
%% ===== \begin{flushleft}
%% ===== y
%% ===== \end{flushleft}
%% ===== 
%% ===== 
%% ===== 
%% ===== 
%% ===== 
%% ===== 3.
%% ===== 
%% ===== 
%% ===== 
%% ===== 
%% ===== 
%% ===== \begin{flushleft}
%% ===== a
%% ===== \end{flushleft}
%% ===== 
%% ===== 
%% ===== 
%% ===== 
%% ===== 
%% ===== \begin{flushleft}
%% ===== ldaq y,au
%% ===== \end{flushleft}
%% ===== 
%% ===== 
%% ===== 
%% ===== 
%% ===== 
%% ===== \begin{flushleft}
%% ===== y + C(A)0,17
%% ===== \end{flushleft}
%% ===== 
%% ===== 
%% ===== 
%% ===== 
%% ===== 
%% ===== 4.
%% ===== 
%% ===== 
%% ===== 
%% ===== 
%% ===== 
%% ===== \begin{flushleft}
%% ===== a
%% ===== \end{flushleft}
%% ===== 
%% ===== 
%% ===== 
%% ===== 
%% ===== 
%% ===== \begin{flushleft}
%% ===== tra
%% ===== \end{flushleft}
%% ===== 
%% ===== 
%% ===== 
%% ===== 
%% ===== 
%% ===== \begin{flushleft}
%% ===== 3,ic
%% ===== \end{flushleft}
%% ===== 
%% ===== 
%% ===== 
%% ===== 
%% ===== 
%% ===== \begin{flushleft}
%% ===== a+3
%% ===== \end{flushleft}
%% ===== 
%% ===== 
%% ===== 
%% ===== 
%% ===== 
%% ===== 5.
%% ===== 
%% ===== 
%% ===== 
%% ===== 
%% ===== 
%% ===== \begin{flushleft}
%% ===== a
%% ===== \end{flushleft}
%% ===== 
%% ===== 
%% ===== 
%% ===== 
%% ===== 
%% ===== \begin{flushleft}
%% ===== ldq
%% ===== \end{flushleft}
%% ===== 
%% ===== 
%% ===== 
%% ===== 
%% ===== 
%% ===== \begin{flushleft}
%% ===== y,du
%% ===== \end{flushleft}
%% ===== 
%% ===== 
%% ===== 
%% ===== 
%% ===== 
%% ===== \begin{flushleft}
%% ===== none; operand has the form y || (00...0)18
%% ===== \end{flushleft}
%% ===== 
%% ===== 
%% ===== 
%% ===== 
%% ===== 
%% ===== 6.
%% ===== 
%% ===== 
%% ===== 
%% ===== 
%% ===== 
%% ===== \begin{flushleft}
%% ===== a
%% ===== \end{flushleft}
%% ===== 
%% ===== 
%% ===== 
%% ===== 
%% ===== 
%% ===== \begin{flushleft}
%% ===== lxl4 y,dl
%% ===== \end{flushleft}
%% ===== 
%% ===== 
%% ===== 
%% ===== 
%% ===== 
%% ===== \begin{flushleft}
%% ===== none; operand has the form (00...0)18 || y
%% ===== \end{flushleft}
%% ===== 
%% ===== 
%% ===== 
%% ===== 
%% ===== 
%% ===== 7.
%% ===== 
%% ===== 
%% ===== 
%% ===== 
%% ===== 
%% ===== \begin{flushleft}
%% ===== a
%% ===== \end{flushleft}
%% ===== 
%% ===== 
%% ===== 
%% ===== 
%% ===== 
%% ===== \begin{flushleft}
%% ===== mpy
%% ===== \end{flushleft}
%% ===== 
%% ===== 
%% ===== 
%% ===== 
%% ===== 
%% ===== \begin{flushleft}
%% ===== y,1
%% ===== \end{flushleft}
%% ===== 
%% ===== 
%% ===== 
%% ===== 
%% ===== 
%% ===== \begin{flushleft}
%% ===== y + C(X1)
%% ===== \end{flushleft}
%% ===== 
%% ===== 
%% ===== 
%% ===== 
%% ===== 
%% ===== 8.
%% ===== 
%% ===== 
%% ===== 
%% ===== 
%% ===== 
%% ===== \begin{flushleft}
%% ===== a
%% ===== \end{flushleft}
%% ===== 
%% ===== 
%% ===== 
%% ===== 
%% ===== 
%% ===== \begin{flushleft}
%% ===== stx4 y,7
%% ===== \end{flushleft}
%% ===== 
%% ===== 
%% ===== 
%% ===== 
%% ===== 
%% ===== \begin{flushleft}
%% ===== y + C(X7)
%% ===== \end{flushleft}
%% ===== 
%% ===== 
%% ===== 
%% ===== 
%% ===== 
%% ===== \begin{flushleft}

\subsubsection{Register Then Indirect (ri) Modifications}

%% ===== \end{flushleft}
%% ===== 
%% ===== 
%% ===== \begin{flushleft}
%% ===== In register then indirect modification (Tm = 1) the value of Td designates a register whose
%% ===== \end{flushleft}
%% ===== 
%% ===== 
%% ===== \begin{flushleft}
%% ===== contents are to be added to C(TPR.CA) to form a modified C(TPR.CA). This modified C(TPR.CA) is
%% ===== \end{flushleft}
%% ===== 
%% ===== 
%% ===== \begin{flushleft}
%% ===== used as a computed address to fetch an indirect word. The ADDRESS field of the indirect word is
%% ===== \end{flushleft}
%% ===== 
%% ===== 
%% ===== \begin{flushleft}
%% ===== loaded into TPR.CA and the TAG field of the indirect word is interpreted in the next step of an
%% ===== \end{flushleft}
%% ===== 
%% ===== 
%% ===== \begin{flushleft}
%% ===== indirect chain. The TALLY field of the indirect word is ignored.
%% ===== \end{flushleft}
%% ===== 
%% ===== 
%% ===== \begin{flushleft}
%% ===== The indirect chain continues until an indirect word TAG field specifies a modification
%% ===== \end{flushleft}
%% ===== 
%% ===== 
%% ===== \begin{flushleft}
%% ===== without indirection.
%% ===== \end{flushleft}
%% ===== 
%% ===== 
%% ===== \begin{flushleft}
%% ===== The coding symbol for register then indirect modification is r* where r is any of the coding
%% ===== \end{flushleft}
%% ===== 
%% ===== 
%% ===== \begin{flushleft}
%% ===== symbols for register modification as given in Table 6-1 above except du and dl. The du and dl
%% ===== \end{flushleft}
%% ===== 
%% ===== 
%% ===== \begin{flushleft}
%% ===== register codes are illegal and and their use causes an illegal procedure, illegal modifier, fault. See
%% ===== \end{flushleft}
%% ===== 
%% ===== 
%% ===== \begin{flushleft}
%% ===== Figure 6-4, Table 6-1, and the examples following.
%% ===== \end{flushleft}
%% ===== 
%% ===== 
%% ===== 
%% ===== 
%% ===== 
%% ===== \begin{flushleft}
%% ===== \newpage
%% ===== RI MOD
%% ===== \end{flushleft}
%% ===== 
%% ===== 
%% ===== 
%% ===== 
%% ===== 
%% ===== \begin{flushleft}
%% ===== Td = 3
%% ===== \end{flushleft}
%% ===== 
%% ===== 
%% ===== \begin{flushleft}
%% ===== or 7?
%% ===== \end{flushleft}
%% ===== 
%% ===== 
%% ===== 
%% ===== 
%% ===== 
%% ===== \begin{flushleft}
%% ===== Yes
%% ===== \end{flushleft}
%% ===== 
%% ===== 
%% ===== 
%% ===== 
%% ===== 
%% ===== \begin{flushleft}
%% ===== No
%% ===== \end{flushleft}
%% ===== 
%% ===== 
%% ===== \begin{flushleft}
%% ===== Td = 0?
%% ===== \end{flushleft}
%% ===== 
%% ===== 
%% ===== 
%% ===== 
%% ===== 
%% ===== \begin{flushleft}
%% ===== ABORT
%% ===== \end{flushleft}
%% ===== 
%% ===== 
%% ===== \begin{flushleft}
%% ===== illegal procedure,
%% ===== \end{flushleft}
%% ===== 
%% ===== 
%% ===== \begin{flushleft}
%% ===== illegal modifier, fault
%% ===== \end{flushleft}
%% ===== 
%% ===== 
%% ===== 
%% ===== 
%% ===== 
%% ===== \begin{flushleft}
%% ===== No
%% ===== \end{flushleft}
%% ===== 
%% ===== 
%% ===== 
%% ===== 
%% ===== 
%% ===== \begin{flushleft}
%% ===== Yes
%% ===== \end{flushleft}
%% ===== 
%% ===== 
%% ===== 
%% ===== 
%% ===== 
%% ===== \begin{flushleft}
%% ===== r = Td
%% ===== \end{flushleft}
%% ===== 
%% ===== 
%% ===== \begin{flushleft}
%% ===== C(TPR.CA) + C(r ) $\rightarrow$ C(TPR.CA)
%% ===== \end{flushleft}
%% ===== 
%% ===== 
%% ===== 
%% ===== 
%% ===== 
%% ===== \begin{flushleft}
%% ===== Indirect word fetch
%% ===== \end{flushleft}
%% ===== 
%% ===== 
%% ===== \begin{flushleft}
%% ===== APPEND CYCLE
%% ===== \end{flushleft}
%% ===== 
%% ===== 
%% ===== \begin{flushleft}
%% ===== (Figure 5-4)
%% ===== \end{flushleft}
%% ===== 
%% ===== 
%% ===== 
%% ===== 
%% ===== 
%% ===== \begin{flushleft}
%% ===== Indirect word ADDRESS
%% ===== \end{flushleft}
%% ===== 
%% ===== 
%% ===== \begin{flushleft}
%% ===== $\rightarrow$ C(TPR.CA)
%% ===== \end{flushleft}
%% ===== 
%% ===== 
%% ===== 
%% ===== 
%% ===== 
%% ===== \begin{flushleft}
%% ===== START CA
%% ===== \end{flushleft}
%% ===== 
%% ===== 
%% ===== \begin{flushleft}
%% ===== (Figure 6-2)
%% ===== \end{flushleft}
%% ===== 
%% ===== 
%% ===== 
%% ===== 
%% ===== 
%% ===== \begin{flushleft}
%% ===== Figure 6-4. Register Then Indirect Modification Flowchart
%% ===== \end{flushleft}
%% ===== 
%% ===== 
%% ===== \begin{flushleft}
%% ===== Examples:
%% ===== \end{flushleft}
%% ===== 
%% ===== 
%% ===== 
%% ===== 
%% ===== 
%% ===== \begin{flushleft}
%% ===== Location
%% ===== \end{flushleft}
%% ===== 
%% ===== 
%% ===== 
%% ===== 
%% ===== 
%% ===== \begin{flushleft}
%% ===== Instruction
%% ===== \end{flushleft}
%% ===== 
%% ===== 
%% ===== 
%% ===== 
%% ===== 
%% ===== \begin{flushleft}
%% ===== Computed address
%% ===== \end{flushleft}
%% ===== 
%% ===== 
%% ===== 
%% ===== 
%% ===== 
%% ===== \begin{flushleft}
%% ===== 1. a
%% ===== \end{flushleft}
%% ===== 
%% ===== 
%% ===== \begin{flushleft}
%% ===== b
%% ===== \end{flushleft}
%% ===== 
%% ===== 
%% ===== 
%% ===== 
%% ===== 
%% ===== \begin{flushleft}
%% ===== lda
%% ===== \end{flushleft}
%% ===== 
%% ===== 
%% ===== \begin{flushleft}
%% ===== arg
%% ===== \end{flushleft}
%% ===== 
%% ===== 
%% ===== 
%% ===== 
%% ===== 
%% ===== \begin{flushleft}
%% ===== b,*
%% ===== \end{flushleft}
%% ===== 
%% ===== 
%% ===== \begin{flushleft}
%% ===== y
%% ===== \end{flushleft}
%% ===== 
%% ===== 
%% ===== 
%% ===== 
%% ===== 
%% ===== \begin{flushleft}
%% ===== (r = null)
%% ===== \end{flushleft}
%% ===== 
%% ===== 
%% ===== \begin{flushleft}
%% ===== y
%% ===== \end{flushleft}
%% ===== 
%% ===== 
%% ===== 
%% ===== 
%% ===== 
%% ===== \begin{flushleft}
%% ===== 2. a
%% ===== \end{flushleft}
%% ===== 
%% ===== 
%% ===== \begin{flushleft}
%% ===== b+C(X1)
%% ===== \end{flushleft}
%% ===== 
%% ===== 
%% ===== 
%% ===== 
%% ===== 
%% ===== \begin{flushleft}
%% ===== ldq
%% ===== \end{flushleft}
%% ===== 
%% ===== 
%% ===== \begin{flushleft}
%% ===== arg
%% ===== \end{flushleft}
%% ===== 
%% ===== 
%% ===== 
%% ===== 
%% ===== 
%% ===== \begin{flushleft}
%% ===== b,1*
%% ===== \end{flushleft}
%% ===== 
%% ===== 
%% ===== \begin{flushleft}
%% ===== y,au
%% ===== \end{flushleft}
%% ===== 
%% ===== 
%% ===== 
%% ===== 
%% ===== 
%% ===== \begin{flushleft}
%% ===== y + C(A)0,17
%% ===== \end{flushleft}
%% ===== 
%% ===== 
%% ===== 
%% ===== 
%% ===== 
%% ===== \begin{flushleft}
%% ===== 3. a
%% ===== \end{flushleft}
%% ===== 
%% ===== 
%% ===== \begin{flushleft}
%% ===== a+4
%% ===== \end{flushleft}
%% ===== 
%% ===== 
%% ===== \begin{flushleft}
%% ===== c
%% ===== \end{flushleft}
%% ===== 
%% ===== 
%% ===== 
%% ===== 
%% ===== 
%% ===== \begin{flushleft}
%% ===== tra
%% ===== \end{flushleft}
%% ===== 
%% ===== 
%% ===== \begin{flushleft}
%% ===== arg
%% ===== \end{flushleft}
%% ===== 
%% ===== 
%% ===== \begin{flushleft}
%% ===== arg
%% ===== \end{flushleft}
%% ===== 
%% ===== 
%% ===== 
%% ===== 
%% ===== 
%% ===== \begin{flushleft}
%% ===== 4,ic*
%% ===== \end{flushleft}
%% ===== 
%% ===== 
%% ===== \begin{flushleft}
%% ===== c,*
%% ===== \end{flushleft}
%% ===== 
%% ===== 
%% ===== \begin{flushleft}
%% ===== y
%% ===== \end{flushleft}
%% ===== 
%% ===== 
%% ===== 
%% ===== 
%% ===== 
%% ===== \begin{flushleft}
%% ===== y
%% ===== \end{flushleft}
%% ===== 
%% ===== 
%% ===== 
%% ===== 
%% ===== 
%% ===== \begin{flushleft}
%% ===== 4. a
%% ===== \end{flushleft}
%% ===== 
%% ===== 
%% ===== \begin{flushleft}
%% ===== b+C(X0)
%% ===== \end{flushleft}
%% ===== 
%% ===== 
%% ===== \begin{flushleft}
%% ===== c+C(X1)
%% ===== \end{flushleft}
%% ===== 
%% ===== 
%% ===== 
%% ===== 
%% ===== 
%% ===== \begin{flushleft}
%% ===== lxl4 b,0*
%% ===== \end{flushleft}
%% ===== 
%% ===== 
%% ===== \begin{flushleft}
%% ===== arg c,1*
%% ===== \end{flushleft}
%% ===== 
%% ===== 
%% ===== \begin{flushleft}
%% ===== arg y,dl
%% ===== \end{flushleft}
%% ===== 
%% ===== 
%% ===== 
%% ===== 
%% ===== 
%% ===== \begin{flushleft}
%% ===== none; operand has the form (00...0)18 || y
%% ===== \end{flushleft}
%% ===== 
%% ===== 
%% ===== 
%% ===== 
%% ===== 
%% ===== \begin{flushleft}

\subsubsection{Indirect Then Register (ir) Modification}

%% ===== \end{flushleft}
%% ===== 
%% ===== 
%% ===== \begin{flushleft}
%% ===== In indirect then register modification (Tm = 3) the value of Td designates a register whose
%% ===== \end{flushleft}
%% ===== 
%% ===== 
%% ===== \begin{flushleft}
%% ===== contents are to be added to C(TPR.CA) to form the final modified C(TPR.CA) during the last step in
%% ===== \end{flushleft}
%% ===== 
%% ===== 
%% ===== \begin{flushleft}
%% ===== the indirect chain. The value of Td is held in a special holding register, CT-HOLD. The initial
%% ===== \end{flushleft}
%% ===== 
%% ===== 
%% ===== \begin{flushleft}
%% ===== C(TPR.CA) is used as computed address to fetch an indirect word. The ADDRESS of the indirect
%% ===== \end{flushleft}
%% ===== 
%% ===== 
%% ===== 
%% ===== 
%% ===== 
%% ===== \begin{flushleft}
%% ===== \newpage
%% ===== word is loaded into TPR.CA and the TAG field of the indirect word is interpreted in the next step of
%% ===== \end{flushleft}
%% ===== 
%% ===== 
%% ===== \begin{flushleft}
%% ===== an indirect chain. The TALLY field of the indirect word is ignored.
%% ===== \end{flushleft}
%% ===== 
%% ===== 
%% ===== \begin{flushleft}
%% ===== If the indirect word TAG field specifies a register then indirect modification, that
%% ===== \end{flushleft}
%% ===== 
%% ===== 
%% ===== \begin{flushleft}
%% ===== modification is performed and the indirect chain continues.
%% ===== \end{flushleft}
%% ===== 
%% ===== 
%% ===== \begin{flushleft}
%% ===== If the indirect word TAG field specifies indirect then register modification, the Td value from
%% ===== \end{flushleft}
%% ===== 
%% ===== 
%% ===== \begin{flushleft}
%% ===== that TAG field replaces the Td value in CT-HOLD and the indirect chain continues.
%% ===== \end{flushleft}
%% ===== 
%% ===== 
%% ===== \begin{flushleft}
%% ===== If the indirect word TAG specifies register or indirect then tally modification, that
%% ===== \end{flushleft}
%% ===== 
%% ===== 
%% ===== \begin{flushleft}
%% ===== modification is replaced with a register modification using the Td value in CT-HOLD and the
%% ===== \end{flushleft}
%% ===== 
%% ===== 
%% ===== \begin{flushleft}
%% ===== indirect chain ends.
%% ===== \end{flushleft}
%% ===== 
%% ===== 
%% ===== \begin{flushleft}
%% ===== The coding symbol for indirect then register modification is * r where r is any of the coding
%% ===== \end{flushleft}
%% ===== 
%% ===== 
%% ===== \begin{flushleft}
%% ===== symbols for register modification as given in Table 6-2 except null. See Figure 6-5, Table 6-1, and
%% ===== \end{flushleft}
%% ===== 
%% ===== 
%% ===== \begin{flushleft}
%% ===== the examples following.
%% ===== \end{flushleft}
%% ===== 
%% ===== 
%% ===== \begin{flushleft}
%% ===== IR MOD
%% ===== \end{flushleft}
%% ===== 
%% ===== 
%% ===== 
%% ===== 
%% ===== 
%% ===== \begin{flushleft}
%% ===== Td $\rightarrow$ CT-HOLD
%% ===== \end{flushleft}
%% ===== 
%% ===== 
%% ===== 
%% ===== 
%% ===== 
%% ===== \begin{flushleft}
%% ===== Indirect word fetch
%% ===== \end{flushleft}
%% ===== 
%% ===== 
%% ===== \begin{flushleft}
%% ===== APPEND CYCLE
%% ===== \end{flushleft}
%% ===== 
%% ===== 
%% ===== \begin{flushleft}
%% ===== (Figure 5-4)
%% ===== \end{flushleft}
%% ===== 
%% ===== 
%% ===== 
%% ===== 
%% ===== 
%% ===== \begin{flushleft}
%% ===== Indirect word ADDRESS
%% ===== \end{flushleft}
%% ===== 
%% ===== 
%% ===== \begin{flushleft}
%% ===== $\rightarrow$ C(TPR.CA)
%% ===== \end{flushleft}
%% ===== 
%% ===== 
%% ===== 
%% ===== 
%% ===== 
%% ===== \begin{flushleft}
%% ===== Interpret
%% ===== \end{flushleft}
%% ===== 
%% ===== 
%% ===== \begin{flushleft}
%% ===== indirect TAG
%% ===== \end{flushleft}
%% ===== 
%% ===== 
%% ===== 
%% ===== 
%% ===== 
%% ===== \begin{flushleft}
%% ===== Tm=ri
%% ===== \end{flushleft}
%% ===== 
%% ===== 
%% ===== 
%% ===== 
%% ===== 
%% ===== \begin{flushleft}
%% ===== Tm=ir
%% ===== \end{flushleft}
%% ===== 
%% ===== 
%% ===== 
%% ===== 
%% ===== 
%% ===== \begin{flushleft}
%% ===== Tm=r`
%% ===== \end{flushleft}
%% ===== 
%% ===== 
%% ===== 
%% ===== 
%% ===== 
%% ===== \begin{flushleft}
%% ===== r = Td
%% ===== \end{flushleft}
%% ===== 
%% ===== 
%% ===== \begin{flushleft}
%% ===== C(TPR.CA) + C(r ) $\rightarrow$ C(TPR.CA)
%% ===== \end{flushleft}
%% ===== 
%% ===== 
%% ===== 
%% ===== 
%% ===== 
%% ===== \begin{flushleft}
%% ===== Tm=it
%% ===== \end{flushleft}
%% ===== 
%% ===== 
%% ===== \begin{flushleft}
%% ===== No
%% ===== \end{flushleft}
%% ===== 
%% ===== 
%% ===== 
%% ===== 
%% ===== 
%% ===== \begin{flushleft}
%% ===== Fault tag
%% ===== \end{flushleft}
%% ===== 
%% ===== 
%% ===== \begin{flushleft}
%% ===== 2 or 3?
%% ===== \end{flushleft}
%% ===== 
%% ===== 
%% ===== \begin{flushleft}
%% ===== Yes
%% ===== \end{flushleft}
%% ===== 
%% ===== 
%% ===== 
%% ===== 
%% ===== 
%% ===== \begin{flushleft}
%% ===== r = CT-HOLD
%% ===== \end{flushleft}
%% ===== 
%% ===== 
%% ===== \begin{flushleft}
%% ===== C(TPR.CA) + C(r ) $\rightarrow$ C(TPR.CA)
%% ===== \end{flushleft}
%% ===== 
%% ===== 
%% ===== 
%% ===== 
%% ===== 
%% ===== \begin{flushleft}
%% ===== ABORT
%% ===== \end{flushleft}
%% ===== 
%% ===== 
%% ===== \begin{flushleft}
%% ===== END CA
%% ===== \end{flushleft}
%% ===== 
%% ===== 
%% ===== 
%% ===== 
%% ===== 
%% ===== \begin{flushleft}
%% ===== fault tag 2
%% ===== \end{flushleft}
%% ===== 
%% ===== 
%% ===== \begin{flushleft}
%% ===== or 3 fault
%% ===== \end{flushleft}
%% ===== 
%% ===== 
%% ===== 
%% ===== 
%% ===== 
%% ===== \begin{flushleft}
%% ===== Figure 6-5. Indirect Then Register Modification Flowchart
%% ===== \end{flushleft}
%% ===== 
%% ===== 
%% ===== 
%% ===== 
%% ===== 
%% ===== \begin{flushleft}
%% ===== \newpage
%% ===== Examples:
%% ===== \end{flushleft}
%% ===== 
%% ===== 
%% ===== 
%% ===== 
%% ===== 
%% ===== \begin{flushleft}
%% ===== Location Instruction
%% ===== \end{flushleft}
%% ===== 
%% ===== 
%% ===== \begin{flushleft}
%% ===== b,*n
%% ===== \end{flushleft}
%% ===== 
%% ===== 
%% ===== \begin{flushleft}
%% ===== y,2
%% ===== \end{flushleft}
%% ===== 
%% ===== 
%% ===== 
%% ===== 
%% ===== 
%% ===== \begin{flushleft}
%% ===== Computed address
%% ===== \end{flushleft}
%% ===== 
%% ===== 
%% ===== 
%% ===== 
%% ===== 
%% ===== \begin{flushleft}
%% ===== 1. a
%% ===== \end{flushleft}
%% ===== 
%% ===== 
%% ===== \begin{flushleft}
%% ===== b
%% ===== \end{flushleft}
%% ===== 
%% ===== 
%% ===== 
%% ===== 
%% ===== 
%% ===== \begin{flushleft}
%% ===== lda
%% ===== \end{flushleft}
%% ===== 
%% ===== 
%% ===== \begin{flushleft}
%% ===== arg
%% ===== \end{flushleft}
%% ===== 
%% ===== 
%% ===== 
%% ===== 
%% ===== 
%% ===== \begin{flushleft}
%% ===== 2. a
%% ===== \end{flushleft}
%% ===== 
%% ===== 
%% ===== \begin{flushleft}
%% ===== b
%% ===== \end{flushleft}
%% ===== 
%% ===== 
%% ===== 
%% ===== 
%% ===== 
%% ===== \begin{flushleft}
%% ===== lxl2 b,*dl
%% ===== \end{flushleft}
%% ===== 
%% ===== 
%% ===== \begin{flushleft}
%% ===== sta y,au
%% ===== \end{flushleft}
%% ===== 
%% ===== 
%% ===== 
%% ===== 
%% ===== 
%% ===== \begin{flushleft}
%% ===== (CT-HOLD = dl)
%% ===== \end{flushleft}
%% ===== 
%% ===== 
%% ===== \begin{flushleft}
%% ===== none; operand has the form (00...0)18 || y
%% ===== \end{flushleft}
%% ===== 
%% ===== 
%% ===== 
%% ===== 
%% ===== 
%% ===== \begin{flushleft}
%% ===== 3. a
%% ===== \end{flushleft}
%% ===== 
%% ===== 
%% ===== \begin{flushleft}
%% ===== b
%% ===== \end{flushleft}
%% ===== 
%% ===== 
%% ===== \begin{flushleft}
%% ===== c
%% ===== \end{flushleft}
%% ===== 
%% ===== 
%% ===== \begin{flushleft}
%% ===== d
%% ===== \end{flushleft}
%% ===== 
%% ===== 
%% ===== 
%% ===== 
%% ===== 
%% ===== \begin{flushleft}
%% ===== lda
%% ===== \end{flushleft}
%% ===== 
%% ===== 
%% ===== \begin{flushleft}
%% ===== arg
%% ===== \end{flushleft}
%% ===== 
%% ===== 
%% ===== \begin{flushleft}
%% ===== arg
%% ===== \end{flushleft}
%% ===== 
%% ===== 
%% ===== \begin{flushleft}
%% ===== arg
%% ===== \end{flushleft}
%% ===== 
%% ===== 
%% ===== 
%% ===== 
%% ===== 
%% ===== \begin{flushleft}
%% ===== (CT-HOLD = x1)
%% ===== \end{flushleft}
%% ===== 
%% ===== 
%% ===== 
%% ===== 
%% ===== 
%% ===== \begin{flushleft}
%% ===== 4. a
%% ===== \end{flushleft}
%% ===== 
%% ===== 
%% ===== \begin{flushleft}
%% ===== b+C(X1)
%% ===== \end{flushleft}
%% ===== 
%% ===== 
%% ===== \begin{flushleft}
%% ===== c
%% ===== \end{flushleft}
%% ===== 
%% ===== 
%% ===== 
%% ===== 
%% ===== 
%% ===== \begin{flushleft}
%% ===== ldx0 b,1*
%% ===== \end{flushleft}
%% ===== 
%% ===== 
%% ===== \begin{flushleft}
%% ===== arg c,*ic
%% ===== \end{flushleft}
%% ===== 
%% ===== 
%% ===== \begin{flushleft}
%% ===== arg 5,dl
%% ===== \end{flushleft}
%% ===== 
%% ===== 
%% ===== 
%% ===== 
%% ===== 
%% ===== \begin{flushleft}
%% ===== b,*1
%% ===== \end{flushleft}
%% ===== 
%% ===== 
%% ===== \begin{flushleft}
%% ===== c,n*
%% ===== \end{flushleft}
%% ===== 
%% ===== 
%% ===== \begin{flushleft}
%% ===== d,*4
%% ===== \end{flushleft}
%% ===== 
%% ===== 
%% ===== \begin{flushleft}
%% ===== y,ql
%% ===== \end{flushleft}
%% ===== 
%% ===== 
%% ===== 
%% ===== 
%% ===== 
%% ===== \begin{flushleft}
%% ===== (CT-HOLD = n)
%% ===== \end{flushleft}
%% ===== 
%% ===== 
%% ===== \begin{flushleft}
%% ===== y
%% ===== \end{flushleft}
%% ===== 
%% ===== 
%% ===== 
%% ===== 
%% ===== 
%% ===== \begin{flushleft}
%% ===== (CT-HOLD = x4)
%% ===== \end{flushleft}
%% ===== 
%% ===== 
%% ===== \begin{flushleft}
%% ===== y + C(X4)
%% ===== \end{flushleft}
%% ===== 
%% ===== 
%% ===== 
%% ===== 
%% ===== 
%% ===== \begin{flushleft}
%% ===== (CT-HOLD = ic)
%% ===== \end{flushleft}
%% ===== 
%% ===== 
%% ===== \begin{flushleft}
%% ===== a+5
%% ===== \end{flushleft}
%% ===== 
%% ===== 
%% ===== 
%% ===== 
%% ===== 
%% ===== \begin{flushleft}

\subsubsection{Indirect Then Tally (it) Modification}

%% ===== \end{flushleft}
%% ===== 
%% ===== 
%% ===== \begin{flushleft}
%% ===== In indirect then tally modification (Tm = 2) the value of Td specifies a variation. The initial
%% ===== \end{flushleft}
%% ===== 
%% ===== 
%% ===== \begin{flushleft}
%% ===== C(TPR.CA) is used an as computed address to fetch an indirect word. The indirect word is
%% ===== \end{flushleft}
%% ===== 
%% ===== 
%% ===== \begin{flushleft}
%% ===== interpreted and possibly altered as the modification is performed. If the specified variation
%% ===== \end{flushleft}
%% ===== 
%% ===== 
%% ===== \begin{flushleft}
%% ===== involves alteration of the indirect word, the indirect word is fetched with a special main memory
%% ===== \end{flushleft}
%% ===== 
%% ===== 
%% ===== \begin{flushleft}
%% ===== cycle that prevents other processors from accessing it until the alteration is complete.
%% ===== \end{flushleft}
%% ===== 
%% ===== 
%% ===== \begin{flushleft}
%% ===== The TALLY field of the indirect word is used to count references made to the indirect word.
%% ===== \end{flushleft}
%% ===== 
%% ===== 
%% ===== \begin{flushleft}
%% ===== It has a maximum range of 4096. If the TALLY field has the value 0 after a reference to the
%% ===== \end{flushleft}
%% ===== 
%% ===== 
%% ===== \begin{flushleft}
%% ===== indirect word, the tally runout indicator will be set ON, otherwise the tally runout indicator is set
%% ===== \end{flushleft}
%% ===== 
%% ===== 
%% ===== \begin{flushleft}
%% ===== OFF. The value of the TALLY field and the state of the tally runout indicator have no effect on
%% ===== \end{flushleft}
%% ===== 
%% ===== 
%% ===== \begin{flushleft}
%% ===== computed address formation.
%% ===== \end{flushleft}
%% ===== 
%% ===== 
%% ===== \begin{flushleft}
%% ===== If there is more than one indirect word in an indirect chain that is referenced by a tally
%% ===== \end{flushleft}
%% ===== 
%% ===== 
%% ===== \begin{flushleft}
%% ===== counting variation, only the state of the TALLY field of the last such word is reflected in the tally
%% ===== \end{flushleft}
%% ===== 
%% ===== 
%% ===== \begin{flushleft}
%% ===== runout indicator.
%% ===== \end{flushleft}
%% ===== 
%% ===== 
%% ===== \begin{flushleft}
%% ===== The variations of the indirect then tally modification are given in Table 6-3 and explained in
%% ===== \end{flushleft}
%% ===== 
%% ===== 
%% ===== \begin{flushleft}
%% ===== detail in the paragraphs following. Those entries given as {``}Undefined'' cause an illegal procedure,
%% ===== \end{flushleft}
%% ===== 
%% ===== 
%% ===== \begin{flushleft}
%% ===== illegal modifier, fault. See Figure 6-6, Table 6-1, and the examples following.
%% ===== \end{flushleft}
%% ===== 
%% ===== 
%% ===== 
%% ===== 
%% ===== 
%% ===== \begin{flushleft}
%% ===== Table 6-3. Variations of Indirect Then Tally Modification
%% ===== \end{flushleft}
%% ===== 
%% ===== 
%% ===== \begin{flushleft}
%% ===== Td
%% ===== \end{flushleft}
%% ===== 
%% ===== 
%% ===== \begin{flushleft}
%% ===== value
%% ===== \end{flushleft}
%% ===== 
%% ===== 
%% ===== 
%% ===== 
%% ===== 
%% ===== \begin{flushleft}
%% ===== Coding
%% ===== \end{flushleft}
%% ===== 
%% ===== 
%% ===== \begin{flushleft}
%% ===== symbol
%% ===== \end{flushleft}
%% ===== 
%% ===== 
%% ===== 
%% ===== 
%% ===== 
%% ===== 0
%% ===== 
%% ===== 
%% ===== 
%% ===== 
%% ===== 
%% ===== \begin{flushleft}
%% ===== f1
%% ===== \end{flushleft}
%% ===== 
%% ===== 
%% ===== 
%% ===== 
%% ===== 
%% ===== \begin{flushleft}
%% ===== Computed address
%% ===== \end{flushleft}
%% ===== 
%% ===== 
%% ===== \begin{flushleft}
%% ===== Fault tag 1
%% ===== \end{flushleft}
%% ===== 
%% ===== 
%% ===== 
%% ===== 
%% ===== 
%% ===== 1
%% ===== 
%% ===== 
%% ===== 
%% ===== 
%% ===== 
%% ===== \begin{flushleft}
%% ===== Undefined (see itp modification later in this section)
%% ===== \end{flushleft}
%% ===== 
%% ===== 
%% ===== 
%% ===== 
%% ===== 
%% ===== 2
%% ===== 
%% ===== 
%% ===== 
%% ===== 
%% ===== 
%% ===== \begin{flushleft}
%% ===== Undefined
%% ===== \end{flushleft}
%% ===== 
%% ===== 
%% ===== 
%% ===== 
%% ===== 
%% ===== 3
%% ===== 
%% ===== 
%% ===== 
%% ===== 
%% ===== 
%% ===== \begin{flushleft}
%% ===== Undefined (see its modification later in this section)
%% ===== \end{flushleft}
%% ===== 
%% ===== 
%% ===== 
%% ===== 
%% ===== 
%% ===== 4
%% ===== 
%% ===== 
%% ===== 
%% ===== 
%% ===== 
%% ===== \begin{flushleft}
%% ===== sd
%% ===== \end{flushleft}
%% ===== 
%% ===== 
%% ===== 
%% ===== 
%% ===== 
%% ===== \begin{flushleft}
%% ===== Subtract delta
%% ===== \end{flushleft}
%% ===== 
%% ===== 
%% ===== 
%% ===== 
%% ===== 
%% ===== \begin{flushleft}
%% ===== \newpage
%% ===== Td
%% ===== \end{flushleft}
%% ===== 
%% ===== 
%% ===== \begin{flushleft}
%% ===== value
%% ===== \end{flushleft}
%% ===== 
%% ===== 
%% ===== 
%% ===== 
%% ===== 
%% ===== \begin{flushleft}
%% ===== Coding
%% ===== \end{flushleft}
%% ===== 
%% ===== 
%% ===== \begin{flushleft}
%% ===== symbol
%% ===== \end{flushleft}
%% ===== 
%% ===== 
%% ===== 
%% ===== 
%% ===== 
%% ===== 5
%% ===== 
%% ===== 
%% ===== 
%% ===== 
%% ===== 
%% ===== \begin{flushleft}
%% ===== scr
%% ===== \end{flushleft}
%% ===== 
%% ===== 
%% ===== 
%% ===== 
%% ===== 
%% ===== \begin{flushleft}
%% ===== Sequence character reverse
%% ===== \end{flushleft}
%% ===== 
%% ===== 
%% ===== 
%% ===== 
%% ===== 
%% ===== 6
%% ===== 
%% ===== 
%% ===== 
%% ===== 
%% ===== 
%% ===== \begin{flushleft}
%% ===== f2
%% ===== \end{flushleft}
%% ===== 
%% ===== 
%% ===== 
%% ===== 
%% ===== 
%% ===== \begin{flushleft}
%% ===== Fault tag 2
%% ===== \end{flushleft}
%% ===== 
%% ===== 
%% ===== 
%% ===== 
%% ===== 
%% ===== 7
%% ===== 
%% ===== 
%% ===== 
%% ===== 
%% ===== 
%% ===== \begin{flushleft}
%% ===== f3
%% ===== \end{flushleft}
%% ===== 
%% ===== 
%% ===== 
%% ===== 
%% ===== 
%% ===== \begin{flushleft}
%% ===== Fault tag 3
%% ===== \end{flushleft}
%% ===== 
%% ===== 
%% ===== 
%% ===== 
%% ===== 
%% ===== 10
%% ===== 
%% ===== 
%% ===== 
%% ===== 
%% ===== 
%% ===== \begin{flushleft}
%% ===== ci
%% ===== \end{flushleft}
%% ===== 
%% ===== 
%% ===== 
%% ===== 
%% ===== 
%% ===== \begin{flushleft}
%% ===== Character indirect
%% ===== \end{flushleft}
%% ===== 
%% ===== 
%% ===== 
%% ===== 
%% ===== 
%% ===== 11
%% ===== 
%% ===== 
%% ===== 
%% ===== 
%% ===== 
%% ===== \begin{flushleft}
%% ===== i
%% ===== \end{flushleft}
%% ===== 
%% ===== 
%% ===== 
%% ===== 
%% ===== 
%% ===== \begin{flushleft}
%% ===== Indirect
%% ===== \end{flushleft}
%% ===== 
%% ===== 
%% ===== 
%% ===== 
%% ===== 
%% ===== 12
%% ===== 
%% ===== 
%% ===== 
%% ===== 
%% ===== 
%% ===== \begin{flushleft}
%% ===== sc
%% ===== \end{flushleft}
%% ===== 
%% ===== 
%% ===== 
%% ===== 
%% ===== 
%% ===== \begin{flushleft}
%% ===== Sequence character
%% ===== \end{flushleft}
%% ===== 
%% ===== 
%% ===== 
%% ===== 
%% ===== 
%% ===== 13
%% ===== 
%% ===== 
%% ===== 
%% ===== 
%% ===== 
%% ===== \begin{flushleft}
%% ===== ad
%% ===== \end{flushleft}
%% ===== 
%% ===== 
%% ===== 
%% ===== 
%% ===== 
%% ===== \begin{flushleft}
%% ===== Add delta
%% ===== \end{flushleft}
%% ===== 
%% ===== 
%% ===== 
%% ===== 
%% ===== 
%% ===== 14
%% ===== 
%% ===== 
%% ===== 
%% ===== 
%% ===== 
%% ===== \begin{flushleft}
%% ===== di
%% ===== \end{flushleft}
%% ===== 
%% ===== 
%% ===== 
%% ===== 
%% ===== 
%% ===== \begin{flushleft}
%% ===== Decrement address, increment tally
%% ===== \end{flushleft}
%% ===== 
%% ===== 
%% ===== 
%% ===== 
%% ===== 
%% ===== 15
%% ===== 
%% ===== 
%% ===== 
%% ===== 
%% ===== 
%% ===== \begin{flushleft}
%% ===== dic
%% ===== \end{flushleft}
%% ===== 
%% ===== 
%% ===== 
%% ===== 
%% ===== 
%% ===== \begin{flushleft}
%% ===== Decrement address, increment tally, and continue
%% ===== \end{flushleft}
%% ===== 
%% ===== 
%% ===== 
%% ===== 
%% ===== 
%% ===== 16
%% ===== 
%% ===== 
%% ===== 
%% ===== 
%% ===== 
%% ===== \begin{flushleft}
%% ===== id
%% ===== \end{flushleft}
%% ===== 
%% ===== 
%% ===== 
%% ===== 
%% ===== 
%% ===== \begin{flushleft}
%% ===== Increment address, decrement tally
%% ===== \end{flushleft}
%% ===== 
%% ===== 
%% ===== 
%% ===== 
%% ===== 
%% ===== 17
%% ===== 
%% ===== 
%% ===== 
%% ===== 
%% ===== 
%% ===== \begin{flushleft}
%% ===== idc
%% ===== \end{flushleft}
%% ===== 
%% ===== 
%% ===== 
%% ===== 
%% ===== 
%% ===== \begin{flushleft}
%% ===== Increment address, decrement tally, and continue
%% ===== \end{flushleft}
%% ===== 
%% ===== 
%% ===== 
%% ===== 
%% ===== 
%% ===== \begin{flushleft}
%% ===== Computed address
%% ===== \end{flushleft}
%% ===== 
%% ===== 
%% ===== 
%% ===== 
%% ===== 
%% ===== \begin{flushleft}
%% ===== Fault tag 1 (Td = 0)
%% ===== \end{flushleft}
%% ===== 
%% ===== 
%% ===== \begin{flushleft}
%% ===== If this variation appears in an indirect word and the TAG of the instruction word or
%% ===== \end{flushleft}
%% ===== 
%% ===== 
%% ===== \begin{flushleft}
%% ===== preceding indirect word is indirect then register (ir), then terminate computed address
%% ===== \end{flushleft}
%% ===== 
%% ===== 
%% ===== \begin{flushleft}
%% ===== formation with a register (r) modification using the register held in CT-HOLD. If this
%% ===== \end{flushleft}
%% ===== 
%% ===== 
%% ===== \begin{flushleft}
%% ===== variation appears in an instruction word or in an indirect word and the TAG of the
%% ===== \end{flushleft}
%% ===== 
%% ===== 
%% ===== \begin{flushleft}
%% ===== instruction word or preceding indirect word is not indirect then register (ir), then generate
%% ===== \end{flushleft}
%% ===== 
%% ===== 
%% ===== \begin{flushleft}
%% ===== a fault tag 1 fault.
%% ===== \end{flushleft}
%% ===== 
%% ===== 
%% ===== \begin{flushleft}
%% ===== C(TPR.CA) at the time of the fault contains the computed address of the word containing
%% ===== \end{flushleft}
%% ===== 
%% ===== 
%% ===== \begin{flushleft}
%% ===== the fault tag 1 variation. Thus, the ADDRESS and TALLY fields of that word may contain
%% ===== \end{flushleft}
%% ===== 
%% ===== 
%% ===== \begin{flushleft}
%% ===== information relative to recovery from the fault.
%% ===== \end{flushleft}
%% ===== 
%% ===== 
%% ===== \begin{flushleft}
%% ===== Subtract delta (Td = 4)
%% ===== \end{flushleft}
%% ===== 
%% ===== 
%% ===== \begin{flushleft}
%% ===== The TAG field of the indirect word is interpreted as a 6-bit, unsigned, positive address
%% ===== \end{flushleft}
%% ===== 
%% ===== 
%% ===== \begin{flushleft}
%% ===== increment value, delta. For each reference to the indirect word, the ADDRESS field is
%% ===== \end{flushleft}
%% ===== 
%% ===== 
%% ===== \begin{flushleft}
%% ===== reduced by delta and the TALLY field is increased by 1 before the computed address is
%% ===== \end{flushleft}
%% ===== 
%% ===== 
%% ===== \begin{flushleft}
%% ===== formed. ADDRESS arithmetic is modulo 218. TALLY arithmetic is modulo 4096. If the
%% ===== \end{flushleft}
%% ===== 
%% ===== 
%% ===== \begin{flushleft}
%% ===== TALLY field overflows to 0, the tally runout indicator is set ON, otherwise it is set OFF. The
%% ===== \end{flushleft}
%% ===== 
%% ===== 
%% ===== \begin{flushleft}
%% ===== computed address is the value of the decremented ADDRESS field of the indirect word.
%% ===== \end{flushleft}
%% ===== 
%% ===== 
%% ===== \begin{flushleft}
%% ===== Example:
%% ===== \end{flushleft}
%% ===== 
%% ===== 
%% ===== 
%% ===== 
%% ===== 
%% ===== \begin{flushleft}
%% ===== Location
%% ===== \end{flushleft}
%% ===== 
%% ===== 
%% ===== 
%% ===== 
%% ===== 
%% ===== \begin{flushleft}
%% ===== Instruction
%% ===== \end{flushleft}
%% ===== 
%% ===== 
%% ===== 
%% ===== 
%% ===== 
%% ===== \begin{flushleft}
%% ===== a
%% ===== \end{flushleft}
%% ===== 
%% ===== 
%% ===== \begin{flushleft}
%% ===== b
%% ===== \end{flushleft}
%% ===== 
%% ===== 
%% ===== 
%% ===== 
%% ===== 
%% ===== \begin{flushleft}
%% ===== lda
%% ===== \end{flushleft}
%% ===== 
%% ===== 
%% ===== \begin{flushleft}
%% ===== vfd
%% ===== \end{flushleft}
%% ===== 
%% ===== 
%% ===== 
%% ===== 
%% ===== 
%% ===== \begin{flushleft}
%% ===== b,sd
%% ===== \end{flushleft}
%% ===== 
%% ===== 
%% ===== \begin{flushleft}
%% ===== 18/c,12/t,6/d
%% ===== \end{flushleft}
%% ===== 
%% ===== 
%% ===== 
%% ===== 
%% ===== 
%% ===== \begin{flushleft}
%% ===== Reference
%% ===== \end{flushleft}
%% ===== 
%% ===== 
%% ===== \begin{flushleft}
%% ===== count
%% ===== \end{flushleft}
%% ===== 
%% ===== 
%% ===== 
%% ===== 
%% ===== 
%% ===== \begin{flushleft}
%% ===== Computed
%% ===== \end{flushleft}
%% ===== 
%% ===== 
%% ===== \begin{flushleft}
%% ===== address
%% ===== \end{flushleft}
%% ===== 
%% ===== 
%% ===== 
%% ===== 
%% ===== 
%% ===== \begin{flushleft}
%% ===== Tally value
%% ===== \end{flushleft}
%% ===== 
%% ===== 
%% ===== 
%% ===== 
%% ===== 
%% ===== 1
%% ===== 
%% ===== 
%% ===== 2
%% ===== 
%% ===== 
%% ===== 3
%% ===== 
%% ===== 
%% ===== ...
%% ===== 
%% ===== 
%% ===== \begin{flushleft}
%% ===== n
%% ===== \end{flushleft}
%% ===== 
%% ===== 
%% ===== 
%% ===== 
%% ===== 
%% ===== \begin{flushleft}
%% ===== c-d
%% ===== \end{flushleft}
%% ===== 
%% ===== 
%% ===== \begin{flushleft}
%% ===== c-2d
%% ===== \end{flushleft}
%% ===== 
%% ===== 
%% ===== \begin{flushleft}
%% ===== c-3d
%% ===== \end{flushleft}
%% ===== 
%% ===== 
%% ===== 
%% ===== 
%% ===== 
%% ===== \begin{flushleft}
%% ===== t+l
%% ===== \end{flushleft}
%% ===== 
%% ===== 
%% ===== \begin{flushleft}
%% ===== t+2
%% ===== \end{flushleft}
%% ===== 
%% ===== 
%% ===== \begin{flushleft}
%% ===== t+3
%% ===== \end{flushleft}
%% ===== 
%% ===== 
%% ===== 
%% ===== 
%% ===== 
%% ===== \begin{flushleft}
%% ===== c-nd
%% ===== \end{flushleft}
%% ===== 
%% ===== 
%% ===== 
%% ===== 
%% ===== 
%% ===== \begin{flushleft}
%% ===== t+n
%% ===== \end{flushleft}
%% ===== 
%% ===== 
%% ===== 
%% ===== 
%% ===== 
%% ===== \begin{flushleft}
%% ===== Sequence character reverse (Td = 5)
%% ===== \end{flushleft}
%% ===== 
%% ===== 
%% ===== \begin{flushleft}
%% ===== Bit 30 of the TAG field of the indirect word is interpreted as a character size flag, tb, with
%% ===== \end{flushleft}
%% ===== 
%% ===== 
%% ===== \begin{flushleft}
%% ===== the value 0 indicating 6-bit characters and the value 1 indicating 9-bit bytes. Bits 33-35 of
%% ===== \end{flushleft}
%% ===== 
%% ===== 
%% ===== 
%% ===== 
%% ===== 
%% ===== \begin{flushleft}
%% ===== \newpage
%% ===== the TAG field are interpreted as a 3-bit character/byte position counter, cf. Bits 31-32 of the
%% ===== \end{flushleft}
%% ===== 
%% ===== 
%% ===== \begin{flushleft}
%% ===== TAG field must be zero.
%% ===== \end{flushleft}
%% ===== 
%% ===== 
%% ===== \begin{flushleft}
%% ===== For each reference to the indirect word, the character counter, cf, is reduced by 1 and the
%% ===== \end{flushleft}
%% ===== 
%% ===== 
%% ===== \begin{flushleft}
%% ===== TALLY field is increased by 1 before the computed address is formed. Character count
%% ===== \end{flushleft}
%% ===== 
%% ===== 
%% ===== \begin{flushleft}
%% ===== arithmetic is modulo 6 for 6-bit characters and modulo 4 for 9-bit bytes. If the character
%% ===== \end{flushleft}
%% ===== 
%% ===== 
%% ===== \begin{flushleft}
%% ===== count, cf, underflows to -1, it is reset to 5 for 6-bit characters or to 3 for 9-bit bytes and
%% ===== \end{flushleft}
%% ===== 
%% ===== 
%% ===== \begin{flushleft}
%% ===== ADDRESS is reduced by 1. ADDRESS arithmetic is modulo 218. TALLY arithmetic is
%% ===== \end{flushleft}
%% ===== 
%% ===== 
%% ===== \begin{flushleft}
%% ===== modulo 4096. If the TALLY field overflows to 0, the tally runout indicator is set ON,
%% ===== \end{flushleft}
%% ===== 
%% ===== 
%% ===== \begin{flushleft}
%% ===== otherwise it is set OFF. The computed address is the (possibly) decremented value of the
%% ===== \end{flushleft}
%% ===== 
%% ===== 
%% ===== \begin{flushleft}
%% ===== ADDRESS field of the indirect word.
%% ===== \end{flushleft}
%% ===== 
%% ===== 
%% ===== \begin{flushleft}
%% ===== The effective character/byte number is the
%% ===== \end{flushleft}
%% ===== 
%% ===== 
%% ===== \begin{flushleft}
%% ===== decremented value of the character position count, cf, field of the indirect word.
%% ===== \end{flushleft}
%% ===== 
%% ===== 
%% ===== \begin{flushleft}
%% ===== A 36-bit operand is formed by high-order zero filling the value of character cf-l of
%% ===== \end{flushleft}
%% ===== 
%% ===== 
%% ===== \begin{flushleft}
%% ===== C(computed address) with an appropriate number of bits .
%% ===== \end{flushleft}
%% ===== 
%% ===== 
%% ===== \begin{flushleft}
%% ===== Examples:
%% ===== \end{flushleft}
%% ===== 
%% ===== 
%% ===== 
%% ===== 
%% ===== 
%% ===== \begin{flushleft}
%% ===== Location
%% ===== \end{flushleft}
%% ===== 
%% ===== 
%% ===== 
%% ===== 
%% ===== 
%% ===== \begin{flushleft}
%% ===== Instruction
%% ===== \end{flushleft}
%% ===== 
%% ===== 
%% ===== 
%% ===== 
%% ===== 
%% ===== \begin{flushleft}
%% ===== a
%% ===== \end{flushleft}
%% ===== 
%% ===== 
%% ===== \begin{flushleft}
%% ===== b
%% ===== \end{flushleft}
%% ===== 
%% ===== 
%% ===== \begin{flushleft}
%% ===== c
%% ===== \end{flushleft}
%% ===== 
%% ===== 
%% ===== 
%% ===== 
%% ===== 
%% ===== \begin{flushleft}
%% ===== lda
%% ===== \end{flushleft}
%% ===== 
%% ===== 
%% ===== \begin{flushleft}
%% ===== vfd
%% ===== \end{flushleft}
%% ===== 
%% ===== 
%% ===== \begin{flushleft}
%% ===== bci
%% ===== \end{flushleft}
%% ===== 
%% ===== 
%% ===== 
%% ===== 
%% ===== 
%% ===== \begin{flushleft}
%% ===== b,scr
%% ===== \end{flushleft}
%% ===== 
%% ===== 
%% ===== \begin{flushleft}
%% ===== 18/c+1,12/t,1/0,5/3
%% ===== \end{flushleft}
%% ===== 
%% ===== 
%% ===== \begin{flushleft}
%% ===== {``}ABCDEFGHIJKL''
%% ===== \end{flushleft}
%% ===== 
%% ===== 
%% ===== 
%% ===== 
%% ===== 
%% ===== \begin{flushleft}
%% ===== a
%% ===== \end{flushleft}
%% ===== 
%% ===== 
%% ===== \begin{flushleft}
%% ===== b
%% ===== \end{flushleft}
%% ===== 
%% ===== 
%% ===== \begin{flushleft}
%% ===== c
%% ===== \end{flushleft}
%% ===== 
%% ===== 
%% ===== 
%% ===== 
%% ===== 
%% ===== \begin{flushleft}
%% ===== lda
%% ===== \end{flushleft}
%% ===== 
%% ===== 
%% ===== \begin{flushleft}
%% ===== vfd
%% ===== \end{flushleft}
%% ===== 
%% ===== 
%% ===== \begin{flushleft}
%% ===== aci
%% ===== \end{flushleft}
%% ===== 
%% ===== 
%% ===== 
%% ===== 
%% ===== 
%% ===== \begin{flushleft}
%% ===== b,scr
%% ===== \end{flushleft}
%% ===== 
%% ===== 
%% ===== \begin{flushleft}
%% ===== 18/c+1,12/t,1/1,5/3
%% ===== \end{flushleft}
%% ===== 
%% ===== 
%% ===== \begin{flushleft}
%% ===== {``}abcdefgh''
%% ===== \end{flushleft}
%% ===== 
%% ===== 
%% ===== 
%% ===== 
%% ===== 
%% ===== \begin{flushleft}
%% ===== Reference
%% ===== \end{flushleft}
%% ===== 
%% ===== 
%% ===== \begin{flushleft}
%% ===== count
%% ===== \end{flushleft}
%% ===== 
%% ===== 
%% ===== 
%% ===== 
%% ===== 
%% ===== \begin{flushleft}
%% ===== cf
%% ===== \end{flushleft}
%% ===== 
%% ===== 
%% ===== 
%% ===== 
%% ===== 
%% ===== \begin{flushleft}
%% ===== Computed
%% ===== \end{flushleft}
%% ===== 
%% ===== 
%% ===== \begin{flushleft}
%% ===== address
%% ===== \end{flushleft}
%% ===== 
%% ===== 
%% ===== 
%% ===== 
%% ===== 
%% ===== \begin{flushleft}
%% ===== Tally
%% ===== \end{flushleft}
%% ===== 
%% ===== 
%% ===== \begin{flushleft}
%% ===== value Operand
%% ===== \end{flushleft}
%% ===== 
%% ===== 
%% ===== 
%% ===== 
%% ===== 
%% ===== 1
%% ===== 
%% ===== 
%% ===== 2
%% ===== 
%% ===== 
%% ===== 3
%% ===== 
%% ===== 
%% ===== 4
%% ===== 
%% ===== 
%% ===== 5
%% ===== 
%% ===== 
%% ===== ...
%% ===== 
%% ===== 
%% ===== 
%% ===== 
%% ===== 
%% ===== 2
%% ===== 
%% ===== 
%% ===== 1
%% ===== 
%% ===== 
%% ===== 0
%% ===== 
%% ===== 
%% ===== 5
%% ===== 
%% ===== 
%% ===== 4
%% ===== 
%% ===== 
%% ===== 
%% ===== 
%% ===== 
%% ===== \begin{flushleft}
%% ===== c+l
%% ===== \end{flushleft}
%% ===== 
%% ===== 
%% ===== \begin{flushleft}
%% ===== c+l
%% ===== \end{flushleft}
%% ===== 
%% ===== 
%% ===== \begin{flushleft}
%% ===== c+l
%% ===== \end{flushleft}
%% ===== 
%% ===== 
%% ===== \begin{flushleft}
%% ===== c
%% ===== \end{flushleft}
%% ===== 
%% ===== 
%% ===== \begin{flushleft}
%% ===== c
%% ===== \end{flushleft}
%% ===== 
%% ===== 
%% ===== 
%% ===== 
%% ===== 
%% ===== \begin{flushleft}
%% ===== t+l
%% ===== \end{flushleft}
%% ===== 
%% ===== 
%% ===== \begin{flushleft}
%% ===== t+2
%% ===== \end{flushleft}
%% ===== 
%% ===== 
%% ===== \begin{flushleft}
%% ===== t+3
%% ===== \end{flushleft}
%% ===== 
%% ===== 
%% ===== \begin{flushleft}
%% ===== t+4
%% ===== \end{flushleft}
%% ===== 
%% ===== 
%% ===== \begin{flushleft}
%% ===== t+5
%% ===== \end{flushleft}
%% ===== 
%% ===== 
%% ===== 
%% ===== 
%% ===== 
%% ===== (00...0)30 ||
%% ===== 
%% ===== 
%% ===== (00...0)30 ||
%% ===== 
%% ===== 
%% ===== (00...0)30 ||
%% ===== 
%% ===== 
%% ===== (00...0)30 ||
%% ===== 
%% ===== 
%% ===== (00...0)30 ||
%% ===== 
%% ===== 
%% ===== 
%% ===== 
%% ===== 
%% ===== \begin{flushleft}
%% ===== {``}I''
%% ===== \end{flushleft}
%% ===== 
%% ===== 
%% ===== \begin{flushleft}
%% ===== {``}H''
%% ===== \end{flushleft}
%% ===== 
%% ===== 
%% ===== \begin{flushleft}
%% ===== {``}G''
%% ===== \end{flushleft}
%% ===== 
%% ===== 
%% ===== \begin{flushleft}
%% ===== {``}F''
%% ===== \end{flushleft}
%% ===== 
%% ===== 
%% ===== \begin{flushleft}
%% ===== {``}E''
%% ===== \end{flushleft}
%% ===== 
%% ===== 
%% ===== 
%% ===== 
%% ===== 
%% ===== 1
%% ===== 
%% ===== 
%% ===== 2
%% ===== 
%% ===== 
%% ===== 3
%% ===== 
%% ===== 
%% ===== 4
%% ===== 
%% ===== 
%% ===== 5
%% ===== 
%% ===== 
%% ===== ...
%% ===== 
%% ===== 
%% ===== 
%% ===== 
%% ===== 
%% ===== 2
%% ===== 
%% ===== 
%% ===== 1
%% ===== 
%% ===== 
%% ===== 0
%% ===== 
%% ===== 
%% ===== 3
%% ===== 
%% ===== 
%% ===== 2
%% ===== 
%% ===== 
%% ===== 
%% ===== 
%% ===== 
%% ===== \begin{flushleft}
%% ===== c+l
%% ===== \end{flushleft}
%% ===== 
%% ===== 
%% ===== \begin{flushleft}
%% ===== c+l
%% ===== \end{flushleft}
%% ===== 
%% ===== 
%% ===== \begin{flushleft}
%% ===== c+l
%% ===== \end{flushleft}
%% ===== 
%% ===== 
%% ===== \begin{flushleft}
%% ===== c
%% ===== \end{flushleft}
%% ===== 
%% ===== 
%% ===== \begin{flushleft}
%% ===== c
%% ===== \end{flushleft}
%% ===== 
%% ===== 
%% ===== 
%% ===== 
%% ===== 
%% ===== \begin{flushleft}
%% ===== t+l
%% ===== \end{flushleft}
%% ===== 
%% ===== 
%% ===== \begin{flushleft}
%% ===== t+2
%% ===== \end{flushleft}
%% ===== 
%% ===== 
%% ===== \begin{flushleft}
%% ===== t+3
%% ===== \end{flushleft}
%% ===== 
%% ===== 
%% ===== \begin{flushleft}
%% ===== t+4
%% ===== \end{flushleft}
%% ===== 
%% ===== 
%% ===== \begin{flushleft}
%% ===== t+5
%% ===== \end{flushleft}
%% ===== 
%% ===== 
%% ===== 
%% ===== 
%% ===== 
%% ===== (00...0)27 ||
%% ===== 
%% ===== 
%% ===== (00...0)27 ||
%% ===== 
%% ===== 
%% ===== (00...0)27 ||
%% ===== 
%% ===== 
%% ===== (00...0)27 ||
%% ===== 
%% ===== 
%% ===== (00...0)27 ||
%% ===== 
%% ===== 
%% ===== 
%% ===== 
%% ===== 
%% ===== \begin{flushleft}
%% ===== {``}g''
%% ===== \end{flushleft}
%% ===== 
%% ===== 
%% ===== \begin{flushleft}
%% ===== ''f''
%% ===== \end{flushleft}
%% ===== 
%% ===== 
%% ===== \begin{flushleft}
%% ===== {``}e''
%% ===== \end{flushleft}
%% ===== 
%% ===== 
%% ===== \begin{flushleft}
%% ===== {``}d''
%% ===== \end{flushleft}
%% ===== 
%% ===== 
%% ===== \begin{flushleft}
%% ===== {``}c''
%% ===== \end{flushleft}
%% ===== 
%% ===== 
%% ===== 
%% ===== 
%% ===== 
%% ===== \begin{flushleft}
%% ===== Fault tag 2 (Td = 6)
%% ===== \end{flushleft}
%% ===== 
%% ===== 
%% ===== \begin{flushleft}
%% ===== Terminate computed address formation immediately and generate a fault tag 2 fault.
%% ===== \end{flushleft}
%% ===== 
%% ===== 
%% ===== \begin{flushleft}
%% ===== C(TPR.CA) at the time of the fault contains the computed address of the word containing
%% ===== \end{flushleft}
%% ===== 
%% ===== 
%% ===== \begin{flushleft}
%% ===== the fault tag 2 variation. Thus, the ADDRESS and TALLY fields of that word may contain
%% ===== \end{flushleft}
%% ===== 
%% ===== 
%% ===== \begin{flushleft}
%% ===== information relative to recovery from the fault.
%% ===== \end{flushleft}
%% ===== 
%% ===== 
%% ===== \begin{flushleft}
%% ===== Fault tag 3 (Td = 7)
%% ===== \end{flushleft}
%% ===== 
%% ===== 
%% ===== \begin{flushleft}
%% ===== Terminate computed address formation immediately and generate a fault tag 3 fault.
%% ===== \end{flushleft}
%% ===== 
%% ===== 
%% ===== \begin{flushleft}
%% ===== C(TPR.CA) at the time of the fault contains the computed address of the word containing
%% ===== \end{flushleft}
%% ===== 
%% ===== 
%% ===== \begin{flushleft}
%% ===== the fault tag 3 variation. Thus, the ADDRESS and TALLY fields of that word may contain
%% ===== \end{flushleft}
%% ===== 
%% ===== 
%% ===== \begin{flushleft}
%% ===== information relative to recovery from the fault.
%% ===== \end{flushleft}
%% ===== 
%% ===== 
%% ===== \begin{flushleft}
%% ===== Character indirect (Td = 10)
%% ===== \end{flushleft}
%% ===== 
%% ===== 
%% ===== \begin{flushleft}
%% ===== Bit 30 of the TAG field of the indirect word is interpreted as a character size flag, tb, with
%% ===== \end{flushleft}
%% ===== 
%% ===== 
%% ===== \begin{flushleft}
%% ===== the value 0 indicating 6-bit characters and the value 1 indicating 9-bit bytes. Bits 33-35 of
%% ===== \end{flushleft}
%% ===== 
%% ===== 
%% ===== \begin{flushleft}
%% ===== the TAG field are interpreted as a 3-bit character/byte position value, cf. Bits 31-32 of the
%% ===== \end{flushleft}
%% ===== 
%% ===== 
%% ===== \begin{flushleft}
%% ===== TAG field must be zero.
%% ===== \end{flushleft}
%% ===== 
%% ===== 
%% ===== 
%% ===== 
%% ===== 
%% ===== \begin{flushleft}
%% ===== \newpage
%% ===== If the character position value is greater than 5 for 6-bit characters or greater than 3 for 9bit bytes, an illegal procedure, illegal modifier, fault will occur. The TALLY field is ignored.
%% ===== \end{flushleft}
%% ===== 
%% ===== 
%% ===== \begin{flushleft}
%% ===== The computed address is the value of the ADDRESS field of the indirect word. The effective
%% ===== \end{flushleft}
%% ===== 
%% ===== 
%% ===== \begin{flushleft}
%% ===== character/byte number is the value of the character position count, cf, field of the indirect
%% ===== \end{flushleft}
%% ===== 
%% ===== 
%% ===== \begin{flushleft}
%% ===== word.
%% ===== \end{flushleft}
%% ===== 
%% ===== 
%% ===== \begin{flushleft}
%% ===== A 36-bit operand is formed by high-order zero filling the value of character cf of
%% ===== \end{flushleft}
%% ===== 
%% ===== 
%% ===== \begin{flushleft}
%% ===== C(computed address) with an appropriate number of bits .
%% ===== \end{flushleft}
%% ===== 
%% ===== 
%% ===== \begin{flushleft}
%% ===== Examples:
%% ===== \end{flushleft}
%% ===== 
%% ===== 
%% ===== 
%% ===== 
%% ===== 
%% ===== \begin{flushleft}
%% ===== Location
%% ===== \end{flushleft}
%% ===== 
%% ===== 
%% ===== 
%% ===== 
%% ===== 
%% ===== \begin{flushleft}
%% ===== Instruction
%% ===== \end{flushleft}
%% ===== 
%% ===== 
%% ===== 
%% ===== 
%% ===== 
%% ===== \begin{flushleft}
%% ===== Operand
%% ===== \end{flushleft}
%% ===== 
%% ===== 
%% ===== 
%% ===== 
%% ===== 
%% ===== \begin{flushleft}
%% ===== a
%% ===== \end{flushleft}
%% ===== 
%% ===== 
%% ===== \begin{flushleft}
%% ===== b
%% ===== \end{flushleft}
%% ===== 
%% ===== 
%% ===== \begin{flushleft}
%% ===== c
%% ===== \end{flushleft}
%% ===== 
%% ===== 
%% ===== 
%% ===== 
%% ===== 
%% ===== \begin{flushleft}
%% ===== lda
%% ===== \end{flushleft}
%% ===== 
%% ===== 
%% ===== \begin{flushleft}
%% ===== vfd
%% ===== \end{flushleft}
%% ===== 
%% ===== 
%% ===== \begin{flushleft}
%% ===== bci
%% ===== \end{flushleft}
%% ===== 
%% ===== 
%% ===== 
%% ===== 
%% ===== 
%% ===== \begin{flushleft}
%% ===== b,ci
%% ===== \end{flushleft}
%% ===== 
%% ===== 
%% ===== \begin{flushleft}
%% ===== 18/c+1,12/0,1/0,5/2
%% ===== \end{flushleft}
%% ===== 
%% ===== 
%% ===== \begin{flushleft}
%% ===== {``}ABCDEFGHIJKL''
%% ===== \end{flushleft}
%% ===== 
%% ===== 
%% ===== 
%% ===== 
%% ===== 
%% ===== \begin{flushleft}
%% ===== (00...0)30 || {``}I''
%% ===== \end{flushleft}
%% ===== 
%% ===== 
%% ===== 
%% ===== 
%% ===== 
%% ===== \begin{flushleft}
%% ===== a
%% ===== \end{flushleft}
%% ===== 
%% ===== 
%% ===== \begin{flushleft}
%% ===== d
%% ===== \end{flushleft}
%% ===== 
%% ===== 
%% ===== 
%% ===== 
%% ===== 
%% ===== \begin{flushleft}
%% ===== lda
%% ===== \end{flushleft}
%% ===== 
%% ===== 
%% ===== \begin{flushleft}
%% ===== vfd
%% ===== \end{flushleft}
%% ===== 
%% ===== 
%% ===== 
%% ===== 
%% ===== 
%% ===== \begin{flushleft}
%% ===== d,ci
%% ===== \end{flushleft}
%% ===== 
%% ===== 
%% ===== \begin{flushleft}
%% ===== 18/c,12/0,1/0,5/1
%% ===== \end{flushleft}
%% ===== 
%% ===== 
%% ===== 
%% ===== 
%% ===== 
%% ===== \begin{flushleft}
%% ===== a
%% ===== \end{flushleft}
%% ===== 
%% ===== 
%% ===== \begin{flushleft}
%% ===== e
%% ===== \end{flushleft}
%% ===== 
%% ===== 
%% ===== \begin{flushleft}
%% ===== f
%% ===== \end{flushleft}
%% ===== 
%% ===== 
%% ===== 
%% ===== 
%% ===== 
%% ===== \begin{flushleft}
%% ===== lda
%% ===== \end{flushleft}
%% ===== 
%% ===== 
%% ===== \begin{flushleft}
%% ===== vfd
%% ===== \end{flushleft}
%% ===== 
%% ===== 
%% ===== \begin{flushleft}
%% ===== aci
%% ===== \end{flushleft}
%% ===== 
%% ===== 
%% ===== 
%% ===== 
%% ===== 
%% ===== \begin{flushleft}
%% ===== e,ci
%% ===== \end{flushleft}
%% ===== 
%% ===== 
%% ===== \begin{flushleft}
%% ===== 18/f,12/0,1/1,5/3
%% ===== \end{flushleft}
%% ===== 
%% ===== 
%% ===== \begin{flushleft}
%% ===== {``}abcdefgh''
%% ===== \end{flushleft}
%% ===== 
%% ===== 
%% ===== 
%% ===== 
%% ===== 
%% ===== \begin{flushleft}
%% ===== a
%% ===== \end{flushleft}
%% ===== 
%% ===== 
%% ===== \begin{flushleft}
%% ===== g
%% ===== \end{flushleft}
%% ===== 
%% ===== 
%% ===== 
%% ===== 
%% ===== 
%% ===== \begin{flushleft}
%% ===== lda
%% ===== \end{flushleft}
%% ===== 
%% ===== 
%% ===== \begin{flushleft}
%% ===== vfd
%% ===== \end{flushleft}
%% ===== 
%% ===== 
%% ===== 
%% ===== 
%% ===== 
%% ===== \begin{flushleft}
%% ===== g,ci
%% ===== \end{flushleft}
%% ===== 
%% ===== 
%% ===== \begin{flushleft}
%% ===== 18/f+1,12/0,1/1,5/0
%% ===== \end{flushleft}
%% ===== 
%% ===== 
%% ===== 
%% ===== 
%% ===== 
%% ===== \begin{flushleft}
%% ===== (00...0)30 || {``}B'''
%% ===== \end{flushleft}
%% ===== 
%% ===== 
%% ===== 
%% ===== 
%% ===== 
%% ===== \begin{flushleft}
%% ===== (00...0)27 || {``}d''
%% ===== \end{flushleft}
%% ===== 
%% ===== 
%% ===== 
%% ===== 
%% ===== 
%% ===== \begin{flushleft}
%% ===== (00...0)27 || {``}e''
%% ===== \end{flushleft}
%% ===== 
%% ===== 
%% ===== 
%% ===== 
%% ===== 
%% ===== \begin{flushleft}
%% ===== Indirect (Td = 11)
%% ===== \end{flushleft}
%% ===== 
%% ===== 
%% ===== \begin{flushleft}
%% ===== The computed address is the value of the ADDRESS field of the indirect word. The TALLY
%% ===== \end{flushleft}
%% ===== 
%% ===== 
%% ===== \begin{flushleft}
%% ===== and TAG fields of the indirect word are ignored.
%% ===== \end{flushleft}
%% ===== 
%% ===== 
%% ===== \begin{flushleft}
%% ===== Sequence character (Td = 12)
%% ===== \end{flushleft}
%% ===== 
%% ===== 
%% ===== \begin{flushleft}
%% ===== Bit 30 of the TAG field of the indirect word is interpreted as a character size flag, tb, with
%% ===== \end{flushleft}
%% ===== 
%% ===== 
%% ===== \begin{flushleft}
%% ===== the value 0 indicating 6-bit characters and the value 1 indicating 9-bit bytes. Bits 33-35 of
%% ===== \end{flushleft}
%% ===== 
%% ===== 
%% ===== \begin{flushleft}
%% ===== the TAG field are interpreted as a 3-bit character position counter, cf. Bits 31-32 of the TAG
%% ===== \end{flushleft}
%% ===== 
%% ===== 
%% ===== \begin{flushleft}
%% ===== field must be zero.
%% ===== \end{flushleft}
%% ===== 
%% ===== 
%% ===== \begin{flushleft}
%% ===== For each reference to the indirect word, the character counter, cf, is increased by 1 and the
%% ===== \end{flushleft}
%% ===== 
%% ===== 
%% ===== \begin{flushleft}
%% ===== TALLY field is reduced by 1 after the computed address is formed. Character count
%% ===== \end{flushleft}
%% ===== 
%% ===== 
%% ===== \begin{flushleft}
%% ===== arithmetic is modulo 6 for 6-bit characters and modulo 4 for 9-bit bytes. If the character
%% ===== \end{flushleft}
%% ===== 
%% ===== 
%% ===== \begin{flushleft}
%% ===== count, cf, overflows to 6 for 6-bit characters or to 4 for 9-bit bytes, it is reset to 0 and
%% ===== \end{flushleft}
%% ===== 
%% ===== 
%% ===== \begin{flushleft}
%% ===== ADDRESS is increased by 1. ADDRESS arithmetic is modulo 218. TALLY arithmetic is
%% ===== \end{flushleft}
%% ===== 
%% ===== 
%% ===== \begin{flushleft}
%% ===== modulo 4096. If the TALLY field is reduced to 0, the tally runout indicator is set ON,
%% ===== \end{flushleft}
%% ===== 
%% ===== 
%% ===== \begin{flushleft}
%% ===== otherwise it is set OFF. The computed address is the unmodified value of the ADDRESS
%% ===== \end{flushleft}
%% ===== 
%% ===== 
%% ===== \begin{flushleft}
%% ===== field. The effective character/byte number is the unmodified value of the character position
%% ===== \end{flushleft}
%% ===== 
%% ===== 
%% ===== \begin{flushleft}
%% ===== counter, cf, field of the indirect word.
%% ===== \end{flushleft}
%% ===== 
%% ===== 
%% ===== \begin{flushleft}
%% ===== A 36-bit operand is formed by high-order zero filling the value of character of of
%% ===== \end{flushleft}
%% ===== 
%% ===== 
%% ===== \begin{flushleft}
%% ===== C(computed address) with an appropriate number of bits .
%% ===== \end{flushleft}
%% ===== 
%% ===== 
%% ===== 
%% ===== 
%% ===== 
%% ===== \begin{flushleft}
%% ===== \newpage
%% ===== Examples:
%% ===== \end{flushleft}
%% ===== 
%% ===== 
%% ===== 
%% ===== 
%% ===== 
%% ===== \begin{flushleft}
%% ===== Reference
%% ===== \end{flushleft}
%% ===== 
%% ===== 
%% ===== \begin{flushleft}
%% ===== count
%% ===== \end{flushleft}
%% ===== 
%% ===== 
%% ===== \begin{flushleft}
%% ===== cf
%% ===== \end{flushleft}
%% ===== 
%% ===== 
%% ===== 
%% ===== 
%% ===== 
%% ===== \begin{flushleft}
%% ===== Location
%% ===== \end{flushleft}
%% ===== 
%% ===== 
%% ===== 
%% ===== 
%% ===== 
%% ===== \begin{flushleft}
%% ===== Instruction
%% ===== \end{flushleft}
%% ===== 
%% ===== 
%% ===== 
%% ===== 
%% ===== 
%% ===== \begin{flushleft}
%% ===== a
%% ===== \end{flushleft}
%% ===== 
%% ===== 
%% ===== \begin{flushleft}
%% ===== b
%% ===== \end{flushleft}
%% ===== 
%% ===== 
%% ===== \begin{flushleft}
%% ===== c
%% ===== \end{flushleft}
%% ===== 
%% ===== 
%% ===== 
%% ===== 
%% ===== 
%% ===== \begin{flushleft}
%% ===== lda
%% ===== \end{flushleft}
%% ===== 
%% ===== 
%% ===== \begin{flushleft}
%% ===== vfd
%% ===== \end{flushleft}
%% ===== 
%% ===== 
%% ===== \begin{flushleft}
%% ===== bci
%% ===== \end{flushleft}
%% ===== 
%% ===== 
%% ===== 
%% ===== 
%% ===== 
%% ===== \begin{flushleft}
%% ===== b,sc
%% ===== \end{flushleft}
%% ===== 
%% ===== 
%% ===== \begin{flushleft}
%% ===== 18/c,12/t,1/0,5/4
%% ===== \end{flushleft}
%% ===== 
%% ===== 
%% ===== \begin{flushleft}
%% ===== {``}ABCDEFGHIJKL''
%% ===== \end{flushleft}
%% ===== 
%% ===== 
%% ===== 
%% ===== 
%% ===== 
%% ===== 1
%% ===== 
%% ===== 
%% ===== 2
%% ===== 
%% ===== 
%% ===== 3
%% ===== 
%% ===== 
%% ===== 4
%% ===== 
%% ===== 
%% ===== 5
%% ===== 
%% ===== 
%% ===== ...
%% ===== 
%% ===== 
%% ===== 
%% ===== 
%% ===== 
%% ===== \begin{flushleft}
%% ===== a
%% ===== \end{flushleft}
%% ===== 
%% ===== 
%% ===== \begin{flushleft}
%% ===== b
%% ===== \end{flushleft}
%% ===== 
%% ===== 
%% ===== \begin{flushleft}
%% ===== c
%% ===== \end{flushleft}
%% ===== 
%% ===== 
%% ===== 
%% ===== 
%% ===== 
%% ===== \begin{flushleft}
%% ===== lda
%% ===== \end{flushleft}
%% ===== 
%% ===== 
%% ===== \begin{flushleft}
%% ===== vfd
%% ===== \end{flushleft}
%% ===== 
%% ===== 
%% ===== \begin{flushleft}
%% ===== aci
%% ===== \end{flushleft}
%% ===== 
%% ===== 
%% ===== 
%% ===== 
%% ===== 
%% ===== \begin{flushleft}
%% ===== b,sc
%% ===== \end{flushleft}
%% ===== 
%% ===== 
%% ===== \begin{flushleft}
%% ===== 18/c,12/t,1/1,5/2
%% ===== \end{flushleft}
%% ===== 
%% ===== 
%% ===== \begin{flushleft}
%% ===== {``}abcdefgh''
%% ===== \end{flushleft}
%% ===== 
%% ===== 
%% ===== 
%% ===== 
%% ===== 
%% ===== 1
%% ===== 
%% ===== 
%% ===== 2
%% ===== 
%% ===== 
%% ===== 3
%% ===== 
%% ===== 
%% ===== 4
%% ===== 
%% ===== 
%% ===== 5
%% ===== 
%% ===== 
%% ===== ...
%% ===== 
%% ===== 
%% ===== 
%% ===== 
%% ===== 
%% ===== \begin{flushleft}
%% ===== Computed
%% ===== \end{flushleft}
%% ===== 
%% ===== 
%% ===== \begin{flushleft}
%% ===== address
%% ===== \end{flushleft}
%% ===== 
%% ===== 
%% ===== 
%% ===== 
%% ===== 
%% ===== \begin{flushleft}
%% ===== Tally
%% ===== \end{flushleft}
%% ===== 
%% ===== 
%% ===== \begin{flushleft}
%% ===== value
%% ===== \end{flushleft}
%% ===== 
%% ===== 
%% ===== 
%% ===== 
%% ===== 
%% ===== 4
%% ===== 
%% ===== 
%% ===== 5
%% ===== 
%% ===== 
%% ===== 0
%% ===== 
%% ===== 
%% ===== 1
%% ===== 
%% ===== 
%% ===== 2
%% ===== 
%% ===== 
%% ===== 
%% ===== 
%% ===== 
%% ===== \begin{flushleft}
%% ===== c
%% ===== \end{flushleft}
%% ===== 
%% ===== 
%% ===== \begin{flushleft}
%% ===== c
%% ===== \end{flushleft}
%% ===== 
%% ===== 
%% ===== \begin{flushleft}
%% ===== c+l
%% ===== \end{flushleft}
%% ===== 
%% ===== 
%% ===== \begin{flushleft}
%% ===== c+l
%% ===== \end{flushleft}
%% ===== 
%% ===== 
%% ===== \begin{flushleft}
%% ===== c+l
%% ===== \end{flushleft}
%% ===== 
%% ===== 
%% ===== 
%% ===== 
%% ===== 
%% ===== \begin{flushleft}
%% ===== t-1
%% ===== \end{flushleft}
%% ===== 
%% ===== 
%% ===== \begin{flushleft}
%% ===== t-2
%% ===== \end{flushleft}
%% ===== 
%% ===== 
%% ===== \begin{flushleft}
%% ===== t-3
%% ===== \end{flushleft}
%% ===== 
%% ===== 
%% ===== \begin{flushleft}
%% ===== t-4
%% ===== \end{flushleft}
%% ===== 
%% ===== 
%% ===== \begin{flushleft}
%% ===== t-5
%% ===== \end{flushleft}
%% ===== 
%% ===== 
%% ===== 
%% ===== 
%% ===== 
%% ===== (00...0)30 ||
%% ===== 
%% ===== 
%% ===== (00...0)30 ||
%% ===== 
%% ===== 
%% ===== (00...0)30 ||
%% ===== 
%% ===== 
%% ===== (00...0)30 ||
%% ===== 
%% ===== 
%% ===== (00...0)30 ||
%% ===== 
%% ===== 
%% ===== 
%% ===== 
%% ===== 
%% ===== \begin{flushleft}
%% ===== {``}E''
%% ===== \end{flushleft}
%% ===== 
%% ===== 
%% ===== \begin{flushleft}
%% ===== {``}F''
%% ===== \end{flushleft}
%% ===== 
%% ===== 
%% ===== \begin{flushleft}
%% ===== {``}G''
%% ===== \end{flushleft}
%% ===== 
%% ===== 
%% ===== \begin{flushleft}
%% ===== {``}H''
%% ===== \end{flushleft}
%% ===== 
%% ===== 
%% ===== \begin{flushleft}
%% ===== {``}I''
%% ===== \end{flushleft}
%% ===== 
%% ===== 
%% ===== 
%% ===== 
%% ===== 
%% ===== 2
%% ===== 
%% ===== 
%% ===== 3
%% ===== 
%% ===== 
%% ===== 0
%% ===== 
%% ===== 
%% ===== 1
%% ===== 
%% ===== 
%% ===== 2
%% ===== 
%% ===== 
%% ===== 
%% ===== 
%% ===== 
%% ===== \begin{flushleft}
%% ===== c
%% ===== \end{flushleft}
%% ===== 
%% ===== 
%% ===== \begin{flushleft}
%% ===== c
%% ===== \end{flushleft}
%% ===== 
%% ===== 
%% ===== \begin{flushleft}
%% ===== c+l
%% ===== \end{flushleft}
%% ===== 
%% ===== 
%% ===== \begin{flushleft}
%% ===== c+l
%% ===== \end{flushleft}
%% ===== 
%% ===== 
%% ===== \begin{flushleft}
%% ===== c+l
%% ===== \end{flushleft}
%% ===== 
%% ===== 
%% ===== 
%% ===== 
%% ===== 
%% ===== \begin{flushleft}
%% ===== t-1
%% ===== \end{flushleft}
%% ===== 
%% ===== 
%% ===== \begin{flushleft}
%% ===== t-2
%% ===== \end{flushleft}
%% ===== 
%% ===== 
%% ===== \begin{flushleft}
%% ===== t-3
%% ===== \end{flushleft}
%% ===== 
%% ===== 
%% ===== \begin{flushleft}
%% ===== t-4
%% ===== \end{flushleft}
%% ===== 
%% ===== 
%% ===== \begin{flushleft}
%% ===== t-5
%% ===== \end{flushleft}
%% ===== 
%% ===== 
%% ===== 
%% ===== 
%% ===== 
%% ===== (00...0)27 ||
%% ===== 
%% ===== 
%% ===== (00...0)27 ||
%% ===== 
%% ===== 
%% ===== (00...0)27 ||
%% ===== 
%% ===== 
%% ===== (00...0)27 ||
%% ===== 
%% ===== 
%% ===== (00...0)27 ||
%% ===== 
%% ===== 
%% ===== 
%% ===== 
%% ===== 
%% ===== \begin{flushleft}
%% ===== {``}c''
%% ===== \end{flushleft}
%% ===== 
%% ===== 
%% ===== \begin{flushleft}
%% ===== {``}d''
%% ===== \end{flushleft}
%% ===== 
%% ===== 
%% ===== \begin{flushleft}
%% ===== {``}e''
%% ===== \end{flushleft}
%% ===== 
%% ===== 
%% ===== \begin{flushleft}
%% ===== {``}f''
%% ===== \end{flushleft}
%% ===== 
%% ===== 
%% ===== \begin{flushleft}
%% ===== {``}g''
%% ===== \end{flushleft}
%% ===== 
%% ===== 
%% ===== 
%% ===== 
%% ===== 
%% ===== \begin{flushleft}
%% ===== Operand
%% ===== \end{flushleft}
%% ===== 
%% ===== 
%% ===== 
%% ===== 
%% ===== 
%% ===== \begin{flushleft}
%% ===== Add delta (Td = 13)
%% ===== \end{flushleft}
%% ===== 
%% ===== 
%% ===== \begin{flushleft}
%% ===== The TAG field of the indirect word is interpreted as a 6-bit, unsigned, positive address
%% ===== \end{flushleft}
%% ===== 
%% ===== 
%% ===== \begin{flushleft}
%% ===== increment value, delta. For each reference to the indirect word, the ADDRESS field is
%% ===== \end{flushleft}
%% ===== 
%% ===== 
%% ===== \begin{flushleft}
%% ===== increased by delta and the TALLY field is reduced by 1 after the computed address is
%% ===== \end{flushleft}
%% ===== 
%% ===== 
%% ===== \begin{flushleft}
%% ===== formed. ADDRESS arithmetic is modulo 218. TALLY arithmetic is modulo 4096. If the
%% ===== \end{flushleft}
%% ===== 
%% ===== 
%% ===== \begin{flushleft}
%% ===== TALLY field is reduced to 0, the tally runout indicator is set ON, otherwise it is set OFF.
%% ===== \end{flushleft}
%% ===== 
%% ===== 
%% ===== \begin{flushleft}
%% ===== The computed address is the value of the unmodified ADDRESS field of the indirect word.
%% ===== \end{flushleft}
%% ===== 
%% ===== 
%% ===== \begin{flushleft}
%% ===== Example:
%% ===== \end{flushleft}
%% ===== 
%% ===== 
%% ===== 
%% ===== 
%% ===== 
%% ===== \begin{flushleft}
%% ===== Location
%% ===== \end{flushleft}
%% ===== 
%% ===== 
%% ===== 
%% ===== 
%% ===== 
%% ===== \begin{flushleft}
%% ===== Instruction
%% ===== \end{flushleft}
%% ===== 
%% ===== 
%% ===== 
%% ===== 
%% ===== 
%% ===== \begin{flushleft}
%% ===== a
%% ===== \end{flushleft}
%% ===== 
%% ===== 
%% ===== \begin{flushleft}
%% ===== b
%% ===== \end{flushleft}
%% ===== 
%% ===== 
%% ===== 
%% ===== 
%% ===== 
%% ===== \begin{flushleft}
%% ===== lda
%% ===== \end{flushleft}
%% ===== 
%% ===== 
%% ===== \begin{flushleft}
%% ===== vfd
%% ===== \end{flushleft}
%% ===== 
%% ===== 
%% ===== 
%% ===== 
%% ===== 
%% ===== \begin{flushleft}
%% ===== Reference count
%% ===== \end{flushleft}
%% ===== 
%% ===== 
%% ===== 
%% ===== 
%% ===== 
%% ===== \begin{flushleft}
%% ===== Computed
%% ===== \end{flushleft}
%% ===== 
%% ===== 
%% ===== \begin{flushleft}
%% ===== address
%% ===== \end{flushleft}
%% ===== 
%% ===== 
%% ===== 
%% ===== 
%% ===== 
%% ===== \begin{flushleft}
%% ===== Tally value
%% ===== \end{flushleft}
%% ===== 
%% ===== 
%% ===== 
%% ===== 
%% ===== 
%% ===== 1
%% ===== 
%% ===== 
%% ===== 2
%% ===== 
%% ===== 
%% ===== 3
%% ===== 
%% ===== 
%% ===== ...
%% ===== 
%% ===== 
%% ===== \begin{flushleft}
%% ===== n
%% ===== \end{flushleft}
%% ===== 
%% ===== 
%% ===== 
%% ===== 
%% ===== 
%% ===== \begin{flushleft}
%% ===== c
%% ===== \end{flushleft}
%% ===== 
%% ===== 
%% ===== \begin{flushleft}
%% ===== c+d
%% ===== \end{flushleft}
%% ===== 
%% ===== 
%% ===== \begin{flushleft}
%% ===== c+2d
%% ===== \end{flushleft}
%% ===== 
%% ===== 
%% ===== 
%% ===== 
%% ===== 
%% ===== \begin{flushleft}
%% ===== t-1
%% ===== \end{flushleft}
%% ===== 
%% ===== 
%% ===== \begin{flushleft}
%% ===== t-2
%% ===== \end{flushleft}
%% ===== 
%% ===== 
%% ===== \begin{flushleft}
%% ===== t-3
%% ===== \end{flushleft}
%% ===== 
%% ===== 
%% ===== 
%% ===== 
%% ===== 
%% ===== \begin{flushleft}
%% ===== c+(n-l)d
%% ===== \end{flushleft}
%% ===== 
%% ===== 
%% ===== 
%% ===== 
%% ===== 
%% ===== \begin{flushleft}
%% ===== t-n
%% ===== \end{flushleft}
%% ===== 
%% ===== 
%% ===== 
%% ===== 
%% ===== 
%% ===== \begin{flushleft}
%% ===== b,ad
%% ===== \end{flushleft}
%% ===== 
%% ===== 
%% ===== \begin{flushleft}
%% ===== 18/c,1/t,6/d
%% ===== \end{flushleft}
%% ===== 
%% ===== 
%% ===== 
%% ===== 
%% ===== 
%% ===== \begin{flushleft}
%% ===== Decrement address, increment tally (Td = 14)
%% ===== \end{flushleft}
%% ===== 
%% ===== 
%% ===== \begin{flushleft}
%% ===== For each reference to the indirect word, the ADDRESS field is reduced by 1 and the TALLY
%% ===== \end{flushleft}
%% ===== 
%% ===== 
%% ===== \begin{flushleft}
%% ===== field is increased by 1 before the computed address is formed. ADDRESS arithmetic is
%% ===== \end{flushleft}
%% ===== 
%% ===== 
%% ===== \begin{flushleft}
%% ===== modulo 218. TALLY arithmetic is modulo 4096. If the TALLY field overflows to 0, the tally
%% ===== \end{flushleft}
%% ===== 
%% ===== 
%% ===== \begin{flushleft}
%% ===== runout indicator is set ON, otherwise it is set OFF. The TAG field of the indirect word is
%% ===== \end{flushleft}
%% ===== 
%% ===== 
%% ===== \begin{flushleft}
%% ===== ignored. The computed address is the value of the decremented ADDRESS field.
%% ===== \end{flushleft}
%% ===== 
%% ===== 
%% ===== \begin{flushleft}
%% ===== Example:
%% ===== \end{flushleft}
%% ===== 
%% ===== 
%% ===== 
%% ===== 
%% ===== 
%% ===== \begin{flushleft}
%% ===== Location
%% ===== \end{flushleft}
%% ===== 
%% ===== 
%% ===== 
%% ===== 
%% ===== 
%% ===== \begin{flushleft}
%% ===== Instruction
%% ===== \end{flushleft}
%% ===== 
%% ===== 
%% ===== 
%% ===== 
%% ===== 
%% ===== \begin{flushleft}
%% ===== a
%% ===== \end{flushleft}
%% ===== 
%% ===== 
%% ===== \begin{flushleft}
%% ===== b
%% ===== \end{flushleft}
%% ===== 
%% ===== 
%% ===== 
%% ===== 
%% ===== 
%% ===== \begin{flushleft}
%% ===== lda
%% ===== \end{flushleft}
%% ===== 
%% ===== 
%% ===== \begin{flushleft}
%% ===== vfd
%% ===== \end{flushleft}
%% ===== 
%% ===== 
%% ===== 
%% ===== 
%% ===== 
%% ===== \begin{flushleft}
%% ===== b,di
%% ===== \end{flushleft}
%% ===== 
%% ===== 
%% ===== \begin{flushleft}
%% ===== 18/c,12/t
%% ===== \end{flushleft}
%% ===== 
%% ===== 
%% ===== 
%% ===== 
%% ===== 
%% ===== \begin{flushleft}
%% ===== Reference
%% ===== \end{flushleft}
%% ===== 
%% ===== 
%% ===== \begin{flushleft}
%% ===== count
%% ===== \end{flushleft}
%% ===== 
%% ===== 
%% ===== 
%% ===== 
%% ===== 
%% ===== \begin{flushleft}
%% ===== Computed
%% ===== \end{flushleft}
%% ===== 
%% ===== 
%% ===== \begin{flushleft}
%% ===== address
%% ===== \end{flushleft}
%% ===== 
%% ===== 
%% ===== 
%% ===== 
%% ===== 
%% ===== \begin{flushleft}
%% ===== Tally value
%% ===== \end{flushleft}
%% ===== 
%% ===== 
%% ===== 
%% ===== 
%% ===== 
%% ===== 1
%% ===== 
%% ===== 
%% ===== 2
%% ===== 
%% ===== 
%% ===== 3
%% ===== 
%% ===== 
%% ===== ...
%% ===== 
%% ===== 
%% ===== \begin{flushleft}
%% ===== n
%% ===== \end{flushleft}
%% ===== 
%% ===== 
%% ===== 
%% ===== 
%% ===== 
%% ===== \begin{flushleft}
%% ===== c-1
%% ===== \end{flushleft}
%% ===== 
%% ===== 
%% ===== \begin{flushleft}
%% ===== c-2
%% ===== \end{flushleft}
%% ===== 
%% ===== 
%% ===== \begin{flushleft}
%% ===== c-3
%% ===== \end{flushleft}
%% ===== 
%% ===== 
%% ===== 
%% ===== 
%% ===== 
%% ===== \begin{flushleft}
%% ===== t+l
%% ===== \end{flushleft}
%% ===== 
%% ===== 
%% ===== \begin{flushleft}
%% ===== t+2
%% ===== \end{flushleft}
%% ===== 
%% ===== 
%% ===== \begin{flushleft}
%% ===== t+3
%% ===== \end{flushleft}
%% ===== 
%% ===== 
%% ===== 
%% ===== 
%% ===== 
%% ===== \begin{flushleft}
%% ===== c-n
%% ===== \end{flushleft}
%% ===== 
%% ===== 
%% ===== 
%% ===== 
%% ===== 
%% ===== \begin{flushleft}
%% ===== t+n
%% ===== \end{flushleft}
%% ===== 
%% ===== 
%% ===== 
%% ===== 
%% ===== 
%% ===== \begin{flushleft}
%% ===== \newpage
%% ===== Decrement address, increment tally, and continue (Td = 15)
%% ===== \end{flushleft}
%% ===== 
%% ===== 
%% ===== \begin{flushleft}
%% ===== The action for this variation is identical to that for the decrement address, increment tally
%% ===== \end{flushleft}
%% ===== 
%% ===== 
%% ===== \begin{flushleft}
%% ===== variation except that the TAG field of the indirect word is interpreted and continuation of
%% ===== \end{flushleft}
%% ===== 
%% ===== 
%% ===== \begin{flushleft}
%% ===== the indirect chain is possible. If the TAG of the indirect word invokes a register, that is,
%% ===== \end{flushleft}
%% ===== 
%% ===== 
%% ===== \begin{flushleft}
%% ===== specifies r, ri, or ir modification, the effective Td value for the register is forced to {``}null''
%% ===== \end{flushleft}
%% ===== 
%% ===== 
%% ===== \begin{flushleft}
%% ===== before the next computed address is formed .
%% ===== \end{flushleft}
%% ===== 
%% ===== 
%% ===== \begin{flushleft}
%% ===== Increment address, decrement tally (Td = 16)
%% ===== \end{flushleft}
%% ===== 
%% ===== 
%% ===== \begin{flushleft}
%% ===== For each reference to the indirect word, the ADDRESS field is increased by 1 and the
%% ===== \end{flushleft}
%% ===== 
%% ===== 
%% ===== \begin{flushleft}
%% ===== TALLY field is reduced by 1 after the computed address is formed. ADDRESS arithmetic is
%% ===== \end{flushleft}
%% ===== 
%% ===== 
%% ===== \begin{flushleft}
%% ===== modulo 218. TALLY arithmetic is modulo 4096. If the TALLY field is reduced to 0, the tally
%% ===== \end{flushleft}
%% ===== 
%% ===== 
%% ===== \begin{flushleft}
%% ===== runout indicator is set ON, otherwise it is set OFF. The TAG field of the indirect word is
%% ===== \end{flushleft}
%% ===== 
%% ===== 
%% ===== \begin{flushleft}
%% ===== ignored. The computed address is the value of the unmodified ADDRESS field.
%% ===== \end{flushleft}
%% ===== 
%% ===== 
%% ===== \begin{flushleft}
%% ===== Example:
%% ===== \end{flushleft}
%% ===== 
%% ===== 
%% ===== 
%% ===== 
%% ===== 
%% ===== \begin{flushleft}
%% ===== Location
%% ===== \end{flushleft}
%% ===== 
%% ===== 
%% ===== 
%% ===== 
%% ===== 
%% ===== \begin{flushleft}
%% ===== Instruction
%% ===== \end{flushleft}
%% ===== 
%% ===== 
%% ===== 
%% ===== 
%% ===== 
%% ===== \begin{flushleft}
%% ===== a
%% ===== \end{flushleft}
%% ===== 
%% ===== 
%% ===== \begin{flushleft}
%% ===== b
%% ===== \end{flushleft}
%% ===== 
%% ===== 
%% ===== 
%% ===== 
%% ===== 
%% ===== \begin{flushleft}
%% ===== lda
%% ===== \end{flushleft}
%% ===== 
%% ===== 
%% ===== \begin{flushleft}
%% ===== vfd
%% ===== \end{flushleft}
%% ===== 
%% ===== 
%% ===== 
%% ===== 
%% ===== 
%% ===== \begin{flushleft}
%% ===== b,id
%% ===== \end{flushleft}
%% ===== 
%% ===== 
%% ===== \begin{flushleft}
%% ===== 18/c,1/t
%% ===== \end{flushleft}
%% ===== 
%% ===== 
%% ===== 
%% ===== 
%% ===== 
%% ===== \begin{flushleft}
%% ===== Reference
%% ===== \end{flushleft}
%% ===== 
%% ===== 
%% ===== \begin{flushleft}
%% ===== count
%% ===== \end{flushleft}
%% ===== 
%% ===== 
%% ===== 
%% ===== 
%% ===== 
%% ===== \begin{flushleft}
%% ===== Computed
%% ===== \end{flushleft}
%% ===== 
%% ===== 
%% ===== \begin{flushleft}
%% ===== address
%% ===== \end{flushleft}
%% ===== 
%% ===== 
%% ===== 
%% ===== 
%% ===== 
%% ===== \begin{flushleft}
%% ===== Tally value
%% ===== \end{flushleft}
%% ===== 
%% ===== 
%% ===== 
%% ===== 
%% ===== 
%% ===== 1
%% ===== 
%% ===== 
%% ===== 2
%% ===== 
%% ===== 
%% ===== 3
%% ===== 
%% ===== 
%% ===== ...
%% ===== 
%% ===== 
%% ===== \begin{flushleft}
%% ===== n
%% ===== \end{flushleft}
%% ===== 
%% ===== 
%% ===== 
%% ===== 
%% ===== 
%% ===== \begin{flushleft}
%% ===== c
%% ===== \end{flushleft}
%% ===== 
%% ===== 
%% ===== \begin{flushleft}
%% ===== c+1
%% ===== \end{flushleft}
%% ===== 
%% ===== 
%% ===== \begin{flushleft}
%% ===== c+2
%% ===== \end{flushleft}
%% ===== 
%% ===== 
%% ===== 
%% ===== 
%% ===== 
%% ===== \begin{flushleft}
%% ===== t-1
%% ===== \end{flushleft}
%% ===== 
%% ===== 
%% ===== \begin{flushleft}
%% ===== t-2
%% ===== \end{flushleft}
%% ===== 
%% ===== 
%% ===== \begin{flushleft}
%% ===== t-3
%% ===== \end{flushleft}
%% ===== 
%% ===== 
%% ===== 
%% ===== 
%% ===== 
%% ===== \begin{flushleft}
%% ===== c+(n-1)
%% ===== \end{flushleft}
%% ===== 
%% ===== 
%% ===== 
%% ===== 
%% ===== 
%% ===== \begin{flushleft}
%% ===== t-n
%% ===== \end{flushleft}
%% ===== 
%% ===== 
%% ===== 
%% ===== 
%% ===== 
%% ===== \begin{flushleft}
%% ===== Increment address, decrement tally, and continue (Td = 17)
%% ===== \end{flushleft}
%% ===== 
%% ===== 
%% ===== \begin{flushleft}
%% ===== The action for this variation is identical to that for the increment address, decrement tally
%% ===== \end{flushleft}
%% ===== 
%% ===== 
%% ===== \begin{flushleft}
%% ===== variation except that the TAG field of the indirect word is interpreted and continuation of
%% ===== \end{flushleft}
%% ===== 
%% ===== 
%% ===== \begin{flushleft}
%% ===== the indirect chain is possible. If the TAG of the indirect word invokes a register, that is,
%% ===== \end{flushleft}
%% ===== 
%% ===== 
%% ===== \begin{flushleft}
%% ===== specifies r, ri, or ir modification, the effective Td value for the register is forced to {``}null''
%% ===== \end{flushleft}
%% ===== 
%% ===== 
%% ===== \begin{flushleft}
%% ===== before the next computed address is formed.
%% ===== \end{flushleft}
%% ===== 
%% ===== 
%% ===== 
%% ===== 
%% ===== 
%% ===== \begin{flushleft}
%% ===== \newpage
%% ===== IT MOD
%% ===== \end{flushleft}
%% ===== 
%% ===== 
%% ===== 
%% ===== 
%% ===== 
%% ===== \begin{flushleft}
%% ===== Interpret
%% ===== \end{flushleft}
%% ===== 
%% ===== 
%% ===== \begin{flushleft}
%% ===== Td
%% ===== \end{flushleft}
%% ===== 
%% ===== 
%% ===== 
%% ===== 
%% ===== 
%% ===== \begin{flushleft}
%% ===== Td = 0, 6, 7
%% ===== \end{flushleft}
%% ===== 
%% ===== 
%% ===== \begin{flushleft}
%% ===== (f1, f2, f3)
%% ===== \end{flushleft}
%% ===== 
%% ===== 
%% ===== 
%% ===== 
%% ===== 
%% ===== \begin{flushleft}
%% ===== Td = 1, 2, 3
%% ===== \end{flushleft}
%% ===== 
%% ===== 
%% ===== \begin{flushleft}
%% ===== (undef)
%% ===== \end{flushleft}
%% ===== 
%% ===== 
%% ===== 
%% ===== 
%% ===== 
%% ===== \begin{flushleft}
%% ===== Td = 10, 12, 5
%% ===== \end{flushleft}
%% ===== 
%% ===== 
%% ===== \begin{flushleft}
%% ===== (ci, sc, scr)
%% ===== \end{flushleft}
%% ===== 
%% ===== 
%% ===== 
%% ===== 
%% ===== 
%% ===== \begin{flushleft}
%% ===== Indirect word fetch
%% ===== \end{flushleft}
%% ===== 
%% ===== 
%% ===== \begin{flushleft}
%% ===== APPEND CYCLE
%% ===== \end{flushleft}
%% ===== 
%% ===== 
%% ===== \begin{flushleft}
%% ===== (Figure 5-4)
%% ===== \end{flushleft}
%% ===== 
%% ===== 
%% ===== 
%% ===== 
%% ===== 
%% ===== \begin{flushleft}
%% ===== ABORT
%% ===== \end{flushleft}
%% ===== 
%% ===== 
%% ===== \begin{flushleft}
%% ===== fault tag 1,
%% ===== \end{flushleft}
%% ===== 
%% ===== 
%% ===== \begin{flushleft}
%% ===== 2, or 3 fault
%% ===== \end{flushleft}
%% ===== 
%% ===== 
%% ===== 
%% ===== 
%% ===== 
%% ===== \begin{flushleft}
%% ===== No
%% ===== \end{flushleft}
%% ===== 
%% ===== 
%% ===== 
%% ===== 
%% ===== 
%% ===== \begin{flushleft}
%% ===== Td = 11, 13, 4, 14, 16
%% ===== \end{flushleft}
%% ===== 
%% ===== 
%% ===== \begin{flushleft}
%% ===== (i, ad, sd, di, id)
%% ===== \end{flushleft}
%% ===== 
%% ===== 
%% ===== \begin{flushleft}
%% ===== Indirect word fetch
%% ===== \end{flushleft}
%% ===== 
%% ===== 
%% ===== \begin{flushleft}
%% ===== APPEND CYCLE
%% ===== \end{flushleft}
%% ===== 
%% ===== 
%% ===== \begin{flushleft}
%% ===== (Figure 5-4)
%% ===== \end{flushleft}
%% ===== 
%% ===== 
%% ===== 
%% ===== 
%% ===== 
%% ===== \begin{flushleft}
%% ===== Td = 15, 17
%% ===== \end{flushleft}
%% ===== 
%% ===== 
%% ===== \begin{flushleft}
%% ===== (dic, idc)
%% ===== \end{flushleft}
%% ===== 
%% ===== 
%% ===== 
%% ===== 
%% ===== 
%% ===== \begin{flushleft}
%% ===== Indirect word fetch
%% ===== \end{flushleft}
%% ===== 
%% ===== 
%% ===== \begin{flushleft}
%% ===== APPEND CYCLE
%% ===== \end{flushleft}
%% ===== 
%% ===== 
%% ===== \begin{flushleft}
%% ===== (Figure 5-4)
%% ===== \end{flushleft}
%% ===== 
%% ===== 
%% ===== 
%% ===== 
%% ===== 
%% ===== \begin{flushleft}
%% ===== is the cf
%% ===== \end{flushleft}
%% ===== 
%% ===== 
%% ===== \begin{flushleft}
%% ===== value legal?
%% ===== \end{flushleft}
%% ===== 
%% ===== 
%% ===== 
%% ===== 
%% ===== 
%% ===== \begin{flushleft}
%% ===== Adjust TALLY
%% ===== \end{flushleft}
%% ===== 
%% ===== 
%% ===== \begin{flushleft}
%% ===== and form
%% ===== \end{flushleft}
%% ===== 
%% ===== 
%% ===== \begin{flushleft}
%% ===== computed address
%% ===== \end{flushleft}
%% ===== 
%% ===== 
%% ===== 
%% ===== 
%% ===== 
%% ===== \begin{flushleft}
%% ===== Yes
%% ===== \end{flushleft}
%% ===== 
%% ===== 
%% ===== \begin{flushleft}
%% ===== ABORT
%% ===== \end{flushleft}
%% ===== 
%% ===== 
%% ===== \begin{flushleft}
%% ===== illegal procedure,
%% ===== \end{flushleft}
%% ===== 
%% ===== 
%% ===== \begin{flushleft}
%% ===== illegal modifier, fault
%% ===== \end{flushleft}
%% ===== 
%% ===== 
%% ===== 
%% ===== 
%% ===== 
%% ===== ???
%% ===== 
%% ===== 
%% ===== \begin{flushleft}
%% ===== cf field, and
%% ===== \end{flushleft}
%% ===== 
%% ===== 
%% ===== \begin{flushleft}
%% ===== ADDRESS. Form
%% ===== \end{flushleft}
%% ===== 
%% ===== 
%% ===== \begin{flushleft}
%% ===== computed address
%% ===== \end{flushleft}
%% ===== 
%% ===== 
%% ===== 
%% ===== 
%% ===== 
%% ===== \begin{flushleft}
%% ===== Interpret
%% ===== \end{flushleft}
%% ===== 
%% ===== 
%% ===== \begin{flushleft}
%% ===== indirect TAG
%% ===== \end{flushleft}
%% ===== 
%% ===== 
%% ===== \begin{flushleft}
%% ===== Tm = r
%% ===== \end{flushleft}
%% ===== 
%% ===== 
%% ===== 
%% ===== 
%% ===== 
%% ===== \begin{flushleft}
%% ===== Tm = ir or it
%% ===== \end{flushleft}
%% ===== 
%% ===== 
%% ===== 
%% ===== 
%% ===== 
%% ===== \begin{flushleft}
%% ===== Tm = ri
%% ===== \end{flushleft}
%% ===== 
%% ===== 
%% ===== 
%% ===== 
%% ===== 
%% ===== \begin{flushleft}
%% ===== Indirect word fetch
%% ===== \end{flushleft}
%% ===== 
%% ===== 
%% ===== \begin{flushleft}
%% ===== APPEND CYCLE
%% ===== \end{flushleft}
%% ===== 
%% ===== 
%% ===== \begin{flushleft}
%% ===== (Figure 5-4)
%% ===== \end{flushleft}
%% ===== 
%% ===== 
%% ===== 
%% ===== 
%% ===== 
%% ===== \begin{flushleft}
%% ===== END CA
%% ===== \end{flushleft}
%% ===== 
%% ===== 
%% ===== 
%% ===== 
%% ===== 
%% ===== \begin{flushleft}
%% ===== START CA
%% ===== \end{flushleft}
%% ===== 
%% ===== 
%% ===== \begin{flushleft}
%% ===== (Figure 6-2)
%% ===== \end{flushleft}
%% ===== 
%% ===== 
%% ===== 
%% ===== 
%% ===== 
%% ===== \begin{flushleft}
%% ===== Figure 6-6. Indirect Then Tally Modification Flowchart
%% ===== \end{flushleft}
%% ===== 
%% ===== 
%% ===== 
%% ===== 
%% ===== 
%% ===== \begin{flushleft}

\subsection{VIRTUAL ADDRESS FORMATION INVOLVING BOTH SEGMENT}

%% ===== \end{flushleft}
%% ===== 
%% ===== 
%% ===== \begin{flushleft}
%% ===== NUMBER AND COMPUTED ADDRESS
%% ===== \end{flushleft}
%% ===== 
%% ===== 
%% ===== \begin{flushleft}
%% ===== The second type of virtual address formation generates an effective segment number and a
%% ===== \end{flushleft}
%% ===== 
%% ===== 
%% ===== \begin{flushleft}
%% ===== computed address simultaneously.
%% ===== \end{flushleft}
%% ===== 
%% ===== 
%% ===== 
%% ===== 
%% ===== 
%% ===== \begin{flushleft}

\subsubsection{The Use of Bit 29 in the Instruction Word}

%% ===== \end{flushleft}
%% ===== 
%% ===== 
%% ===== \begin{flushleft}
%% ===== The reader is reminded that there is a preliminary step of loading TPR.CA with the
%% ===== \end{flushleft}
%% ===== 
%% ===== 
%% ===== \begin{flushleft}
%% ===== ADDRESS field of the instruction word during instruction decode.
%% ===== \end{flushleft}
%% ===== 
%% ===== 
%% ===== \begin{flushleft}
%% ===== If bit 29 of the instruction word is set to 1, modification by pointer register is invoked and
%% ===== \end{flushleft}
%% ===== 
%% ===== 
%% ===== \begin{flushleft}
%% ===== the preliminary step is executed as follows:
%% ===== \end{flushleft}
%% ===== 
%% ===== 
%% ===== \begin{flushleft}
%% ===== 1. The ADDRESS field of the instruction word is interpreted as shown in Figure 6-7 below.
%% ===== \end{flushleft}
%% ===== 
%% ===== 
%% ===== \begin{flushleft}
%% ===== 2. C(PRn.SNR) $\rightarrow$ C(TPR.TSR)
%% ===== \end{flushleft}
%% ===== 
%% ===== 
%% ===== 
%% ===== 
%% ===== 
%% ===== \begin{flushleft}
%% ===== \newpage
%% ===== 3. maximum of
%% ===== \end{flushleft}
%% ===== 
%% ===== 
%% ===== 
%% ===== 
%% ===== 
%% ===== (
%% ===== 
%% ===== 
%% ===== 
%% ===== 
%% ===== 
%% ===== \begin{flushleft}
%% ===== C(PRn.RNR), C(TPR.TRR), C(PPR.PRR)
%% ===== \end{flushleft}
%% ===== 
%% ===== 
%% ===== 
%% ===== 
%% ===== 
%% ===== )
%% ===== 
%% ===== 
%% ===== 
%% ===== 
%% ===== 
%% ===== \begin{flushleft}
%% ===== $\rightarrow$ C(TPR.TRR)
%% ===== \end{flushleft}
%% ===== 
%% ===== 
%% ===== 
%% ===== 
%% ===== 
%% ===== \begin{flushleft}
%% ===== 4. C(PRn.WORDNO) + OFFSET $\rightarrow$ C(TPR.CA)
%% ===== \end{flushleft}
%% ===== 
%% ===== 
%% ===== \begin{flushleft}
%% ===== (NOTE: OFFSET is a signed binary number.)
%% ===== \end{flushleft}
%% ===== 
%% ===== 
%% ===== \begin{flushleft}
%% ===== 5. C(PRn.BITNO) $\rightarrow$ TPR.BITNO
%% ===== \end{flushleft}
%% ===== 
%% ===== 
%% ===== 0
%% ===== 
%% ===== 
%% ===== 0
%% ===== 
%% ===== 
%% ===== 
%% ===== 
%% ===== 
%% ===== 0 0
%% ===== 
%% ===== 
%% ===== 2 3
%% ===== 
%% ===== 
%% ===== 
%% ===== 
%% ===== 
%% ===== 1
%% ===== 
%% ===== 
%% ===== 7
%% ===== 
%% ===== 
%% ===== 
%% ===== 
%% ===== 
%% ===== \begin{flushleft}
%% ===== PRn
%% ===== \end{flushleft}
%% ===== 
%% ===== 
%% ===== 
%% ===== 
%% ===== 
%% ===== \begin{flushleft}
%% ===== OFFSET
%% ===== \end{flushleft}
%% ===== 
%% ===== 
%% ===== 3
%% ===== 
%% ===== 
%% ===== 
%% ===== 
%% ===== 
%% ===== 15
%% ===== 
%% ===== 
%% ===== 
%% ===== 
%% ===== 
%% ===== \begin{flushleft}
%% ===== Figure 6-7. Format of Instruction Word ADDRESS When Bit 29 = 1
%% ===== \end{flushleft}
%% ===== 
%% ===== 
%% ===== \begin{flushleft}
%% ===== After this preliminary step is performed, virtual address formation proceeds as discussed
%% ===== \end{flushleft}
%% ===== 
%% ===== 
%% ===== \begin{flushleft}
%% ===== above or as discussed for the special address modifiers below.
%% ===== \end{flushleft}
%% ===== 
%% ===== 
%% ===== 
%% ===== 
%% ===== 
%% ===== \begin{flushleft}

\subsubsection{Special Address Modifiers}

%% ===== \end{flushleft}
%% ===== 
%% ===== 
%% ===== \begin{flushleft}
%% ===== Whenever the processor is forming a virtual address two special address modifiers may be
%% ===== \end{flushleft}
%% ===== 
%% ===== 
%% ===== \begin{flushleft}
%% ===== specified and are effective under certain restrictive conditions. The special address modifiers are
%% ===== \end{flushleft}
%% ===== 
%% ===== 
%% ===== \begin{flushleft}
%% ===== shown in Table 6-4 and discussed in the paragraphs below.
%% ===== \end{flushleft}
%% ===== 
%% ===== 
%% ===== \begin{flushleft}
%% ===== The conditions for which the special address modifiers are effective are as follows:
%% ===== \end{flushleft}
%% ===== 
%% ===== 
%% ===== \begin{flushleft}
%% ===== 1. The instruction word (or preceding indirect word) must specify indirect then register or
%% ===== \end{flushleft}
%% ===== 
%% ===== 
%% ===== \begin{flushleft}
%% ===== register then indirect modification.
%% ===== \end{flushleft}
%% ===== 
%% ===== 
%% ===== \begin{flushleft}
%% ===== 2. The computed address for the indirect word must be even.
%% ===== \end{flushleft}
%% ===== 
%% ===== 
%% ===== \begin{flushleft}
%% ===== If these conditions are satisfied, the processor examines the indirect word TAG field for the
%% ===== \end{flushleft}
%% ===== 
%% ===== 
%% ===== \begin{flushleft}
%% ===== special address modifiers.
%% ===== \end{flushleft}
%% ===== 
%% ===== 
%% ===== \begin{flushleft}
%% ===== If either condition is violated, the indirect word TAG field is interpreted as a normal address
%% ===== \end{flushleft}
%% ===== 
%% ===== 
%% ===== \begin{flushleft}
%% ===== modifier and the presence of a special address modifier will cause an illegal procedure, illegal
%% ===== \end{flushleft}
%% ===== 
%% ===== 
%% ===== \begin{flushleft}
%% ===== modifier, fault.
%% ===== \end{flushleft}
%% ===== 
%% ===== 
%% ===== 
%% ===== 
%% ===== 
%% ===== \begin{flushleft}
%% ===== Table 6-4. Special Address Modifiers
%% ===== \end{flushleft}
%% ===== 
%% ===== 
%% ===== \begin{flushleft}
%% ===== TAG Value
%% ===== \end{flushleft}
%% ===== 
%% ===== 
%% ===== 
%% ===== 
%% ===== 
%% ===== \begin{flushleft}
%% ===== Coding Symbol
%% ===== \end{flushleft}
%% ===== 
%% ===== 
%% ===== 
%% ===== 
%% ===== 
%% ===== \begin{flushleft}
%% ===== Name
%% ===== \end{flushleft}
%% ===== 
%% ===== 
%% ===== 
%% ===== 
%% ===== 
%% ===== 41
%% ===== 
%% ===== 
%% ===== 
%% ===== 
%% ===== 
%% ===== \begin{flushleft}
%% ===== itp
%% ===== \end{flushleft}
%% ===== 
%% ===== 
%% ===== 
%% ===== 
%% ===== 
%% ===== \begin{flushleft}

\subsubsubsection{Indirect to pointer}

%% ===== \end{flushleft}
%% ===== 
%% ===== 
%% ===== 
%% ===== 
%% ===== 
%% ===== 43
%% ===== 
%% ===== 
%% ===== 
%% ===== 
%% ===== 
%% ===== \begin{flushleft}
%% ===== its
%% ===== \end{flushleft}
%% ===== 
%% ===== 
%% ===== 
%% ===== 
%% ===== 
%% ===== \begin{flushleft}

\subsubsubsection{Indirect to segment}

%% ===== \end{flushleft}
%% ===== 
%% ===== 
%% ===== 
%% ===== 
%% ===== 
%% ===== \begin{flushleft}
%% ===== Indirect to Pointer (ITP) Modification
%% ===== \end{flushleft}
%% ===== 
%% ===== 
%% ===== \begin{flushleft}
%% ===== If the value for indirect to pointer modification is found in the test for special modifiers, the
%% ===== \end{flushleft}
%% ===== 
%% ===== 
%% ===== \begin{flushleft}
%% ===== indirect word-pair is interpreted as an ITP pointer pair (see Figure 6-8 for format) and the
%% ===== \end{flushleft}
%% ===== 
%% ===== 
%% ===== \begin{flushleft}
%% ===== following actions take place:
%% ===== \end{flushleft}
%% ===== 
%% ===== 
%% ===== \begin{flushleft}
%% ===== For n = C(ITP.PRNUM):
%% ===== \end{flushleft}
%% ===== 
%% ===== 
%% ===== 
%% ===== 
%% ===== 
%% ===== \begin{flushleft}
%% ===== \newpage
%% ===== C(PRn.SNR) $\rightarrow$ C(TPR.TSR)
%% ===== \end{flushleft}
%% ===== 
%% ===== 
%% ===== \begin{flushleft}
%% ===== maximum of
%% ===== \end{flushleft}
%% ===== 
%% ===== 
%% ===== 
%% ===== 
%% ===== 
%% ===== (
%% ===== 
%% ===== 
%% ===== 
%% ===== 
%% ===== 
%% ===== \begin{flushleft}
%% ===== C(PRn.RNR), C(SDW.R1), C(TPR.TRR)
%% ===== \end{flushleft}
%% ===== 
%% ===== 
%% ===== 
%% ===== 
%% ===== 
%% ===== \begin{flushleft}
%% ===== ) $\rightarrow$ C(TPR.TRR)
%% ===== \end{flushleft}
%% ===== 
%% ===== 
%% ===== 
%% ===== 
%% ===== 
%% ===== \begin{flushleft}
%% ===== C(ITP.BITNO) $\rightarrow$ C(TPR.TBR)
%% ===== \end{flushleft}
%% ===== 
%% ===== 
%% ===== \begin{flushleft}
%% ===== C(PRn.WORDNO) + C(ITP.WORDNO) + C(r) $\rightarrow$ C(TPR.CA)
%% ===== \end{flushleft}
%% ===== 
%% ===== 
%% ===== \begin{flushleft}
%% ===== where:
%% ===== \end{flushleft}
%% ===== 
%% ===== 
%% ===== \begin{flushleft}
%% ===== 1. r = C(CT-HOLD) if the instruction word or preceding indirect word specified
%% ===== \end{flushleft}
%% ===== 
%% ===== 
%% ===== \begin{flushleft}
%% ===== indirect then register modification, or
%% ===== \end{flushleft}
%% ===== 
%% ===== 
%% ===== \begin{flushleft}
%% ===== 2. r = C(ITP.MOD.Td) if the instruction word or preceding indirect word specified
%% ===== \end{flushleft}
%% ===== 
%% ===== 
%% ===== \begin{flushleft}
%% ===== register then indirect modification and ITP.MOD.Tm specifies either register or
%% ===== \end{flushleft}
%% ===== 
%% ===== 
%% ===== \begin{flushleft}
%% ===== register then indirect modification.
%% ===== \end{flushleft}
%% ===== 
%% ===== 
%% ===== \begin{flushleft}
%% ===== 3. SDW.R1 is the upper limit of the read/write ring bracket for the segment
%% ===== \end{flushleft}
%% ===== 
%% ===== 
%% ===== \begin{flushleft}
%% ===== C(TPR.TSR) (see Section 8).
%% ===== \end{flushleft}
%% ===== 
%% ===== 
%% ===== \begin{flushleft}
%% ===== Even word
%% ===== \end{flushleft}
%% ===== 
%% ===== 
%% ===== 0
%% ===== 
%% ===== 
%% ===== 0
%% ===== 
%% ===== 
%% ===== 
%% ===== 
%% ===== 
%% ===== 0 0
%% ===== 
%% ===== 
%% ===== 2 3
%% ===== 
%% ===== 
%% ===== 
%% ===== 
%% ===== 
%% ===== 2 3
%% ===== 
%% ===== 
%% ===== 9 0
%% ===== 
%% ===== 
%% ===== 
%% ===== 
%% ===== 
%% ===== \begin{flushleft}
%% ===== PRNUM 0 0 0 0 0 0 0 0 0 0 0 0 0 0 0 0 0 0 0 0 0 0 0 0 0 0 0
%% ===== \end{flushleft}
%% ===== 
%% ===== 
%% ===== 3
%% ===== 
%% ===== 
%% ===== 
%% ===== 
%% ===== 
%% ===== 3
%% ===== 
%% ===== 
%% ===== 5
%% ===== 
%% ===== 
%% ===== 418
%% ===== 
%% ===== 
%% ===== 
%% ===== 
%% ===== 
%% ===== 27
%% ===== 
%% ===== 
%% ===== 
%% ===== 
%% ===== 
%% ===== 6
%% ===== 
%% ===== 
%% ===== 
%% ===== 
%% ===== 
%% ===== \begin{flushleft}
%% ===== Odd word
%% ===== \end{flushleft}
%% ===== 
%% ===== 
%% ===== 0
%% ===== 
%% ===== 
%% ===== 0
%% ===== 
%% ===== 
%% ===== 
%% ===== 
%% ===== 
%% ===== 1 1
%% ===== 
%% ===== 
%% ===== 7 8
%% ===== 
%% ===== 
%% ===== \begin{flushleft}
%% ===== WORDNO
%% ===== \end{flushleft}
%% ===== 
%% ===== 
%% ===== 
%% ===== 
%% ===== 
%% ===== 2 2
%% ===== 
%% ===== 
%% ===== 0 1
%% ===== 
%% ===== 
%% ===== 
%% ===== 
%% ===== 
%% ===== 0 0 0
%% ===== 
%% ===== 
%% ===== 18
%% ===== 
%% ===== 
%% ===== 
%% ===== 
%% ===== 
%% ===== 3
%% ===== 
%% ===== 
%% ===== 
%% ===== 
%% ===== 
%% ===== 2 2
%% ===== 
%% ===== 
%% ===== 6 7
%% ===== 
%% ===== 
%% ===== \begin{flushleft}
%% ===== BITNO
%% ===== \end{flushleft}
%% ===== 
%% ===== 
%% ===== 
%% ===== 
%% ===== 
%% ===== 2 3 3 3
%% ===== 
%% ===== 
%% ===== 9 0 1 2
%% ===== 
%% ===== 
%% ===== 
%% ===== 
%% ===== 
%% ===== 0 0 0
%% ===== 
%% ===== 
%% ===== 6
%% ===== 
%% ===== 
%% ===== 
%% ===== 
%% ===== 
%% ===== 3
%% ===== 
%% ===== 
%% ===== 
%% ===== 
%% ===== 
%% ===== 3
%% ===== 
%% ===== 
%% ===== 5
%% ===== 
%% ===== 
%% ===== 
%% ===== 
%% ===== 
%% ===== \begin{flushleft}
%% ===== MOD
%% ===== \end{flushleft}
%% ===== 
%% ===== 
%% ===== \begin{flushleft}
%% ===== Tm
%% ===== \end{flushleft}
%% ===== 
%% ===== 
%% ===== 2
%% ===== 
%% ===== 
%% ===== 
%% ===== 
%% ===== 
%% ===== \begin{flushleft}
%% ===== Td
%% ===== \end{flushleft}
%% ===== 
%% ===== 
%% ===== 4
%% ===== 
%% ===== 
%% ===== 
%% ===== 
%% ===== 
%% ===== \begin{flushleft}
%% ===== Figure 6-8. ITP Pointer Pair Format
%% ===== \end{flushleft}
%% ===== 
%% ===== 
%% ===== \begin{flushleft}
%% ===== Field Name Meaning
%% ===== \end{flushleft}
%% ===== 
%% ===== 
%% ===== \begin{flushleft}
%% ===== PRNUM
%% ===== \end{flushleft}
%% ===== 
%% ===== 
%% ===== 
%% ===== 
%% ===== 
%% ===== \begin{flushleft}
%% ===== The number of the pointer register through which to make the
%% ===== \end{flushleft}
%% ===== 
%% ===== 
%% ===== \begin{flushleft}
%% ===== segment reference
%% ===== \end{flushleft}
%% ===== 
%% ===== 
%% ===== 
%% ===== 
%% ===== 
%% ===== \begin{flushleft}
%% ===== WORDNO
%% ===== \end{flushleft}
%% ===== 
%% ===== 
%% ===== 
%% ===== 
%% ===== 
%% ===== \begin{flushleft}
%% ===== A word offset value to be added to C(PRn.WORDNO)
%% ===== \end{flushleft}
%% ===== 
%% ===== 
%% ===== 
%% ===== 
%% ===== 
%% ===== \begin{flushleft}
%% ===== BITNO
%% ===== \end{flushleft}
%% ===== 
%% ===== 
%% ===== 
%% ===== 
%% ===== 
%% ===== \begin{flushleft}
%% ===== A bit offset value for the data item
%% ===== \end{flushleft}
%% ===== 
%% ===== 
%% ===== 
%% ===== 
%% ===== 
%% ===== \begin{flushleft}
%% ===== MOD
%% ===== \end{flushleft}
%% ===== 
%% ===== 
%% ===== 
%% ===== 
%% ===== 
%% ===== \begin{flushleft}
%% ===== Any normal address modifier (not ITP or ITS)
%% ===== \end{flushleft}
%% ===== 
%% ===== 
%% ===== 
%% ===== 
%% ===== 
%% ===== \begin{flushleft}
%% ===== Indirect to Segment (ITS) Modification
%% ===== \end{flushleft}
%% ===== 
%% ===== 
%% ===== \begin{flushleft}
%% ===== If the value for indirect to segment modification is found in the test for special modifiers,
%% ===== \end{flushleft}
%% ===== 
%% ===== 
%% ===== \begin{flushleft}
%% ===== the indirect word-pair is interpreted as an ITS pointer pair (see Figure 6-9 for format) and the
%% ===== \end{flushleft}
%% ===== 
%% ===== 
%% ===== \begin{flushleft}
%% ===== following actions take place:
%% ===== \end{flushleft}
%% ===== 
%% ===== 
%% ===== \begin{flushleft}
%% ===== C(ITS.SEGNO) $\rightarrow$ C(TPR.TSR)
%% ===== \end{flushleft}
%% ===== 
%% ===== 
%% ===== 
%% ===== 
%% ===== 
%% ===== \begin{flushleft}
%% ===== \newpage
%% ===== maximum of
%% ===== \end{flushleft}
%% ===== 
%% ===== 
%% ===== 
%% ===== 
%% ===== 
%% ===== (
%% ===== 
%% ===== 
%% ===== 
%% ===== 
%% ===== 
%% ===== \begin{flushleft}
%% ===== C(ITS. RN), C(SDW.R1), C(TPR.TRR)
%% ===== \end{flushleft}
%% ===== 
%% ===== 
%% ===== 
%% ===== 
%% ===== 
%% ===== \begin{flushleft}
%% ===== ) $\rightarrow$ C(TPR.TRR)
%% ===== \end{flushleft}
%% ===== 
%% ===== 
%% ===== 
%% ===== 
%% ===== 
%% ===== \begin{flushleft}
%% ===== C(ITS.BITNO) $\rightarrow$ C(TPR.TBR)
%% ===== \end{flushleft}
%% ===== 
%% ===== 
%% ===== \begin{flushleft}
%% ===== C(ITS.WORDNO) + C(r) $\rightarrow$ C(TPR.CA)
%% ===== \end{flushleft}
%% ===== 
%% ===== 
%% ===== \begin{flushleft}
%% ===== where:
%% ===== \end{flushleft}
%% ===== 
%% ===== 
%% ===== \begin{flushleft}
%% ===== 1. r = C(CT-HOLD) if the instruction word or preceding indirect word specified
%% ===== \end{flushleft}
%% ===== 
%% ===== 
%% ===== \begin{flushleft}
%% ===== indirect then register modification, or
%% ===== \end{flushleft}
%% ===== 
%% ===== 
%% ===== \begin{flushleft}
%% ===== 2. r = C(ITS.MOD.Td) if the instruction word or preceding indirect word specified
%% ===== \end{flushleft}
%% ===== 
%% ===== 
%% ===== \begin{flushleft}
%% ===== register then indirect modification and ITS.MOD.Tm specifies either register or
%% ===== \end{flushleft}
%% ===== 
%% ===== 
%% ===== \begin{flushleft}
%% ===== register then indirect modification.
%% ===== \end{flushleft}
%% ===== 
%% ===== 
%% ===== \begin{flushleft}
%% ===== 3. SDW.R1 is the upper limit of the read/write ring bracket for the segment
%% ===== \end{flushleft}
%% ===== 
%% ===== 
%% ===== \begin{flushleft}
%% ===== C(TPR.TSR) (see Section 8).
%% ===== \end{flushleft}
%% ===== 
%% ===== 
%% ===== \begin{flushleft}
%% ===== Even word
%% ===== \end{flushleft}
%% ===== 
%% ===== 
%% ===== 0
%% ===== 
%% ===== 
%% ===== 0
%% ===== 
%% ===== 
%% ===== 
%% ===== 
%% ===== 
%% ===== 0 0
%% ===== 
%% ===== 
%% ===== 2 3
%% ===== 
%% ===== 
%% ===== 
%% ===== 
%% ===== 
%% ===== 1 1
%% ===== 
%% ===== 
%% ===== 7 8
%% ===== 
%% ===== 
%% ===== 
%% ===== 
%% ===== 
%% ===== 0 0 0
%% ===== 
%% ===== 
%% ===== 
%% ===== 
%% ===== 
%% ===== \begin{flushleft}
%% ===== SEGNO
%% ===== \end{flushleft}
%% ===== 
%% ===== 
%% ===== 
%% ===== 
%% ===== 
%% ===== 3
%% ===== 
%% ===== 
%% ===== 
%% ===== 
%% ===== 
%% ===== 2 2
%% ===== 
%% ===== 
%% ===== 0 1
%% ===== 
%% ===== 
%% ===== \begin{flushleft}
%% ===== RN
%% ===== \end{flushleft}
%% ===== 
%% ===== 
%% ===== 
%% ===== 
%% ===== 
%% ===== 15
%% ===== 
%% ===== 
%% ===== 
%% ===== 
%% ===== 
%% ===== 2 3
%% ===== 
%% ===== 
%% ===== 9 0
%% ===== 
%% ===== 
%% ===== 
%% ===== 
%% ===== 
%% ===== 0 0 0 0 0 0 0 0 0
%% ===== 
%% ===== 
%% ===== 3
%% ===== 
%% ===== 
%% ===== 
%% ===== 
%% ===== 
%% ===== 3
%% ===== 
%% ===== 
%% ===== 5
%% ===== 
%% ===== 
%% ===== 438
%% ===== 
%% ===== 
%% ===== 
%% ===== 
%% ===== 
%% ===== 9
%% ===== 
%% ===== 
%% ===== 
%% ===== 
%% ===== 
%% ===== 6
%% ===== 
%% ===== 
%% ===== 
%% ===== 
%% ===== 
%% ===== 2 3 3 3
%% ===== 
%% ===== 
%% ===== 9 0 1 2
%% ===== 
%% ===== 
%% ===== 
%% ===== 
%% ===== 
%% ===== 3
%% ===== 
%% ===== 
%% ===== 5
%% ===== 
%% ===== 
%% ===== 
%% ===== 
%% ===== 
%% ===== \begin{flushleft}
%% ===== Odd word
%% ===== \end{flushleft}
%% ===== 
%% ===== 
%% ===== 0
%% ===== 
%% ===== 
%% ===== 0
%% ===== 
%% ===== 
%% ===== 
%% ===== 
%% ===== 
%% ===== 1 1
%% ===== 
%% ===== 
%% ===== 7 8
%% ===== 
%% ===== 
%% ===== \begin{flushleft}
%% ===== WORDNO
%% ===== \end{flushleft}
%% ===== 
%% ===== 
%% ===== 
%% ===== 
%% ===== 
%% ===== 2 2
%% ===== 
%% ===== 
%% ===== 0 1
%% ===== 
%% ===== 
%% ===== 
%% ===== 
%% ===== 
%% ===== 0 0 0
%% ===== 
%% ===== 
%% ===== 18
%% ===== 
%% ===== 
%% ===== 
%% ===== 
%% ===== 
%% ===== 3
%% ===== 
%% ===== 
%% ===== 
%% ===== 
%% ===== 
%% ===== 2 2
%% ===== 
%% ===== 
%% ===== 6 7
%% ===== 
%% ===== 
%% ===== \begin{flushleft}
%% ===== BITNO
%% ===== \end{flushleft}
%% ===== 
%% ===== 
%% ===== 
%% ===== 
%% ===== 
%% ===== 0 0 0
%% ===== 
%% ===== 
%% ===== 6
%% ===== 
%% ===== 
%% ===== 
%% ===== 
%% ===== 
%% ===== 3
%% ===== 
%% ===== 
%% ===== 
%% ===== 
%% ===== 
%% ===== \begin{flushleft}
%% ===== MOD
%% ===== \end{flushleft}
%% ===== 
%% ===== 
%% ===== \begin{flushleft}
%% ===== Tm
%% ===== \end{flushleft}
%% ===== 
%% ===== 
%% ===== 2
%% ===== 
%% ===== 
%% ===== 
%% ===== 
%% ===== 
%% ===== \begin{flushleft}
%% ===== Td
%% ===== \end{flushleft}
%% ===== 
%% ===== 
%% ===== 4
%% ===== 
%% ===== 
%% ===== 
%% ===== 
%% ===== 
%% ===== \begin{flushleft}
%% ===== Figure 6-9. ITS Pointer Pair Format
%% ===== \end{flushleft}
%% ===== 
%% ===== 
%% ===== \begin{flushleft}
%% ===== Field Name
%% ===== \end{flushleft}
%% ===== 
%% ===== 
%% ===== 
%% ===== 
%% ===== 
%% ===== \begin{flushleft}
%% ===== Meaning
%% ===== \end{flushleft}
%% ===== 
%% ===== 
%% ===== 
%% ===== 
%% ===== 
%% ===== \begin{flushleft}
%% ===== SEGNO
%% ===== \end{flushleft}
%% ===== 
%% ===== 
%% ===== 
%% ===== 
%% ===== 
%% ===== \begin{flushleft}
%% ===== The number of the segment to be referenced
%% ===== \end{flushleft}
%% ===== 
%% ===== 
%% ===== 
%% ===== 
%% ===== 
%% ===== \begin{flushleft}
%% ===== WORDNO
%% ===== \end{flushleft}
%% ===== 
%% ===== 
%% ===== 
%% ===== 
%% ===== 
%% ===== \begin{flushleft}
%% ===== Word offset to be used in the computed address formation
%% ===== \end{flushleft}
%% ===== 
%% ===== 
%% ===== 
%% ===== 
%% ===== 
%% ===== \begin{flushleft}
%% ===== BITNO
%% ===== \end{flushleft}
%% ===== 
%% ===== 
%% ===== 
%% ===== 
%% ===== 
%% ===== \begin{flushleft}
%% ===== The bit offset for the data item
%% ===== \end{flushleft}
%% ===== 
%% ===== 
%% ===== 
%% ===== 
%% ===== 
%% ===== \begin{flushleft}
%% ===== MOD
%% ===== \end{flushleft}
%% ===== 
%% ===== 
%% ===== 
%% ===== 
%% ===== 
%% ===== \begin{flushleft}
%% ===== Any valid normal address modifier (not ITS or ITP)
%% ===== \end{flushleft}
%% ===== 
%% ===== 
%% ===== 
%% ===== 
%% ===== 
%% ===== \begin{flushleft}

\subsubsection{Effective Segment Number Generation}

%% ===== \end{flushleft}
%% ===== 
%% ===== 
%% ===== \begin{flushleft}
%% ===== A simplified flowchart for effective segment number generation is given in Figure 6-10.
%% ===== \end{flushleft}
%% ===== 
%% ===== 
%% ===== \begin{flushleft}
%% ===== Although effective ring number generation and access checking are an integral part of this
%% ===== \end{flushleft}
%% ===== 
%% ===== 
%% ===== \begin{flushleft}
%% ===== process, their treatment is deferred to Section 8.
%% ===== \end{flushleft}
%% ===== 
%% ===== 
%% ===== 
%% ===== 
%% ===== 
%% ===== \begin{flushleft}
%% ===== \newpage
%% ===== START ESN
%% ===== \end{flushleft}
%% ===== 
%% ===== 
%% ===== 
%% ===== 
%% ===== 
%% ===== \begin{flushleft}
%% ===== Yes
%% ===== \end{flushleft}
%% ===== 
%% ===== 
%% ===== 
%% ===== 
%% ===== 
%% ===== \begin{flushleft}
%% ===== Was last cycle an
%% ===== \end{flushleft}
%% ===== 
%% ===== 
%% ===== \begin{flushleft}
%% ===== indirect word?
%% ===== \end{flushleft}
%% ===== 
%% ===== 
%% ===== \begin{flushleft}
%% ===== No
%% ===== \end{flushleft}
%% ===== 
%% ===== 
%% ===== \begin{flushleft}
%% ===== Was it a
%% ===== \end{flushleft}
%% ===== 
%% ===== 
%% ===== \begin{flushleft}
%% ===== No
%% ===== \end{flushleft}
%% ===== 
%% ===== 
%% ===== \begin{flushleft}
%% ===== sequential instruction
%% ===== \end{flushleft}
%% ===== 
%% ===== 
%% ===== \begin{flushleft}
%% ===== fetch?
%% ===== \end{flushleft}
%% ===== 
%% ===== 
%% ===== \begin{flushleft}
%% ===== Yes
%% ===== \end{flushleft}
%% ===== 
%% ===== 
%% ===== 
%% ===== 
%% ===== 
%% ===== \begin{flushleft}
%% ===== Is bit
%% ===== \end{flushleft}
%% ===== 
%% ===== 
%% ===== \begin{flushleft}
%% ===== 29 ON?
%% ===== \end{flushleft}
%% ===== 
%% ===== 
%% ===== 
%% ===== 
%% ===== 
%% ===== \begin{flushleft}
%% ===== Yes
%% ===== \end{flushleft}
%% ===== 
%% ===== 
%% ===== 
%% ===== 
%% ===== 
%% ===== \begin{flushleft}
%% ===== No
%% ===== \end{flushleft}
%% ===== 
%% ===== 
%% ===== 
%% ===== 
%% ===== 
%% ===== \begin{flushleft}
%% ===== n = C(instruction word)0,2
%% ===== \end{flushleft}
%% ===== 
%% ===== 
%% ===== \begin{flushleft}
%% ===== C(PRn .SNR) $\rightarrow$ C(TPR.TSR)
%% ===== \end{flushleft}
%% ===== 
%% ===== 
%% ===== 
%% ===== 
%% ===== 
%% ===== \begin{flushleft}
%% ===== C(PPR.PSR) $\rightarrow$ C(TPR.TSR)
%% ===== \end{flushleft}
%% ===== 
%% ===== 
%% ===== 
%% ===== 
%% ===== 
%% ===== \begin{flushleft}
%% ===== CA CYCLE
%% ===== \end{flushleft}
%% ===== 
%% ===== 
%% ===== \begin{flushleft}
%% ===== (Figure 6-2)
%% ===== \end{flushleft}
%% ===== 
%% ===== 
%% ===== 
%% ===== 
%% ===== 
%% ===== \begin{flushleft}
%% ===== Indirect
%% ===== \end{flushleft}
%% ===== 
%% ===== 
%% ===== \begin{flushleft}
%% ===== word fetch?
%% ===== \end{flushleft}
%% ===== 
%% ===== 
%% ===== \begin{flushleft}
%% ===== No
%% ===== \end{flushleft}
%% ===== 
%% ===== 
%% ===== 
%% ===== 
%% ===== 
%% ===== \begin{flushleft}
%% ===== Yes
%% ===== \end{flushleft}
%% ===== 
%% ===== 
%% ===== \begin{flushleft}
%% ===== ri or ir \&
%% ===== \end{flushleft}
%% ===== 
%% ===== 
%% ===== \begin{flushleft}
%% ===== TPR.CA even?
%% ===== \end{flushleft}
%% ===== 
%% ===== 
%% ===== 
%% ===== 
%% ===== 
%% ===== \begin{flushleft}
%% ===== Yes
%% ===== \end{flushleft}
%% ===== 
%% ===== 
%% ===== 
%% ===== 
%% ===== 
%% ===== \begin{flushleft}
%% ===== No
%% ===== \end{flushleft}
%% ===== 
%% ===== 
%% ===== 
%% ===== 
%% ===== 
%% ===== \begin{flushleft}
%% ===== A
%% ===== \end{flushleft}
%% ===== 
%% ===== 
%% ===== 
%% ===== 
%% ===== 
%% ===== \begin{flushleft}
%% ===== B
%% ===== \end{flushleft}
%% ===== 
%% ===== 
%% ===== 
%% ===== 
%% ===== 
%% ===== \begin{flushleft}
%% ===== Figure 6-10. Effective Segment Generation Flowchart
%% ===== \end{flushleft}
%% ===== 
%% ===== 
%% ===== 
%% ===== 
%% ===== 
%% ===== \begin{flushleft}
%% ===== \newpage
%% ===== A
%% ===== \end{flushleft}
%% ===== 
%% ===== 
%% ===== 
%% ===== 
%% ===== 
%% ===== \begin{flushleft}
%% ===== B
%% ===== \end{flushleft}
%% ===== 
%% ===== 
%% ===== 
%% ===== 
%% ===== 
%% ===== \begin{flushleft}
%% ===== No
%% ===== \end{flushleft}
%% ===== 
%% ===== 
%% ===== \begin{flushleft}
%% ===== TAG =
%% ===== \end{flushleft}
%% ===== 
%% ===== 
%% ===== \begin{flushleft}
%% ===== ITP?
%% ===== \end{flushleft}
%% ===== 
%% ===== 
%% ===== 
%% ===== 
%% ===== 
%% ===== \begin{flushleft}
%% ===== TAG =
%% ===== \end{flushleft}
%% ===== 
%% ===== 
%% ===== \begin{flushleft}
%% ===== ITS?
%% ===== \end{flushleft}
%% ===== 
%% ===== 
%% ===== 
%% ===== 
%% ===== 
%% ===== \begin{flushleft}
%% ===== C(Y)3,17 $\rightarrow$ C(TPR.TSR)
%% ===== \end{flushleft}
%% ===== 
%% ===== 
%% ===== \begin{flushleft}
%% ===== C(Y+1)0,17 $\rightarrow$ C(TPR.CA)
%% ===== \end{flushleft}
%% ===== 
%% ===== 
%% ===== \begin{flushleft}
%% ===== C(ITS.MOD) $\rightarrow$ TAG
%% ===== \end{flushleft}
%% ===== 
%% ===== 
%% ===== 
%% ===== 
%% ===== 
%% ===== \begin{flushleft}
%% ===== Yes
%% ===== \end{flushleft}
%% ===== 
%% ===== 
%% ===== 
%% ===== 
%% ===== 
%% ===== \begin{flushleft}
%% ===== No
%% ===== \end{flushleft}
%% ===== 
%% ===== 
%% ===== 
%% ===== 
%% ===== 
%% ===== \begin{flushleft}
%% ===== Need an
%% ===== \end{flushleft}
%% ===== 
%% ===== 
%% ===== \begin{flushleft}
%% ===== indirect word?
%% ===== \end{flushleft}
%% ===== 
%% ===== 
%% ===== 
%% ===== 
%% ===== 
%% ===== \begin{flushleft}
%% ===== n = C(Y)0,2
%% ===== \end{flushleft}
%% ===== 
%% ===== 
%% ===== \begin{flushleft}
%% ===== C(PRn .SNR) $\rightarrow$ C(TPR.TSR)
%% ===== \end{flushleft}
%% ===== 
%% ===== 
%% ===== \begin{flushleft}
%% ===== C(PRn .WORDNO) + C(Y+1)0,17
%% ===== \end{flushleft}
%% ===== 
%% ===== 
%% ===== \begin{flushleft}
%% ===== $\rightarrow$ C(TPR.CA)
%% ===== \end{flushleft}
%% ===== 
%% ===== 
%% ===== \begin{flushleft}
%% ===== C(ITP.MOD) $\rightarrow$ TAG
%% ===== \end{flushleft}
%% ===== 
%% ===== 
%% ===== 
%% ===== 
%% ===== 
%% ===== \begin{flushleft}
%% ===== Yes
%% ===== \end{flushleft}
%% ===== 
%% ===== 
%% ===== 
%% ===== 
%% ===== 
%% ===== \begin{flushleft}
%% ===== No
%% ===== \end{flushleft}
%% ===== 
%% ===== 
%% ===== 
%% ===== 
%% ===== 
%% ===== \begin{flushleft}
%% ===== rtcd
%% ===== \end{flushleft}
%% ===== 
%% ===== 
%% ===== \begin{flushleft}
%% ===== operand?
%% ===== \end{flushleft}
%% ===== 
%% ===== 
%% ===== 
%% ===== 
%% ===== 
%% ===== \begin{flushleft}
%% ===== Yes
%% ===== \end{flushleft}
%% ===== 
%% ===== 
%% ===== 
%% ===== 
%% ===== 
%% ===== \begin{flushleft}
%% ===== No
%% ===== \end{flushleft}
%% ===== 
%% ===== 
%% ===== 
%% ===== 
%% ===== 
%% ===== \begin{flushleft}
%% ===== Yes
%% ===== \end{flushleft}
%% ===== 
%% ===== 
%% ===== 
%% ===== 
%% ===== 
%% ===== \begin{flushleft}
%% ===== C(Y)3,17 $\rightarrow$ C(PPR.PSR)
%% ===== \end{flushleft}
%% ===== 
%% ===== 
%% ===== \begin{flushleft}
%% ===== C(Y+1)0,17 $\rightarrow$ C(PPR.IC)
%% ===== \end{flushleft}
%% ===== 
%% ===== 
%% ===== 
%% ===== 
%% ===== 
%% ===== \begin{flushleft}
%% ===== call6 or
%% ===== \end{flushleft}
%% ===== 
%% ===== 
%% ===== \begin{flushleft}
%% ===== transfer
%% ===== \end{flushleft}
%% ===== 
%% ===== 
%% ===== \begin{flushleft}
%% ===== operand?
%% ===== \end{flushleft}
%% ===== 
%% ===== 
%% ===== \begin{flushleft}
%% ===== Yes
%% ===== \end{flushleft}
%% ===== 
%% ===== 
%% ===== 
%% ===== 
%% ===== 
%% ===== \begin{flushleft}
%% ===== START ESN
%% ===== \end{flushleft}
%% ===== 
%% ===== 
%% ===== \begin{flushleft}
%% ===== No
%% ===== \end{flushleft}
%% ===== 
%% ===== 
%% ===== 
%% ===== 
%% ===== 
%% ===== \begin{flushleft}
%% ===== C(TPR.TSR) $\rightarrow$ C(PPR.PSR)
%% ===== \end{flushleft}
%% ===== 
%% ===== 
%% ===== \begin{flushleft}
%% ===== C(TPR.CA) $\rightarrow$ C(PPR.IC)`
%% ===== \end{flushleft}
%% ===== 
%% ===== 
%% ===== 
%% ===== 
%% ===== 
%% ===== \begin{flushleft}
%% ===== appending
%% ===== \end{flushleft}
%% ===== 
%% ===== 
%% ===== \begin{flushleft}
%% ===== unit data
%% ===== \end{flushleft}
%% ===== 
%% ===== 
%% ===== \begin{flushleft}
%% ===== movement?
%% ===== \end{flushleft}
%% ===== 
%% ===== 
%% ===== \begin{flushleft}
%% ===== No
%% ===== \end{flushleft}
%% ===== 
%% ===== 
%% ===== 
%% ===== 
%% ===== 
%% ===== \begin{flushleft}
%% ===== Yes
%% ===== \end{flushleft}
%% ===== 
%% ===== 
%% ===== 
%% ===== 
%% ===== 
%% ===== \begin{flushleft}
%% ===== EXECUTE
%% ===== \end{flushleft}
%% ===== 
%% ===== 
%% ===== \begin{flushleft}
%% ===== END ESN
%% ===== \end{flushleft}
%% ===== 
%% ===== 
%% ===== 
%% ===== 
%% ===== 
%% ===== \begin{flushleft}
%% ===== (Not shown)
%% ===== \end{flushleft}
%% ===== 
%% ===== 
%% ===== 
%% ===== 
%% ===== 
%% ===== \begin{flushleft}
%% ===== Figure 6-10(cont). Effective Segment Number Generation Flowchart
%% ===== \end{flushleft}
%% ===== 
%% ===== 
%% ===== 
%% ===== 
%% ===== 
%% ===== \begin{flushleft}

\subsection{VIRTUAL ADDRESS FORMATION FOR EXTENDED INSTRUCTION}

%% ===== \end{flushleft}
%% ===== 
%% ===== 
%% ===== \begin{flushleft}
%% ===== SET
%% ===== \end{flushleft}
%% ===== 
%% ===== 
%% ===== \begin{flushleft}
%% ===== The steps involved in virtual address formation for the operand of an EIS instruction are
%% ===== \end{flushleft}
%% ===== 
%% ===== 
%% ===== \begin{flushleft}
%% ===== shown in Figure 6-11. The flowchart depicts the virtual address formation for operand k as
%% ===== \end{flushleft}
%% ===== 
%% ===== 
%% ===== \begin{flushleft}
%% ===== described by its modification field, MFk. This virtual address formation is performed for each
%% ===== \end{flushleft}
%% ===== 
%% ===== 
%% ===== \begin{flushleft}
%% ===== operand as its operand descriptor is decoded.
%% ===== \end{flushleft}
%% ===== 
%% ===== 
%% ===== 
%% ===== 
%% ===== 
%% ===== \begin{flushleft}
%% ===== \newpage
%% ===== START EIS CA
%% ===== \end{flushleft}
%% ===== 
%% ===== 
%% ===== 
%% ===== 
%% ===== 
%% ===== \begin{flushleft}
%% ===== No
%% ===== \end{flushleft}
%% ===== 
%% ===== 
%% ===== 
%% ===== 
%% ===== 
%% ===== \begin{flushleft}
%% ===== MFk .ID
%% ===== \end{flushleft}
%% ===== 
%% ===== 
%% ===== = 1?
%% ===== 
%% ===== 
%% ===== 
%% ===== 
%% ===== 
%% ===== \begin{flushleft}
%% ===== Yes
%% ===== \end{flushleft}
%% ===== 
%% ===== 
%% ===== \begin{flushleft}
%% ===== ESN CYCLE
%% ===== \end{flushleft}
%% ===== 
%% ===== 
%% ===== \begin{flushleft}
%% ===== (Figure 6-10)
%% ===== \end{flushleft}
%% ===== 
%% ===== 
%% ===== 
%% ===== 
%% ===== 
%% ===== \begin{flushleft}
%% ===== operand descriptor
%% ===== \end{flushleft}
%% ===== 
%% ===== 
%% ===== \begin{flushleft}
%% ===== APPEND CYCLE
%% ===== \end{flushleft}
%% ===== 
%% ===== 
%% ===== \begin{flushleft}
%% ===== (Figure 5-4)
%% ===== \end{flushleft}
%% ===== 
%% ===== 
%% ===== 
%% ===== 
%% ===== 
%% ===== \begin{flushleft}
%% ===== No
%% ===== \end{flushleft}
%% ===== 
%% ===== 
%% ===== 
%% ===== 
%% ===== 
%% ===== \begin{flushleft}
%% ===== MFk .AR
%% ===== \end{flushleft}
%% ===== 
%% ===== 
%% ===== = 1?
%% ===== 
%% ===== 
%% ===== 
%% ===== 
%% ===== 
%% ===== \begin{flushleft}
%% ===== Yes
%% ===== \end{flushleft}
%% ===== 
%% ===== 
%% ===== 
%% ===== 
%% ===== 
%% ===== \begin{flushleft}
%% ===== n = C(Y)0,2
%% ===== \end{flushleft}
%% ===== 
%% ===== 
%% ===== \begin{flushleft}
%% ===== (NOTE 1)
%% ===== \end{flushleft}
%% ===== 
%% ===== 
%% ===== 
%% ===== 
%% ===== 
%% ===== \begin{flushleft}
%% ===== n = null
%% ===== \end{flushleft}
%% ===== 
%% ===== 
%% ===== 
%% ===== 
%% ===== 
%% ===== \begin{flushleft}
%% ===== Yes
%% ===== \end{flushleft}
%% ===== 
%% ===== 
%% ===== 
%% ===== 
%% ===== 
%% ===== \begin{flushleft}
%% ===== MFk .REG
%% ===== \end{flushleft}
%% ===== 
%% ===== 
%% ===== = 0?
%% ===== 
%% ===== 
%% ===== 
%% ===== 
%% ===== 
%% ===== \begin{flushleft}
%% ===== No
%% ===== \end{flushleft}
%% ===== 
%% ===== 
%% ===== 
%% ===== 
%% ===== 
%% ===== \begin{flushleft}
%% ===== r = null
%% ===== \end{flushleft}
%% ===== 
%% ===== 
%% ===== 
%% ===== 
%% ===== 
%% ===== \begin{flushleft}
%% ===== r = Mfk .REG
%% ===== \end{flushleft}
%% ===== 
%% ===== 
%% ===== 
%% ===== 
%% ===== 
%% ===== \begin{flushleft}
%% ===== Form effective word/char/bit
%% ===== \end{flushleft}
%% ===== 
%% ===== 
%% ===== \begin{flushleft}
%% ===== address from
%% ===== \end{flushleft}
%% ===== 
%% ===== 
%% ===== \begin{flushleft}
%% ===== Y, CN, C, B, C(PRn), C(r )
%% ===== \end{flushleft}
%% ===== 
%% ===== 
%% ===== \begin{flushleft}
%% ===== (NOTE 1, 2)
%% ===== \end{flushleft}
%% ===== 
%% ===== 
%% ===== 
%% ===== 
%% ===== 
%% ===== \begin{flushleft}
%% ===== END EIS CA
%% ===== \end{flushleft}
%% ===== 
%% ===== 
%% ===== 
%% ===== 
%% ===== 
%% ===== \begin{flushleft}
%% ===== Figure 6-11. EIS Virtual Address Formation Flowchart
%% ===== \end{flushleft}
%% ===== 
%% ===== 
%% ===== \begin{flushleft}
%% ===== NOTE 1: The symbol {``}Y'' stands for the contents of the ADDRESS field of the operand descriptor.
%% ===== \end{flushleft}
%% ===== 
%% ===== 
%% ===== \begin{flushleft}
%% ===== The symbols '{``}CN'' and {``}C'' stand for the contents of the character number field. The
%% ===== \end{flushleft}
%% ===== 
%% ===== 
%% ===== \begin{flushleft}
%% ===== symbol {``}B'' stands for the contents of the bit number field.
%% ===== \end{flushleft}
%% ===== 
%% ===== 
%% ===== \begin{flushleft}
%% ===== NOTE 2: The algorithms used in the formation of the effective word/char/bit address are described
%% ===== \end{flushleft}
%% ===== 
%% ===== 
%% ===== \begin{flushleft}
%% ===== below.
%% ===== \end{flushleft}
%% ===== 
%% ===== 
%% ===== 
%% ===== 
%% ===== 
%% ===== \begin{flushleft}

\subsubsection{Character- and Bit-String Addressing}

%% ===== \end{flushleft}
%% ===== 
%% ===== 
%% ===== \begin{flushleft}
%% ===== The processor represents the effective address of a character- or bit-string operand in three
%% ===== \end{flushleft}
%% ===== 
%% ===== 
%% ===== \begin{flushleft}
%% ===== different forms as follows:
%% ===== \end{flushleft}
%% ===== 
%% ===== 
%% ===== \begin{flushleft}
%% ===== 1. Pointer register form
%% ===== \end{flushleft}
%% ===== 
%% ===== 
%% ===== \begin{flushleft}
%% ===== This form consists of a word value (PRn.WORDNO) and a bit value (PRn.BITNO). The
%% ===== \end{flushleft}
%% ===== 
%% ===== 
%% ===== \begin{flushleft}
%% ===== word value is the word offset of the word containing the first character or bit of the
%% ===== \end{flushleft}
%% ===== 
%% ===== 
%% ===== \begin{flushleft}
%% ===== operand and the bit value is the bit position of that character or bit within the word. This
%% ===== \end{flushleft}
%% ===== 
%% ===== 
%% ===== 
%% ===== 
%% ===== 
%% ===== \begin{flushleft}
%% ===== \newpage
%% ===== form is seen when C(PRn) are stored as an ITS pointer pair or as a packed pointer (see
%% ===== \end{flushleft}
%% ===== 
%% ===== 
%% ===== \begin{flushleft}
%% ===== discussion of ITS pointers earlier in this section and the Store Pointer Register n Packed
%% ===== \end{flushleft}
%% ===== 
%% ===== 
%% ===== \begin{flushleft}
%% ===== (sprpn) instruction in Section 4).
%% ===== \end{flushleft}
%% ===== 
%% ===== 
%% ===== \begin{flushleft}
%% ===== 2. Address register form
%% ===== \end{flushleft}
%% ===== 
%% ===== 
%% ===== \begin{flushleft}
%% ===== This form consists of a word value (ARn.WORDNO), a byte number (ARn.CHAR), and a
%% ===== \end{flushleft}
%% ===== 
%% ===== 
%% ===== \begin{flushleft}
%% ===== bit value (ARn.BITNO). The word value is the word offset of the word containing the first
%% ===== \end{flushleft}
%% ===== 
%% ===== 
%% ===== \begin{flushleft}
%% ===== character or bit of the operand. The byte number is the number of the 9-bit byte
%% ===== \end{flushleft}
%% ===== 
%% ===== 
%% ===== \begin{flushleft}
%% ===== containing the first character or bit. The bit value is the bit position within AR n.CHAR of
%% ===== \end{flushleft}
%% ===== 
%% ===== 
%% ===== \begin{flushleft}
%% ===== the first character or bit. This form is seen when C(AR n) are stored with the Store
%% ===== \end{flushleft}
%% ===== 
%% ===== 
%% ===== \begin{flushleft}
%% ===== Address Register n (sarn) instruction (see Section 4).
%% ===== \end{flushleft}
%% ===== 
%% ===== 
%% ===== \begin{flushleft}
%% ===== 3. Operand Descriptor Form
%% ===== \end{flushleft}
%% ===== 
%% ===== 
%% ===== \begin{flushleft}
%% ===== This form is valid for character-string operands only. It consists of a word value
%% ===== \end{flushleft}
%% ===== 
%% ===== 
%% ===== \begin{flushleft}
%% ===== (ADDRESS) and a character number (CN). The word value is the word offset of the word
%% ===== \end{flushleft}
%% ===== 
%% ===== 
%% ===== \begin{flushleft}
%% ===== containing the first character of the operand and the character number is the number of
%% ===== \end{flushleft}
%% ===== 
%% ===== 
%% ===== \begin{flushleft}
%% ===== that character within the word. This form is seen when C(ARn) are stored with the
%% ===== \end{flushleft}
%% ===== 
%% ===== 
%% ===== \begin{flushleft}
%% ===== Address Register n to Alphanumeric Descriptor (aran) or Address Register n to Numeric
%% ===== \end{flushleft}
%% ===== 
%% ===== 
%% ===== \begin{flushleft}
%% ===== Descriptor (arnn) instructions. (The operand descriptor form for bit-string operands is
%% ===== \end{flushleft}
%% ===== 
%% ===== 
%% ===== \begin{flushleft}
%% ===== identical to the address register form.)
%% ===== \end{flushleft}
%% ===== 
%% ===== 
%% ===== \begin{flushleft}
%% ===== The terms {``}pointer register'' and {``}address register'' both apply to the same physical
%% ===== \end{flushleft}
%% ===== 
%% ===== 
%% ===== \begin{flushleft}
%% ===== hardware. The distinction arises from the manner in which the register is used and in the
%% ===== \end{flushleft}
%% ===== 
%% ===== 
%% ===== \begin{flushleft}
%% ===== interpretation of the register contents. {``}Pointer register'' refers to the register as used by the
%% ===== \end{flushleft}
%% ===== 
%% ===== 
%% ===== \begin{flushleft}
%% ===== appending unit and {``}address register'' refers to the register as used by the decimal unit.
%% ===== \end{flushleft}
%% ===== 
%% ===== 
%% ===== \begin{flushleft}
%% ===== The three forms are compatible and may be freely intermixed. For example, PR n may be
%% ===== \end{flushleft}
%% ===== 
%% ===== 
%% ===== \begin{flushleft}
%% ===== loaded in pointer register form with the Effective Pointer to Pointer Register n (eppn) instruction,
%% ===== \end{flushleft}
%% ===== 
%% ===== 
%% ===== \begin{flushleft}
%% ===== then modified in pointer register form with the Effective Address to Word/Bit Number of Pointer
%% ===== \end{flushleft}
%% ===== 
%% ===== 
%% ===== \begin{flushleft}
%% ===== Register n (eawpn) instruction, then further modified in address register form (assuming character
%% ===== \end{flushleft}
%% ===== 
%% ===== 
%% ===== \begin{flushleft}
%% ===== size k) with the Add k-Bit Displacement to Address Register (akbd) instruction, and finally invoked
%% ===== \end{flushleft}
%% ===== 
%% ===== 
%% ===== \begin{flushleft}
%% ===== in operand descriptor form by the use of MF.AR in an EIS multiword instruction .
%% ===== \end{flushleft}
%% ===== 
%% ===== 
%% ===== 
%% ===== 
%% ===== 
%% ===== \begin{flushleft}

\subsubsection{Character- and Bit-String Address Arithmetic Algorithms}

%% ===== \end{flushleft}
%% ===== 
%% ===== 
%% ===== \begin{flushleft}
%% ===== The arithmetic algorithms for calculating character- and bit-string addresses are presented below.
%% ===== \end{flushleft}
%% ===== 
%% ===== 
%% ===== \begin{flushleft}
%% ===== The symbols {``}ADDRESS'' and {``}CN'' represent the ADDRESS and CN fields of the operand
%% ===== \end{flushleft}
%% ===== 
%% ===== 
%% ===== \begin{flushleft}
%% ===== descriptor being decoded. {``}r'' and {``}n'' are set according to the flowchart in Figure 6-11. If either
%% ===== \end{flushleft}
%% ===== 
%% ===== 
%% ===== \begin{flushleft}
%% ===== has the value {``}null'', the contents of all associated fields are identically zero.
%% ===== \end{flushleft}
%% ===== 
%% ===== 
%% ===== 
%% ===== 
%% ===== 
%% ===== \begin{flushleft}

\subsubsubsection{9-bit Byte String Address Arithmetic}

%% ===== \end{flushleft}
%% ===== 
%% ===== 
%% ===== \begin{flushleft}
%% ===== Effective BITNO
%% ===== \end{flushleft}
%% ===== 
%% ===== 
%% ===== 
%% ===== 
%% ===== 
%% ===== = 0000
%% ===== 
%% ===== 
%% ===== 
%% ===== 
%% ===== 
%% ===== \begin{flushleft}
%% ===== Effective CHAR
%% ===== \end{flushleft}
%% ===== 
%% ===== 
%% ===== 
%% ===== 
%% ===== 
%% ===== \begin{flushleft}
%% ===== = (CN + C(ARn.CHAR) + C(r))[4]
%% ===== \end{flushleft}
%% ===== 
%% ===== 
%% ===== 
%% ===== 
%% ===== 
%% ===== \begin{flushleft}
%% ===== Effective WORDNO
%% ===== \end{flushleft}
%% ===== 
%% ===== 
%% ===== 
%% ===== 
%% ===== 
%% ===== \begin{flushleft}
%% ===== = ADDRESS + C(ARn.WORDNO) +
%% ===== \end{flushleft}
%% ===== 
%% ===== 
%% ===== \begin{flushleft}
%% ===== (CN + C(ARn.CHAR) + C(r)) / 4
%% ===== \end{flushleft}
%% ===== 
%% ===== 
%% ===== 
%% ===== 
%% ===== 
%% ===== \begin{flushleft}

\subsubsubsection{6-bit Character String Address Arithmetic}

%% ===== \end{flushleft}
%% ===== 
%% ===== 
%% ===== \begin{flushleft}
%% ===== Effective BITNO
%% ===== \end{flushleft}
%% ===== 
%% ===== 
%% ===== 
%% ===== 
%% ===== 
%% ===== \begin{flushleft}
%% ===== = (9*C(ARn.CHAR) + 6*C(r) + C(ARn.BITNO))[9]
%% ===== \end{flushleft}
%% ===== 
%% ===== 
%% ===== 
%% ===== 
%% ===== 
%% ===== \begin{flushleft}
%% ===== Effective CHAR
%% ===== \end{flushleft}
%% ===== 
%% ===== 
%% ===== 
%% ===== 
%% ===== 
%% ===== \begin{flushleft}
%% ===== = ((9*C(ARn.CHAR) + 6*C(r) + C(ARn.BITNO))[36]) / 9
%% ===== \end{flushleft}
%% ===== 
%% ===== 
%% ===== 
%% ===== 
%% ===== 
%% ===== \begin{flushleft}
%% ===== Effective WORDNO
%% ===== \end{flushleft}
%% ===== 
%% ===== 
%% ===== 
%% ===== 
%% ===== 
%% ===== \begin{flushleft}
%% ===== = ADDRESS + C(ARn.WORDNO) +
%% ===== \end{flushleft}
%% ===== 
%% ===== 
%% ===== \begin{flushleft}
%% ===== (9*C(ARn.CHAR) + 6*C(r) + C(ARn.BITNO)) / 36
%% ===== \end{flushleft}
%% ===== 
%% ===== 
%% ===== 
%% ===== 
%% ===== 
%% ===== \begin{flushleft}
%% ===== \newpage

\subsubsubsection{4-bit Byte String Address Arithmetic}

%% ===== \end{flushleft}
%% ===== 
%% ===== 
%% ===== \begin{flushleft}
%% ===== Effective BITNO
%% ===== \end{flushleft}
%% ===== 
%% ===== 
%% ===== 
%% ===== 
%% ===== 
%% ===== \begin{flushleft}
%% ===== = 4 * (C(ARn.CHAR) + 2*C(r) + C(ARn.BITNO)/4)[2] + 1
%% ===== \end{flushleft}
%% ===== 
%% ===== 
%% ===== 
%% ===== 
%% ===== 
%% ===== \begin{flushleft}
%% ===== Effective CHAR
%% ===== \end{flushleft}
%% ===== 
%% ===== 
%% ===== 
%% ===== 
%% ===== 
%% ===== \begin{flushleft}
%% ===== = ((9*C(ARn.CHAR) + 4*C(r) + C(ARn.BITNO))[36] / 9
%% ===== \end{flushleft}
%% ===== 
%% ===== 
%% ===== 
%% ===== 
%% ===== 
%% ===== \begin{flushleft}
%% ===== Effective WORDNO
%% ===== \end{flushleft}
%% ===== 
%% ===== 
%% ===== 
%% ===== 
%% ===== 
%% ===== \begin{flushleft}
%% ===== = ADDRESS + C(ARn.WORDNO) +
%% ===== \end{flushleft}
%% ===== 
%% ===== 
%% ===== \begin{flushleft}
%% ===== (9*C(ARn.CHAR) + 4*C(r) + C(ARn.BITNO)) / 36
%% ===== \end{flushleft}
%% ===== 
%% ===== 
%% ===== 
%% ===== 
%% ===== 
%% ===== \begin{flushleft}

\subsubsubsection{Bit String Address Arithmetic}

%% ===== \end{flushleft}
%% ===== 
%% ===== 
%% ===== \begin{flushleft}
%% ===== Effective BITNO
%% ===== \end{flushleft}
%% ===== 
%% ===== 
%% ===== 
%% ===== 
%% ===== 
%% ===== \begin{flushleft}
%% ===== = (9*C(ARn.CHAR) + 36*C(r) + C(ARn.BITNO))[9]
%% ===== \end{flushleft}
%% ===== 
%% ===== 
%% ===== 
%% ===== 
%% ===== 
%% ===== \begin{flushleft}
%% ===== Effective CHAR
%% ===== \end{flushleft}
%% ===== 
%% ===== 
%% ===== 
%% ===== 
%% ===== 
%% ===== \begin{flushleft}
%% ===== = ((9*C(ARn.CHAR) + 36*C(r) + C(ARn.BITNO))[36]) / 9
%% ===== \end{flushleft}
%% ===== 
%% ===== 
%% ===== 
%% ===== 
%% ===== 
%% ===== \begin{flushleft}
%% ===== Effective WORDNO
%% ===== \end{flushleft}
%% ===== 
%% ===== 
%% ===== 
%% ===== 
%% ===== 
%% ===== \begin{flushleft}
%% ===== = ADDRESS + C(ARn.WORDNO) +
%% ===== \end{flushleft}
%% ===== 
%% ===== 
%% ===== \begin{flushleft}
%% ===== (9*C(ARn.CHAR) + 36*C(r) + C(ARn.BITNO)) / 36
%% ===== \end{flushleft}
%% ===== 
%% ===== 
%% ===== 
%% ===== 
%% ===== 
%% ===== \begin{flushleft}
%% ===== \newpage
