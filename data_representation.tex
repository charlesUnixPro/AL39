\section{DATA REPRESENTATION}


\subsection{INFORMATION ORGANIZATION}


The processor, like the rest of the Multics system, is organized to deal with
information in basic units of 36-bit words. Other units of 4-, 6-, 9-bit
characters or bytes, 18-bit half words, and 72-bit word pairs can be
manipulated within the processor by use of the instruction set. These bit
groupings are used by the hardware and software to represent a variety of forms
of coded data.  Certain processor functions appear to manipulate larger units
of 144, 288, 576, and 1152 bits, but these functions are performed by means of
repeated use of 72-bit word pairs. All information is transmitted, stored, and
processed as strings of binary bits. The data values are derived when the bit
strings are interpreted according to the various formats discussed in this
section.  

\subsection{POSITION NUMBERING}

The numbering of bit positions, character and byte positions, and words
increases from 0 in the direction of conventional reading and writing: from the
most significant to the least significant digit of a number, and from left to
right in conventional alphanumeric text.

Graphic presentations in this manual show registers and data with position
numbers increasing from left to right.


\subsection{NUMBER SYSTEM}


The binary arithmetic functions of the processor are implemented in the twos
complement, binary number system. One of the primary properties of this number
system is that a field (or register) having width n bits may be interpreted in
two different ways; the logical case and the arithmetic or algebraic case.

In the logical case, the number is unsigned, positive, and lies in the range
$[0,2^n-1]$ where n is the size of the register or the length of the field. The
results of arithmetic operations on numbers for this case are interpreted as
modulo $2^n$ numbers. Overflow is not defined for this case since the range of
the
field or register cannot be exceeded. The numbers 0 and $2^n-1$ are consecutive
(not separated) in the set of numbers defined for the field or register.

In the arithmetic case, the number is signed and lies in the range
$[-2^{(n-1)},2^{(n-1)}-1]$. Overflow is defined for this case since the range
can be
exceeded in either direction (positive or negative).  The left-hand-most bit of
the field or register (bit 0) serves as the sign bit and does not contribute to
the magnitude of the number.

The main advantage of this implementation is that the hardware arithmetic
algorithms for the two cases are identical; the only distinction lying in the
interpretation of the results by the user. Instruction set features are
provided for performing binary arithmetic with overflow disabled (the so-called
logical instructions) and for comparing numbers in either sense.

Subtraction is performed by adding the twos complement of the subtrahend to the
minuend.  (Note that when the subtrahend is zero the algorithm for forming the
twos complement is still carried out, but, since the twos complement of zero is
zero, the result is correct.)

Another important feature of the twos complement number system (with respect to
comparison of numeric values) is that the no borrow condition in true
subtraction is identical to the carry condition in true addition and vice
versa.

A statement on the assumed location of the binary point has significance only
for multiplication and division. These two operations are implemented for the
arithmetic case in both integer and fraction modes. Integer means that the
position of the binary point is assumed to the right of the least significant
bit position, that is, to the right of the right-hand-most bit of the field or
register, and fraction means that the position of the binary point is assumed
to the left of the most significant bit position, that is, between bit 0 and
bit 1 of the field or register (recall that bit 0 is the sign bit).


\subsection {INFORMATION FORMATS}

The figures that follow show the unstructured formats (templates) for the
various information units defined for the processor. Data transfer between the
processor and main memory is word oriented; a 36-bit machine word is
transferred for single-precision operands and subfields of machine words, and a
72-bit word pair is transferred for all other cases (multiword operands,
instruction fetches, bit- and character-string operands, etc.). The information
unit to be used and the data transfer mode are determined by the processor
according to the function to be performed.

The 36-bit unstructured machine word shown in Figure \ref{f2.1} is the minimum
addressable information unit in main memory. Its location is uniquely
determined by its main memory address, Y. All other information units are
defined relative to the 36-bit machine word.


\begin{figure}[H]
\begin{center}
\begin{bytefield}{36}
\\
\bitheader{0,35} \\
\bitbox{36}{} \\
\bitbox[]{36}{\hfill\tiny 36}
\end{bytefield}
\caption{Unstructured Machine Word Format}
\label{f2.1}
\end{center}
\end{figure}



Two consecutive machine words as shown in Figure \ref{f2.2}, the first having
an even main memory address, form a 72-bit word pair. In 72-bit word pair data
transfer mode, the word pair is uniquely located by the main memory address of
either of its constituent 36-bit machine words.  Thus, if Y is even, the word
pair at (Y,Y+1) is selected. If Y is odd, the word pair at (Y-1,Y) is selected.
The term Y-pair is used when referring to such a word pair.  


\begin{figure}[H]
\begin{center}
\begin{bytefield}[bitwidth=0.0138\linewidth]{72}
\\
\bitheader{0,35,36,71} \\
\bitbox{36}{Even word}
\bitbox{36}{Odd word} \\
\bitbox[]{36}{\hfill\tiny 36}
\bitbox[]{36}{\hfill\tiny 36}
\end{bytefield}
\caption{Unstructured Word Pair Format}
\label{f2.2}
\end{center}
\end{figure}


Four-bit bytes are mapped onto 36-bit machine words as shown in Figure
\ref{f2.3}. The 0 bits at bit positions 0, 9, 18, and 27 are forced to be 0 by
the processor on data transfers to main memory and are ignored on data
transfers from main memory.

\begin{figure}[H]
\begin{center}
\begin{bytefield}{36}
\\
\bitheader{0,1,4,5,8,9,10,13,14,17,18,19,22,23,26,27,28,31,31,35} \\

\bitbox{1}{0}
\bitbox{4}{}
\bitbox{4}{}
\bitbox{1}{0}
\bitbox{4}{}
\bitbox{4}{}
\bitbox{1}{0}
\bitbox{4}{}
\bitbox{4}{}
\bitbox{1}{0}
\bitbox{4}{}
\bitbox{4}{} \\

\bitbox[]{1}{\hfill\tiny 1}
\bitbox[]{4}{\hfill\tiny 4}
\bitbox[]{4}{\hfill\tiny 4}
\bitbox[]{1}{\hfill\tiny 1}
\bitbox[]{4}{\hfill\tiny 4}
\bitbox[]{4}{\hfill\tiny 4}
\bitbox[]{1}{\hfill\tiny 1}
\bitbox[]{4}{\hfill\tiny 4}
\bitbox[]{4}{\hfill\tiny 4}
\bitbox[]{1}{\hfill\tiny 1}
\bitbox[]{4}{\hfill\tiny 4}
\bitbox[]{4}{\hfill\tiny 4}

\end{bytefield}
\caption{Unstructured 4-bit Byte Format}
\label{f2.3}
\end{center}
\end{figure}

Six-bit characters are mapped onto 36-bit machine words as shown in Figure \ref{f2.4}.


\begin{figure}[H]
\begin{center}
\begin{bytefield}{36}
\\
\bitheader{0,5,6,11,12,17,18,23,24,29,30,35} \\

\bitbox{6}{}
\bitbox{6}{}
\bitbox{6}{}
\bitbox{6}{}
\bitbox{6}{}
\bitbox{6}{} \\

\bitbox[]{6}{\hfill\tiny 6}
\bitbox[]{6}{\hfill\tiny 6}
\bitbox[]{6}{\hfill\tiny 6}
\bitbox[]{6}{\hfill\tiny 6}
\bitbox[]{6}{\hfill\tiny 6}
\bitbox[]{6}{\hfill\tiny 6} \\

\end{bytefield}
\caption{Unstructured 6-bit Character Format}
\label{f2.4}
\end{center}
\end{figure}


Nine-bit bytes are mapped onto 36-bit machine words as shown in Figure \ref{f2.5}.

\begin{figure}[H]
\begin{center}
\begin{bytefield}{36}
\\
\bitheader{0,8,9,17,18,26,27,35} \\

\bitbox{9}{}
\bitbox{9}{}
\bitbox{9}{}
\bitbox{9}{} \\

\bitbox[]{9}{\hfill\tiny 9}
\bitbox[]{9}{\hfill\tiny 9}
\bitbox[]{9}{\hfill\tiny 9}
\bitbox[]{9}{\hfill\tiny 9} \\

\end{bytefield}
\caption{Unstructured 9-bit Byte Format}
\label{f2.5}
\end{center}
\end{figure}

Eighteen-bit half words are mapped onto 36-bit machine words as shown in Figure
\ref{f2.6}.

\begin{figure}[H]
\begin{center}
\begin{bytefield}{36}
\\
\bitheader{0,17,18,35} \\

\bitbox{18}{}
\bitbox{18}{} \\

\bitbox[]{18}{\hfill\tiny 18}
\bitbox[]{18}{\hfill\tiny 18} \\

\end{bytefield}
\caption{Unstructured 18-bit Half Word Format}
\label{f2.6}
\end{center}
\end{figure}

\subsection{DATA PARITY}

Odd parity on each 36-bit machine word transferred to main memory is generated
as it leaves the processor, is verified at several points along the
transmission path, and is held in main memory either as an extra bit in the
case of magnetic core memory or as part of the error detecting and correcting
(EDAC) code in the case of magnetic oxide semiconductor (MOS) memory. If an
incorrect parity is detected at any of the various parity check points, the
main memory returns an illegal action signal and a code appropriate to the
check point.

On data transfers from main memory, the parity information is retrieved and
transmitted with the data information. The same verification checks are made
and illegal action signalled for errors. The processor makes a final parity
check as the data enters the processor.

Any detected parity error causes the processor parity indicator to be set ON
and (if enabled) a parity fault occurs.


\subsection{REPRESENTATION OF DATA}

Data is defined by imposing an operand structure on the information units just
described.  Data is represented in two forms: numeric or alphanumeric. The form
is determined by the processor according to the function to be performed.

In the definitions below, a\textsubscript{i} is the value of the bit in the i\textsuperscript{th} bit
position, either 0 or 1.


\subsubsection{Numeric Data}

Numeric data is represented in three modes: fixed-point binary, floating-point
binary, and decimal. The mode is determined by the processor according to the
function being performed.





\subsubsubsection{Fixed-point Binary Data}

\paragraph{Fixed-point Binary Integers}
\paragraph{}
Fixed-point binary integer data is defined by imposing either of the bit
position value expressions shown below on an information unit of n bits.


Logical value:

%\begin{equation*}
%a0\times2(n-1) + a1\times2(n-2) + ... + ai\times2(n-i-l) + ... + an-1
%\end{equation*}
\hspace{1em}a\tsb{0}$\times$2\tsp{(n-1)} + a\tsb{1}$\times$2\tsp{(n-2)} + \ldots + a\tsb{i}$\times$2\tsp{(n-i-1)} + \ldots + a\tsb{(n-1)}

Arithmetic value:

%\begin{equation*}
%-a0\times2(n-l) + a1\times2(n-2) + ... + ai\times2(n-i-l) + ... + an-l
%\end{equation*}
\hspace{1em}$-$a\tsb{0}$\times$2\tsp{(n-1)} + a\tsb{1}$\times$2\tsp{(n-2)} + \ldots + a\tsb{i}$\times$2\tsp{(n-i-1)} + \ldots + a\tsb{(n-1)}


The following fixed-point binary integer data items are defined (also see Table
\ref{t2.1} for values):

\begin{tabular}{ c l }
\textbf{\textit{Operand size (bits)}} & \textbf{\textit{Operand name}} \\
6 & 6-bit character operand \\
9 & 9-bit byte operand \\
18 & Half word operand \\
36 & Single-precision operand \\
72 & Double-precision operand \\
\end{tabular}


Note that a 4-bit operand is not defined. This data item is defined only for
decimal data.  (See discussion of decimal data later in this section).


The proper operand and its position with respect to a 36-bit machine word are
determined by the processor during preparation of the main memory address for
the operand. If the data width of the operand selected is smaller than the
register involved, the operand is high- or loworder zero filled as necessary.
The values in Table \ref{t2.1} are given in terms of the operand sizes. The value an
operand contributes to a larger field or register depends on the alignment of
the operand with respect to the field or register.  


\begin{table}[H]
\begin{center}
\caption{Fixed-Point Binary Integer Values}
\label{t2.1}
\begin{tabular}{ l c c c c c }
\\
 & & & & \textit{\textbf{36-bit}} & \textit{\textbf{72-bit}} \\

 &
\textit{\textbf{6-bit}} &
 \textit{\textbf{9-bit}} & 
\textit{\textbf{18-bit half}} & 
\textit{\textbf{single}} & 
\textit{\textbf{double}} \\

\textit{\textbf{Operand}} & 
\textit{\textbf{character}} & 
\textit{\textbf{byte}} & 
\textit{\textbf{word}} & 
\textit{\textbf{precision}} & 
\textit{\textbf{precision}} \\
Logical & & & & & \\
\hspace{1em}minimum & 0 & 0 & 0 & 0 & 0 \\
\hspace{1em}maximum & 2\tsp6-1 & 2\tsp9-1 & 2\tsp{18}-1 & 2\tsp{36}-1 & 2\tsp{72}-1 \\
\hspace{1em}resolution & 1 & 1 & 1 & 1 & 1 \\
Arithmetic & & & & & \\
\hspace{1em}minimum & 0 & 0 & 0 & 0 & 0 \\
\hspace{1em}maxima & & & & & \\
\hspace{2em}negative & -2\tsp5 & -2\tsp8 & -2\tsp{17} & -2\tsp{35} & -2\tsp{71} \\
\hspace{2em}positive & 2\tsp5-1 & 2\tsp8-1 & 2\tsp{17}-1 & 2\tsp{35}-1 & 2\tsp{71}-1 \\
\hspace{1em}resolution & 1 & 1 & 1 & 1 & 1 \\
\end{tabular}
\end{center}
\end{table}

\paragraph{Fixed-point Binary Fractions}
\paragraph{}

Fixed-point binary fraction data is defined by imposing the bit position value
expression below on an information unit of n bits.

Arithmetic value:

%\begin{equation*}
%-a0 + a1\times2-1 + a2\times2-2 + ... + ai\times2-i + ... + an-l\times2-(n-l)
%\end{equation*}
\hspace{1em} -a\tsb0 + a\tsb1$\times$2\tsp{-1} + a\tsb2$\times$2\tsp{-2} + ... + a\tsb{i}$\times$2\tsp{-i} + ... + a\tsb{n-l}$\times$2\tsp{-(n-l)}

Note that logical values are not defined for fixed-point binary fraction data.

The following fixed-point binary fraction data items are defined (also see
Table \ref{t2.2} for values):

\begin{tabular}{ c l }
\textbf{\textit{Operand size (bits)}} & \textbf{\textit{Operand name}} \\
6 & 6-bit character operand \\
9 & 9-bit byte operand \\
18 & Half word operand \\
36 & Single-precision operand \\
72 & Double-precision operand \\
\end{tabular}

Note that a 4-bit operand is not defined. This data item is defined only for
decimal data.  (See discussion of decimal data later in this section.)
Fixed-point binary fraction operands are used by the Divide Fraction (dvf) and
Multiply Fraction (mpf) instructions only.

The proper operand and its position with respect to a 36-bit machine word are
determined by the processor during preparation of the main memory address for
the operand. If the data width of the operand selected is smaller than the
register involved, the operand is high- or loworder zero filled as necessary.  

The values in Table \ref{t2.2} are given in terms of the operand sizes. The
value an operand contributes to a larger field or register depends on the
alignment of the operand with respect to the field or register.  


\begin{table}[H]
\begin{center}
\caption{Fixed-Point Binary Fraction Values}
\label{t2.2}
\begin{tabular}{ l c c c c c }
\\
 & & & & \textit{\textbf{36-bit}} & \textit{\textbf{72-bit}} \\

 &
\textit{\textbf{6-bit}} &
 \textit{\textbf{9-bit}} & 
\textit{\textbf{18-bit half}} & 
\textit{\textbf{single}} & 
\textit{\textbf{double}} \\

\textit{\textbf{Operand}} & 
\textit{\textbf{character}} & 
\textit{\textbf{byte}} & 
\textit{\textbf{word}} & 
\textit{\textbf{precision}} & 
\textit{\textbf{precision}} \\
Arithmetic & & & & & \\
\hspace{1em}minimum & 0 & 0 & 0 & 0 & 0 \\
\hspace{1em}maxima & & & & & \\
\hspace{2em}negative & $-$1.0 & $-$1.0 & $-$1.0 & $-$1.0 & $-$1.0 \\
\hspace{2em}positive & 1.0$-$2\tsp{-5} & 1.0$-$2\tsp{-8} & 1.0$-$2\tsp{-17} & 1.0$-$2\tsp{-35} & 1.0$-$2\tsp{-71} \\
\hspace{1em}resolution & 2\tsp{-5} & 2\tsp{-8} & 2\tsp{-17} & 2\tsp{-35} & 2\tsp{-71} \\
\end{tabular}
\end{center}
\end{table}

\subsubsubsection{Floating-point Binary Data}


A floating-point binary number is expressed as:


%\begin{equation*}
%Z = M \times 2^E
%\end{equation*}
\hspace{1em}Z $=$ M $\times$ 2\tsp{E}

where:

\begin{description}
\item M is a fixed-point binary fraction; the mantissa
\item E is a fixed-point binary integer; the exponent
\end{description}


A floating-point binary number is defined by partitioning an information unit
of n bits into two pieces; an 8-bit fixed-point binary integer exponent and an
(n-8)-bit fixed-point binary fraction mantissa.


The following floating-point data items are defined.


\begin{tabular}{ c l }
\textbf{\textit{Operand size (bits)}} & \textbf{\textit{Operand name}} \\
18 & Half word operand \\
36 & Single-precision operand \\
72 & Double-precision operand \\
\end{tabular}

For clarity, the formats of these operands are shown in Figure \ref{f2.7}
through Figure \ref{f2.9} In the figures, the fields labeled S hold sign bits
associated with the exponent, E, and the mantissa, M.


The floating-point binary operands are used only by the floating-point binary
arithmetic instructions (see Section \ref{s4}). The 18-bit half word operand has
meaning only when used in conjunction with the direct upper (du) address
modification (see Section \ref{s6} for a discussion of address modification).



\begin{figure}[H]
\begin{center}
\begin{bytefield}{18}
\\
\bitheader{0,1,7,8,9,17} \\

\bitbox{1}{S}
\bitbox{7}{E}
\bitbox{1}{S}
\bitbox{9}{M} \\

\bitbox[]{1}{\hfill\tiny 1}
\bitbox[]{7}{\hfill\tiny 7}
\bitbox[]{1}{\hfill\tiny 1}
\bitbox[]{9}{\hfill\tiny 9} \\

\end{bytefield}
\caption{Eighteen-bit Half Word Floating-Point Binary Operand Format}
\label{f2.7}
\end{center}
\end{figure}

\begin{figure}[H]
\begin{center}
\begin{bytefield}{36}
\\
\bitheader{0,1,7,8,9,35} \\

\bitbox{1}{S}
\bitbox{7}{E}
\bitbox{1}{S}
\bitbox{27}{M} \\

\bitbox[]{1}{\hfill\tiny 1}
\bitbox[]{7}{\hfill\tiny 7}
\bitbox[]{1}{\hfill\tiny 1}
\bitbox[]{27}{\hfill\tiny 27} \\

\end{bytefield}
\caption{Single-Precision Floating-Point Binary Operand Format}
\label{f2.8}
\end{center}
\end{figure}

\begin{figure}[H]
\begin{center}
\begin{bytefield}[bitwidth=0.0138\linewidth]{72}
\\
\bitheader{0,1,7,8,9,71} \\

\bitbox{1}{S}
\bitbox{7}{E}
\bitbox{1}{S}
\bitbox{63}{M} \\

\bitbox[]{1}{\hfill\tiny 1}
\bitbox[]{7}{\hfill\tiny 7}
\bitbox[]{1}{\hfill\tiny 1}
\bitbox[]{63}{\hfill\tiny 63} \\

\end{bytefield}
\caption{DOuble-Precision Floating-Point Binary Operand Format}
\label{f2.9}
\end{center}
\end{figure}

The proper operand is selected by the processor during preparation of the main
memory address for the operand.


\paragraph{Overlength Registers}
\paragraph{}

The AQ-register is used to hold the mantissa of all floating-point binary
numbers. The AQregister is said to be overlength with respect to the operands
since it has more bits than are provided by the operands. Operands are
low-order zero filled when loaded and low-order truncated (or rounded,
depending on the instruction) when stored. Thus, the result of all
floatingpoint instructions has more bits of precision in the AQ-register than
may be stored.

Users are cautioned that calculations involving floating-point operands may
suffer from propagation of truncation errors even if the computation algorithms
are designed to hold mantissas in the AQ-register as long as possible. It is
possible to retain full AQ-register precision of intermediate results if they
are saved with the Store AQ (\texttt{staq}) and Store Exponent (\texttt{ste})
instructions but such saved data are not usable as a floating-point operand.  




\paragraph{Normalized Numbers}
\paragraph{}


A floating-point binary number is said to be normalized if the relation
%\begin{equation*}
%-0.5 > M > -1 \text{\ or\ } 0.5 \leq M < 1 \text{\ or \ } [M=0 \text{\ and\ } E=-128]
%\end{equation*}
\begin{quotation}
\hspace{2em}$-$0.5 $>$ M $>$ -1 or 0.5 $\leq$ M < 1 or [M$=$0 and E$=$$-$128]
\end{quotation}
is satisfied. This is a result of using a 2's complement mantissa. Bits 8 and 9
are different unless the number is zero. The presence of unnormalized numbers
in any finite mantissa arithmetic can only degrade the accuracy of results. For
example, in an arithmetic allowing only two digits in the mantissa, the number
0.005$\times$10\tsp2 has the value zero instead of the value one-half.  

Normalization is a process of shifting the mantissa and adjusting the exponent
until the relation above is satisfied. Normalization may be used to recover
some or all of the extra bits of the overlength AQ-register after a
floating-point operation.  

There are cases where the limits of the registers force the use of unnormalized
numbers.  For example, in an arithmetic allowing three digits of mantissa and
one digit of exponent, the calculation 0.3$\times$10\tsp{-10} $-$ 0.1$\times$10\tsp{-11} (the
normalized case) may not be made, but 0.03$\times$10\tsp{-9} $-$ 0.001$\times$10\tsp{-9} =
0.029$\times$10\tsp{-9} (the unnormalized case) is a valid result.  

Some examples of normalized and unnormalized floating-point binary numbers are:


\begin{description}
\item Unnormalized positive binary 0.00011010 $\times$ 2\tsp7
\item Same number normalized 0.11010000 $\times$ 2\tsp4
\item Unnormalized negative binary 1.11010111 $\times$ 2\tsp{-4}
\item Same number normalized 1.01011100 $\times$ 2\tsp{-6}
\end{description}

The minimum normalized nonzero floating-point binary number is 2\tsp{-128} in
all cases. Table \ref{t2.3} gives the values for the floating-point binary
operands.  



\begin{table}[H]
\begin{center}
\caption{Floating-Point Binary Operand Values}
\label{t2.3}
\begin{tabular}{ l c c c }
\\
&
\textit{\textbf{18-bit half}} &
\textit{\textbf{36-bit single}} &
\textit{\textbf{72-bit double}} \\

\textit{\textbf{Operand}} & 
\textit{\textbf{word}} & 
\textit{\textbf{precision}} & 
\textit{\textbf{precision}} \\
Unnormalized & & & \\
\hspace{1em}minimum & 0\tsp{(a)} & 0\tsp{(a)} & 0\tsp{(a)} \\
\hspace{1em}maxima & & & \\
\hspace{2em}negative & $-$1.0$\times$2\tsp{127} & $-$1.0$\times$2\tsp{127} & $-$1.0$\times$2\tsp{127} \\
\hspace{2em}positive & (1-2\tsp{-9}$\times$2\tsp{127} & (1-2\tsp{-27}$\times$2\tsp{127} & (1-2\tsp{-63}$\times$2\tsp{127} \\
\hspace{1em}resolution & 1:9\tsp{(b)} & 1:9\tsp{(b)} & 1:9\tsp{(b)} \\
\end{tabular}
\end{center}
\vskip 1pc
(a)There is no unique representation for the value zero in floating-point
binary numbers; any number with mantissa zero has the value zero. However, the
processor treats a zero mantissa as a special case in order to preserve
precision in later calculations with a zero intermediate result. Whenever the
processor detects a zero mantissa as the result of a floating-point binary
operation, the AQ-register is cleared to zeros and the E register is set to
-128. This representation is known as a floating-point normalized zero. The
unnormalized zero (any zero mantissa) will be handled correctly if encountered
in an operand but precision may be lost. For example, A$\times$10\tsp{-14} +
0$\times$10\tsp{38} will not produce desired results since all the precision of A
will be lost when it is aligned to match the 10\tsp{38} exponent of the 0.

\vskip 1pc
(b)A value cannot be given for resolution in these cases since such a value
depends on the value of the exponent, E. The notation used, l:m, indicates
resolution to 1 bit in a field of m. Thus, the following general statement on
resolution may be made:

\vskip 1pc
\begin{description}
\item The resolution of a floating-point binary operand with mantissa length m and exponent
value E is 2i\tsp{(E-m)}.
\end{description}

\end{table}



\subsubsubsection{Decimal Data}

Decimal numbers are expressed in the following forms:


Fixed-point, no sign MMMMMM.

Fixed-point, leading sign $\pm$MMMMMM.

Fixed-point, trailing sign MMMMMM.$\pm$

Floating-point $\pm$MMMMMM.$\times$10\tsp{E}

The form is specified by control information in the operand descriptor for the
operand as used by the Extended Instruction Set (EIS) instructions (see Section
\ref{s4} for a discussion of the EIS instructions).

A decimal number is defined by imposing any of the byte position value
expressions below on a 4- or 9-bit byte information unit of length $n$ bytes.


Fixed-point, no sign:

%\begin{equation*}
%c_0\times10^{(n-1)} + c_1\times10^{(n-2)} + ... + c_{(n-1)}
%\end{equation*}
\hspace{1em}c\tsb{0} $\times$ 10\tsp{(n-1)} + c\tsb{1} $\times$ 10\tsp{(n-2)} + \ldots + c\tsb{(n-1)}


Fixed-point, leading sign:

%\begin{equation*}
%[\text{sign}=c_0] c_1\times10^{(n-2)} + c_2\times10^{(n-3)} + ... + c_{(n-1)}
%\end{equation*}
\hspace{1em}[sign=c\tsb{0}] c\tsb{1} $\times$ 10\tsp{(n-2)} + c\tsb{2} $\times$ 10\tsp{(n-3)} + \ldots + c\tsb{(n-1)}

Fixed-point, trailing sign:

%\begin{equation*}
%c_0\times10^{(n-2)} + c_1\times10{(n-3)} + ... + c_{(n-2)} [%\text{sign}=c_{(n-1)}]
%\end{equation*}
\hspace{1em}c\tsb{0} $\times$ 10\tsp{(n-2)} + c\tsb{1} $\times$ 10\tsp{(n-3)} + \ldots + c\tsb{(n-2)} [sign\tsb{(n-1)}]

Floating-point:

%\begin{equation*}
%[\text{sign}=c_0] c_1\times10_{(n-3)} + c_2\times10_{(n-4)} + ... + c_{(n-3)} [\text{exponent}=8 bits]
%\end{equation*}
\hspace{1em}[sign=c\tsb{0}] c\tsb{1} $\times$ 10\tsp{(n-3)} + c\tsb{2} $\times$ 10\tsp{(n-4)} + \ldots + c\tsb{(n-3)} [exponent=8 bits]

where:

\begin{description}

\item
c\tsb{i} is the decimal value of the byte in the i\tsp{th} byte position.

\item
$[$sign=c\tsb{i}$]$ indicates that c\tsb{i} is interpreted as a sign byte.

\item
$[$exponent=8 bits$]$ indicates that the exponent value is taken from the last 8
bits of the string. If the data is in 9-bit bytes, the exponent is bits 1-8 of
c\tsb{(n-1)}. If the data is in 4bit bytes, the exponent is the binary value of the
concatenation of c\tsb{(n-2)} and c\tsb{(n-1)}.  

\end{description}

The decimal number as described above is the only decimal data item defined. It
may begin on any legal byte boundary (without regard to word boundaries) and
has a maximum extent of 63 bytes.


The processor handles decimal data as 4-bit bytes internally. Thus, 9-bit bytes
are highorder truncated as they are transferred from main memory and high-order
filled as they are transferred to main memory. The fill pattern is {``}00011''b
for digit bytes and {``}00010''b for sign bytes. The floating-point exponent is
a special case and is treated as a fixed-point binary integer.

\cac {Changed {``}00010'' to {``}00010''b.}

The processor performs validity checking on decimal data. Only the byte values
[0,11]\tsb{8} are legal in digit positions and only the byte values $[12,17]_8$ are
legal in sign positions. Detection of an illegal byte value causes an illegal
procedure fault. The interpretation of decimal sign bytes is shown in Table
\ref{t2.4}.

\begin{table}[H]
\begin{center}
\caption{Decimal Sign Character Interpretation}
\label{t2.4}
\begin{tabular}{ c c c }
\\
\textit{\textbf{9-bit}} & \textit{\textbf{4-bit}} & \\

\textit{\textbf{bytes}} & 
\textit{\textbf{bytes}} & 
\textit{\textbf{Interpretation}} \\
52\tsb{8}          & 12\tsb{8} & + \\
53\tsb{8}\tsp{(a)} & 13\tsb{8}\tsp{(b)} & + \\
54\tsb{8}          & 14\tsb{8}\tsp{(a)} & + \\
55\tsb{8}\tsp{(a)} & 15\tsb{8}\tsp{(a)} & - \\
56\tsb{8}          & 16\tsb{8} & + \\
57\tsb{8}          & 17\tsb{8} & + \\
\end{tabular}
\end{center}


\vskip 1pc
(a)This value is used as the default sign byte for storage of results. The
presence of other values will yield correct results according to the
interpretation.

\vskip 1pc
(b)An optional control bit in the EIS decimal arithmetic instructions (see
Section 4) allows the selection of 13\tsb{8} for the plus sign byte for storage of
results in 4-bit data mode.
\end{table}


\paragraph{Decimal Data Values}
\paragraph{}

The operand descriptors for decimal data operands have a 6-bit fixed-point
binary integer field for specification of a scaling factor (SF). This scaling
factor has the same effect as the value of E in floating-point decimal
operands; a negative value moves the assumed decimal point to the left; a
positive value, to the right. The use of the scaling factor extends the range
and resolution of decimal data operands. The range of the scaling factor is
[-32,31]10. See Table 2-5 for decimal data operand values.

Table 2-5. Decimal Data Values

(a)As in floating-point binary arithmetic, there is no unique representation of
the value zero except in the case of fixed-point, unsigned data. Therefore, the
processor detects a zero result and forces a value of +0. for fixed-point,
signed data and +0.$\times$10127 for floating-point data. Again, as in
floating-point binary arithmetic, other representations of the value zero will
be handled correctly except for possible loss of precision during operand
alignment.  

(b)A value cannot be given for resolution in these cases since such a value
depends on the value of the scaling factor, SF. The notation used, 1:SF,
indicates resolution to 1 part in 10 (SF). Thus, the following general
statement on resolution may be made: 

The resolution of a fixed-point decimal operand with scaling factor SF is 10SF.


(c)A value cannot be given for resolution in these cases since such a value
depends on the value of the exponent, E. The notation used, 1:E, indicates
resolution to 1 part in 10(E). Thus, the following general statement on
resolution may be made: 

The resolution of a floating-point decimal operand with exponent E is 10(E).

The scaling factor is ignored by the hardware.


\subsubsection{Alphanumeric Data}

Alphanumeric data is represented in two modes; character-string and bit-string.
The mode is determined by the processor according to the function being
performed.


\subsubsubsection{Character String Data}

Character string data is defined by imposing the character position structure
below on a 4bit, 6-bit, or 9-bit information unit of length n bytes or
characters.  
\begin{quotation}
c\tsb{0} $||$ c\tsb{1} $||$ ... $||$ c\tsb{(n-1)}
\end{quotation}
where:
\begin{quotation}
c\tsb{i} is the character in the ith character position.

$||$ indicates the concatenation operation.
\end{quotation}


The character string described above is the only character string data item
defined. It may begin on any legal character boundary (without regard to word
boundaries) and has a maximum extent as shown in Table 2-6.





Table 2-6. Character String Data Length Limits

No interpretation of the characters is made except as specified for the instruction being executed (see Section 4).

\subsubsubsection{Bit String Data}

Bit string data is defined by imposing the bit position structure below on a
bit information unit of length n bits.
\begin{quotation}
b\tsb{0} $||$ b\tsb{1} $||$ ... $||$ b\tsb{(n-1)}
\end{quotation}
where:
\begin{quotation}
b\tsb{i} is the character in the ith character position.
$||$ indicates the concatenation operation.
\end{quotation}

The bit string described above is the only bit string data item defined. It may
begin at any bit position (without regard to character or word boundaries) and
has a maximum extent of 9437184 bits.  

