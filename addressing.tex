
\section{ADDRESSING -- SEGMENTATION AND PAGING}
\label {s5}



\subsection{ADDRESSING MODES}



The Multics processor is able to access the main memory in either absolute mode
or append mode. The processor prepares an 18-bit computed address (TPR.CA) for
each main memory reference for instructions or operands using the address
preparation algorithms described in Section 6. This computed address is a
scalar index into a virtual memory with an extent of 262,144 words.  


\subsubsection{Absolute Mode}


In absolute mode, the appending unit is bypassed for instruction fetches and
most operand fetches and the final 18-bit computed address (TPR.CA) from
address preparation becomes the absolute main memory address.


Thus, all instructions to be executed in absolute mode must reside in the
low-order 262,144 words of main memory, that is, main memory addresses 0
through 262,143. Operands normally also reside in the low-order 262,144 words
of main memory but, by specifying in an instruction word that the appending
unit be used for the main memory access, operands may reside anywhere in main
memory. An appended operand fetch may be specified by:


1. Specifying register then indirect (ri) address modification in the
instruction word and indirect to segment (its) or indirect to pointer (itp)
address modification in the indirect word.


2. Specifying pointer register modification in the instruction word (bit 29 =
1) and giving a pointer register number in the instruction address C(y)0,2.


3. Specifying pointer register modification (MFk.AR = 1) in the modification
field for an EIS operand descriptor.


The use of any of the above constructs in absolute mode places the processor in
append mode for one or more address preparation cycles. All necessary registers
must be properly loaded, all tables of segment descriptor words (SDWs) and page
table words (PTWs) expected by the appending unit must exist and be properly
described, and all fault conditions must be considered (see append mode below).


If a transfer of control is made with any of the above constructs, the
processor remains in append mode after the transfer and subsequent instruction
fetches are made in append mode.


Although no segment is defined for absolute mode, it may be helpful to
visualize a virtual, unpaged segment overlaying the first 262,144 words of main
memory.



\subsubsection{Append Mode}

In append mode, the appending unit is employed for all main memory references.
The appending unit is described later in this section.  



\subsection{SEGMENTATION}

In Multics, a segment is defined as an array of arbitrary (but limited) size of
machine words containing arbitrary data. A segment is identified within the
processor by a segment number (segno) unique to the segment.


To simplify this discussion, the operation of the hardware ring mechanism is
not described although it is an integral part of address preparation. See
Section 8 for a discussion of the ring mechanism hardware.


A virtual memory address in the processor consists of a pair of integers,
(segno,offset) The range of segno is [0,215-1] and the range of offset is
[0,218-1]. The description of the segment whose segno value is n is kept in the
nth word-pair in a table known as the descriptor segment.  The location of the
descriptor segment is held by the processor in the descriptor segment base
register (DSBR) (see Section 3). Each word-pair of a descriptor segment is
known as a segment descriptor word (SDW) and is 72 bits long (see Figure 5-5).


A bit in the SDW for a segment (SDW.U) specifies whether the segment is paged
or unpaged. The following is a simplified description of the appending process
for unpaged segments (also using an unpaged descriptor segment) (refer to
Figures 3-15 and 5-5).


1. If 2 * segno $>$= 16 * (DSBR.BND + 1), then generate an access violation,
out of segment bounds, fault.


2. Fetch the target segment SDW from DSBR.ADDR + 2 * segno.


3. If SDW.F = 0, then generate directed fault n where n is given in SDW.FC. The
value of n used here is the value assigned to define a missing segment fault
or, simply, a segment fault.


4. If offset $>$= 16 * (SDW.BOUND + 1), then generate an access violation, out
of segment bounds, fault.


5. If the access bits (SDW.R, SDW.E, etc.) of the segment are incompatible with
the reference, generate the appropriate access violation fault.


6. Generate 24-bit absolute main memory address SDW.ADDR + offset.


Figure 5-1 depicts the relationships just described.


Figure 5-1. Main Memory Address Generation for Unpaged Segments


\subsection{PAGING}

In Multics, a page is defined as a block of virtual memory with a size of 210
machine words.  The processor is designed in such a way that the page size is
adjustable over the range [2 6, 212] but no basis has been found to justify an
assertion that any page size is more efficient than 2 10 or 1024 words.


The processor divides a k-bit offset or segno value into two parts; the
high-order (k-n) bits forming a page number, x, and the low-order n bits
forming a word number, y. This may be stated as:


y = (value) modulo (page size)

x = (value - y) / (page size)


The symbols x and y are used in this context throughout this section. An example of page number formation is shown in Figure 5-2.

Figure 5-2. Page Number Formation


A bit in the SDW for a segment (SDW.U) specifies whether the segment is paged
or unpaged. A paged segment may be defined as an array of arbitrary (but
limited) size of pages and a page may be defined as an array of 1024 machine
words. Thus, x is a scalar index into the array of pages, y is a scalar index
into the page, and a reference to a word of a paged segment may be treated as a
reference to word y of page x of the segment.


Multics subdivides the virtual memory into page size blocks of 1024 words each.
Such a subdivision of space allows a segment page to be handled as a physical
block independently from the other pages of the segment and from other
segments. In main memory, the blocks are known as frames; on secondary storage,
they are known as records. When a reference to a word in a paged segment is
required (and the page containing the word is not already in main memory), a
main memory frame is allocated and the page is read in from secondary storage.
Unneeded pages need not occupy space in main memory.


The location and status of page x of a paged segment is kept in the xth word of
a table known as the page table for the segment. The words in this table are
known as page table words (PTWs) (see Figure 5-6).


Any segment may be paged as appropriate and convenient. The address field of
the segment descriptor word (SDW.ADDR) for a paged segment contains the 24-bit
absolute main memory address of the page table for the segment instead of the
address of the origin of the segment. If the descriptor segment is paged, the
address field of the descriptor segment base register (DSBR.ADDR) contains the
24-bit absolute main memory address of the page table for the descriptor
segment.


The full algorithm used by the processor to access word offset of paged segment
segno (including descriptor segment paging) is as follows. (Refer to Figures
3-15, 5-5, and 5-6.)


1. If 2 * segno $>$= 16 * (DSBR.BND + 1), then generate an access violation,
out of segment bounds, fault.


2. Form the quantities:


y1 = (2 * segno) modulo 1024

x1 = (2 * segno -- y1) / 1024

3. Fetch the descriptor segment PTW(x1) from DSBR.ADR + x1.

4. If PTW(x1).F = 0, then generate directed fault n where n is given in
PTW(x1).FC. The value of n used here is the value assigned to define a missing
page fault or, simply, a page fault.

5. Fetch the target segment SDW, SDW(segno), from the descriptor segment page
at PTW(x1).ADDR + y1.

6. If SDW(segno).F = 0, then generate directed fault n where n is given in
SDW(segno).FC.  This is a segment fault as discussed earlier in this section.

7. If offset $>$= 16 * (SDW(segno).BOUND + 1), then generate an access
violation, out of segment bounds, fault.

8. If the access bits (SDW(segno).R, SDW(segno).E, etc.) of the segment are
incompatible with the reference, generate the appropriate access violation
fault.

9. Form the quantities:

y2 = offset modulo 1024

x2 = (offset -- y2) / 1024

10.Fetch the target segment PTW(x2) from SDW(segno).ADDR + x2.


11.If PTW(x2).F = 0, then generate directed fault n where n is given in
PTW(x2).FC. This is a page fault as in Step 4 above.


12.Generate the 24-bit absolute main memory address PTW(x2).ADDR + y2.


Figure 5-3 depicts the relationships described above.


Figure 5-3. Main Memory Address Generation for Paged Segments

\subsection{CHANGING ADDRESSING MODES}

The processor is placed in absolute mode by the initialize, initialize and
clear, or system initialize functions. The first response to faults and
interrupts is in absolute mode and the mode thereafter is determined by the
instruction sequence entered through the fault or interrupt trap pair. The
processor remains in absolute mode until a transfer of control via the
appending unit takes place. Note that a Return (ret) or Restore Control Unit
(rcu) instruction that sets the absolute indicator OFF (see Section 3 for a
discussion of the indicators) or a Return Control Double (rtcd) instruction
also places the processor in append mode.  

When it responds to a fault or interrupt, the processor enters absolute mode
temporarily for the fetch and execution of the trap pair. If an unappended
transfer is executed while in the trap pair, the processor remains in absolute
mode, otherwise it returns to append mode.

\subsection{ADDRESS APPENDING}

At the completion of the formation of the virtual memory address (see Section
6) an effective segment number (segno) is in the segment number register of the
temporary pointer register (TPR.SNR) and a computed address (offset) is in the
computed address register of the temporary pointer register (TPR.CA). (See
Section 3 for a discussion of the temporary pointer register.)

\subsubsection{Address Appending Sequences}

Once segno and offset are formed in TPR.SNR and TPR.CA, respectively, the
process of generating the 24-bit absolute main memory address can involve a
number of different and distinct appending unit cycles.

The operation of the appending unit is shown in the flowchart in Figure 5-4.
This flowchart assumes that directed faults, store faults, and parity faults do
not occur.

A segment boundary check is made in every cycle except PSDW. If a boundary
violation is detected, an access violation, out of segment bounds, fault is
generated and the execution of the instruction interrupted. The occurrence of
any fault interrupts the sequence at the point of occurrence. The operating
system software should store the control unit data for possible later
continuation and attempt to resolve the fault condition.


The value of the associative memories may be seen in the flowchart by observing
the number of appending unit cycles bypassed if an SDW or PTW is found in the
associative memories.

There are nine different appending unit cycles that involve accesses to main
memory. Two of these (FANP, FAP) generate the 24-bit absolute main memory
address and initiate a main memory access for the operand, indirect word, or
instruction pair; five (NSDW, PSDW, PTW, PTW2, and DSPTW) generate a main
memory access to fetch an SDW or PTW; and two (MDSPTW and MPTW) generate a main
memory access to update page status bits (PTW.U and PTW.M) in a PTW. The cycles
are defined in Table 5-1.

Table 5-1. Appending Unit Cycle Definitions

Cycle name Function

FANP Final address nonpaged 

Generates the 24-bit absolute main memory address and initiates a main memory
access to an unpaged segment for operands, indirect words, or instructions.

FAP Final address paged

Generates the 24-bit absolute main memory address and initiates a main memory 
access to a paged segment for operands, indirect words, or instructions.

NSDW Nonpaged SDW Fetch

Fetches an SDW from an unpaged descriptor segment.

PSDW Paged SDW Fetch

Fetches an SDW from a paged descriptor segment.

PTW PTW fetch

Fetches a PTW from a page table other than a descriptor segment page table and 
sets the page accessed bit (PTW.U).

PTW2 Prepage PTW fetch

Fetches the next PTW from a page table other than a descriptor segment page
table during hardware prepaging for certain uninterruptible EIS instructions.
This cycle does not load the next PTW into the appending unit. It merely
assures that the PTW is not faulted (PTW.F = 1) and that the target page will
be in main memory when and if needed by the instruction.

DSPTW Descriptor segment PTW fetch

Fetches a PTW from a descriptor segment page table.

MDSPTW Modify DSPTW

Sets the page accessed bit (PTW.U) in the PTW for a page in a descriptor
segment page table. This cycle always immediately follows a DSPTW cycle.

MPTW Modify PTW

Sets the page modified bit (PTW.M) in the PTW for a page in other than a
descriptor segment page table.

Figure 5-4. Appending Unit Operation Flowchart

\subsection{APPENDING UNIT DATA WORD FORMATS}

\subsubsection{Segment Descriptor Word (SDW) Format}

The segment descriptor word (SDW) pair contains information that controls the
access to a segment. The SDW for segment n is located at offset 2n in the
descriptor segment whose description is currently loaded into the descriptor
segment base register (DSBR).

Even word

Odd word

Figure 5-5. Segment Descriptor Word (SDW) Format


Field Name Description

ADDR 24-bit absolute main memory address of unpaged segment (U=1) or segment
page table (U=0)

R1,R2,R3 Ring brackets (see Section 8)

F Directed fault flag

0 = execute the directed fault specified in FC

1 = the unpaged segment or segment page table is in main memory

FC The number of the directed fault (df0-df3) to be executed if F=0

BOUND 14 high-order bits of the largest 18-bit modulo 16 offset that may be
accessed without causing a descriptor violation, out of segment bounds, fault.  

R Read permission bit

E Execute permission bit (xec and xed instructions excluded)

W Write permission bit

P Privileged mode bit

0 = privileged instructions cannot be executed

1 = privileged instructions may be executed if in ring 0

U Paged/unpaged control bit

0 = segment is paged; ADDR is the 24-bit main memory address of the page table

1 = segment is unpaged; ADDR is the 24-bit main memory address of the origin of the segment

G Gate indicator bit

0 = any call into the segment must be to an offset less than the value of EB

1 = any legal segment offset may be called

C Cache control bit

0 = words (operands or instructions) from this segment may not be placed in the
cache memory

1 = words from this segment may be placed in the cache memory

EB Entry bound

Any call into this segment must be to an offset less than EB if G=0


\subsubsection{Page Table Word (PTW) Format}

The page table word (PTW) contains main memory address and status information
for a page of a paged segment.

Figure 5-6. Page Table Word (PTW) Format


Bits pictured as {``}x'' are ignored by the hardware and may be used by the
operating system software.

Field Name Description

ADDR 18-bit modulo 64 absolute main memory address of page


The hardware ignores low order bits of the main memory page address according
to page size based on the following:


Page Size in words ADDR Bits ignored

64 none

128 17

256 16-17

512 15-17

1024 14-17

2048 13-17

4096 12-17

U 1 = page has been used (referenced)

M 1 = page has been modified

F Directed fault flag

0 = page not in main memory; execute directed fault FC

1 = page is in main memory

FC directed fault number for page fault.





