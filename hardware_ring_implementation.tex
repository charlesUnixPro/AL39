
\section{HARDWARE RING IMPLEMENTATION}

The philosophy of ring protection is based on the existence of a set of
hierarchical levels of protection. This concept can be illustrated by a set of
N concentric circles, numbered 0, 1, 2, ..., N-1 from the inside out. The space
included in circle 0 is called ring 0, the space included between circle i-1
and i is called ring i. Any segment in the system is placed in one and only one
ring. The closer a segment to the center, the greater its protection and
privilege.  

When a program is executing a procedure segment placed in ring R, the program
is said to be in ring R, or that the ring of execution or current ring is ring
R. A program in ring R potentially has access to any segment located in ring R
and in outer rings. The word {``}potentially'' is used because the final
decision is subject to what access rights the user has for the target segment.
This same program in ring R has no access to any segment located in inner
rings, except to special procedures called gates.


Gates are procedures residing in a given ring and intended to provide
controlled access to the ring. A program that is in ring R can enter an inner
ring r only by calling one of the gate procedures associated with this inner
ring r. Gates must be carefully coded and must not trust any data that has been
manufactured or modified by the caller in a less privileged ring. In
particular, gates must validate all arguments passed to them by the caller so
as not to compromise the protection of any segment residing in the inner ring.

Calls from an outer ring to an inner ring are referred to as inward calls. They
are associated with an increase in the access capability of the program and are
controlled by gates.  Calls from an inner ring to an outer ring, referred to as
outward calls, are associated with a decrease in the access capability of the
program and do not need to be controlled.  

\subsection{RING PROTECTION IN MULTICS}

The ring protection designed for Multics uses the foregoing philosophy,
extended to obtain more flexibility and better efficiency.

First, the assignment of a segment to one and only one ring is inconvenient for
a class of procedure segments, such as library routines. Such procedures
operate in whatever the ring of execution the program is at the time they are
called; they need no more access than the caller.  One solution could have been
to have a copy of the library in each ring. Instead, the solution adopted by
Multics is to relax the condition that a segment can be assigned to only one
ring and allow a procedure segment to be assigned to a set of consecutive rings
defined by two integers (r1, r2), with r1 $<$= r2. If such a segment is called
from ring R such that r1 $<$= R $<$= r2, it behaves as if it were in ring R,
and executes without changing the current ring of the program. If it is called
from ring R such that R $>$ r2, it behaves likes a gate associated with ring
r2, accepting the call as an inward call and decreasing the current ring of the
program from R to r2. Upon return to the caller, the current ring is restored
to R.  

Second, the maximum ring number from which a gate can be called may be
specified. A third integer, r3, is added to the pair of integers already
associated with a segment. Any procedure segment has associated with it three
ring numbers (r1, r2, r3), called its ring brackets, such that r1 $<$= r2 $<$=
r3. If r3 $>$ r2, the procedure is a gate for ring r2, accessible from rings no
higher than r3; if r2 = r3, the procedure is not a gate. Because outward calls
are declared illegal in Multics, a segment may be called from a ring R only if
r1 $<$= R $<$= r3. Such a segment is said to have the call bracket [r1,r3].


Third, data segments may also be used in more than one ring. A segment resides
in ring r1 for write purposes but resides in a less privileged ring r2 for read
purposes. Such a segment is said to have the write bracket [0,r1] and the read
bracket [0,r2].


In summary, the operations that are potentially permitted to a program in ring
R on a segment whose ring brackets are (r1, r2, r3) are as follows:

Write if 0 $<$= R $<$= r1

Read if 0 $<$= R $<$= r2

Execute if r1 $<$= R $<$= r2 (execution in ring R)

Inward call if r2 $<$ R $<$= r3 (execution in ring r2) 

\subsection{RING PROTECTION IN THE PROCESSOR}

The processor provides hardware support for the implementation of Multics ring
protection.  A particular effort was made to minimize the overhead associated
with all authorized ring crossings, which the processor performs without
operating system intervention; and also to minimize the overhead associated
with the validation of arguments, for which the processor provides assistance.

The number of rings available in the processor is eight, numbered from 0 to 7.
The current ring R of a program is recorded in the procedure ring register
(PPR.PRR).  

The ring brackets (r1, r2, r3) of a segment are recorded in the segment
descriptor word (SDW) used by the hardware to access the segment. In addition,
the SDW contains the number of legal gate entries (SDW.CL) existing in the
segment. The hardware assumes that all gate entries are located from word 0 to
word (CL-1) and does not permit an inward call to the segment if the word
number specified in the call is greater than (CL-1). The SDW also contains the
access rights for the user on the segment. If the same segment is used by
several users, who may have different access rights to the segment, there is an
SDW describing the segment in the descriptor segment for each user.


In order to provide assistance in argument validation, any pointer being stored
into an ITS pointer pair or loaded into a pointer register also contains a ring
number. A program in ring R may write any value into the ring number field of
an ITS pointer pair; the hardware assures that, when a pointer register is
loaded from an ITS pointer pair, the ring number loaded is equal to or greater
than R, but never smaller.


During the execution of an instruction, the hardware may examine several SDWs,
ITS pointer pairs and pointer registers. For any given examination, the
hardware records the maximum of the current ring, the r1 value found in an SDW,
the ring number found in an ITS pointer pair, and the ring number found in a
pointer register. This maximum is kept in the temporary ring register (TPR.TRR)
and is updated at each such examination. The reason for having this temporary
ring number available at any point of instruction execution is that it
represents the highest ring (least privileged) that might have created or
modified any information that led the hardware to the target segment it is
about to reference. Although the current ring is R, the hardware evaluates
references as if the current ring were C(TPR.TRR), which is always equal to or
greater than R. The hardware uses C(TPR.TRR) instead of R in all comparisons
with the ring brackets involved in the enforcement of the ring protection rules
given in the previous paragraph.


The use of C(TPR.TRR) by the hardware allows gate procedures to rely on the
hardware to perform the validation of all addresses passed to the gate by the
less privileged ring. The rule enforced by the hardware regarding argument
validation can be stated as follows: Whenever an inner ring performs an
operation on a given segment and references that segment through pointers
manufactured by an outer ring, the operation is considered valid only if it
could have been performed while in the outer ring.


\subsection{APPENDING UNIT OPERATION WITH RING MECHANISM}

The complete flow chart for effective segment number generation, including the
hardware ring mechanism, is shown in Figure 8-1 below. See the description of
the access violation fault in Section 7 for the meanings of the coded faults.
The current instruction is in the instruction working buffer (IWB).


\begin{figure}[H]
\begin{tikzpicture}[node distance=2cm]
o
\node (start) [startstop] {START APPEND};
\node (wasind) [decision, below of=start] {Was the last cycle an indirect word fetch?};
\node (corner) [coordinate, right=2.5cm of wasind] {};
\node (edge) [coordinate, below=1cm of corner] {};
\node (wasrtcd) [decision, below left=0.5cm and 0.5cm of wasind] {Was is an \texttt{rtcd} operand fetch?};
\node (wasinst) [decision, below left=0.5cm and 1cm of wasrtcd] {Was it a sequential instruction fetch?};
\node (isb29) [decision, below right=0.5cm and .5cm of wasinst] {Is bit 29 ON?};
\node (neq) [process, below right=0.5cm and 0.5cm of isb29] {\textsl{n} = C(IWB)\tsb{0,2}};
\node (isring) [decision, below=0.5cm of neq] {C(PR\textsl{n}.RNR) $>$ C(PPR.PRR)?};
\node (pprring) [process, below left=0.5cm and 0.2cm of isring] {C(PPR.PRR) $\rightarrow$ C(TPR.TRR)};
\node (prnring) [process, below right=0.5cm and 0.2cm of isring] {C(PR\textsl{n}.RNR) $\rightarrow$ C(TPR.TRR)};
\node (settsr) [process, below=2cm of isring] {C(PRn .SNR) $\rightarrow$ C(TPR.TSR)};
\node (useppr) [process, below=5cm of wasinst] {C(PPR.PRR) $\rightarrow$ C(TPR.TRR)\\ C(PPR.PSR) $\rightarrow$ C(TPR.TSR)};
\node (a) [startstop, below=1cm of settsr] {A};

\draw [arrow] (start) -- (wasind);
\draw [arrow] (wasind.west) -- ++ (-1,0) node[anchor=south,pos=0.5] {No} -| (wasrtcd);
\draw [arrow] (wasind.east) -- ++ (1,0) node[anchor=south,pos=0.5] {Yes} -- (corner) |- (a);
\draw [arrow] (wasrtcd.west) -- ++ (-1,0) node[anchor=south,pos=0.5] {No} -| (wasinst);
\draw (wasrtcd.east) -- ++ (1,0) node[anchor=south,pos=0.5] {Yes} -- (wasrtcd-|corner.south);
\draw [arrow] (wasinst.east) -- ++ (1,0) node[anchor=south,pos=0.5] {No} -| (isb29);
\draw [arrow] (isb29.east) -- ++ (1,0) node[anchor=south,pos=0.5] {No} -| (neq);
\draw [arrow] (neq.south) -- (isring);
\draw [arrow] (isring.west) -- ++ (-1,0) node[anchor=south,pos=0.5] {No} -| (pprring);
\draw [arrow] (isring.east) -- ++ (1,0) node[anchor=south,pos=0.5] {Yes} -| (prnring);
\draw [arrow] (pprring.south) |-(settsr);
\draw [arrow] (prnring.south) |- (settsr);
\draw [arrow] (wasinst.south) -- ++ (0,-1) node[anchor=east,pos=0.5] {No} -- (useppr);
\draw [arrow] (isb29.west) -- ++ (-1,0) node[anchor=south,pos=0.5] {Yes} -| (useppr);
\draw [arrow] (useppr.south) |- (a);
\draw [arrow] (settsr.south) -- (a);
\end{tikzpicture}
\caption{ Complete Appending Unit Operation Flowchart}
\label{f8.1}
\end{figure}

