
\section{HARDWARE RING IMPLEMENTATION}

%% ===== \end{flushleft}
%% ===== 
%% ===== 
%% ===== \begin{flushleft}
%% ===== The philosophy of ring protection is based on the existence of a set of hierarchical levels of
%% ===== \end{flushleft}
%% ===== 
%% ===== 
%% ===== \begin{flushleft}
%% ===== protection. This concept can be illustrated by a set of N concentric circles, numbered 0, 1, 2, ...,
%% ===== \end{flushleft}
%% ===== 
%% ===== 
%% ===== \begin{flushleft}
%% ===== N-1 from the inside out. The space included in circle 0 is called ring 0, the space included between
%% ===== \end{flushleft}
%% ===== 
%% ===== 
%% ===== \begin{flushleft}
%% ===== circle i-1 and i is called ring i. Any segment in the system is placed in one and only one ring. The
%% ===== \end{flushleft}
%% ===== 
%% ===== 
%% ===== \begin{flushleft}
%% ===== closer a segment to the center, the greater its protection and privilege.
%% ===== \end{flushleft}
%% ===== 
%% ===== 
%% ===== \begin{flushleft}
%% ===== When a program is executing a procedure segment placed in ring R, the program is said to
%% ===== \end{flushleft}
%% ===== 
%% ===== 
%% ===== \begin{flushleft}
%% ===== be in ring R, or that the ring of execution or current ring is ring R. A program in ring R potentially
%% ===== \end{flushleft}
%% ===== 
%% ===== 
%% ===== \begin{flushleft}
%% ===== has access to any segment located in ring R and in outer rings. The word {``}potentially'' is used
%% ===== \end{flushleft}
%% ===== 
%% ===== 
%% ===== \begin{flushleft}
%% ===== because the final decision is subject to what access rights the user has for the target segment.
%% ===== \end{flushleft}
%% ===== 
%% ===== 
%% ===== \begin{flushleft}
%% ===== This same program in ring R has no access to any segment located in inner rings, except to special
%% ===== \end{flushleft}
%% ===== 
%% ===== 
%% ===== \begin{flushleft}
%% ===== procedures called gates.
%% ===== \end{flushleft}
%% ===== 
%% ===== 
%% ===== \begin{flushleft}
%% ===== Gates are procedures residing in a given ring and intended to provide controlled access to
%% ===== \end{flushleft}
%% ===== 
%% ===== 
%% ===== \begin{flushleft}
%% ===== the ring. A program that is in ring R can enter an inner ring r only by calling one of the gate
%% ===== \end{flushleft}
%% ===== 
%% ===== 
%% ===== \begin{flushleft}
%% ===== procedures associated with this inner ring r. Gates must be carefully coded and must not trust any
%% ===== \end{flushleft}
%% ===== 
%% ===== 
%% ===== \begin{flushleft}
%% ===== data that has been manufactured or modified by the caller in a less privileged ring. In particular,
%% ===== \end{flushleft}
%% ===== 
%% ===== 
%% ===== \begin{flushleft}
%% ===== gates must validate all arguments passed to them by the caller so as not to compromise the
%% ===== \end{flushleft}
%% ===== 
%% ===== 
%% ===== \begin{flushleft}
%% ===== protection of any segment residing in the inner ring.
%% ===== \end{flushleft}
%% ===== 
%% ===== 
%% ===== \begin{flushleft}
%% ===== Calls from an outer ring to an inner ring are referred to as inward calls. They are
%% ===== \end{flushleft}
%% ===== 
%% ===== 
%% ===== \begin{flushleft}
%% ===== associated with an increase in the access capability of the program and are controlled by gates.
%% ===== \end{flushleft}
%% ===== 
%% ===== 
%% ===== \begin{flushleft}
%% ===== Calls from an inner ring to an outer ring, referred to as outward calls, are associated with a
%% ===== \end{flushleft}
%% ===== 
%% ===== 
%% ===== \begin{flushleft}
%% ===== decrease in the access capability of the program and do not need to be controlled.
%% ===== \end{flushleft}
%% ===== 
%% ===== 
%% ===== 
%% ===== 
%% ===== 
%% ===== \begin{flushleft}

\subsection{RING PROTECTION IN MULTICS}

%% ===== \end{flushleft}
%% ===== 
%% ===== 
%% ===== \begin{flushleft}
%% ===== The ring protection designed for Multics uses the foregoing philosophy, extended to obtain
%% ===== \end{flushleft}
%% ===== 
%% ===== 
%% ===== \begin{flushleft}
%% ===== more flexibility and better efficiency.
%% ===== \end{flushleft}
%% ===== 
%% ===== 
%% ===== \begin{flushleft}
%% ===== First, the assignment of a segment to one and only one ring is inconvenient for a class of
%% ===== \end{flushleft}
%% ===== 
%% ===== 
%% ===== \begin{flushleft}
%% ===== procedure segments, such as library routines. Such procedures operate in whatever the ring of
%% ===== \end{flushleft}
%% ===== 
%% ===== 
%% ===== \begin{flushleft}
%% ===== execution the program is at the time they are called; they need no more access than the caller.
%% ===== \end{flushleft}
%% ===== 
%% ===== 
%% ===== \begin{flushleft}
%% ===== One solution could have been to have a copy of the library in each ring. Instead, the solution
%% ===== \end{flushleft}
%% ===== 
%% ===== 
%% ===== \begin{flushleft}
%% ===== adopted by Multics is to relax the condition that a segment can be assigned to only one ring and
%% ===== \end{flushleft}
%% ===== 
%% ===== 
%% ===== \begin{flushleft}
%% ===== allow a procedure segment to be assigned to a set of consecutive rings defined by two integers (r1,
%% ===== \end{flushleft}
%% ===== 
%% ===== 
%% ===== \begin{flushleft}
%% ===== r2), with r1 $<$= r2. If such a segment is called from ring R such that r1 $<$= R $<$= r2, it behaves as
%% ===== \end{flushleft}
%% ===== 
%% ===== 
%% ===== \begin{flushleft}
%% ===== if it were in ring R, and executes without changing the current ring of the program. If it is called
%% ===== \end{flushleft}
%% ===== 
%% ===== 
%% ===== \begin{flushleft}
%% ===== from ring R such that R $>$ r2, it behaves likes a gate associated with ring r2, accepting the call as
%% ===== \end{flushleft}
%% ===== 
%% ===== 
%% ===== \begin{flushleft}
%% ===== an inward call and decreasing the current ring of the program from R to r2. Upon return to the
%% ===== \end{flushleft}
%% ===== 
%% ===== 
%% ===== \begin{flushleft}
%% ===== caller, the current ring is restored to R.
%% ===== \end{flushleft}
%% ===== 
%% ===== 
%% ===== \begin{flushleft}
%% ===== Second, the maximum ring number from which a gate can be called may be specified. A
%% ===== \end{flushleft}
%% ===== 
%% ===== 
%% ===== \begin{flushleft}
%% ===== third integer, r3, is added to the pair of integers already associated with a segment. Any
%% ===== \end{flushleft}
%% ===== 
%% ===== 
%% ===== \begin{flushleft}
%% ===== procedure segment has associated with it three ring numbers (r1, r2, r3), called its ring brackets,
%% ===== \end{flushleft}
%% ===== 
%% ===== 
%% ===== \begin{flushleft}
%% ===== such that r1 $<$= r2 $<$= r3. If r3 $>$ r2, the procedure is a gate for ring r2, accessible from rings no
%% ===== \end{flushleft}
%% ===== 
%% ===== 
%% ===== \begin{flushleft}
%% ===== higher than r3; if r2 = r3, the procedure is not a gate. Because outward calls are declared illegal
%% ===== \end{flushleft}
%% ===== 
%% ===== 
%% ===== \begin{flushleft}
%% ===== in Multics, a segment may be called from a ring R only if r1 $<$= R $<$= r3. Such a segment is said to
%% ===== \end{flushleft}
%% ===== 
%% ===== 
%% ===== \begin{flushleft}
%% ===== have the call bracket [r1,r3].
%% ===== \end{flushleft}
%% ===== 
%% ===== 
%% ===== \begin{flushleft}
%% ===== Third, data segments may also be used in more than one ring. A segment resides in ring r1
%% ===== \end{flushleft}
%% ===== 
%% ===== 
%% ===== \begin{flushleft}
%% ===== for write purposes but resides in a less privileged ring r2 for read purposes. Such a segment is
%% ===== \end{flushleft}
%% ===== 
%% ===== 
%% ===== \begin{flushleft}
%% ===== said to have the write bracket [0,r1] and the read bracket [0,r2].
%% ===== \end{flushleft}
%% ===== 
%% ===== 
%% ===== \begin{flushleft}
%% ===== In summary, the operations that are potentially permitted to a program in ring R on a
%% ===== \end{flushleft}
%% ===== 
%% ===== 
%% ===== \begin{flushleft}
%% ===== segment whose ring brackets are (r1, r2, r3) are as follows:
%% ===== \end{flushleft}
%% ===== 
%% ===== 
%% ===== 
%% ===== 
%% ===== 
%% ===== \begin{flushleft}
%% ===== \newpage
%% ===== Write
%% ===== \end{flushleft}
%% ===== 
%% ===== 
%% ===== 
%% ===== 
%% ===== 
%% ===== \begin{flushleft}
%% ===== if 0 $<$= R $<$= r1
%% ===== \end{flushleft}
%% ===== 
%% ===== 
%% ===== 
%% ===== 
%% ===== 
%% ===== \begin{flushleft}
%% ===== Read
%% ===== \end{flushleft}
%% ===== 
%% ===== 
%% ===== 
%% ===== 
%% ===== 
%% ===== \begin{flushleft}
%% ===== if 0 $<$= R $<$= r2
%% ===== \end{flushleft}
%% ===== 
%% ===== 
%% ===== 
%% ===== 
%% ===== 
%% ===== \begin{flushleft}
%% ===== Execute
%% ===== \end{flushleft}
%% ===== 
%% ===== 
%% ===== 
%% ===== 
%% ===== 
%% ===== \begin{flushleft}
%% ===== if r1 $<$= R $<$= r2 (execution in ring R)
%% ===== \end{flushleft}
%% ===== 
%% ===== 
%% ===== 
%% ===== 
%% ===== 
%% ===== \begin{flushleft}
%% ===== Inward call
%% ===== \end{flushleft}
%% ===== 
%% ===== 
%% ===== 
%% ===== 
%% ===== 
%% ===== \begin{flushleft}
%% ===== if r2 $<$ R $<$= r3 (execution in ring r2)
%% ===== \end{flushleft}
%% ===== 
%% ===== 
%% ===== 
%% ===== 
%% ===== 
%% ===== \begin{flushleft}

\subsection{RING PROTECTION IN THE PROCESSOR}

%% ===== \end{flushleft}
%% ===== 
%% ===== 
%% ===== \begin{flushleft}
%% ===== The processor provides hardware support for the implementation of Multics ring protection.
%% ===== \end{flushleft}
%% ===== 
%% ===== 
%% ===== \begin{flushleft}
%% ===== A particular effort was made to minimize the overhead associated with all authorized ring
%% ===== \end{flushleft}
%% ===== 
%% ===== 
%% ===== \begin{flushleft}
%% ===== crossings, which the processor performs without operating system intervention; and also to
%% ===== \end{flushleft}
%% ===== 
%% ===== 
%% ===== \begin{flushleft}
%% ===== minimize the overhead associated with the validation of arguments, for which the processor
%% ===== \end{flushleft}
%% ===== 
%% ===== 
%% ===== \begin{flushleft}
%% ===== provides assistance.
%% ===== \end{flushleft}
%% ===== 
%% ===== 
%% ===== \begin{flushleft}
%% ===== The number of rings available in the processor is eight, numbered from 0 to 7. The current
%% ===== \end{flushleft}
%% ===== 
%% ===== 
%% ===== \begin{flushleft}
%% ===== ring R of a program is recorded in the procedure ring register (PPR.PRR).
%% ===== \end{flushleft}
%% ===== 
%% ===== 
%% ===== \begin{flushleft}
%% ===== The ring brackets (r1, r2, r3) of a segment are recorded in the segment descriptor word
%% ===== \end{flushleft}
%% ===== 
%% ===== 
%% ===== \begin{flushleft}
%% ===== (SDW) used by the hardware to access the segment. In addition, the SDW contains the number of
%% ===== \end{flushleft}
%% ===== 
%% ===== 
%% ===== \begin{flushleft}
%% ===== legal gate entries (SDW.CL) existing in the segment. The hardware assumes that all gate entries
%% ===== \end{flushleft}
%% ===== 
%% ===== 
%% ===== \begin{flushleft}
%% ===== are located from word 0 to word (CL-1) and does not permit an inward call to the segment if the
%% ===== \end{flushleft}
%% ===== 
%% ===== 
%% ===== \begin{flushleft}
%% ===== word number specified in the call is greater than (CL-1). The SDW also contains the access rights
%% ===== \end{flushleft}
%% ===== 
%% ===== 
%% ===== \begin{flushleft}
%% ===== for the user on the segment. If the same segment is used by several users, who may have different
%% ===== \end{flushleft}
%% ===== 
%% ===== 
%% ===== \begin{flushleft}
%% ===== access rights to the segment, there is an SDW describing the segment in the descriptor segment
%% ===== \end{flushleft}
%% ===== 
%% ===== 
%% ===== \begin{flushleft}
%% ===== for each user.
%% ===== \end{flushleft}
%% ===== 
%% ===== 
%% ===== \begin{flushleft}
%% ===== In order to provide assistance in argument validation, any pointer being stored into an ITS
%% ===== \end{flushleft}
%% ===== 
%% ===== 
%% ===== \begin{flushleft}
%% ===== pointer pair or loaded into a pointer register also contains a ring number. A program in ring R
%% ===== \end{flushleft}
%% ===== 
%% ===== 
%% ===== \begin{flushleft}
%% ===== may write any value into the ring number field of an ITS pointer pair; the hardware assures that,
%% ===== \end{flushleft}
%% ===== 
%% ===== 
%% ===== \begin{flushleft}
%% ===== when a pointer register is loaded from an ITS pointer pair, the ring number loaded is equal to or
%% ===== \end{flushleft}
%% ===== 
%% ===== 
%% ===== \begin{flushleft}
%% ===== greater than R, but never smaller.
%% ===== \end{flushleft}
%% ===== 
%% ===== 
%% ===== \begin{flushleft}
%% ===== During the execution of an instruction, the hardware may examine several SDWs, ITS
%% ===== \end{flushleft}
%% ===== 
%% ===== 
%% ===== \begin{flushleft}
%% ===== pointer pairs and pointer registers. For any given examination, the hardware records the
%% ===== \end{flushleft}
%% ===== 
%% ===== 
%% ===== \begin{flushleft}
%% ===== maximum of the current ring, the r1 value found in an SDW, the ring number found in an ITS
%% ===== \end{flushleft}
%% ===== 
%% ===== 
%% ===== \begin{flushleft}
%% ===== pointer pair, and the ring number found in a pointer register. This maximum is kept in the
%% ===== \end{flushleft}
%% ===== 
%% ===== 
%% ===== \begin{flushleft}
%% ===== temporary ring register (TPR.TRR) and is updated at each such examination. The reason for
%% ===== \end{flushleft}
%% ===== 
%% ===== 
%% ===== \begin{flushleft}
%% ===== having this temporary ring number available at any point of instruction execution is that it
%% ===== \end{flushleft}
%% ===== 
%% ===== 
%% ===== \begin{flushleft}
%% ===== represents the highest ring (least privileged) that might have created or modified any information
%% ===== \end{flushleft}
%% ===== 
%% ===== 
%% ===== \begin{flushleft}
%% ===== that led the hardware to the target segment it is about to reference. Although the current ring is
%% ===== \end{flushleft}
%% ===== 
%% ===== 
%% ===== \begin{flushleft}
%% ===== R, the hardware evaluates references as if the current ring were C(TPR.TRR), which is always
%% ===== \end{flushleft}
%% ===== 
%% ===== 
%% ===== \begin{flushleft}
%% ===== equal to or greater than R. The hardware uses C(TPR.TRR) instead of R in all comparisons with
%% ===== \end{flushleft}
%% ===== 
%% ===== 
%% ===== \begin{flushleft}
%% ===== the ring brackets involved in the enforcement of the ring protection rules given in the previous
%% ===== \end{flushleft}
%% ===== 
%% ===== 
%% ===== \begin{flushleft}
%% ===== paragraph.
%% ===== \end{flushleft}
%% ===== 
%% ===== 
%% ===== \begin{flushleft}
%% ===== The use of C(TPR.TRR) by the hardware allows gate procedures to rely on the hardware to
%% ===== \end{flushleft}
%% ===== 
%% ===== 
%% ===== \begin{flushleft}
%% ===== perform the validation of all addresses passed to the gate by the less privileged ring. The rule
%% ===== \end{flushleft}
%% ===== 
%% ===== 
%% ===== \begin{flushleft}
%% ===== enforced by the hardware regarding argument validation can be stated as follows:
%% ===== \end{flushleft}
%% ===== 
%% ===== 
%% ===== \begin{flushleft}
%% ===== Whenever an inner ring performs an operation on a given segment and references that
%% ===== \end{flushleft}
%% ===== 
%% ===== 
%% ===== \begin{flushleft}
%% ===== segment through pointers manufactured by an outer ring, the operation is considered valid
%% ===== \end{flushleft}
%% ===== 
%% ===== 
%% ===== \begin{flushleft}
%% ===== only if it could have been performed while in the outer ring.
%% ===== \end{flushleft}
%% ===== 
%% ===== 
%% ===== 
%% ===== 
%% ===== 
%% ===== \begin{flushleft}

\subsection{APPENDING UNIT OPERATION WITH RING MECHANISM}

%% ===== \end{flushleft}
%% ===== 
%% ===== 
%% ===== \begin{flushleft}
%% ===== The complete flow chart for effective segment number generation, including the hardware
%% ===== \end{flushleft}
%% ===== 
%% ===== 
%% ===== \begin{flushleft}
%% ===== ring mechanism, is shown in Figure 8-1 below. See the description of the access violation fault in
%% ===== \end{flushleft}
%% ===== 
%% ===== 
%% ===== 
%% ===== 
%% ===== 
%% ===== \begin{flushleft}
%% ===== \newpage
%% ===== Section 7 for the meanings of the coded faults.
%% ===== \end{flushleft}
%% ===== 
%% ===== 
%% ===== \begin{flushleft}
%% ===== working buffer (IWB).
%% ===== \end{flushleft}
%% ===== 
%% ===== 
%% ===== 
%% ===== 
%% ===== 
%% ===== \begin{flushleft}
%% ===== The current instruction is in the instruction
%% ===== \end{flushleft}
%% ===== 
%% ===== 
%% ===== 
%% ===== 
%% ===== 
%% ===== \begin{flushleft}



\begin{figure}[H]
\begin{tikzpicture}[node distance=2cm]
o
\node (start) [startstop] {START APPEND};
\node (wasind) [decision, below of=start] {Was the last cycle an indirect word fetch?};
\node (corner) [coordinate, right=2.5cm of wasind] {};
\node (edge) [coordinate, below=1cm of corner] {};
\node (wasrtcd) [decision, below left=0.5cm and 0.5cm of wasind] {Was is an \texttt{rtcd} operand fetch?};
\node (wasinst) [decision, below left=0.5cm and 1cm of wasrtcd] {Was it a sequential instruction fetch?};
\node (isb29) [decision, below right=0.5cm and .5cm of wasinst] {Is bit 29 ON?};
\node (neq) [process, below right=0.5cm and 0.5cm of isb29] {\textsl{n} = C(IWB)\tsb{0,2}};
\node (isring) [decision, below=0.5cm of neq] {C(PR\textsl{n}.RNR) $>$ C(PPR.PRR)?};
\node (pprring) [process, below left=0.5cm and 0.2cm of isring] {C(PPR.PRR) $\rightarrow$ C(TPR.TRR)};
\node (prnring) [process, below right=0.5cm and 0.2cm of isring] {C(PR\textsl{n}.RNR) $\rightarrow$ C(TPR.TRR)};
\node (settsr) [process, below=2cm of isring] {C(PRn .SNR) $\rightarrow$ C(TPR.TSR)};
\node (useppr) [process, below=5cm of wasinst] {C(PPR.PRR) $\rightarrow$ C(TPR.TRR)\\ C(PPR.PSR) $\rightarrow$ C(TPR.TSR)};
\node (a) [startstop, below=1cm of settsr] {A};

\draw [arrow] (start) -- (wasind);
\draw [arrow] (wasind.west) -- ++ (-1,0) node[anchor=south,pos=0.5] {No} -| (wasrtcd);
\draw [arrow] (wasind.east) -- ++ (1,0) node[anchor=south,pos=0.5] {Yes} -- (corner) |- (a);
\draw [arrow] (wasrtcd.west) -- ++ (-1,0) node[anchor=south,pos=0.5] {No} -| (wasinst);
\draw (wasrtcd.east) -- ++ (1,0) node[anchor=south,pos=0.5] {Yes} -- (wasrtcd-|corner.south);
\draw [arrow] (wasinst.east) -- ++ (1,0) node[anchor=south,pos=0.5] {No} -| (isb29);
\draw [arrow] (isb29.east) -- ++ (1,0) node[anchor=south,pos=0.5] {No} -| (neq);
\draw [arrow] (neq.south) -- (isring);
\draw [arrow] (isring.west) -- ++ (-1,0) node[anchor=south,pos=0.5] {No} -| (pprring);
\draw [arrow] (isring.east) -- ++ (1,0) node[anchor=south,pos=0.5] {Yes} -| (prnring);
\draw [arrow] (pprring.south) |-(settsr);
\draw [arrow] (prnring.south) |- (settsr);
\draw [arrow] (wasinst.south) -- ++ (0,-1) node[anchor=east,pos=0.5] {No} -- (useppr);
\draw [arrow] (isb29.west) -- ++ (-1,0) node[anchor=south,pos=0.5] {Yes} -| (useppr);
\draw [arrow] (useppr.south) |- (a);
\draw [arrow] (settsr.south) -- (a);
\end{tikzpicture}
\caption{ Complete Appending Unit Operation Flowchart}
\label{f8.1}
\end{figure}

%% ===== START APPEND
%% ===== \end{flushleft}
%% ===== 
%% ===== 
%% ===== 
%% ===== 
%% ===== 
%% ===== \begin{flushleft}
%% ===== No
%% ===== \end{flushleft}
%% ===== 
%% ===== 
%% ===== \begin{flushleft}
%% ===== No
%% ===== \end{flushleft}
%% ===== 
%% ===== 
%% ===== \begin{flushleft}
%% ===== Was it a
%% ===== \end{flushleft}
%% ===== 
%% ===== 
%% ===== \begin{flushleft}
%% ===== No
%% ===== \end{flushleft}
%% ===== 
%% ===== 
%% ===== \begin{flushleft}
%% ===== sequential instruction
%% ===== \end{flushleft}
%% ===== 
%% ===== 
%% ===== \begin{flushleft}
%% ===== fetch?
%% ===== \end{flushleft}
%% ===== 
%% ===== 
%% ===== \begin{flushleft}
%% ===== Yes
%% ===== \end{flushleft}
%% ===== 
%% ===== 
%% ===== 
%% ===== 
%% ===== 
%% ===== \begin{flushleft}
%% ===== Was it an
%% ===== \end{flushleft}
%% ===== 
%% ===== 
%% ===== \begin{flushleft}
%% ===== rtcd operand
%% ===== \end{flushleft}
%% ===== 
%% ===== 
%% ===== \begin{flushleft}
%% ===== fetch?
%% ===== \end{flushleft}
%% ===== 
%% ===== 
%% ===== 
%% ===== 
%% ===== 
%% ===== \begin{flushleft}
%% ===== Is bit
%% ===== \end{flushleft}
%% ===== 
%% ===== 
%% ===== \begin{flushleft}
%% ===== 29 ON?
%% ===== \end{flushleft}
%% ===== 
%% ===== 
%% ===== 
%% ===== 
%% ===== 
%% ===== \begin{flushleft}
%% ===== Was the last
%% ===== \end{flushleft}
%% ===== 
%% ===== 
%% ===== \begin{flushleft}
%% ===== Yes
%% ===== \end{flushleft}
%% ===== 
%% ===== 
%% ===== \begin{flushleft}
%% ===== cycle an indirect
%% ===== \end{flushleft}
%% ===== 
%% ===== 
%% ===== \begin{flushleft}
%% ===== word fetch?
%% ===== \end{flushleft}
%% ===== 
%% ===== 
%% ===== 
%% ===== 
%% ===== 
%% ===== \begin{flushleft}
%% ===== Yes
%% ===== \end{flushleft}
%% ===== 
%% ===== 
%% ===== 
%% ===== 
%% ===== 
%% ===== \begin{flushleft}
%% ===== Yes
%% ===== \end{flushleft}
%% ===== 
%% ===== 
%% ===== 
%% ===== 
%% ===== 
%% ===== \begin{flushleft}
%% ===== n = C(IWB)0,2
%% ===== \end{flushleft}
%% ===== 
%% ===== 
%% ===== 
%% ===== 
%% ===== 
%% ===== \begin{flushleft}
%% ===== No
%% ===== \end{flushleft}
%% ===== 
%% ===== 
%% ===== 
%% ===== 
%% ===== 
%% ===== \begin{flushleft}
%% ===== No
%% ===== \end{flushleft}
%% ===== 
%% ===== 
%% ===== 
%% ===== 
%% ===== 
%% ===== \begin{flushleft}
%% ===== C(PRn .RNR) $>$
%% ===== \end{flushleft}
%% ===== 
%% ===== 
%% ===== \begin{flushleft}
%% ===== C(PPR.PRR)?
%% ===== \end{flushleft}
%% ===== 
%% ===== 
%% ===== 
%% ===== 
%% ===== 
%% ===== \begin{flushleft}
%% ===== C(PRn .RNR) $\rightarrow$ C(TPR.TRR)
%% ===== \end{flushleft}
%% ===== 
%% ===== 
%% ===== 
%% ===== 
%% ===== 
%% ===== \begin{flushleft}
%% ===== C(PPR.PRR) $\rightarrow$ C(TPR.TRR)
%% ===== \end{flushleft}
%% ===== 
%% ===== 
%% ===== 
%% ===== 
%% ===== 
%% ===== \begin{flushleft}
%% ===== C(PPR.PRR) $\rightarrow$ C(TPR.TRR)
%% ===== \end{flushleft}
%% ===== 
%% ===== 
%% ===== \begin{flushleft}
%% ===== C(PPR.PSR) $\rightarrow$ C(TPR.TSR)
%% ===== \end{flushleft}
%% ===== 
%% ===== 
%% ===== 
%% ===== 
%% ===== 
%% ===== \begin{flushleft}
%% ===== Yes
%% ===== \end{flushleft}
%% ===== 
%% ===== 
%% ===== 
%% ===== 
%% ===== 
%% ===== \begin{flushleft}
%% ===== C(PRn .SNR) $\rightarrow$ C(TPR.TSR)
%% ===== \end{flushleft}
%% ===== 
%% ===== 
%% ===== 
%% ===== 
%% ===== 
%% ===== \begin{flushleft}
%% ===== A
%% ===== \end{flushleft}
%% ===== 
%% ===== 
%% ===== 
%% ===== 
%% ===== 
%% ===== \begin{flushleft}
%% ===== Figure 8-1. Complete Appending Unit Operation Flowchart
%% ===== \end{flushleft}
%% ===== 
%% ===== 
%% ===== 
%% ===== 
%% ===== 
%% ===== \begin{flushleft}
%% ===== \newpage
%% ===== A
%% ===== \end{flushleft}
%% ===== 
%% ===== 
%% ===== 
%% ===== 
%% ===== 
%% ===== \begin{flushleft}
%% ===== is SDW for
%% ===== \end{flushleft}
%% ===== 
%% ===== 
%% ===== \begin{flushleft}
%% ===== C(TPR.TSR)
%% ===== \end{flushleft}
%% ===== 
%% ===== 
%% ===== \begin{flushleft}
%% ===== in SDWAM?
%% ===== \end{flushleft}
%% ===== 
%% ===== 
%% ===== \begin{flushleft}
%% ===== Yes
%% ===== \end{flushleft}
%% ===== 
%% ===== 
%% ===== 
%% ===== 
%% ===== 
%% ===== \begin{flushleft}
%% ===== No
%% ===== \end{flushleft}
%% ===== 
%% ===== 
%% ===== \begin{flushleft}
%% ===== No
%% ===== \end{flushleft}
%% ===== 
%% ===== 
%% ===== 
%% ===== 
%% ===== 
%% ===== \begin{flushleft}
%% ===== Yes
%% ===== \end{flushleft}
%% ===== 
%% ===== 
%% ===== 
%% ===== 
%% ===== 
%% ===== \begin{flushleft}
%% ===== DSBR.U
%% ===== \end{flushleft}
%% ===== 
%% ===== 
%% ===== =0
%% ===== 
%% ===== 
%% ===== 
%% ===== 
%% ===== 
%% ===== \begin{flushleft}
%% ===== DSPTW
%% ===== \end{flushleft}
%% ===== 
%% ===== 
%% ===== \begin{flushleft}
%% ===== cycle
%% ===== \end{flushleft}
%% ===== 
%% ===== 
%% ===== \begin{flushleft}
%% ===== Yes
%% ===== \end{flushleft}
%% ===== 
%% ===== 
%% ===== \begin{flushleft}
%% ===== DSPTW.U
%% ===== \end{flushleft}
%% ===== 
%% ===== 
%% ===== \begin{flushleft}
%% ===== set ON?
%% ===== \end{flushleft}
%% ===== 
%% ===== 
%% ===== 
%% ===== 
%% ===== 
%% ===== \begin{flushleft}
%% ===== No
%% ===== \end{flushleft}
%% ===== 
%% ===== 
%% ===== 
%% ===== 
%% ===== 
%% ===== \begin{flushleft}
%% ===== No
%% ===== \end{flushleft}
%% ===== 
%% ===== 
%% ===== 
%% ===== 
%% ===== 
%% ===== \begin{flushleft}
%% ===== Initiate a
%% ===== \end{flushleft}
%% ===== 
%% ===== 
%% ===== \begin{flushleft}
%% ===== directed fault
%% ===== \end{flushleft}
%% ===== 
%% ===== 
%% ===== \begin{flushleft}
%% ===== MDSPTW
%% ===== \end{flushleft}
%% ===== 
%% ===== 
%% ===== \begin{flushleft}
%% ===== cycle
%% ===== \end{flushleft}
%% ===== 
%% ===== 
%% ===== 
%% ===== 
%% ===== 
%% ===== \begin{flushleft}
%% ===== Yes
%% ===== \end{flushleft}
%% ===== 
%% ===== 
%% ===== 
%% ===== 
%% ===== 
%% ===== \begin{flushleft}
%% ===== NSDW
%% ===== \end{flushleft}
%% ===== 
%% ===== 
%% ===== \begin{flushleft}
%% ===== cycle
%% ===== \end{flushleft}
%% ===== 
%% ===== 
%% ===== 
%% ===== 
%% ===== 
%% ===== \begin{flushleft}
%% ===== DSPTW.F
%% ===== \end{flushleft}
%% ===== 
%% ===== 
%% ===== \begin{flushleft}
%% ===== set ON?
%% ===== \end{flushleft}
%% ===== 
%% ===== 
%% ===== 
%% ===== 
%% ===== 
%% ===== \begin{flushleft}
%% ===== PSDW
%% ===== \end{flushleft}
%% ===== 
%% ===== 
%% ===== \begin{flushleft}
%% ===== cycle
%% ===== \end{flushleft}
%% ===== 
%% ===== 
%% ===== 
%% ===== 
%% ===== 
%% ===== \begin{flushleft}
%% ===== Yes
%% ===== \end{flushleft}
%% ===== 
%% ===== 
%% ===== 
%% ===== 
%% ===== 
%% ===== \begin{flushleft}
%% ===== Load SDWAM
%% ===== \end{flushleft}
%% ===== 
%% ===== 
%% ===== 
%% ===== 
%% ===== 
%% ===== \begin{flushleft}
%% ===== SDW.F
%% ===== \end{flushleft}
%% ===== 
%% ===== 
%% ===== \begin{flushleft}
%% ===== set ON?
%% ===== \end{flushleft}
%% ===== 
%% ===== 
%% ===== 
%% ===== 
%% ===== 
%% ===== \begin{flushleft}
%% ===== No
%% ===== \end{flushleft}
%% ===== 
%% ===== 
%% ===== 
%% ===== 
%% ===== 
%% ===== \begin{flushleft}
%% ===== Initiate a
%% ===== \end{flushleft}
%% ===== 
%% ===== 
%% ===== \begin{flushleft}
%% ===== directed fault
%% ===== \end{flushleft}
%% ===== 
%% ===== 
%% ===== 
%% ===== 
%% ===== 
%% ===== \begin{flushleft}
%% ===== C(SDW.R1) $\rightarrow$
%% ===== \end{flushleft}
%% ===== 
%% ===== 
%% ===== \begin{flushleft}
%% ===== C(RSDWH.R1)
%% ===== \end{flushleft}
%% ===== 
%% ===== 
%% ===== 
%% ===== 
%% ===== 
%% ===== \begin{flushleft}
%% ===== B
%% ===== \end{flushleft}
%% ===== 
%% ===== 
%% ===== 
%% ===== 
%% ===== 
%% ===== \begin{flushleft}
%% ===== Figure 8-1(cont). Complete Appending Unit Operation Flowchart
%% ===== \end{flushleft}
%% ===== 
%% ===== 
%% ===== 
%% ===== 
%% ===== 
%% ===== \begin{flushleft}
%% ===== \newpage
%% ===== B
%% ===== \end{flushleft}
%% ===== 
%% ===== 
%% ===== 
%% ===== 
%% ===== 
%% ===== \begin{flushleft}
%% ===== C(SDW.R1) $\leq$
%% ===== \end{flushleft}
%% ===== 
%% ===== 
%% ===== \begin{flushleft}
%% ===== C(SDW.R2) $\leq$
%% ===== \end{flushleft}
%% ===== 
%% ===== 
%% ===== \begin{flushleft}
%% ===== C(SDW.R3)?
%% ===== \end{flushleft}
%% ===== 
%% ===== 
%% ===== \begin{flushleft}
%% ===== Yes
%% ===== \end{flushleft}
%% ===== 
%% ===== 
%% ===== 
%% ===== 
%% ===== 
%% ===== \begin{flushleft}
%% ===== No
%% ===== \end{flushleft}
%% ===== 
%% ===== 
%% ===== \begin{flushleft}
%% ===== Set fault
%% ===== \end{flushleft}
%% ===== 
%% ===== 
%% ===== \begin{flushleft}
%% ===== ACV0 = IRO
%% ===== \end{flushleft}
%% ===== 
%% ===== 
%% ===== 
%% ===== 
%% ===== 
%% ===== \begin{flushleft}
%% ===== Was last
%% ===== \end{flushleft}
%% ===== 
%% ===== 
%% ===== \begin{flushleft}
%% ===== cycle an rtcd
%% ===== \end{flushleft}
%% ===== 
%% ===== 
%% ===== \begin{flushleft}
%% ===== operand fetch?
%% ===== \end{flushleft}
%% ===== 
%% ===== 
%% ===== 
%% ===== 
%% ===== 
%% ===== \begin{flushleft}
%% ===== No
%% ===== \end{flushleft}
%% ===== 
%% ===== 
%% ===== \begin{flushleft}
%% ===== Is
%% ===== \end{flushleft}
%% ===== 
%% ===== 
%% ===== \begin{flushleft}
%% ===== OPCODE
%% ===== \end{flushleft}
%% ===== 
%% ===== 
%% ===== \begin{flushleft}
%% ===== call6?
%% ===== \end{flushleft}
%% ===== 
%% ===== 
%% ===== 
%% ===== 
%% ===== 
%% ===== \begin{flushleft}
%% ===== No
%% ===== \end{flushleft}
%% ===== 
%% ===== 
%% ===== \begin{flushleft}
%% ===== Transfer or
%% ===== \end{flushleft}
%% ===== 
%% ===== 
%% ===== \begin{flushleft}
%% ===== instruction
%% ===== \end{flushleft}
%% ===== 
%% ===== 
%% ===== \begin{flushleft}
%% ===== fetch?
%% ===== \end{flushleft}
%% ===== 
%% ===== 
%% ===== \begin{flushleft}
%% ===== No
%% ===== \end{flushleft}
%% ===== 
%% ===== 
%% ===== \begin{flushleft}
%% ===== No
%% ===== \end{flushleft}
%% ===== 
%% ===== 
%% ===== 
%% ===== 
%% ===== 
%% ===== \begin{flushleft}
%% ===== No
%% ===== \end{flushleft}
%% ===== 
%% ===== 
%% ===== 
%% ===== 
%% ===== 
%% ===== \begin{flushleft}
%% ===== SDW.R
%% ===== \end{flushleft}
%% ===== 
%% ===== 
%% ===== \begin{flushleft}
%% ===== ON?
%% ===== \end{flushleft}
%% ===== 
%% ===== 
%% ===== 
%% ===== 
%% ===== 
%% ===== \begin{flushleft}
%% ===== F
%% ===== \end{flushleft}
%% ===== 
%% ===== 
%% ===== \begin{flushleft}
%% ===== Yes
%% ===== \end{flushleft}
%% ===== 
%% ===== 
%% ===== \begin{flushleft}
%% ===== C(TPR.TRR) $>$
%% ===== \end{flushleft}
%% ===== 
%% ===== 
%% ===== \begin{flushleft}
%% ===== C(SDW.R2)?
%% ===== \end{flushleft}
%% ===== 
%% ===== 
%% ===== 
%% ===== 
%% ===== 
%% ===== \begin{flushleft}
%% ===== Yes
%% ===== \end{flushleft}
%% ===== 
%% ===== 
%% ===== 
%% ===== 
%% ===== 
%% ===== \begin{flushleft}
%% ===== No
%% ===== \end{flushleft}
%% ===== 
%% ===== 
%% ===== 
%% ===== 
%% ===== 
%% ===== \begin{flushleft}
%% ===== No
%% ===== \end{flushleft}
%% ===== 
%% ===== 
%% ===== 
%% ===== 
%% ===== 
%% ===== \begin{flushleft}
%% ===== SDW.W
%% ===== \end{flushleft}
%% ===== 
%% ===== 
%% ===== \begin{flushleft}
%% ===== ON?
%% ===== \end{flushleft}
%% ===== 
%% ===== 
%% ===== \begin{flushleft}
%% ===== C(PPR.PSR) =
%% ===== \end{flushleft}
%% ===== 
%% ===== 
%% ===== \begin{flushleft}
%% ===== C(TPR.TSR)?
%% ===== \end{flushleft}
%% ===== 
%% ===== 
%% ===== 
%% ===== 
%% ===== 
%% ===== \begin{flushleft}
%% ===== C
%% ===== \end{flushleft}
%% ===== 
%% ===== 
%% ===== \begin{flushleft}
%% ===== E
%% ===== \end{flushleft}
%% ===== 
%% ===== 
%% ===== 
%% ===== 
%% ===== 
%% ===== \begin{flushleft}
%% ===== Set fault
%% ===== \end{flushleft}
%% ===== 
%% ===== 
%% ===== \begin{flushleft}
%% ===== ACV3 = ORB
%% ===== \end{flushleft}
%% ===== 
%% ===== 
%% ===== 
%% ===== 
%% ===== 
%% ===== \begin{flushleft}
%% ===== No
%% ===== \end{flushleft}
%% ===== 
%% ===== 
%% ===== 
%% ===== 
%% ===== 
%% ===== \begin{flushleft}
%% ===== Yes
%% ===== \end{flushleft}
%% ===== 
%% ===== 
%% ===== 
%% ===== 
%% ===== 
%% ===== \begin{flushleft}
%% ===== Yes
%% ===== \end{flushleft}
%% ===== 
%% ===== 
%% ===== 
%% ===== 
%% ===== 
%% ===== \begin{flushleft}
%% ===== Yes
%% ===== \end{flushleft}
%% ===== 
%% ===== 
%% ===== 
%% ===== 
%% ===== 
%% ===== \begin{flushleft}
%% ===== C(TPR.TRR) $>$
%% ===== \end{flushleft}
%% ===== 
%% ===== 
%% ===== \begin{flushleft}
%% ===== C(SDW.R2)?
%% ===== \end{flushleft}
%% ===== 
%% ===== 
%% ===== 
%% ===== 
%% ===== 
%% ===== \begin{flushleft}
%% ===== Yes
%% ===== \end{flushleft}
%% ===== 
%% ===== 
%% ===== 
%% ===== 
%% ===== 
%% ===== \begin{flushleft}
%% ===== Is it a
%% ===== \end{flushleft}
%% ===== 
%% ===== 
%% ===== \begin{flushleft}
%% ===== STR-OP?
%% ===== \end{flushleft}
%% ===== 
%% ===== 
%% ===== 
%% ===== 
%% ===== 
%% ===== \begin{flushleft}
%% ===== Yes
%% ===== \end{flushleft}
%% ===== 
%% ===== 
%% ===== 
%% ===== 
%% ===== 
%% ===== \begin{flushleft}
%% ===== Yes
%% ===== \end{flushleft}
%% ===== 
%% ===== 
%% ===== 
%% ===== 
%% ===== 
%% ===== \begin{flushleft}
%% ===== Yes
%% ===== \end{flushleft}
%% ===== 
%% ===== 
%% ===== 
%% ===== 
%% ===== 
%% ===== \begin{flushleft}
%% ===== Set fault
%% ===== \end{flushleft}
%% ===== 
%% ===== 
%% ===== \begin{flushleft}
%% ===== ACV5 = OWB
%% ===== \end{flushleft}
%% ===== 
%% ===== 
%% ===== 
%% ===== 
%% ===== 
%% ===== \begin{flushleft}
%% ===== No
%% ===== \end{flushleft}
%% ===== 
%% ===== 
%% ===== \begin{flushleft}
%% ===== Set fault
%% ===== \end{flushleft}
%% ===== 
%% ===== 
%% ===== \begin{flushleft}
%% ===== ACV6 = W-OFF
%% ===== \end{flushleft}
%% ===== 
%% ===== 
%% ===== 
%% ===== 
%% ===== 
%% ===== \begin{flushleft}
%% ===== Set fault
%% ===== \end{flushleft}
%% ===== 
%% ===== 
%% ===== \begin{flushleft}
%% ===== ACV4 = R-OFF
%% ===== \end{flushleft}
%% ===== 
%% ===== 
%% ===== 
%% ===== 
%% ===== 
%% ===== \begin{flushleft}
%% ===== G
%% ===== \end{flushleft}
%% ===== 
%% ===== 
%% ===== 
%% ===== 
%% ===== 
%% ===== \begin{flushleft}
%% ===== Figure 8-1(cont). Complete Appending Unit Operation Flowchart
%% ===== \end{flushleft}
%% ===== 
%% ===== 
%% ===== 
%% ===== 
%% ===== 
%% ===== \begin{flushleft}
%% ===== \newpage
%% ===== D
%% ===== \end{flushleft}
%% ===== 
%% ===== 
%% ===== 
%% ===== 
%% ===== 
%% ===== \begin{flushleft}
%% ===== C
%% ===== \end{flushleft}
%% ===== 
%% ===== 
%% ===== 
%% ===== 
%% ===== 
%% ===== \begin{flushleft}
%% ===== (instruction fetch)
%% ===== \end{flushleft}
%% ===== 
%% ===== 
%% ===== 
%% ===== 
%% ===== 
%% ===== \begin{flushleft}
%% ===== No
%% ===== \end{flushleft}
%% ===== 
%% ===== 
%% ===== \begin{flushleft}
%% ===== C(TPR.TRR) $>$
%% ===== \end{flushleft}
%% ===== 
%% ===== 
%% ===== \begin{flushleft}
%% ===== C(SDW.R2)?
%% ===== \end{flushleft}
%% ===== 
%% ===== 
%% ===== 
%% ===== 
%% ===== 
%% ===== \begin{flushleft}
%% ===== (rtcd operand)
%% ===== \end{flushleft}
%% ===== 
%% ===== 
%% ===== 
%% ===== 
%% ===== 
%% ===== \begin{flushleft}
%% ===== C(TPR.TRR) $<$
%% ===== \end{flushleft}
%% ===== 
%% ===== 
%% ===== \begin{flushleft}
%% ===== C(SDW.R1)?
%% ===== \end{flushleft}
%% ===== 
%% ===== 
%% ===== 
%% ===== 
%% ===== 
%% ===== \begin{flushleft}
%% ===== Yes
%% ===== \end{flushleft}
%% ===== 
%% ===== 
%% ===== 
%% ===== 
%% ===== 
%% ===== \begin{flushleft}
%% ===== Yes
%% ===== \end{flushleft}
%% ===== 
%% ===== 
%% ===== 
%% ===== 
%% ===== 
%% ===== \begin{flushleft}
%% ===== No
%% ===== \end{flushleft}
%% ===== 
%% ===== 
%% ===== 
%% ===== 
%% ===== 
%% ===== \begin{flushleft}
%% ===== SDW.E
%% ===== \end{flushleft}
%% ===== 
%% ===== 
%% ===== \begin{flushleft}
%% ===== set ON?
%% ===== \end{flushleft}
%% ===== 
%% ===== 
%% ===== 
%% ===== 
%% ===== 
%% ===== \begin{flushleft}
%% ===== Set fault
%% ===== \end{flushleft}
%% ===== 
%% ===== 
%% ===== \begin{flushleft}
%% ===== ACV1 = OEB
%% ===== \end{flushleft}
%% ===== 
%% ===== 
%% ===== 
%% ===== 
%% ===== 
%% ===== \begin{flushleft}
%% ===== No
%% ===== \end{flushleft}
%% ===== 
%% ===== 
%% ===== 
%% ===== 
%% ===== 
%% ===== \begin{flushleft}
%% ===== Yes
%% ===== \end{flushleft}
%% ===== 
%% ===== 
%% ===== 
%% ===== 
%% ===== 
%% ===== \begin{flushleft}
%% ===== C(TPR.TRR) $\geq$
%% ===== \end{flushleft}
%% ===== 
%% ===== 
%% ===== \begin{flushleft}
%% ===== C(PPR.PRR)
%% ===== \end{flushleft}
%% ===== 
%% ===== 
%% ===== 
%% ===== 
%% ===== 
%% ===== \begin{flushleft}
%% ===== Set fault
%% ===== \end{flushleft}
%% ===== 
%% ===== 
%% ===== \begin{flushleft}
%% ===== ACV2 = E-OFF
%% ===== \end{flushleft}
%% ===== 
%% ===== 
%% ===== 
%% ===== 
%% ===== 
%% ===== \begin{flushleft}
%% ===== No
%% ===== \end{flushleft}
%% ===== 
%% ===== 
%% ===== 
%% ===== 
%% ===== 
%% ===== \begin{flushleft}
%% ===== Yes
%% ===== \end{flushleft}
%% ===== 
%% ===== 
%% ===== 
%% ===== 
%% ===== 
%% ===== \begin{flushleft}
%% ===== RALR
%% ===== \end{flushleft}
%% ===== 
%% ===== 
%% ===== = 0?
%% ===== 
%% ===== 
%% ===== \begin{flushleft}
%% ===== Yes
%% ===== \end{flushleft}
%% ===== 
%% ===== 
%% ===== 
%% ===== 
%% ===== 
%% ===== \begin{flushleft}
%% ===== Set fault
%% ===== \end{flushleft}
%% ===== 
%% ===== 
%% ===== \begin{flushleft}
%% ===== ACV11 = INRET
%% ===== \end{flushleft}
%% ===== 
%% ===== 
%% ===== 
%% ===== 
%% ===== 
%% ===== \begin{flushleft}
%% ===== No
%% ===== \end{flushleft}
%% ===== 
%% ===== 
%% ===== \begin{flushleft}
%% ===== C(PPR.PRR)
%% ===== \end{flushleft}
%% ===== 
%% ===== 
%% ===== \begin{flushleft}
%% ===== $<$ RALR?
%% ===== \end{flushleft}
%% ===== 
%% ===== 
%% ===== \begin{flushleft}
%% ===== Yes
%% ===== \end{flushleft}
%% ===== 
%% ===== 
%% ===== 
%% ===== 
%% ===== 
%% ===== \begin{flushleft}
%% ===== No
%% ===== \end{flushleft}
%% ===== 
%% ===== 
%% ===== \begin{flushleft}
%% ===== Set fault
%% ===== \end{flushleft}
%% ===== 
%% ===== 
%% ===== \begin{flushleft}
%% ===== ACV13 = RALR
%% ===== \end{flushleft}
%% ===== 
%% ===== 
%% ===== 
%% ===== 
%% ===== 
%% ===== \begin{flushleft}
%% ===== G
%% ===== \end{flushleft}
%% ===== 
%% ===== 
%% ===== 
%% ===== 
%% ===== 
%% ===== \begin{flushleft}
%% ===== Figure 8-1(cont). Complete Appending Unit Operation Flowchart
%% ===== \end{flushleft}
%% ===== 
%% ===== 
%% ===== 
%% ===== 
%% ===== 
%% ===== \begin{flushleft}
%% ===== \newpage
%% ===== E
%% ===== \end{flushleft}
%% ===== 
%% ===== 
%% ===== 
%% ===== 
%% ===== 
%% ===== \begin{flushleft}
%% ===== (call6)
%% ===== \end{flushleft}
%% ===== 
%% ===== 
%% ===== 
%% ===== 
%% ===== 
%% ===== \begin{flushleft}
%% ===== SDW.E
%% ===== \end{flushleft}
%% ===== 
%% ===== 
%% ===== \begin{flushleft}
%% ===== set ON?
%% ===== \end{flushleft}
%% ===== 
%% ===== 
%% ===== 
%% ===== 
%% ===== 
%% ===== \begin{flushleft}
%% ===== No
%% ===== \end{flushleft}
%% ===== 
%% ===== 
%% ===== 
%% ===== 
%% ===== 
%% ===== \begin{flushleft}
%% ===== Yes
%% ===== \end{flushleft}
%% ===== 
%% ===== 
%% ===== 
%% ===== 
%% ===== 
%% ===== \begin{flushleft}
%% ===== No
%% ===== \end{flushleft}
%% ===== 
%% ===== 
%% ===== \begin{flushleft}
%% ===== No
%% ===== \end{flushleft}
%% ===== 
%% ===== 
%% ===== 
%% ===== 
%% ===== 
%% ===== \begin{flushleft}
%% ===== C(TPR.CA)4,17
%% ===== \end{flushleft}
%% ===== 
%% ===== 
%% ===== \begin{flushleft}
%% ===== $\geq$ SDW.CL?
%% ===== \end{flushleft}
%% ===== 
%% ===== 
%% ===== 
%% ===== 
%% ===== 
%% ===== \begin{flushleft}
%% ===== SDW.G
%% ===== \end{flushleft}
%% ===== 
%% ===== 
%% ===== \begin{flushleft}
%% ===== set ON?
%% ===== \end{flushleft}
%% ===== 
%% ===== 
%% ===== \begin{flushleft}
%% ===== Yes
%% ===== \end{flushleft}
%% ===== 
%% ===== 
%% ===== 
%% ===== 
%% ===== 
%% ===== \begin{flushleft}
%% ===== C(PPR.PSR) =
%% ===== \end{flushleft}
%% ===== 
%% ===== 
%% ===== \begin{flushleft}
%% ===== C(TPR.TSR)?
%% ===== \end{flushleft}
%% ===== 
%% ===== 
%% ===== 
%% ===== 
%% ===== 
%% ===== \begin{flushleft}
%% ===== No
%% ===== \end{flushleft}
%% ===== 
%% ===== 
%% ===== 
%% ===== 
%% ===== 
%% ===== \begin{flushleft}
%% ===== Set fault
%% ===== \end{flushleft}
%% ===== 
%% ===== 
%% ===== \begin{flushleft}
%% ===== ACV2 = E-OFF
%% ===== \end{flushleft}
%% ===== 
%% ===== 
%% ===== 
%% ===== 
%% ===== 
%% ===== \begin{flushleft}
%% ===== Yes
%% ===== \end{flushleft}
%% ===== 
%% ===== 
%% ===== 
%% ===== 
%% ===== 
%% ===== \begin{flushleft}
%% ===== Yes
%% ===== \end{flushleft}
%% ===== 
%% ===== 
%% ===== \begin{flushleft}
%% ===== Set fault
%% ===== \end{flushleft}
%% ===== 
%% ===== 
%% ===== \begin{flushleft}
%% ===== ACV7 = NO GA
%% ===== \end{flushleft}
%% ===== 
%% ===== 
%% ===== \begin{flushleft}
%% ===== C(TPR.TRR) $>$
%% ===== \end{flushleft}
%% ===== 
%% ===== 
%% ===== \begin{flushleft}
%% ===== C(PPR.PRR)?
%% ===== \end{flushleft}
%% ===== 
%% ===== 
%% ===== \begin{flushleft}
%% ===== C(TPR.TRR)
%% ===== \end{flushleft}
%% ===== 
%% ===== 
%% ===== \begin{flushleft}
%% ===== $>$ SDW.R3?
%% ===== \end{flushleft}
%% ===== 
%% ===== 
%% ===== \begin{flushleft}
%% ===== No
%% ===== \end{flushleft}
%% ===== 
%% ===== 
%% ===== 
%% ===== 
%% ===== 
%% ===== \begin{flushleft}
%% ===== C(TPR.TRR)
%% ===== \end{flushleft}
%% ===== 
%% ===== 
%% ===== \begin{flushleft}
%% ===== $<$ SDW.R1?
%% ===== \end{flushleft}
%% ===== 
%% ===== 
%% ===== 
%% ===== 
%% ===== 
%% ===== \begin{flushleft}
%% ===== No
%% ===== \end{flushleft}
%% ===== 
%% ===== 
%% ===== 
%% ===== 
%% ===== 
%% ===== \begin{flushleft}
%% ===== Yes
%% ===== \end{flushleft}
%% ===== 
%% ===== 
%% ===== 
%% ===== 
%% ===== 
%% ===== \begin{flushleft}
%% ===== Yes
%% ===== \end{flushleft}
%% ===== 
%% ===== 
%% ===== \begin{flushleft}
%% ===== C(PPR.PRR)
%% ===== \end{flushleft}
%% ===== 
%% ===== 
%% ===== \begin{flushleft}
%% ===== $<$ SDW.R2?
%% ===== \end{flushleft}
%% ===== 
%% ===== 
%% ===== \begin{flushleft}
%% ===== No
%% ===== \end{flushleft}
%% ===== 
%% ===== 
%% ===== 
%% ===== 
%% ===== 
%% ===== \begin{flushleft}
%% ===== Set fault
%% ===== \end{flushleft}
%% ===== 
%% ===== 
%% ===== \begin{flushleft}
%% ===== ACV8 = OCB
%% ===== \end{flushleft}
%% ===== 
%% ===== 
%% ===== 
%% ===== 
%% ===== 
%% ===== \begin{flushleft}
%% ===== No
%% ===== \end{flushleft}
%% ===== 
%% ===== 
%% ===== 
%% ===== 
%% ===== 
%% ===== \begin{flushleft}
%% ===== C(TPR.TRR)
%% ===== \end{flushleft}
%% ===== 
%% ===== 
%% ===== \begin{flushleft}
%% ===== $>$ SDW.R2?
%% ===== \end{flushleft}
%% ===== 
%% ===== 
%% ===== 
%% ===== 
%% ===== 
%% ===== \begin{flushleft}
%% ===== Yes
%% ===== \end{flushleft}
%% ===== 
%% ===== 
%% ===== 
%% ===== 
%% ===== 
%% ===== \begin{flushleft}
%% ===== No
%% ===== \end{flushleft}
%% ===== 
%% ===== 
%% ===== 
%% ===== 
%% ===== 
%% ===== \begin{flushleft}
%% ===== Yes
%% ===== \end{flushleft}
%% ===== 
%% ===== 
%% ===== \begin{flushleft}
%% ===== Set fault
%% ===== \end{flushleft}
%% ===== 
%% ===== 
%% ===== \begin{flushleft}
%% ===== ACV10 = BOC
%% ===== \end{flushleft}
%% ===== 
%% ===== 
%% ===== 
%% ===== 
%% ===== 
%% ===== \begin{flushleft}
%% ===== Yes
%% ===== \end{flushleft}
%% ===== 
%% ===== 
%% ===== 
%% ===== 
%% ===== 
%% ===== \begin{flushleft}
%% ===== SDW.R2 $\rightarrow$ C(TPR.TRR)
%% ===== \end{flushleft}
%% ===== 
%% ===== 
%% ===== 
%% ===== 
%% ===== 
%% ===== \begin{flushleft}
%% ===== Set fault
%% ===== \end{flushleft}
%% ===== 
%% ===== 
%% ===== \begin{flushleft}
%% ===== ACV9 = OCALL
%% ===== \end{flushleft}
%% ===== 
%% ===== 
%% ===== \begin{flushleft}
%% ===== G
%% ===== \end{flushleft}
%% ===== 
%% ===== 
%% ===== 
%% ===== 
%% ===== 
%% ===== \begin{flushleft}
%% ===== Figure 8-1(cont). Complete Appending Unit Operation Flowchart
%% ===== \end{flushleft}
%% ===== 
%% ===== 
%% ===== 
%% ===== 
%% ===== 
%% ===== \begin{flushleft}
%% ===== \newpage
%% ===== F
%% ===== \end{flushleft}
%% ===== 
%% ===== 
%% ===== 
%% ===== 
%% ===== 
%% ===== \begin{flushleft}
%% ===== No
%% ===== \end{flushleft}
%% ===== 
%% ===== 
%% ===== \begin{flushleft}
%% ===== C(TPR.TRR) $>$
%% ===== \end{flushleft}
%% ===== 
%% ===== 
%% ===== \begin{flushleft}
%% ===== C(SDW.R2)?
%% ===== \end{flushleft}
%% ===== 
%% ===== 
%% ===== 
%% ===== 
%% ===== 
%% ===== \begin{flushleft}
%% ===== (transfer or instruction fetch)
%% ===== \end{flushleft}
%% ===== 
%% ===== 
%% ===== 
%% ===== 
%% ===== 
%% ===== \begin{flushleft}
%% ===== C(TPR.TRR) $<$
%% ===== \end{flushleft}
%% ===== 
%% ===== 
%% ===== \begin{flushleft}
%% ===== C(SDW.R1)?
%% ===== \end{flushleft}
%% ===== 
%% ===== 
%% ===== 
%% ===== 
%% ===== 
%% ===== \begin{flushleft}
%% ===== Yes
%% ===== \end{flushleft}
%% ===== 
%% ===== 
%% ===== \begin{flushleft}
%% ===== Set fault
%% ===== \end{flushleft}
%% ===== 
%% ===== 
%% ===== \begin{flushleft}
%% ===== ACV1 = OEB
%% ===== \end{flushleft}
%% ===== 
%% ===== 
%% ===== 
%% ===== 
%% ===== 
%% ===== \begin{flushleft}
%% ===== No
%% ===== \end{flushleft}
%% ===== 
%% ===== 
%% ===== 
%% ===== 
%% ===== 
%% ===== \begin{flushleft}
%% ===== SDW.E
%% ===== \end{flushleft}
%% ===== 
%% ===== 
%% ===== \begin{flushleft}
%% ===== set ON?
%% ===== \end{flushleft}
%% ===== 
%% ===== 
%% ===== 
%% ===== 
%% ===== 
%% ===== \begin{flushleft}
%% ===== Yes
%% ===== \end{flushleft}
%% ===== 
%% ===== 
%% ===== 
%% ===== 
%% ===== 
%% ===== \begin{flushleft}
%% ===== No
%% ===== \end{flushleft}
%% ===== 
%% ===== 
%% ===== \begin{flushleft}
%% ===== Set fault
%% ===== \end{flushleft}
%% ===== 
%% ===== 
%% ===== \begin{flushleft}
%% ===== ACV2 = E-OFF
%% ===== \end{flushleft}
%% ===== 
%% ===== 
%% ===== 
%% ===== 
%% ===== 
%% ===== \begin{flushleft}
%% ===== Yes
%% ===== \end{flushleft}
%% ===== 
%% ===== 
%% ===== 
%% ===== 
%% ===== 
%% ===== \begin{flushleft}
%% ===== No
%% ===== \end{flushleft}
%% ===== 
%% ===== 
%% ===== 
%% ===== 
%% ===== 
%% ===== \begin{flushleft}
%% ===== C(PPR.PRR) =
%% ===== \end{flushleft}
%% ===== 
%% ===== 
%% ===== \begin{flushleft}
%% ===== C(TPR.TRR)?
%% ===== \end{flushleft}
%% ===== 
%% ===== 
%% ===== 
%% ===== 
%% ===== 
%% ===== \begin{flushleft}
%% ===== Set fault
%% ===== \end{flushleft}
%% ===== 
%% ===== 
%% ===== \begin{flushleft}
%% ===== ACV12 = CRT
%% ===== \end{flushleft}
%% ===== 
%% ===== 
%% ===== 
%% ===== 
%% ===== 
%% ===== \begin{flushleft}
%% ===== Yes
%% ===== \end{flushleft}
%% ===== 
%% ===== 
%% ===== 
%% ===== 
%% ===== 
%% ===== \begin{flushleft}
%% ===== D
%% ===== \end{flushleft}
%% ===== 
%% ===== 
%% ===== 
%% ===== 
%% ===== 
%% ===== \begin{flushleft}
%% ===== Figure 8-1(cont). Complete Appending Unit Operation Flowchart
%% ===== \end{flushleft}
%% ===== 
%% ===== 
%% ===== 
%% ===== 
%% ===== 
%% ===== \begin{flushleft}
%% ===== \newpage
%% ===== G
%% ===== \end{flushleft}
%% ===== 
%% ===== 
%% ===== 
%% ===== 
%% ===== 
%% ===== \begin{flushleft}
%% ===== C(TPR.CA)0,13
%% ===== \end{flushleft}
%% ===== 
%% ===== 
%% ===== \begin{flushleft}
%% ===== $>$ SDW.BOUND?
%% ===== \end{flushleft}
%% ===== 
%% ===== 
%% ===== 
%% ===== 
%% ===== 
%% ===== \begin{flushleft}
%% ===== Yes
%% ===== \end{flushleft}
%% ===== 
%% ===== 
%% ===== \begin{flushleft}
%% ===== Set fault
%% ===== \end{flushleft}
%% ===== 
%% ===== 
%% ===== \begin{flushleft}
%% ===== ACV15 = OOSB
%% ===== \end{flushleft}
%% ===== 
%% ===== 
%% ===== 
%% ===== 
%% ===== 
%% ===== \begin{flushleft}
%% ===== No
%% ===== \end{flushleft}
%% ===== 
%% ===== 
%% ===== 
%% ===== 
%% ===== 
%% ===== \begin{flushleft}
%% ===== No
%% ===== \end{flushleft}
%% ===== 
%% ===== 
%% ===== \begin{flushleft}
%% ===== No
%% ===== \end{flushleft}
%% ===== 
%% ===== 
%% ===== 
%% ===== 
%% ===== 
%% ===== \begin{flushleft}
%% ===== is segment
%% ===== \end{flushleft}
%% ===== 
%% ===== 
%% ===== \begin{flushleft}
%% ===== C(TPR.TSR)
%% ===== \end{flushleft}
%% ===== 
%% ===== 
%% ===== \begin{flushleft}
%% ===== paged?
%% ===== \end{flushleft}
%% ===== 
%% ===== 
%% ===== 
%% ===== 
%% ===== 
%% ===== \begin{flushleft}
%% ===== Yes
%% ===== \end{flushleft}
%% ===== 
%% ===== 
%% ===== \begin{flushleft}
%% ===== Initiate an access
%% ===== \end{flushleft}
%% ===== 
%% ===== 
%% ===== \begin{flushleft}
%% ===== viloation fault
%% ===== \end{flushleft}
%% ===== 
%% ===== 
%% ===== 
%% ===== 
%% ===== 
%% ===== \begin{flushleft}
%% ===== Yes
%% ===== \end{flushleft}
%% ===== 
%% ===== 
%% ===== \begin{flushleft}
%% ===== Yes
%% ===== \end{flushleft}
%% ===== 
%% ===== 
%% ===== 
%% ===== 
%% ===== 
%% ===== \begin{flushleft}
%% ===== H
%% ===== \end{flushleft}
%% ===== 
%% ===== 
%% ===== 
%% ===== 
%% ===== 
%% ===== \begin{flushleft}
%% ===== Any ACV
%% ===== \end{flushleft}
%% ===== 
%% ===== 
%% ===== \begin{flushleft}
%% ===== faults?
%% ===== \end{flushleft}
%% ===== 
%% ===== 
%% ===== 
%% ===== 
%% ===== 
%% ===== \begin{flushleft}
%% ===== is PTW for
%% ===== \end{flushleft}
%% ===== 
%% ===== 
%% ===== \begin{flushleft}
%% ===== C(TPR.CA)
%% ===== \end{flushleft}
%% ===== 
%% ===== 
%% ===== \begin{flushleft}
%% ===== in PTWAM?
%% ===== \end{flushleft}
%% ===== 
%% ===== 
%% ===== 
%% ===== 
%% ===== 
%% ===== \begin{flushleft}
%% ===== No
%% ===== \end{flushleft}
%% ===== 
%% ===== 
%% ===== \begin{flushleft}
%% ===== PTW
%% ===== \end{flushleft}
%% ===== 
%% ===== 
%% ===== \begin{flushleft}
%% ===== cycle
%% ===== \end{flushleft}
%% ===== 
%% ===== 
%% ===== \begin{flushleft}
%% ===== Yes
%% ===== \end{flushleft}
%% ===== 
%% ===== 
%% ===== 
%% ===== 
%% ===== 
%% ===== \begin{flushleft}
%% ===== Is PTW.F
%% ===== \end{flushleft}
%% ===== 
%% ===== 
%% ===== \begin{flushleft}
%% ===== set ON?
%% ===== \end{flushleft}
%% ===== 
%% ===== 
%% ===== 
%% ===== 
%% ===== 
%% ===== \begin{flushleft}
%% ===== Initiate a
%% ===== \end{flushleft}
%% ===== 
%% ===== 
%% ===== \begin{flushleft}
%% ===== directed fault
%% ===== \end{flushleft}
%% ===== 
%% ===== 
%% ===== 
%% ===== 
%% ===== 
%% ===== \begin{flushleft}
%% ===== Load
%% ===== \end{flushleft}
%% ===== 
%% ===== 
%% ===== \begin{flushleft}
%% ===== PTWAM
%% ===== \end{flushleft}
%% ===== 
%% ===== 
%% ===== 
%% ===== 
%% ===== 
%% ===== \begin{flushleft}
%% ===== Prepage
%% ===== \end{flushleft}
%% ===== 
%% ===== 
%% ===== \begin{flushleft}
%% ===== Mode?
%% ===== \end{flushleft}
%% ===== 
%% ===== 
%% ===== 
%% ===== 
%% ===== 
%% ===== \begin{flushleft}
%% ===== Yes
%% ===== \end{flushleft}
%% ===== 
%% ===== 
%% ===== \begin{flushleft}
%% ===== PTW2
%% ===== \end{flushleft}
%% ===== 
%% ===== 
%% ===== \begin{flushleft}
%% ===== cycle
%% ===== \end{flushleft}
%% ===== 
%% ===== 
%% ===== 
%% ===== 
%% ===== 
%% ===== \begin{flushleft}
%% ===== No
%% ===== \end{flushleft}
%% ===== 
%% ===== 
%% ===== 
%% ===== 
%% ===== 
%% ===== \begin{flushleft}
%% ===== Yes
%% ===== \end{flushleft}
%% ===== 
%% ===== 
%% ===== 
%% ===== 
%% ===== 
%% ===== \begin{flushleft}
%% ===== I
%% ===== \end{flushleft}
%% ===== 
%% ===== 
%% ===== 
%% ===== 
%% ===== 
%% ===== \begin{flushleft}
%% ===== No
%% ===== \end{flushleft}
%% ===== 
%% ===== 
%% ===== 
%% ===== 
%% ===== 
%% ===== \begin{flushleft}
%% ===== Is PTW.F
%% ===== \end{flushleft}
%% ===== 
%% ===== 
%% ===== \begin{flushleft}
%% ===== set ON?
%% ===== \end{flushleft}
%% ===== 
%% ===== 
%% ===== 
%% ===== 
%% ===== 
%% ===== \begin{flushleft}
%% ===== No
%% ===== \end{flushleft}
%% ===== 
%% ===== 
%% ===== \begin{flushleft}
%% ===== Initiate a
%% ===== \end{flushleft}
%% ===== 
%% ===== 
%% ===== \begin{flushleft}
%% ===== directed fault
%% ===== \end{flushleft}
%% ===== 
%% ===== 
%% ===== 
%% ===== 
%% ===== 
%% ===== \begin{flushleft}
%% ===== Figure 8-1(cont). Complete Appending Unit Operation Flowchart
%% ===== \end{flushleft}
%% ===== 
%% ===== 
%% ===== 
%% ===== 
%% ===== 
%% ===== \begin{flushleft}
%% ===== \newpage
%% ===== H
%% ===== \end{flushleft}
%% ===== 
%% ===== 
%% ===== 
%% ===== 
%% ===== 
%% ===== \begin{flushleft}
%% ===== I
%% ===== \end{flushleft}
%% ===== 
%% ===== 
%% ===== 
%% ===== 
%% ===== 
%% ===== \begin{flushleft}
%% ===== Yes
%% ===== \end{flushleft}
%% ===== 
%% ===== 
%% ===== 
%% ===== 
%% ===== 
%% ===== \begin{flushleft}
%% ===== STR-OP \&
%% ===== \end{flushleft}
%% ===== 
%% ===== 
%% ===== \begin{flushleft}
%% ===== PTW.M = 0?
%% ===== \end{flushleft}
%% ===== 
%% ===== 
%% ===== \begin{flushleft}
%% ===== No
%% ===== \end{flushleft}
%% ===== 
%% ===== 
%% ===== 
%% ===== 
%% ===== 
%% ===== \begin{flushleft}
%% ===== MPTW
%% ===== \end{flushleft}
%% ===== 
%% ===== 
%% ===== \begin{flushleft}
%% ===== cycle
%% ===== \end{flushleft}
%% ===== 
%% ===== 
%% ===== 
%% ===== 
%% ===== 
%% ===== \begin{flushleft}
%% ===== FANP
%% ===== \end{flushleft}
%% ===== 
%% ===== 
%% ===== \begin{flushleft}
%% ===== cycle
%% ===== \end{flushleft}
%% ===== 
%% ===== 
%% ===== 
%% ===== 
%% ===== 
%% ===== \begin{flushleft}
%% ===== Was this
%% ===== \end{flushleft}
%% ===== 
%% ===== 
%% ===== \begin{flushleft}
%% ===== an indirect
%% ===== \end{flushleft}
%% ===== 
%% ===== 
%% ===== \begin{flushleft}
%% ===== word fetch?
%% ===== \end{flushleft}
%% ===== 
%% ===== 
%% ===== \begin{flushleft}
%% ===== Yes
%% ===== \end{flushleft}
%% ===== 
%% ===== 
%% ===== 
%% ===== 
%% ===== 
%% ===== \begin{flushleft}
%% ===== J
%% ===== \end{flushleft}
%% ===== 
%% ===== 
%% ===== 
%% ===== 
%% ===== 
%% ===== \begin{flushleft}
%% ===== FAP
%% ===== \end{flushleft}
%% ===== 
%% ===== 
%% ===== \begin{flushleft}
%% ===== cycle
%% ===== \end{flushleft}
%% ===== 
%% ===== 
%% ===== 
%% ===== 
%% ===== 
%% ===== \begin{flushleft}
%% ===== No
%% ===== \end{flushleft}
%% ===== 
%% ===== 
%% ===== \begin{flushleft}
%% ===== Was it an
%% ===== \end{flushleft}
%% ===== 
%% ===== 
%% ===== \begin{flushleft}
%% ===== rtcd operand
%% ===== \end{flushleft}
%% ===== 
%% ===== 
%% ===== \begin{flushleft}
%% ===== fetch?
%% ===== \end{flushleft}
%% ===== 
%% ===== 
%% ===== \begin{flushleft}
%% ===== Yes
%% ===== \end{flushleft}
%% ===== 
%% ===== 
%% ===== 
%% ===== 
%% ===== 
%% ===== \begin{flushleft}
%% ===== No
%% ===== \end{flushleft}
%% ===== 
%% ===== 
%% ===== 
%% ===== 
%% ===== 
%% ===== \begin{flushleft}
%% ===== Is OPCODE
%% ===== \end{flushleft}
%% ===== 
%% ===== 
%% ===== \begin{flushleft}
%% ===== call6?
%% ===== \end{flushleft}
%% ===== 
%% ===== 
%% ===== \begin{flushleft}
%% ===== Yes
%% ===== \end{flushleft}
%% ===== 
%% ===== 
%% ===== 
%% ===== 
%% ===== 
%% ===== \begin{flushleft}
%% ===== K
%% ===== \end{flushleft}
%% ===== 
%% ===== 
%% ===== \begin{flushleft}
%% ===== N
%% ===== \end{flushleft}
%% ===== 
%% ===== 
%% ===== 
%% ===== 
%% ===== 
%% ===== \begin{flushleft}
%% ===== No
%% ===== \end{flushleft}
%% ===== 
%% ===== 
%% ===== \begin{flushleft}
%% ===== Transfer or
%% ===== \end{flushleft}
%% ===== 
%% ===== 
%% ===== \begin{flushleft}
%% ===== instruction
%% ===== \end{flushleft}
%% ===== 
%% ===== 
%% ===== \begin{flushleft}
%% ===== fetch?
%% ===== \end{flushleft}
%% ===== 
%% ===== 
%% ===== \begin{flushleft}
%% ===== Yes
%% ===== \end{flushleft}
%% ===== 
%% ===== 
%% ===== 
%% ===== 
%% ===== 
%% ===== \begin{flushleft}
%% ===== L
%% ===== \end{flushleft}
%% ===== 
%% ===== 
%% ===== 
%% ===== 
%% ===== 
%% ===== \begin{flushleft}
%% ===== No
%% ===== \end{flushleft}
%% ===== 
%% ===== 
%% ===== \begin{flushleft}
%% ===== APU data
%% ===== \end{flushleft}
%% ===== 
%% ===== 
%% ===== \begin{flushleft}
%% ===== movement?
%% ===== \end{flushleft}
%% ===== 
%% ===== 
%% ===== 
%% ===== 
%% ===== 
%% ===== \begin{flushleft}
%% ===== No
%% ===== \end{flushleft}
%% ===== 
%% ===== 
%% ===== 
%% ===== 
%% ===== 
%% ===== \begin{flushleft}
%% ===== Yes
%% ===== \end{flushleft}
%% ===== 
%% ===== 
%% ===== \begin{flushleft}
%% ===== Load/store
%% ===== \end{flushleft}
%% ===== 
%% ===== 
%% ===== \begin{flushleft}
%% ===== APU data
%% ===== \end{flushleft}
%% ===== 
%% ===== 
%% ===== 
%% ===== 
%% ===== 
%% ===== \begin{flushleft}
%% ===== END APPEND
%% ===== \end{flushleft}
%% ===== 
%% ===== 
%% ===== 
%% ===== 
%% ===== 
%% ===== \begin{flushleft}
%% ===== Figure 8-1(cont). Complete Appending Unit Operation Flowchart
%% ===== \end{flushleft}
%% ===== 
%% ===== 
%% ===== 
%% ===== 
%% ===== 
%% ===== \begin{flushleft}
%% ===== \newpage
%% ===== J
%% ===== \end{flushleft}
%% ===== 
%% ===== 
%% ===== 
%% ===== 
%% ===== 
%% ===== \begin{flushleft}
%% ===== ri or ir \&
%% ===== \end{flushleft}
%% ===== 
%% ===== 
%% ===== \begin{flushleft}
%% ===== TPR.CA even?
%% ===== \end{flushleft}
%% ===== 
%% ===== 
%% ===== \begin{flushleft}
%% ===== No
%% ===== \end{flushleft}
%% ===== 
%% ===== 
%% ===== 
%% ===== 
%% ===== 
%% ===== \begin{flushleft}
%% ===== Yes
%% ===== \end{flushleft}
%% ===== 
%% ===== 
%% ===== \begin{flushleft}
%% ===== C(Y)30,35
%% ===== \end{flushleft}
%% ===== 
%% ===== 
%% ===== = 438
%% ===== 
%% ===== 
%% ===== \begin{flushleft}
%% ===== Yes
%% ===== \end{flushleft}
%% ===== 
%% ===== 
%% ===== 
%% ===== 
%% ===== 
%% ===== \begin{flushleft}
%% ===== No
%% ===== \end{flushleft}
%% ===== 
%% ===== 
%% ===== \begin{flushleft}
%% ===== C(Y)30,35
%% ===== \end{flushleft}
%% ===== 
%% ===== 
%% ===== = 418
%% ===== 
%% ===== 
%% ===== 
%% ===== 
%% ===== 
%% ===== \begin{flushleft}
%% ===== O
%% ===== \end{flushleft}
%% ===== 
%% ===== 
%% ===== 
%% ===== 
%% ===== 
%% ===== \begin{flushleft}
%% ===== No
%% ===== \end{flushleft}
%% ===== 
%% ===== 
%% ===== 
%% ===== 
%% ===== 
%% ===== \begin{flushleft}
%% ===== Yes
%% ===== \end{flushleft}
%% ===== 
%% ===== 
%% ===== \begin{flushleft}
%% ===== P
%% ===== \end{flushleft}
%% ===== 
%% ===== 
%% ===== 
%% ===== 
%% ===== 
%% ===== \begin{flushleft}
%% ===== Yes
%% ===== \end{flushleft}
%% ===== 
%% ===== 
%% ===== \begin{flushleft}
%% ===== C(Y)0,17 $\rightarrow$ C(IWB)0,17
%% ===== \end{flushleft}
%% ===== 
%% ===== 
%% ===== \begin{flushleft}
%% ===== C(Y)30,35 $\rightarrow$ C(IWB)30,35
%% ===== \end{flushleft}
%% ===== 
%% ===== 
%% ===== \begin{flushleft}
%% ===== 0 $\rightarrow$ C(IWB)29
%% ===== \end{flushleft}
%% ===== 
%% ===== 
%% ===== 
%% ===== 
%% ===== 
%% ===== \begin{flushleft}
%% ===== C(Y)30,35 =
%% ===== \end{flushleft}
%% ===== 
%% ===== 
%% ===== \begin{flushleft}
%% ===== other indirect?
%% ===== \end{flushleft}
%% ===== 
%% ===== 
%% ===== \begin{flushleft}
%% ===== No
%% ===== \end{flushleft}
%% ===== 
%% ===== 
%% ===== 
%% ===== 
%% ===== 
%% ===== \begin{flushleft}
%% ===== END APPEND
%% ===== \end{flushleft}
%% ===== 
%% ===== 
%% ===== 
%% ===== 
%% ===== 
%% ===== \begin{flushleft}
%% ===== Figure 8-1(cont). Complete Appending Unit Operation Flowchart
%% ===== \end{flushleft}
%% ===== 
%% ===== 
%% ===== 
%% ===== 
%% ===== 
%% ===== \begin{flushleft}
%% ===== \newpage
%% ===== M
%% ===== \end{flushleft}
%% ===== 
%% ===== 
%% ===== 
%% ===== 
%% ===== 
%% ===== \begin{flushleft}
%% ===== K
%% ===== \end{flushleft}
%% ===== 
%% ===== 
%% ===== 
%% ===== 
%% ===== 
%% ===== \begin{flushleft}
%% ===== L
%% ===== \end{flushleft}
%% ===== 
%% ===== 
%% ===== 
%% ===== 
%% ===== 
%% ===== \begin{flushleft}
%% ===== C(Y)3,17 $\rightarrow$ C(TPR.TSR)
%% ===== \end{flushleft}
%% ===== 
%% ===== 
%% ===== \begin{flushleft}
%% ===== C(Y+1)0,17 $\rightarrow$ C(TPR.CA)
%% ===== \end{flushleft}
%% ===== 
%% ===== 
%% ===== \begin{flushleft}
%% ===== C(TPR.TRR) $\geq$
%% ===== \end{flushleft}
%% ===== 
%% ===== 
%% ===== \begin{flushleft}
%% ===== C(PPR.PRR)?
%% ===== \end{flushleft}
%% ===== 
%% ===== 
%% ===== \begin{flushleft}
%% ===== No
%% ===== \end{flushleft}
%% ===== 
%% ===== 
%% ===== 
%% ===== 
%% ===== 
%% ===== \begin{flushleft}
%% ===== Is
%% ===== \end{flushleft}
%% ===== 
%% ===== 
%% ===== \begin{flushleft}
%% ===== OPCODE
%% ===== \end{flushleft}
%% ===== 
%% ===== 
%% ===== \begin{flushleft}
%% ===== tspn?
%% ===== \end{flushleft}
%% ===== 
%% ===== 
%% ===== \begin{flushleft}
%% ===== No
%% ===== \end{flushleft}
%% ===== 
%% ===== 
%% ===== 
%% ===== 
%% ===== 
%% ===== \begin{flushleft}
%% ===== Yes
%% ===== \end{flushleft}
%% ===== 
%% ===== 
%% ===== \begin{flushleft}
%% ===== C(TPR.TRR) $\rightarrow$ C(PRi .RNR)
%% ===== \end{flushleft}
%% ===== 
%% ===== 
%% ===== \begin{flushleft}
%% ===== for i = 0, 7
%% ===== \end{flushleft}
%% ===== 
%% ===== 
%% ===== 
%% ===== 
%% ===== 
%% ===== \begin{flushleft}
%% ===== Yes
%% ===== \end{flushleft}
%% ===== 
%% ===== 
%% ===== \begin{flushleft}
%% ===== C(PPR.PRR) $\rightarrow$ C(PRn .RNR)
%% ===== \end{flushleft}
%% ===== 
%% ===== 
%% ===== \begin{flushleft}
%% ===== C(PPR.PSR) $\rightarrow$ C(PRn .SNR)
%% ===== \end{flushleft}
%% ===== 
%% ===== 
%% ===== \begin{flushleft}
%% ===== C(PPR.IC) $\rightarrow$ C(PRn .WORDNO)
%% ===== \end{flushleft}
%% ===== 
%% ===== 
%% ===== \begin{flushleft}
%% ===== 000000 $\rightarrow$ C(PRn .BITNO)
%% ===== \end{flushleft}
%% ===== 
%% ===== 
%% ===== 
%% ===== 
%% ===== 
%% ===== \begin{flushleft}
%% ===== C(TPR.TRR) $\rightarrow$ C(PPR.PRR)
%% ===== \end{flushleft}
%% ===== 
%% ===== 
%% ===== 
%% ===== 
%% ===== 
%% ===== \begin{flushleft}
%% ===== C(TPR.TSR) $\rightarrow$ C(PPR.PSR)
%% ===== \end{flushleft}
%% ===== 
%% ===== 
%% ===== \begin{flushleft}
%% ===== C(TPR.CA) $\rightarrow$ C(PPR.IC)
%% ===== \end{flushleft}
%% ===== 
%% ===== 
%% ===== 
%% ===== 
%% ===== 
%% ===== \begin{flushleft}
%% ===== Yes
%% ===== \end{flushleft}
%% ===== 
%% ===== 
%% ===== 
%% ===== 
%% ===== 
%% ===== \begin{flushleft}
%% ===== C(TPR.TRR)
%% ===== \end{flushleft}
%% ===== 
%% ===== 
%% ===== = 0?
%% ===== 
%% ===== 
%% ===== 
%% ===== 
%% ===== 
%% ===== \begin{flushleft}
%% ===== C(SDW.P) $\rightarrow$ C(PPR.P)
%% ===== \end{flushleft}
%% ===== 
%% ===== 
%% ===== 
%% ===== 
%% ===== 
%% ===== \begin{flushleft}
%% ===== No
%% ===== \end{flushleft}
%% ===== 
%% ===== 
%% ===== \begin{flushleft}
%% ===== 0 $\rightarrow$ C(PPR.P)
%% ===== \end{flushleft}
%% ===== 
%% ===== 
%% ===== 
%% ===== 
%% ===== 
%% ===== \begin{flushleft}
%% ===== Is this an
%% ===== \end{flushleft}
%% ===== 
%% ===== 
%% ===== \begin{flushleft}
%% ===== Yes
%% ===== \end{flushleft}
%% ===== 
%% ===== 
%% ===== \begin{flushleft}
%% ===== rtcd operand
%% ===== \end{flushleft}
%% ===== 
%% ===== 
%% ===== \begin{flushleft}
%% ===== fetch?
%% ===== \end{flushleft}
%% ===== 
%% ===== 
%% ===== \begin{flushleft}
%% ===== No
%% ===== \end{flushleft}
%% ===== 
%% ===== 
%% ===== 
%% ===== 
%% ===== 
%% ===== \begin{flushleft}
%% ===== O
%% ===== \end{flushleft}
%% ===== 
%% ===== 
%% ===== 
%% ===== 
%% ===== 
%% ===== \begin{flushleft}
%% ===== END APPEND
%% ===== \end{flushleft}
%% ===== 
%% ===== 
%% ===== 
%% ===== 
%% ===== 
%% ===== \begin{flushleft}
%% ===== Figure 8-1(cont). Complete Appending Unit Operation Flowchart
%% ===== \end{flushleft}
%% ===== 
%% ===== 
%% ===== 
%% ===== 
%% ===== 
%% ===== \begin{flushleft}
%% ===== N
%% ===== \end{flushleft}
%% ===== 
%% ===== 
%% ===== 
%% ===== 
%% ===== 
%% ===== \begin{flushleft}
%% ===== Yes
%% ===== \end{flushleft}
%% ===== 
%% ===== 
%% ===== 
%% ===== 
%% ===== 
%% ===== \begin{flushleft}
%% ===== C(PR6.SNR) $\rightarrow$ C(PR7.SNR)
%% ===== \end{flushleft}
%% ===== 
%% ===== 
%% ===== 
%% ===== 
%% ===== 
%% ===== \begin{flushleft}
%% ===== C(TPR.TRR) =
%% ===== \end{flushleft}
%% ===== 
%% ===== 
%% ===== \begin{flushleft}
%% ===== C(PPR.PRR)?
%% ===== \end{flushleft}
%% ===== 
%% ===== 
%% ===== 
%% ===== 
%% ===== 
%% ===== \begin{flushleft}
%% ===== No
%% ===== \end{flushleft}
%% ===== 
%% ===== 
%% ===== \begin{flushleft}
%% ===== C(DSBR.STACK) || C(TPR.TRR)
%% ===== \end{flushleft}
%% ===== 
%% ===== 
%% ===== \begin{flushleft}
%% ===== $\rightarrow$ C(PR7.SNR)
%% ===== \end{flushleft}
%% ===== 
%% ===== 
%% ===== 
%% ===== 
%% ===== 
%% ===== \begin{flushleft}
%% ===== C(TPR.TRR) $\rightarrow$ C(PR7.RNR)
%% ===== \end{flushleft}
%% ===== 
%% ===== 
%% ===== \begin{flushleft}
%% ===== 00...0 $\rightarrow$ C(PR7.WORDNO)
%% ===== \end{flushleft}
%% ===== 
%% ===== 
%% ===== \begin{flushleft}
%% ===== 000000 $\rightarrow$ C(PR7.BITNO)
%% ===== \end{flushleft}
%% ===== 
%% ===== 
%% ===== \begin{flushleft}
%% ===== C(TPR.TRR) $\rightarrow$ C(PPR.PRR)
%% ===== \end{flushleft}
%% ===== 
%% ===== 
%% ===== \begin{flushleft}
%% ===== C(TPR.TSR) $\rightarrow$ C(PPR.PSR)
%% ===== \end{flushleft}
%% ===== 
%% ===== 
%% ===== \begin{flushleft}
%% ===== C(TPR.CA) $\rightarrow$ C(PPR.IC)
%% ===== \end{flushleft}
%% ===== 
%% ===== 
%% ===== 
%% ===== 
%% ===== 
%% ===== \begin{flushleft}
%% ===== M
%% ===== \end{flushleft}
%% ===== 
%% ===== 
%% ===== 
%% ===== 
%% ===== 
%% ===== \begin{flushleft}
%% ===== Figure 8-1(cont). Complete Appending Unit Operation Flowchart
%% ===== \end{flushleft}
%% ===== 
%% ===== 
%% ===== 
%% ===== 
%% ===== 
%% ===== \begin{flushleft}
%% ===== \newpage
%% ===== O
%% ===== \end{flushleft}
%% ===== 
%% ===== 
%% ===== 
%% ===== 
%% ===== 
%% ===== \begin{flushleft}
%% ===== No
%% ===== \end{flushleft}
%% ===== 
%% ===== 
%% ===== \begin{flushleft}
%% ===== No
%% ===== \end{flushleft}
%% ===== 
%% ===== 
%% ===== 
%% ===== 
%% ===== 
%% ===== \begin{flushleft}
%% ===== C(Y)18,20 $\geq$
%% ===== \end{flushleft}
%% ===== 
%% ===== 
%% ===== \begin{flushleft}
%% ===== RSDWH.R1?
%% ===== \end{flushleft}
%% ===== 
%% ===== 
%% ===== 
%% ===== 
%% ===== 
%% ===== \begin{flushleft}
%% ===== Yes
%% ===== \end{flushleft}
%% ===== 
%% ===== 
%% ===== 
%% ===== 
%% ===== 
%% ===== \begin{flushleft}
%% ===== C(TPR.TRR) $\geq$
%% ===== \end{flushleft}
%% ===== 
%% ===== 
%% ===== \begin{flushleft}
%% ===== RSDWH.R1?
%% ===== \end{flushleft}
%% ===== 
%% ===== 
%% ===== \begin{flushleft}
%% ===== Yes
%% ===== \end{flushleft}
%% ===== 
%% ===== 
%% ===== 
%% ===== 
%% ===== 
%% ===== \begin{flushleft}
%% ===== C(TPR.TRR) $\geq$
%% ===== \end{flushleft}
%% ===== 
%% ===== 
%% ===== \begin{flushleft}
%% ===== C(Y)18,20?
%% ===== \end{flushleft}
%% ===== 
%% ===== 
%% ===== 
%% ===== 
%% ===== 
%% ===== \begin{flushleft}
%% ===== No
%% ===== \end{flushleft}
%% ===== 
%% ===== 
%% ===== 
%% ===== 
%% ===== 
%% ===== \begin{flushleft}
%% ===== Yes
%% ===== \end{flushleft}
%% ===== 
%% ===== 
%% ===== \begin{flushleft}
%% ===== C(Y)18,20 $\rightarrow$ C(TPR.TRR)
%% ===== \end{flushleft}
%% ===== 
%% ===== 
%% ===== 
%% ===== 
%% ===== 
%% ===== \begin{flushleft}
%% ===== RSDWH.R1 $\rightarrow$ C(TPR.TRR)
%% ===== \end{flushleft}
%% ===== 
%% ===== 
%% ===== 
%% ===== 
%% ===== 
%% ===== \begin{flushleft}
%% ===== END APPEND
%% ===== \end{flushleft}
%% ===== 
%% ===== 
%% ===== 
%% ===== 
%% ===== 
%% ===== \begin{flushleft}
%% ===== P
%% ===== \end{flushleft}
%% ===== 
%% ===== 
%% ===== 
%% ===== 
%% ===== 
%% ===== \begin{flushleft}
%% ===== No
%% ===== \end{flushleft}
%% ===== 
%% ===== 
%% ===== \begin{flushleft}
%% ===== No
%% ===== \end{flushleft}
%% ===== 
%% ===== 
%% ===== 
%% ===== 
%% ===== 
%% ===== \begin{flushleft}
%% ===== C(PRn .RNR) $\geq$
%% ===== \end{flushleft}
%% ===== 
%% ===== 
%% ===== \begin{flushleft}
%% ===== RSDWH.R1?
%% ===== \end{flushleft}
%% ===== 
%% ===== 
%% ===== 
%% ===== 
%% ===== 
%% ===== \begin{flushleft}
%% ===== Yes
%% ===== \end{flushleft}
%% ===== 
%% ===== 
%% ===== 
%% ===== 
%% ===== 
%% ===== \begin{flushleft}
%% ===== C(TPR.TRR) $\geq$
%% ===== \end{flushleft}
%% ===== 
%% ===== 
%% ===== \begin{flushleft}
%% ===== RSDWH.R1?
%% ===== \end{flushleft}
%% ===== 
%% ===== 
%% ===== \begin{flushleft}
%% ===== Yes
%% ===== \end{flushleft}
%% ===== 
%% ===== 
%% ===== 
%% ===== 
%% ===== 
%% ===== \begin{flushleft}
%% ===== No
%% ===== \end{flushleft}
%% ===== 
%% ===== 
%% ===== 
%% ===== 
%% ===== 
%% ===== \begin{flushleft}
%% ===== C(TPR.TRR) $\geq$
%% ===== \end{flushleft}
%% ===== 
%% ===== 
%% ===== \begin{flushleft}
%% ===== C(PRn .RNR)?
%% ===== \end{flushleft}
%% ===== 
%% ===== 
%% ===== \begin{flushleft}
%% ===== Yes
%% ===== \end{flushleft}
%% ===== 
%% ===== 
%% ===== 
%% ===== 
%% ===== 
%% ===== \begin{flushleft}
%% ===== RSDWH.R1 $\rightarrow$ C(TPR.TRR)
%% ===== \end{flushleft}
%% ===== 
%% ===== 
%% ===== 
%% ===== 
%% ===== 
%% ===== \begin{flushleft}
%% ===== C(PRn .RNR) $\rightarrow$ C(TPR.TRR)
%% ===== \end{flushleft}
%% ===== 
%% ===== 
%% ===== 
%% ===== 
%% ===== 
%% ===== \begin{flushleft}
%% ===== END APPEND
%% ===== \end{flushleft}
%% ===== 
%% ===== 
%% ===== 
%% ===== 
%% ===== 
%% ===== \begin{flushleft}
%% ===== Figure 8-1(cont). Complete Appending Unit Operation Flowchart
%% ===== \end{flushleft}
%% ===== 
%% ===== 
%% ===== 
%% ===== 
%% ===== 
%% ===== \begin{flushleft}
%% ===== \newpage
