
\section{FAULTS AND INTERRUPTS}

%% ===== \end{flushleft}
%% ===== 
%% ===== 
%% ===== \begin{flushleft}
%% ===== Faults and interrupts both result in an interruption of normal sequential processing, but
%% ===== \end{flushleft}
%% ===== 
%% ===== 
%% ===== \begin{flushleft}
%% ===== there is a difference in how they originate. Generally, faults are caused by events or conditions
%% ===== \end{flushleft}
%% ===== 
%% ===== 
%% ===== \begin{flushleft}
%% ===== that are internal to the processor and interrupts are caused by events or conditions that are
%% ===== \end{flushleft}
%% ===== 
%% ===== 
%% ===== \begin{flushleft}
%% ===== external to the processor. Faults and interrupts enable the processor to respond promptly when
%% ===== \end{flushleft}
%% ===== 
%% ===== 
%% ===== \begin{flushleft}
%% ===== conditions occur that require system attention.
%% ===== \end{flushleft}
%% ===== 
%% ===== 
%% ===== \begin{flushleft}
%% ===== A unique word-pair is dedicated for the instructions to service each fault and interrupt
%% ===== \end{flushleft}
%% ===== 
%% ===== 
%% ===== \begin{flushleft}
%% ===== condition. The instruction pair associated with a fault or interrupt is called the trap pair for that
%% ===== \end{flushleft}
%% ===== 
%% ===== 
%% ===== \begin{flushleft}
%% ===== fault or interrupt. The set of all interrupt trap pairs is called the interrupt vector and is located at
%% ===== \end{flushleft}
%% ===== 
%% ===== 
%% ===== \begin{flushleft}
%% ===== absolute main memory address 0. The set of all fault trap pairs is called the fault vector and is
%% ===== \end{flushleft}
%% ===== 
%% ===== 
%% ===== \begin{flushleft}
%% ===== located at a 0 modulo 32 absolute main memory address whose high-order bits are given by the
%% ===== \end{flushleft}
%% ===== 
%% ===== 
%% ===== \begin{flushleft}
%% ===== setting of the FAULT BASE switches on the processor configuration panel. The fault vector is
%% ===== \end{flushleft}
%% ===== 
%% ===== 
%% ===== \begin{flushleft}
%% ===== constrained to lie within the lowest 4096 words of main memory.
%% ===== \end{flushleft}
%% ===== 
%% ===== 
%% ===== 
%% ===== 
%% ===== 
%% ===== \begin{flushleft}

\subsection{FAULT CYCLE SEQUENCE}

%% ===== \end{flushleft}
%% ===== 
%% ===== 
%% ===== \begin{flushleft}
%% ===== Following the detection of a fault condition, the control unit determines the proper time to
%% ===== \end{flushleft}
%% ===== 
%% ===== 
%% ===== \begin{flushleft}
%% ===== initiate the fault sequence according to the fault group (Fault groups are discussed later in this
%% ===== \end{flushleft}
%% ===== 
%% ===== 
%% ===== \begin{flushleft}
%% ===== section). At that time, the control unit interrupts normal sequential processing with an ABORT
%% ===== \end{flushleft}
%% ===== 
%% ===== 
%% ===== \begin{flushleft}
%% ===== CYCLE.
%% ===== \end{flushleft}
%% ===== 
%% ===== 
%% ===== \begin{flushleft}
%% ===== The ABORT CYCLE brings all overlapped and asynchronous functions within the
%% ===== \end{flushleft}
%% ===== 
%% ===== 
%% ===== \begin{flushleft}
%% ===== processor to an orderly halt. At the end of the ABORT CYCLE, the control unit initiates a FAULT
%% ===== \end{flushleft}
%% ===== 
%% ===== 
%% ===== \begin{flushleft}
%% ===== CYCLE.
%% ===== \end{flushleft}
%% ===== 
%% ===== 
%% ===== \begin{flushleft}
%% ===== In the FAULT CYCLE, the processor safe-stores the Control Unit Data (see Section 3) into
%% ===== \end{flushleft}
%% ===== 
%% ===== 
%% ===== \begin{flushleft}
%% ===== program-invisible holding registers in preparation for a Store Control Unit (scu) instruction, then
%% ===== \end{flushleft}
%% ===== 
%% ===== 
%% ===== \begin{flushleft}
%% ===== enters temporary absolute mode, forces the current ring of execution C(PPR.PRR) to 0, and
%% ===== \end{flushleft}
%% ===== 
%% ===== 
%% ===== \begin{flushleft}
%% ===== generates a computed address for the fault trap pair by concatenating the setting of the FAULT
%% ===== \end{flushleft}
%% ===== 
%% ===== 
%% ===== \begin{flushleft}
%% ===== BASE switches on the processor configuration panel with twice the fault number (see Table 7-1).
%% ===== \end{flushleft}
%% ===== 
%% ===== 
%% ===== \begin{flushleft}
%% ===== This computed address and the operation code for the Execute Double (xed) instruction are forced
%% ===== \end{flushleft}
%% ===== 
%% ===== 
%% ===== \begin{flushleft}
%% ===== into the instruction register and executed as an instruction. Note that the execution of the
%% ===== \end{flushleft}
%% ===== 
%% ===== 
%% ===== \begin{flushleft}
%% ===== instruction is not done in a normal EXECUTE CYCLE but in the FAULT CYCLE with the processor
%% ===== \end{flushleft}
%% ===== 
%% ===== 
%% ===== \begin{flushleft}
%% ===== in temporary absolute mode.
%% ===== \end{flushleft}
%% ===== 
%% ===== 
%% ===== \begin{flushleft}
%% ===== If the attempt to fetch and execute the instruction pair at the fault trap pair results in
%% ===== \end{flushleft}
%% ===== 
%% ===== 
%% ===== \begin{flushleft}
%% ===== another fault, the current FAULT CYCLE is aborted and a new FAULT CYCLE for the trouble fault
%% ===== \end{flushleft}
%% ===== 
%% ===== 
%% ===== \begin{flushleft}
%% ===== (fault number 31) is initiated. In the FAULT CYCLE for a trouble fault, the processor does not
%% ===== \end{flushleft}
%% ===== 
%% ===== 
%% ===== \begin{flushleft}
%% ===== safe-store the Control Unit Data. Therefore, it may be possible to recover the conditions for the
%% ===== \end{flushleft}
%% ===== 
%% ===== 
%% ===== \begin{flushleft}
%% ===== original fault (except the fault number) by use of the Store Control Unit (scu) instruction. The fault
%% ===== \end{flushleft}
%% ===== 
%% ===== 
%% ===== \begin{flushleft}
%% ===== number may usually be recovered by analysis of the computed address for the original fault trap
%% ===== \end{flushleft}
%% ===== 
%% ===== 
%% ===== \begin{flushleft}
%% ===== pair stored in the control unit history registers.
%% ===== \end{flushleft}
%% ===== 
%% ===== 
%% ===== \begin{flushleft}
%% ===== If either of the two instructions in the fault trap pair results in a transfer of control to a
%% ===== \end{flushleft}
%% ===== 
%% ===== 
%% ===== \begin{flushleft}
%% ===== computed address generated in absolute mode, the absolute mode indicator is set ON for the
%% ===== \end{flushleft}
%% ===== 
%% ===== 
%% ===== \begin{flushleft}
%% ===== transfer and remains ON thereafter until changed by program action.
%% ===== \end{flushleft}
%% ===== 
%% ===== 
%% ===== \begin{flushleft}
%% ===== If either of the two instructions in the fault trap pair results in a transfer of control to a
%% ===== \end{flushleft}
%% ===== 
%% ===== 
%% ===== \begin{flushleft}
%% ===== computed address generated in append mode (through the use of bit 29 of the instruction word or
%% ===== \end{flushleft}
%% ===== 
%% ===== 
%% ===== \begin{flushleft}
%% ===== by use of the its or itp modifiers), the transfer is made in the append mode and the processor
%% ===== \end{flushleft}
%% ===== 
%% ===== 
%% ===== \begin{flushleft}
%% ===== remains in append mode thereafter.
%% ===== \end{flushleft}
%% ===== 
%% ===== 
%% ===== \begin{flushleft}
%% ===== If no transfer of control takes place, the processor returns to the mode in effect at the time
%% ===== \end{flushleft}
%% ===== 
%% ===== 
%% ===== \begin{flushleft}
%% ===== of the fault and resumes normal sequential execution with the instruction following the faulting
%% ===== \end{flushleft}
%% ===== 
%% ===== 
%% ===== \begin{flushleft}
%% ===== instruction (C(PPR.IC) + 1). Note that the current ring of execution C(PPR.PRR) was forced to 0
%% ===== \end{flushleft}
%% ===== 
%% ===== 
%% ===== \begin{flushleft}
%% ===== during the FAULT CYCLE and that normal sequential execution will resume in ring 0.
%% ===== \end{flushleft}
%% ===== 
%% ===== 
%% ===== 
%% ===== 
%% ===== 
%% ===== \begin{flushleft}
%% ===== \newpage
%% ===== Many of the fault conditions are deliberately or inadvertently caused by the software and do
%% ===== \end{flushleft}
%% ===== 
%% ===== 
%% ===== \begin{flushleft}
%% ===== not necessarily involve error conditions. The operating supervisor determines the proper action
%% ===== \end{flushleft}
%% ===== 
%% ===== 
%% ===== \begin{flushleft}
%% ===== for each fault condition by analyzing the detailed state of the processor at the time of the fault. In
%% ===== \end{flushleft}
%% ===== 
%% ===== 
%% ===== \begin{flushleft}
%% ===== order to accomplish this analysis, it is necessary that the first instruction in each of the fault trap
%% ===== \end{flushleft}
%% ===== 
%% ===== 
%% ===== \begin{flushleft}
%% ===== pairs be the Store Control Unit (scu) instruction and the second be a transfer to a fault analysis
%% ===== \end{flushleft}
%% ===== 
%% ===== 
%% ===== \begin{flushleft}
%% ===== routine. If a fault condition is to be intentionally ignored, the fault trap pair for that condition
%% ===== \end{flushleft}
%% ===== 
%% ===== 
%% ===== \begin{flushleft}
%% ===== should contain an scu/rcu pair referencing a unique Y-block8. By using this pair to ignore a fault,
%% ===== \end{flushleft}
%% ===== 
%% ===== 
%% ===== \begin{flushleft}
%% ===== the state of the processor for the ignored fault condition may be recovered if the ignored fault
%% ===== \end{flushleft}
%% ===== 
%% ===== 
%% ===== \begin{flushleft}
%% ===== causes a trouble fault. The use of the scu/rcu pair also ensures that execution is resumed in the
%% ===== \end{flushleft}
%% ===== 
%% ===== 
%% ===== \begin{flushleft}
%% ===== original ring of execution.
%% ===== \end{flushleft}
%% ===== 
%% ===== 
%% ===== 
%% ===== 
%% ===== 
%% ===== \begin{flushleft}
%% ===== Table 7-1. List of Faults
%% ===== \end{flushleft}
%% ===== 
%% ===== 
%% ===== 
%% ===== 
%% ===== 
%% ===== \begin{flushleft}
%% ===== Decimal fault
%% ===== \end{flushleft}
%% ===== 
%% ===== 
%% ===== \begin{flushleft}
%% ===== number
%% ===== \end{flushleft}
%% ===== 
%% ===== 
%% ===== 
%% ===== 
%% ===== 
%% ===== \begin{flushleft}
%% ===== Octal (1)
%% ===== \end{flushleft}
%% ===== 
%% ===== 
%% ===== \begin{flushleft}
%% ===== fault
%% ===== \end{flushleft}
%% ===== 
%% ===== 
%% ===== \begin{flushleft}
%% ===== address
%% ===== \end{flushleft}
%% ===== 
%% ===== 
%% ===== 
%% ===== 
%% ===== 
%% ===== \begin{flushleft}
%% ===== Fault
%% ===== \end{flushleft}
%% ===== 
%% ===== 
%% ===== \begin{flushleft}
%% ===== mnemonic
%% ===== \end{flushleft}
%% ===== 
%% ===== 
%% ===== 
%% ===== 
%% ===== 
%% ===== 0
%% ===== 
%% ===== 
%% ===== 
%% ===== 
%% ===== 
%% ===== 0
%% ===== 
%% ===== 
%% ===== 
%% ===== 
%% ===== 
%% ===== \begin{flushleft}
%% ===== sdf
%% ===== \end{flushleft}
%% ===== 
%% ===== 
%% ===== 
%% ===== 
%% ===== 
%% ===== \begin{flushleft}
%% ===== Shutdown
%% ===== \end{flushleft}
%% ===== 
%% ===== 
%% ===== 
%% ===== 
%% ===== 
%% ===== 27
%% ===== 
%% ===== 
%% ===== 
%% ===== 
%% ===== 
%% ===== 7
%% ===== 
%% ===== 
%% ===== 
%% ===== 
%% ===== 
%% ===== 1
%% ===== 
%% ===== 
%% ===== 
%% ===== 
%% ===== 
%% ===== 2
%% ===== 
%% ===== 
%% ===== 
%% ===== 
%% ===== 
%% ===== \begin{flushleft}
%% ===== str
%% ===== \end{flushleft}
%% ===== 
%% ===== 
%% ===== 
%% ===== 
%% ===== 
%% ===== \begin{flushleft}
%% ===== Store
%% ===== \end{flushleft}
%% ===== 
%% ===== 
%% ===== 
%% ===== 
%% ===== 
%% ===== 10
%% ===== 
%% ===== 
%% ===== 
%% ===== 
%% ===== 
%% ===== 4
%% ===== 
%% ===== 
%% ===== 
%% ===== 
%% ===== 
%% ===== 2
%% ===== 
%% ===== 
%% ===== 
%% ===== 
%% ===== 
%% ===== 4
%% ===== 
%% ===== 
%% ===== 
%% ===== 
%% ===== 
%% ===== \begin{flushleft}
%% ===== mme
%% ===== \end{flushleft}
%% ===== 
%% ===== 
%% ===== 
%% ===== 
%% ===== 
%% ===== \begin{flushleft}
%% ===== Master mode entry 1
%% ===== \end{flushleft}
%% ===== 
%% ===== 
%% ===== 
%% ===== 
%% ===== 
%% ===== 11
%% ===== 
%% ===== 
%% ===== 
%% ===== 
%% ===== 
%% ===== 5
%% ===== 
%% ===== 
%% ===== 
%% ===== 
%% ===== 
%% ===== 3
%% ===== 
%% ===== 
%% ===== 
%% ===== 
%% ===== 
%% ===== 6
%% ===== 
%% ===== 
%% ===== 
%% ===== 
%% ===== 
%% ===== \begin{flushleft}
%% ===== f1
%% ===== \end{flushleft}
%% ===== 
%% ===== 
%% ===== 
%% ===== 
%% ===== 
%% ===== \begin{flushleft}
%% ===== Fault tag 1
%% ===== \end{flushleft}
%% ===== 
%% ===== 
%% ===== 
%% ===== 
%% ===== 
%% ===== 17
%% ===== 
%% ===== 
%% ===== 
%% ===== 
%% ===== 
%% ===== 5
%% ===== 
%% ===== 
%% ===== 
%% ===== 
%% ===== 
%% ===== 4
%% ===== 
%% ===== 
%% ===== 
%% ===== 
%% ===== 
%% ===== 10
%% ===== 
%% ===== 
%% ===== 
%% ===== 
%% ===== 
%% ===== \begin{flushleft}
%% ===== tro
%% ===== \end{flushleft}
%% ===== 
%% ===== 
%% ===== 
%% ===== 
%% ===== 
%% ===== \begin{flushleft}
%% ===== Timer runout
%% ===== \end{flushleft}
%% ===== 
%% ===== 
%% ===== 
%% ===== 
%% ===== 
%% ===== 26
%% ===== 
%% ===== 
%% ===== 
%% ===== 
%% ===== 
%% ===== 7
%% ===== 
%% ===== 
%% ===== 
%% ===== 
%% ===== 
%% ===== 5
%% ===== 
%% ===== 
%% ===== 
%% ===== 
%% ===== 
%% ===== 12
%% ===== 
%% ===== 
%% ===== 
%% ===== 
%% ===== 
%% ===== \begin{flushleft}
%% ===== cmd
%% ===== \end{flushleft}
%% ===== 
%% ===== 
%% ===== 
%% ===== 
%% ===== 
%% ===== 9
%% ===== 
%% ===== 
%% ===== 
%% ===== 
%% ===== 
%% ===== 4
%% ===== 
%% ===== 
%% ===== 
%% ===== 
%% ===== 
%% ===== 6
%% ===== 
%% ===== 
%% ===== 
%% ===== 
%% ===== 
%% ===== 14
%% ===== 
%% ===== 
%% ===== 
%% ===== 
%% ===== 
%% ===== \begin{flushleft}
%% ===== drl
%% ===== \end{flushleft}
%% ===== 
%% ===== 
%% ===== 
%% ===== 
%% ===== 
%% ===== \begin{flushleft}
%% ===== Derail
%% ===== \end{flushleft}
%% ===== 
%% ===== 
%% ===== 
%% ===== 
%% ===== 
%% ===== 15
%% ===== 
%% ===== 
%% ===== 
%% ===== 
%% ===== 
%% ===== 5
%% ===== 
%% ===== 
%% ===== 
%% ===== 
%% ===== 
%% ===== 7
%% ===== 
%% ===== 
%% ===== 
%% ===== 
%% ===== 
%% ===== 16
%% ===== 
%% ===== 
%% ===== 
%% ===== 
%% ===== 
%% ===== \begin{flushleft}
%% ===== luf
%% ===== \end{flushleft}
%% ===== 
%% ===== 
%% ===== 
%% ===== 
%% ===== 
%% ===== \begin{flushleft}
%% ===== Lockup
%% ===== \end{flushleft}
%% ===== 
%% ===== 
%% ===== 
%% ===== 
%% ===== 
%% ===== 5
%% ===== 
%% ===== 
%% ===== 
%% ===== 
%% ===== 
%% ===== 4
%% ===== 
%% ===== 
%% ===== 
%% ===== 
%% ===== 
%% ===== 8
%% ===== 
%% ===== 
%% ===== 
%% ===== 
%% ===== 
%% ===== 20
%% ===== 
%% ===== 
%% ===== 
%% ===== 
%% ===== 
%% ===== \begin{flushleft}
%% ===== con
%% ===== \end{flushleft}
%% ===== 
%% ===== 
%% ===== 
%% ===== 
%% ===== 
%% ===== \begin{flushleft}
%% ===== Connect
%% ===== \end{flushleft}
%% ===== 
%% ===== 
%% ===== 
%% ===== 
%% ===== 
%% ===== 25
%% ===== 
%% ===== 
%% ===== 
%% ===== 
%% ===== 
%% ===== 7
%% ===== 
%% ===== 
%% ===== 
%% ===== 
%% ===== 
%% ===== 9
%% ===== 
%% ===== 
%% ===== 
%% ===== 
%% ===== 
%% ===== 22
%% ===== 
%% ===== 
%% ===== 
%% ===== 
%% ===== 
%% ===== \begin{flushleft}
%% ===== par
%% ===== \end{flushleft}
%% ===== 
%% ===== 
%% ===== 
%% ===== 
%% ===== 
%% ===== \begin{flushleft}
%% ===== Parity
%% ===== \end{flushleft}
%% ===== 
%% ===== 
%% ===== 
%% ===== 
%% ===== 
%% ===== 8
%% ===== 
%% ===== 
%% ===== 
%% ===== 
%% ===== 
%% ===== 4
%% ===== 
%% ===== 
%% ===== 
%% ===== 
%% ===== 
%% ===== 10
%% ===== 
%% ===== 
%% ===== 
%% ===== 
%% ===== 
%% ===== 24
%% ===== 
%% ===== 
%% ===== 
%% ===== 
%% ===== 
%% ===== \begin{flushleft}
%% ===== ipr
%% ===== \end{flushleft}
%% ===== 
%% ===== 
%% ===== 
%% ===== 
%% ===== 
%% ===== \begin{flushleft}
%% ===== Illegal procedure
%% ===== \end{flushleft}
%% ===== 
%% ===== 
%% ===== 
%% ===== 
%% ===== 
%% ===== 16
%% ===== 
%% ===== 
%% ===== 
%% ===== 
%% ===== 
%% ===== 5
%% ===== 
%% ===== 
%% ===== 
%% ===== 
%% ===== 
%% ===== 11
%% ===== 
%% ===== 
%% ===== 
%% ===== 
%% ===== 
%% ===== 26
%% ===== 
%% ===== 
%% ===== 
%% ===== 
%% ===== 
%% ===== \begin{flushleft}
%% ===== onc
%% ===== \end{flushleft}
%% ===== 
%% ===== 
%% ===== 
%% ===== 
%% ===== 
%% ===== \begin{flushleft}
%% ===== Operation not
%% ===== \end{flushleft}
%% ===== 
%% ===== 
%% ===== \begin{flushleft}
%% ===== complete
%% ===== \end{flushleft}
%% ===== 
%% ===== 
%% ===== 
%% ===== 
%% ===== 
%% ===== 4
%% ===== 
%% ===== 
%% ===== 
%% ===== 
%% ===== 
%% ===== 2
%% ===== 
%% ===== 
%% ===== 
%% ===== 
%% ===== 
%% ===== 12
%% ===== 
%% ===== 
%% ===== 
%% ===== 
%% ===== 
%% ===== 30
%% ===== 
%% ===== 
%% ===== 
%% ===== 
%% ===== 
%% ===== \begin{flushleft}
%% ===== suf
%% ===== \end{flushleft}
%% ===== 
%% ===== 
%% ===== 
%% ===== 
%% ===== 
%% ===== \begin{flushleft}
%% ===== Startup
%% ===== \end{flushleft}
%% ===== 
%% ===== 
%% ===== 
%% ===== 
%% ===== 
%% ===== 1
%% ===== 
%% ===== 
%% ===== 
%% ===== 
%% ===== 
%% ===== 1
%% ===== 
%% ===== 
%% ===== 
%% ===== 
%% ===== 
%% ===== 13
%% ===== 
%% ===== 
%% ===== 
%% ===== 
%% ===== 
%% ===== 32
%% ===== 
%% ===== 
%% ===== 
%% ===== 
%% ===== 
%% ===== \begin{flushleft}
%% ===== ofl
%% ===== \end{flushleft}
%% ===== 
%% ===== 
%% ===== 
%% ===== 
%% ===== 
%% ===== \begin{flushleft}
%% ===== Overflow
%% ===== \end{flushleft}
%% ===== 
%% ===== 
%% ===== 
%% ===== 
%% ===== 
%% ===== 7
%% ===== 
%% ===== 
%% ===== 
%% ===== 
%% ===== 
%% ===== 3
%% ===== 
%% ===== 
%% ===== 
%% ===== 
%% ===== 
%% ===== 14
%% ===== 
%% ===== 
%% ===== 
%% ===== 
%% ===== 
%% ===== 34
%% ===== 
%% ===== 
%% ===== 
%% ===== 
%% ===== 
%% ===== \begin{flushleft}
%% ===== div
%% ===== \end{flushleft}
%% ===== 
%% ===== 
%% ===== 
%% ===== 
%% ===== 
%% ===== \begin{flushleft}
%% ===== Divide check
%% ===== \end{flushleft}
%% ===== 
%% ===== 
%% ===== 
%% ===== 
%% ===== 
%% ===== 6
%% ===== 
%% ===== 
%% ===== 
%% ===== 
%% ===== 
%% ===== 3
%% ===== 
%% ===== 
%% ===== 
%% ===== 
%% ===== 
%% ===== 15
%% ===== 
%% ===== 
%% ===== 
%% ===== 
%% ===== 
%% ===== 36
%% ===== 
%% ===== 
%% ===== 
%% ===== 
%% ===== 
%% ===== \begin{flushleft}
%% ===== exf
%% ===== \end{flushleft}
%% ===== 
%% ===== 
%% ===== 
%% ===== 
%% ===== 
%% ===== \begin{flushleft}
%% ===== Execute
%% ===== \end{flushleft}
%% ===== 
%% ===== 
%% ===== 
%% ===== 
%% ===== 
%% ===== 2
%% ===== 
%% ===== 
%% ===== 
%% ===== 
%% ===== 
%% ===== 1
%% ===== 
%% ===== 
%% ===== 
%% ===== 
%% ===== 
%% ===== 16
%% ===== 
%% ===== 
%% ===== 
%% ===== 
%% ===== 
%% ===== 40
%% ===== 
%% ===== 
%% ===== 
%% ===== 
%% ===== 
%% ===== \begin{flushleft}
%% ===== df0
%% ===== \end{flushleft}
%% ===== 
%% ===== 
%% ===== 
%% ===== 
%% ===== 
%% ===== \begin{flushleft}
%% ===== Directed fault 0
%% ===== \end{flushleft}
%% ===== 
%% ===== 
%% ===== 
%% ===== 
%% ===== 
%% ===== 20
%% ===== 
%% ===== 
%% ===== 
%% ===== 
%% ===== 
%% ===== 6
%% ===== 
%% ===== 
%% ===== 
%% ===== 
%% ===== 
%% ===== 17
%% ===== 
%% ===== 
%% ===== 
%% ===== 
%% ===== 
%% ===== 42
%% ===== 
%% ===== 
%% ===== 
%% ===== 
%% ===== 
%% ===== \begin{flushleft}
%% ===== df1
%% ===== \end{flushleft}
%% ===== 
%% ===== 
%% ===== 
%% ===== 
%% ===== 
%% ===== \begin{flushleft}
%% ===== Directed fault 1
%% ===== \end{flushleft}
%% ===== 
%% ===== 
%% ===== 
%% ===== 
%% ===== 
%% ===== 21
%% ===== 
%% ===== 
%% ===== 
%% ===== 
%% ===== 
%% ===== 6
%% ===== 
%% ===== 
%% ===== 
%% ===== 
%% ===== 
%% ===== 18
%% ===== 
%% ===== 
%% ===== 
%% ===== 
%% ===== 
%% ===== 44
%% ===== 
%% ===== 
%% ===== 
%% ===== 
%% ===== 
%% ===== \begin{flushleft}
%% ===== df2
%% ===== \end{flushleft}
%% ===== 
%% ===== 
%% ===== 
%% ===== 
%% ===== 
%% ===== \begin{flushleft}
%% ===== Directed fault 2
%% ===== \end{flushleft}
%% ===== 
%% ===== 
%% ===== 
%% ===== 
%% ===== 
%% ===== 22
%% ===== 
%% ===== 
%% ===== 
%% ===== 
%% ===== 
%% ===== 6
%% ===== 
%% ===== 
%% ===== 
%% ===== 
%% ===== 
%% ===== 19
%% ===== 
%% ===== 
%% ===== 
%% ===== 
%% ===== 
%% ===== 46
%% ===== 
%% ===== 
%% ===== 
%% ===== 
%% ===== 
%% ===== \begin{flushleft}
%% ===== df3
%% ===== \end{flushleft}
%% ===== 
%% ===== 
%% ===== 
%% ===== 
%% ===== 
%% ===== \begin{flushleft}
%% ===== Directed fault 3
%% ===== \end{flushleft}
%% ===== 
%% ===== 
%% ===== 
%% ===== 
%% ===== 
%% ===== 23
%% ===== 
%% ===== 
%% ===== 
%% ===== 
%% ===== 
%% ===== 6
%% ===== 
%% ===== 
%% ===== 
%% ===== 
%% ===== 
%% ===== 20
%% ===== 
%% ===== 
%% ===== 
%% ===== 
%% ===== 
%% ===== 50
%% ===== 
%% ===== 
%% ===== 
%% ===== 
%% ===== 
%% ===== \begin{flushleft}
%% ===== acv
%% ===== \end{flushleft}
%% ===== 
%% ===== 
%% ===== 
%% ===== 
%% ===== 
%% ===== \begin{flushleft}
%% ===== Access violation
%% ===== \end{flushleft}
%% ===== 
%% ===== 
%% ===== 
%% ===== 
%% ===== 
%% ===== 24
%% ===== 
%% ===== 
%% ===== 
%% ===== 
%% ===== 
%% ===== 6
%% ===== 
%% ===== 
%% ===== 
%% ===== 
%% ===== 
%% ===== 21
%% ===== 
%% ===== 
%% ===== 
%% ===== 
%% ===== 
%% ===== 52
%% ===== 
%% ===== 
%% ===== 
%% ===== 
%% ===== 
%% ===== \begin{flushleft}
%% ===== mme2
%% ===== \end{flushleft}
%% ===== 
%% ===== 
%% ===== 
%% ===== 
%% ===== 
%% ===== \begin{flushleft}
%% ===== Master mode entry 2
%% ===== \end{flushleft}
%% ===== 
%% ===== 
%% ===== 
%% ===== 
%% ===== 
%% ===== 12
%% ===== 
%% ===== 
%% ===== 
%% ===== 
%% ===== 
%% ===== 5
%% ===== 
%% ===== 
%% ===== 
%% ===== 
%% ===== 
%% ===== 22
%% ===== 
%% ===== 
%% ===== 
%% ===== 
%% ===== 
%% ===== 54
%% ===== 
%% ===== 
%% ===== 
%% ===== 
%% ===== 
%% ===== \begin{flushleft}
%% ===== mme3
%% ===== \end{flushleft}
%% ===== 
%% ===== 
%% ===== 
%% ===== 
%% ===== 
%% ===== \begin{flushleft}
%% ===== Master mode entry 3
%% ===== \end{flushleft}
%% ===== 
%% ===== 
%% ===== 
%% ===== 
%% ===== 
%% ===== 13
%% ===== 
%% ===== 
%% ===== 
%% ===== 
%% ===== 
%% ===== 5
%% ===== 
%% ===== 
%% ===== 
%% ===== 
%% ===== 
%% ===== 23
%% ===== 
%% ===== 
%% ===== 
%% ===== 
%% ===== 
%% ===== 56
%% ===== 
%% ===== 
%% ===== 
%% ===== 
%% ===== 
%% ===== \begin{flushleft}
%% ===== mme4
%% ===== \end{flushleft}
%% ===== 
%% ===== 
%% ===== 
%% ===== 
%% ===== 
%% ===== \begin{flushleft}
%% ===== Master mode entry 4
%% ===== \end{flushleft}
%% ===== 
%% ===== 
%% ===== 
%% ===== 
%% ===== 
%% ===== 14
%% ===== 
%% ===== 
%% ===== 
%% ===== 
%% ===== 
%% ===== 5
%% ===== 
%% ===== 
%% ===== 
%% ===== 
%% ===== 
%% ===== 24
%% ===== 
%% ===== 
%% ===== 
%% ===== 
%% ===== 
%% ===== 60
%% ===== 
%% ===== 
%% ===== 
%% ===== 
%% ===== 
%% ===== \begin{flushleft}
%% ===== f2
%% ===== \end{flushleft}
%% ===== 
%% ===== 
%% ===== 
%% ===== 
%% ===== 
%% ===== \begin{flushleft}
%% ===== Fault tag 2
%% ===== \end{flushleft}
%% ===== 
%% ===== 
%% ===== 
%% ===== 
%% ===== 
%% ===== 18
%% ===== 
%% ===== 
%% ===== 
%% ===== 
%% ===== 
%% ===== 5
%% ===== 
%% ===== 
%% ===== 
%% ===== 
%% ===== 
%% ===== 25
%% ===== 
%% ===== 
%% ===== 
%% ===== 
%% ===== 
%% ===== 62
%% ===== 
%% ===== 
%% ===== 
%% ===== 
%% ===== 
%% ===== \begin{flushleft}
%% ===== f3
%% ===== \end{flushleft}
%% ===== 
%% ===== 
%% ===== 
%% ===== 
%% ===== 
%% ===== \begin{flushleft}
%% ===== Fault tag 3
%% ===== \end{flushleft}
%% ===== 
%% ===== 
%% ===== 
%% ===== 
%% ===== 
%% ===== 19
%% ===== 
%% ===== 
%% ===== 
%% ===== 
%% ===== 
%% ===== 5
%% ===== 
%% ===== 
%% ===== 
%% ===== 
%% ===== 
%% ===== 26
%% ===== 
%% ===== 
%% ===== 
%% ===== 
%% ===== 
%% ===== 64
%% ===== 
%% ===== 
%% ===== 
%% ===== 
%% ===== 
%% ===== \begin{flushleft}
%% ===== Unassigned
%% ===== \end{flushleft}
%% ===== 
%% ===== 
%% ===== 
%% ===== 
%% ===== 
%% ===== 27
%% ===== 
%% ===== 
%% ===== 
%% ===== 
%% ===== 
%% ===== 66
%% ===== 
%% ===== 
%% ===== 
%% ===== 
%% ===== 
%% ===== \begin{flushleft}
%% ===== Unassigned
%% ===== \end{flushleft}
%% ===== 
%% ===== 
%% ===== 
%% ===== 
%% ===== 
%% ===== \begin{flushleft}
%% ===== Fault name
%% ===== \end{flushleft}
%% ===== 
%% ===== 
%% ===== 
%% ===== 
%% ===== 
%% ===== \begin{flushleft}
%% ===== Command
%% ===== \end{flushleft}
%% ===== 
%% ===== 
%% ===== 
%% ===== 
%% ===== 
%% ===== \begin{flushleft}
%% ===== Priority Group
%% ===== \end{flushleft}
%% ===== 
%% ===== 
%% ===== 
%% ===== 
%% ===== 
%% ===== \begin{flushleft}
%% ===== \newpage
%% ===== Decimal fault
%% ===== \end{flushleft}
%% ===== 
%% ===== 
%% ===== \begin{flushleft}
%% ===== number
%% ===== \end{flushleft}
%% ===== 
%% ===== 
%% ===== 
%% ===== 
%% ===== 
%% ===== \begin{flushleft}
%% ===== Octal (1)
%% ===== \end{flushleft}
%% ===== 
%% ===== 
%% ===== \begin{flushleft}
%% ===== fault
%% ===== \end{flushleft}
%% ===== 
%% ===== 
%% ===== \begin{flushleft}
%% ===== address
%% ===== \end{flushleft}
%% ===== 
%% ===== 
%% ===== 
%% ===== 
%% ===== 
%% ===== 28
%% ===== 
%% ===== 
%% ===== 
%% ===== 
%% ===== 
%% ===== 70
%% ===== 
%% ===== 
%% ===== 
%% ===== 
%% ===== 
%% ===== \begin{flushleft}
%% ===== Unassigned
%% ===== \end{flushleft}
%% ===== 
%% ===== 
%% ===== 
%% ===== 
%% ===== 
%% ===== 29
%% ===== 
%% ===== 
%% ===== 
%% ===== 
%% ===== 
%% ===== 72
%% ===== 
%% ===== 
%% ===== 
%% ===== 
%% ===== 
%% ===== \begin{flushleft}
%% ===== Unassigned
%% ===== \end{flushleft}
%% ===== 
%% ===== 
%% ===== 
%% ===== 
%% ===== 
%% ===== 30
%% ===== 
%% ===== 
%% ===== 
%% ===== 
%% ===== 
%% ===== 74
%% ===== 
%% ===== 
%% ===== 
%% ===== 
%% ===== 
%% ===== \begin{flushleft}
%% ===== Unassigned
%% ===== \end{flushleft}
%% ===== 
%% ===== 
%% ===== 
%% ===== 
%% ===== 
%% ===== 31
%% ===== 
%% ===== 
%% ===== 
%% ===== 
%% ===== 
%% ===== 76
%% ===== 
%% ===== 
%% ===== 
%% ===== 
%% ===== 
%% ===== \begin{flushleft}
%% ===== Fault
%% ===== \end{flushleft}
%% ===== 
%% ===== 
%% ===== \begin{flushleft}
%% ===== mnemonic
%% ===== \end{flushleft}
%% ===== 
%% ===== 
%% ===== 
%% ===== 
%% ===== 
%% ===== \begin{flushleft}
%% ===== trb
%% ===== \end{flushleft}
%% ===== 
%% ===== 
%% ===== 
%% ===== 
%% ===== 
%% ===== \begin{flushleft}
%% ===== Fault name
%% ===== \end{flushleft}
%% ===== 
%% ===== 
%% ===== 
%% ===== 
%% ===== 
%% ===== \begin{flushleft}
%% ===== Trouble
%% ===== \end{flushleft}
%% ===== 
%% ===== 
%% ===== 
%% ===== 
%% ===== 
%% ===== \begin{flushleft}
%% ===== Priority Group
%% ===== \end{flushleft}
%% ===== 
%% ===== 
%% ===== 
%% ===== 
%% ===== 
%% ===== 3
%% ===== 
%% ===== 
%% ===== 
%% ===== 
%% ===== 
%% ===== 2
%% ===== 
%% ===== 
%% ===== 
%% ===== 
%% ===== 
%% ===== \begin{flushleft}
%% ===== (1)The octal fault address value is the value concatenated with the FAULT BASE switch setting in
%% ===== \end{flushleft}
%% ===== 
%% ===== 
%% ===== \begin{flushleft}
%% ===== forming the computed address for the fault trap pair.
%% ===== \end{flushleft}
%% ===== 
%% ===== 
%% ===== 
%% ===== 
%% ===== 
%% ===== \begin{flushleft}

\subsection{FAULT PRIORITY}

%% ===== \end{flushleft}
%% ===== 
%% ===== 
%% ===== \begin{flushleft}
%% ===== The processor has provision for 32 faults of which 27 are implemented. The faults are
%% ===== \end{flushleft}
%% ===== 
%% ===== 
%% ===== \begin{flushleft}
%% ===== classified into seven fault priority groups that roughly correspond to the severity of the faults.
%% ===== \end{flushleft}
%% ===== 
%% ===== 
%% ===== \begin{flushleft}
%% ===== Fault priority groups are defined so that fault recognition precedence may be established when
%% ===== \end{flushleft}
%% ===== 
%% ===== 
%% ===== \begin{flushleft}
%% ===== two or more faults exist concurrently. Overlapped and asynchronous functions in the processor
%% ===== \end{flushleft}
%% ===== 
%% ===== 
%% ===== \begin{flushleft}
%% ===== allow the simultaneous occurrence of faults. Group 1 has the highest priority and group 7 has the
%% ===== \end{flushleft}
%% ===== 
%% ===== 
%% ===== \begin{flushleft}
%% ===== lowest. In groups 1 through 6, only one fault within each group is allowed to be active at any one
%% ===== \end{flushleft}
%% ===== 
%% ===== 
%% ===== \begin{flushleft}
%% ===== time. The first fault within a group occurring through the normal program sequence is the one
%% ===== \end{flushleft}
%% ===== 
%% ===== 
%% ===== \begin{flushleft}
%% ===== serviced.
%% ===== \end{flushleft}
%% ===== 
%% ===== 
%% ===== \begin{flushleft}
%% ===== Group 7 faults are saved by the hardware for eventual recognition. In the case of
%% ===== \end{flushleft}
%% ===== 
%% ===== 
%% ===== \begin{flushleft}
%% ===== simultaneous faults within group 7, shutdown has the highest priority with timer runout next and
%% ===== \end{flushleft}
%% ===== 
%% ===== 
%% ===== \begin{flushleft}
%% ===== connect the lowest.
%% ===== \end{flushleft}
%% ===== 
%% ===== 
%% ===== \begin{flushleft}
%% ===== There is a single exception to the handling of faults in priority group order. If an operand
%% ===== \end{flushleft}
%% ===== 
%% ===== 
%% ===== \begin{flushleft}
%% ===== fetch generates a parity fault and the use of the operand in {``}closing out'' instruction execution
%% ===== \end{flushleft}
%% ===== 
%% ===== 
%% ===== \begin{flushleft}
%% ===== generates an overflow fault or a divide check fault, these faults are considered simultaneous but
%% ===== \end{flushleft}
%% ===== 
%% ===== 
%% ===== \begin{flushleft}
%% ===== the parity fault takes precedence.
%% ===== \end{flushleft}
%% ===== 
%% ===== 
%% ===== 
%% ===== 
%% ===== 
%% ===== \begin{flushleft}

\subsection{FAULT RECOGNITION}

%% ===== \end{flushleft}
%% ===== 
%% ===== 
%% ===== \begin{flushleft}
%% ===== For the discussion following, the term {``}function'' is defined as a major processor functional
%% ===== \end{flushleft}
%% ===== 
%% ===== 
%% ===== \begin{flushleft}
%% ===== cycle. Examples are: APPEND CYCLE, CA CYCLE, INSTRUCTION FETCH CYCLE, OPERAND
%% ===== \end{flushleft}
%% ===== 
%% ===== 
%% ===== \begin{flushleft}
%% ===== STORE CYCLE, DIVIDE EXECUTION CYCLE. Some of these cycles are discussed in various
%% ===== \end{flushleft}
%% ===== 
%% ===== 
%% ===== \begin{flushleft}
%% ===== sections of this manual.
%% ===== \end{flushleft}
%% ===== 
%% ===== 
%% ===== \begin{flushleft}
%% ===== Faults in groups 1 and 2 cause the processor to abort all functions immediately by entering
%% ===== \end{flushleft}
%% ===== 
%% ===== 
%% ===== \begin{flushleft}
%% ===== a FAULT CYCLE.
%% ===== \end{flushleft}
%% ===== 
%% ===== 
%% ===== \begin{flushleft}
%% ===== Faults in group 3 cause the processor to {``}close out'' current functions without taking any
%% ===== \end{flushleft}
%% ===== 
%% ===== 
%% ===== \begin{flushleft}
%% ===== irrevocable action (such as setting PTW.U in an APPEND CYCLE or modifying an indirect word in a
%% ===== \end{flushleft}
%% ===== 
%% ===== 
%% ===== \begin{flushleft}
%% ===== CA CYCLE), then to discard any pending functions (such as an APPEND CYCLE needed during a
%% ===== \end{flushleft}
%% ===== 
%% ===== 
%% ===== \begin{flushleft}
%% ===== CA CYCLE), and to enter a FAULT CYCLE.
%% ===== \end{flushleft}
%% ===== 
%% ===== 
%% ===== \begin{flushleft}
%% ===== Faults in group 4 cause the processor to suspend overlapped operation, to complete current
%% ===== \end{flushleft}
%% ===== 
%% ===== 
%% ===== \begin{flushleft}
%% ===== and pending functions for the current instruction, and then to enter a FAULT CYCLE.
%% ===== \end{flushleft}
%% ===== 
%% ===== 
%% ===== \begin{flushleft}
%% ===== Faults in groups 5 or 6 are normally detected during virtual address formation and
%% ===== \end{flushleft}
%% ===== 
%% ===== 
%% ===== \begin{flushleft}
%% ===== instruction decode. These faults cause the processor to suspend overlapped operation, to complete
%% ===== \end{flushleft}
%% ===== 
%% ===== 
%% ===== \begin{flushleft}
%% ===== the current and pending instructions, and to enter a FAULT CYCLE. If a fault in a higher priority
%% ===== \end{flushleft}
%% ===== 
%% ===== 
%% ===== \begin{flushleft}
%% ===== group is generated by the execution of the current or pending instructions, that higher priority
%% ===== \end{flushleft}
%% ===== 
%% ===== 
%% ===== \begin{flushleft}
%% ===== fault will take precedence and the group 5 or 6 fault will be lost. If a group 5 or 6 fault is detected
%% ===== \end{flushleft}
%% ===== 
%% ===== 
%% ===== \begin{flushleft}
%% ===== during execution of the current instruction (e.g., an access violation, out of segment bounds, fault
%% ===== \end{flushleft}
%% ===== 
%% ===== 
%% ===== 
%% ===== 
%% ===== 
%% ===== \begin{flushleft}
%% ===== \newpage
%% ===== during certain interruptible EIS instructions), the instruction is considered {``}complete'' upon
%% ===== \end{flushleft}
%% ===== 
%% ===== 
%% ===== \begin{flushleft}
%% ===== detection of the fault.
%% ===== \end{flushleft}
%% ===== 
%% ===== 
%% ===== \begin{flushleft}
%% ===== Faults in group 7 are held and processed (with interrupts) at the completion of the current
%% ===== \end{flushleft}
%% ===== 
%% ===== 
%% ===== \begin{flushleft}
%% ===== instruction pair. Group 7 faults are inhibitable by setting bit 28 of the instruction word.
%% ===== \end{flushleft}
%% ===== 
%% ===== 
%% ===== \begin{flushleft}
%% ===== Faults in groups 3 through 6 must wait for the system controller to acknowledge the last
%% ===== \end{flushleft}
%% ===== 
%% ===== 
%% ===== \begin{flushleft}
%% ===== access request before entering the FAULT CYCLE.
%% ===== \end{flushleft}
%% ===== 
%% ===== 
%% ===== 
%% ===== 
%% ===== 
%% ===== \begin{flushleft}

\subsection{FAULT DESCRIPTIONS}

%% ===== \end{flushleft}
%% ===== 
%% ===== 
%% ===== \begin{flushleft}

\subsubsection{Group 1 Faults}

%% ===== \end{flushleft}
%% ===== 
%% ===== 
%% ===== \begin{flushleft}
%% ===== Startup
%% ===== \end{flushleft}
%% ===== 
%% ===== 
%% ===== \begin{flushleft}
%% ===== DC POWER has been turned on. When the POWER ON button is pressed, the
%% ===== \end{flushleft}
%% ===== 
%% ===== 
%% ===== \begin{flushleft}
%% ===== processor is first initialized and then the startup fault is generated.
%% ===== \end{flushleft}
%% ===== 
%% ===== 
%% ===== \begin{flushleft}
%% ===== Execute
%% ===== \end{flushleft}
%% ===== 
%% ===== 
%% ===== \begin{flushleft}
%% ===== 1. The EXECUTE pushbutton on the processor maintenance panel has been pressed.
%% ===== \end{flushleft}
%% ===== 
%% ===== 
%% ===== \begin{flushleft}
%% ===== 2. An external gate signal has been substituted for the EXECUTE pushbutton.
%% ===== \end{flushleft}
%% ===== 
%% ===== 
%% ===== \begin{flushleft}
%% ===== The selection between the above conditions is made by settings of various switches on
%% ===== \end{flushleft}
%% ===== 
%% ===== 
%% ===== \begin{flushleft}
%% ===== the processor maintenance panel.
%% ===== \end{flushleft}
%% ===== 
%% ===== 
%% ===== 
%% ===== 
%% ===== 
%% ===== \begin{flushleft}

\subsubsection{Group 2 Faults}

%% ===== \end{flushleft}
%% ===== 
%% ===== 
%% ===== \begin{flushleft}
%% ===== Operation Not Complete
%% ===== \end{flushleft}
%% ===== 
%% ===== 
%% ===== \begin{flushleft}
%% ===== Any of the following will cause an operation not complete fault:
%% ===== \end{flushleft}
%% ===== 
%% ===== 
%% ===== \begin{flushleft}
%% ===== 1. The processor has addressed a system controller to which it is not attached, that is,
%% ===== \end{flushleft}
%% ===== 
%% ===== 
%% ===== \begin{flushleft}
%% ===== there is no main memory interface port having its ADDRESS ASSIGNMENT
%% ===== \end{flushleft}
%% ===== 
%% ===== 
%% ===== \begin{flushleft}
%% ===== switches set to a value including the main memory address desired.
%% ===== \end{flushleft}
%% ===== 
%% ===== 
%% ===== \begin{flushleft}
%% ===== 2. The addressed system controller has failed to acknowledge the processor.
%% ===== \end{flushleft}
%% ===== 
%% ===== 
%% ===== \begin{flushleft}
%% ===== 3. The processor has not generated a main memory access request or a direct
%% ===== \end{flushleft}
%% ===== 
%% ===== 
%% ===== \begin{flushleft}
%% ===== operand within 1 to 2 milliseconds and is not executing the Delay Until Interrupt
%% ===== \end{flushleft}
%% ===== 
%% ===== 
%% ===== \begin{flushleft}
%% ===== Signal (dis) instruction.
%% ===== \end{flushleft}
%% ===== 
%% ===== 
%% ===== \begin{flushleft}
%% ===== 4. A main memory interface port received a data strobe without a preceding
%% ===== \end{flushleft}
%% ===== 
%% ===== 
%% ===== \begin{flushleft}
%% ===== acknowledgment from the system controller that it had received the access
%% ===== \end{flushleft}
%% ===== 
%% ===== 
%% ===== \begin{flushleft}
%% ===== request.
%% ===== \end{flushleft}
%% ===== 
%% ===== 
%% ===== \begin{flushleft}
%% ===== 5. A main memory interface port received a data strobe before the data previously
%% ===== \end{flushleft}
%% ===== 
%% ===== 
%% ===== \begin{flushleft}
%% ===== sent to it was unloaded.
%% ===== \end{flushleft}
%% ===== 
%% ===== 
%% ===== \begin{flushleft}
%% ===== Trouble
%% ===== \end{flushleft}
%% ===== 
%% ===== 
%% ===== \begin{flushleft}
%% ===== The trouble fault is defined as the occurrence of a fault during the fetch or execution
%% ===== \end{flushleft}
%% ===== 
%% ===== 
%% ===== \begin{flushleft}
%% ===== of a fault trap pair or interrupt trap pair. Such faults may be hardware generated (for
%% ===== \end{flushleft}
%% ===== 
%% ===== 
%% ===== \begin{flushleft}
%% ===== example, operation not complete or parity), or operating system generated (e.g., the
%% ===== \end{flushleft}
%% ===== 
%% ===== 
%% ===== \begin{flushleft}
%% ===== page containing a trap pair instruction operand is missing).
%% ===== \end{flushleft}
%% ===== 
%% ===== 
%% ===== 
%% ===== 
%% ===== 
%% ===== \begin{flushleft}
%% ===== \newpage

\subsubsection{Group 3 Faults}

%% ===== \end{flushleft}
%% ===== 
%% ===== 
%% ===== \begin{flushleft}
%% ===== Overflow
%% ===== \end{flushleft}
%% ===== 
%% ===== 
%% ===== \begin{flushleft}
%% ===== An arithmetic overflow, exponent overflow, exponent underflow, or EIS truncation
%% ===== \end{flushleft}
%% ===== 
%% ===== 
%% ===== \begin{flushleft}
%% ===== fault has been generated. The generation of this fault is inhibited when the overflow
%% ===== \end{flushleft}
%% ===== 
%% ===== 
%% ===== \begin{flushleft}
%% ===== mask indicator is ON. Resetting of the overflow mask indicator to OFF does not
%% ===== \end{flushleft}
%% ===== 
%% ===== 
%% ===== \begin{flushleft}
%% ===== generate a fault from previously set indicators. The overflow mask state does not
%% ===== \end{flushleft}
%% ===== 
%% ===== 
%% ===== \begin{flushleft}
%% ===== affect the setting, testing or storing of indicators. The determination of the specific
%% ===== \end{flushleft}
%% ===== 
%% ===== 
%% ===== \begin{flushleft}
%% ===== overflow condition is by indicator testing by the operating supervisor.
%% ===== \end{flushleft}
%% ===== 
%% ===== 
%% ===== \begin{flushleft}
%% ===== Divide Check
%% ===== \end{flushleft}
%% ===== 
%% ===== 
%% ===== \begin{flushleft}
%% ===== A divide check fault occurs when the actual division cannot be carried out for one of
%% ===== \end{flushleft}
%% ===== 
%% ===== 
%% ===== \begin{flushleft}
%% ===== the reasons specified with individual divide instructions.
%% ===== \end{flushleft}
%% ===== 
%% ===== 
%% ===== 
%% ===== 
%% ===== 
%% ===== \begin{flushleft}

\subsubsection{Group 4 Faults}

%% ===== \end{flushleft}
%% ===== 
%% ===== 
%% ===== \begin{flushleft}
%% ===== Store
%% ===== \end{flushleft}
%% ===== 
%% ===== 
%% ===== \begin{flushleft}
%% ===== The processor attempted to select a disabled port, an out-of-bounds address was
%% ===== \end{flushleft}
%% ===== 
%% ===== 
%% ===== \begin{flushleft}
%% ===== generated in the BAR mode or absolute mode, or an attempt was made to access a
%% ===== \end{flushleft}
%% ===== 
%% ===== 
%% ===== \begin{flushleft}
%% ===== store unit that was not ready.
%% ===== \end{flushleft}
%% ===== 
%% ===== 
%% ===== \begin{flushleft}
%% ===== Command
%% ===== \end{flushleft}
%% ===== 
%% ===== 
%% ===== \begin{flushleft}
%% ===== 1. The processor attempted to load or read the interrupt mask register in a system
%% ===== \end{flushleft}
%% ===== 
%% ===== 
%% ===== \begin{flushleft}
%% ===== controller in which it did not have an interrupt mask assigned.
%% ===== \end{flushleft}
%% ===== 
%% ===== 
%% ===== \begin{flushleft}
%% ===== 2. The processor issued an XEC system controller command to a system controller in
%% ===== \end{flushleft}
%% ===== 
%% ===== 
%% ===== \begin{flushleft}
%% ===== which it did not have an interrupt mask assigned.
%% ===== \end{flushleft}
%% ===== 
%% ===== 
%% ===== \begin{flushleft}
%% ===== 3. The processor issued a connect to a system controller port that is masked OFF.
%% ===== \end{flushleft}
%% ===== 
%% ===== 
%% ===== \begin{flushleft}
%% ===== 4. The selected system controller is in TEST mode and a condition determined by
%% ===== \end{flushleft}
%% ===== 
%% ===== 
%% ===== \begin{flushleft}
%% ===== certain system controller maintenance panel switches has been trapped.
%% ===== \end{flushleft}
%% ===== 
%% ===== 
%% ===== \begin{flushleft}
%% ===== 5. An attempt was made to load a pointer register with packed pointer data in which
%% ===== \end{flushleft}
%% ===== 
%% ===== 
%% ===== \begin{flushleft}
%% ===== the BITNO field value was greater than or equal to 60(8).
%% ===== \end{flushleft}
%% ===== 
%% ===== 
%% ===== \begin{flushleft}
%% ===== Lockup
%% ===== \end{flushleft}
%% ===== 
%% ===== 
%% ===== \begin{flushleft}
%% ===== The program is in a code sequence which has inhibited sampling for interrupts
%% ===== \end{flushleft}
%% ===== 
%% ===== 
%% ===== \begin{flushleft}
%% ===== (whether present or not) and group 7 faults for longer than the prescribed time. In
%% ===== \end{flushleft}
%% ===== 
%% ===== 
%% ===== \begin{flushleft}
%% ===== absolute mode or privileged mode the lockup time is 32 milliseconds. In normal mode
%% ===== \end{flushleft}
%% ===== 
%% ===== 
%% ===== \begin{flushleft}
%% ===== or BAR mode the lockup time is specified by the setting for the lockup time in the
%% ===== \end{flushleft}
%% ===== 
%% ===== 
%% ===== \begin{flushleft}
%% ===== cache mode register. The lockup time is program settable to 2, 4, 8, or 16
%% ===== \end{flushleft}
%% ===== 
%% ===== 
%% ===== \begin{flushleft}
%% ===== milliseconds.
%% ===== \end{flushleft}
%% ===== 
%% ===== 
%% ===== \begin{flushleft}
%% ===== While in absolute mode or privileged mode the lockup fault is signalled at the end of
%% ===== \end{flushleft}
%% ===== 
%% ===== 
%% ===== \begin{flushleft}
%% ===== the time limit set in the lockup timer but is not recognized until the 32 millisecond
%% ===== \end{flushleft}
%% ===== 
%% ===== 
%% ===== \begin{flushleft}
%% ===== limit. If the processor returns to normal mode or BAR mode after the fault has been
%% ===== \end{flushleft}
%% ===== 
%% ===== 
%% ===== \begin{flushleft}
%% ===== signalled but before the 32 millisecond limit, the fault is recognized before any
%% ===== \end{flushleft}
%% ===== 
%% ===== 
%% ===== \begin{flushleft}
%% ===== instruction in the new mode is executed.
%% ===== \end{flushleft}
%% ===== 
%% ===== 
%% ===== \begin{flushleft}
%% ===== Parity
%% ===== \end{flushleft}
%% ===== 
%% ===== 
%% ===== \begin{flushleft}
%% ===== 1. The selected system controller has returned an illegal action signal with an illegal
%% ===== \end{flushleft}
%% ===== 
%% ===== 
%% ===== \begin{flushleft}
%% ===== action code for one of the various main memory parity error conditions.
%% ===== \end{flushleft}
%% ===== 
%% ===== 
%% ===== 
%% ===== 
%% ===== 
%% ===== \begin{flushleft}
%% ===== \newpage
%% ===== 2. A cache memory data or directory parity error has occurred either for read, write,
%% ===== \end{flushleft}
%% ===== 
%% ===== 
%% ===== \begin{flushleft}
%% ===== or block load. Cache status bits for the condition have been set in the cache mode
%% ===== \end{flushleft}
%% ===== 
%% ===== 
%% ===== \begin{flushleft}
%% ===== register.
%% ===== \end{flushleft}
%% ===== 
%% ===== 
%% ===== \begin{flushleft}
%% ===== 3. The processor has detected a parity error in the system controller interface port
%% ===== \end{flushleft}
%% ===== 
%% ===== 
%% ===== \begin{flushleft}
%% ===== while either generating outgoing parity or verifying incoming parity.
%% ===== \end{flushleft}
%% ===== 
%% ===== 
%% ===== 
%% ===== 
%% ===== 
%% ===== \begin{flushleft}

\subsubsection{Group 5 Faults}

%% ===== \end{flushleft}
%% ===== 
%% ===== 
%% ===== \begin{flushleft}
%% ===== Master Mode Entries 1-4
%% ===== \end{flushleft}
%% ===== 
%% ===== 
%% ===== \begin{flushleft}
%% ===== The corresponding Master Mode Entry instruction has been decoded.
%% ===== \end{flushleft}
%% ===== 
%% ===== 
%% ===== \begin{flushleft}
%% ===== Fault Tags 1-3
%% ===== \end{flushleft}
%% ===== 
%% ===== 
%% ===== \begin{flushleft}
%% ===== The corresponding indirect then tally variation has been detected during virtual
%% ===== \end{flushleft}
%% ===== 
%% ===== 
%% ===== \begin{flushleft}
%% ===== address formation.
%% ===== \end{flushleft}
%% ===== 
%% ===== 
%% ===== \begin{flushleft}
%% ===== Derail
%% ===== \end{flushleft}
%% ===== 
%% ===== 
%% ===== \begin{flushleft}
%% ===== The Derail instruction has been decoded.
%% ===== \end{flushleft}
%% ===== 
%% ===== 
%% ===== \begin{flushleft}
%% ===== Illegal Procedure
%% ===== \end{flushleft}
%% ===== 
%% ===== 
%% ===== \begin{flushleft}
%% ===== 1. An illegal operation code has been decoded or an illegal instruction sequence has
%% ===== \end{flushleft}
%% ===== 
%% ===== 
%% ===== \begin{flushleft}
%% ===== been encountered.
%% ===== \end{flushleft}
%% ===== 
%% ===== 
%% ===== \begin{flushleft}
%% ===== 2. An illegal modifier or modifier sequence has been encountered during virtual
%% ===== \end{flushleft}
%% ===== 
%% ===== 
%% ===== \begin{flushleft}
%% ===== address formation.
%% ===== \end{flushleft}
%% ===== 
%% ===== 
%% ===== \begin{flushleft}
%% ===== 3. An illegal address has been given in an instruction for which the ADDRESS field is
%% ===== \end{flushleft}
%% ===== 
%% ===== 
%% ===== \begin{flushleft}
%% ===== used for register selection.
%% ===== \end{flushleft}
%% ===== 
%% ===== 
%% ===== \begin{flushleft}
%% ===== 4. An attempt was made to execute a privileged instruction in normal mode or BAR
%% ===== \end{flushleft}
%% ===== 
%% ===== 
%% ===== \begin{flushleft}
%% ===== mode.
%% ===== \end{flushleft}
%% ===== 
%% ===== 
%% ===== \begin{flushleft}
%% ===== 5. An illegal digit was encountered in a decimal numeric operand.
%% ===== \end{flushleft}
%% ===== 
%% ===== 
%% ===== \begin{flushleft}
%% ===== 6. An illegal specification was found in an EIS operand descriptor.
%% ===== \end{flushleft}
%% ===== 
%% ===== 
%% ===== \begin{flushleft}
%% ===== The conditions for the fault will be set in the fault register, word 1 of the Control Unit
%% ===== \end{flushleft}
%% ===== 
%% ===== 
%% ===== \begin{flushleft}
%% ===== Data, or in both.
%% ===== \end{flushleft}
%% ===== 
%% ===== 
%% ===== 
%% ===== 
%% ===== 
%% ===== \begin{flushleft}

\subsubsection{Group 6 Faults}

%% ===== \end{flushleft}
%% ===== 
%% ===== 
%% ===== \begin{flushleft}
%% ===== Directed Faults 0-3
%% ===== \end{flushleft}
%% ===== 
%% ===== 
%% ===== \begin{flushleft}
%% ===== A faulted segment descriptor word (SDW) or page table word (PTW) with the
%% ===== \end{flushleft}
%% ===== 
%% ===== 
%% ===== \begin{flushleft}
%% ===== corresponding directed fault number has been fetched by the appending unit.
%% ===== \end{flushleft}
%% ===== 
%% ===== 
%% ===== \begin{flushleft}
%% ===== Access Violation
%% ===== \end{flushleft}
%% ===== 
%% ===== 
%% ===== \begin{flushleft}
%% ===== The appending unit has detected one of the several access violations below. Word 1 of
%% ===== \end{flushleft}
%% ===== 
%% ===== 
%% ===== \begin{flushleft}
%% ===== the Control Unit Data contains status bits for the condition.
%% ===== \end{flushleft}
%% ===== 
%% ===== 
%% ===== \begin{flushleft}
%% ===== 1. Not in read bracket (ACV3=ORB)
%% ===== \end{flushleft}
%% ===== 
%% ===== 
%% ===== \begin{flushleft}
%% ===== 2. Not in write bracket (ACV5=OWB)
%% ===== \end{flushleft}
%% ===== 
%% ===== 
%% ===== 
%% ===== 
%% ===== 
%% ===== \begin{flushleft}
%% ===== \newpage
%% ===== 3. Not in execute bracket (ACV1=OEB)
%% ===== \end{flushleft}
%% ===== 
%% ===== 
%% ===== \begin{flushleft}
%% ===== 4. No read permission (ACV4=R-OFF)
%% ===== \end{flushleft}
%% ===== 
%% ===== 
%% ===== \begin{flushleft}
%% ===== 5. No write permission (ACV6=W-OFF)
%% ===== \end{flushleft}
%% ===== 
%% ===== 
%% ===== \begin{flushleft}
%% ===== 6. No execute permission (ACV2=E-OFF)
%% ===== \end{flushleft}
%% ===== 
%% ===== 
%% ===== \begin{flushleft}
%% ===== 7. Invalid ring crossing (ACV12=CRT)
%% ===== \end{flushleft}
%% ===== 
%% ===== 
%% ===== \begin{flushleft}
%% ===== 8. Call limiter fault (ACV7=NO GA)
%% ===== \end{flushleft}
%% ===== 
%% ===== 
%% ===== \begin{flushleft}
%% ===== 9. Outward call (ACV9=OCALL)
%% ===== \end{flushleft}
%% ===== 
%% ===== 
%% ===== \begin{flushleft}
%% ===== 10.Bad outward call (ACV10=BOC)
%% ===== \end{flushleft}
%% ===== 
%% ===== 
%% ===== \begin{flushleft}
%% ===== 11.Inward return (ACV11=INRET)
%% ===== \end{flushleft}
%% ===== 
%% ===== 
%% ===== \begin{flushleft}
%% ===== 12.Ring alarm (ACV13=RALR)
%% ===== \end{flushleft}
%% ===== 
%% ===== 
%% ===== \begin{flushleft}
%% ===== 13.Associative memory error
%% ===== \end{flushleft}
%% ===== 
%% ===== 
%% ===== \begin{flushleft}
%% ===== 14.Out of segment bounds (ACV15=OOSB)
%% ===== \end{flushleft}
%% ===== 
%% ===== 
%% ===== \begin{flushleft}
%% ===== 15.Illegal ring order (ACV0=IRO)
%% ===== \end{flushleft}
%% ===== 
%% ===== 
%% ===== \begin{flushleft}
%% ===== 16.Out of call brackets (ACV8=OCB)
%% ===== \end{flushleft}
%% ===== 
%% ===== 
%% ===== 
%% ===== 
%% ===== 
%% ===== \begin{flushleft}

\subsubsection{Group 7 Faults}

%% ===== \end{flushleft}
%% ===== 
%% ===== 
%% ===== \begin{flushleft}
%% ===== Shutdown
%% ===== \end{flushleft}
%% ===== 
%% ===== 
%% ===== \begin{flushleft}
%% ===== An external power shutdown condition has been detected. DC POWER shutdown will
%% ===== \end{flushleft}
%% ===== 
%% ===== 
%% ===== \begin{flushleft}
%% ===== occur in approximately one millisecond.
%% ===== \end{flushleft}
%% ===== 
%% ===== 
%% ===== \begin{flushleft}
%% ===== Timer Runout
%% ===== \end{flushleft}
%% ===== 
%% ===== 
%% ===== \begin{flushleft}
%% ===== The timer register has decremented to or through the value zero. If the processor is
%% ===== \end{flushleft}
%% ===== 
%% ===== 
%% ===== \begin{flushleft}
%% ===== in privileged mode or absolute mode, recognition of this fault is delayed until a return
%% ===== \end{flushleft}
%% ===== 
%% ===== 
%% ===== \begin{flushleft}
%% ===== to normal mode or BAR mode. Counting in the timer register continues.
%% ===== \end{flushleft}
%% ===== 
%% ===== 
%% ===== \begin{flushleft}
%% ===== Connect
%% ===== \end{flushleft}
%% ===== 
%% ===== 
%% ===== \begin{flushleft}
%% ===== A connect signal (\$CON strobe) has been received from a system controller. This
%% ===== \end{flushleft}
%% ===== 
%% ===== 
%% ===== \begin{flushleft}
%% ===== event is to be distinguished from a Connect Input/Output Channel (cioc) instruction
%% ===== \end{flushleft}
%% ===== 
%% ===== 
%% ===== \begin{flushleft}
%% ===== encountered in the program sequence.
%% ===== \end{flushleft}
%% ===== 
%% ===== 
%% ===== \begin{flushleft}
%% ===== (See the discussion of the floating faults in Section 3).
%% ===== \end{flushleft}
%% ===== 
%% ===== 
%% ===== 
%% ===== 
%% ===== 
%% ===== \begin{flushleft}

\subsection{INTERRUPTS AND EXTERNAL FAULTS}

%% ===== \end{flushleft}
%% ===== 
%% ===== 
%% ===== \begin{flushleft}
%% ===== Each system controller contains 32 interrupt cells that are used for communication among
%% ===== \end{flushleft}
%% ===== 
%% ===== 
%% ===== \begin{flushleft}
%% ===== the active system modules (processors, I/O multiplexers, etc.). The interrupt cells are organized in
%% ===== \end{flushleft}
%% ===== 
%% ===== 
%% ===== \begin{flushleft}
%% ===== a numbered priority chain. Any active system module connected to a system controller port may
%% ===== \end{flushleft}
%% ===== 
%% ===== 
%% ===== \begin{flushleft}
%% ===== request the setting of an interrupt cell with the SXC system controller command.
%% ===== \end{flushleft}
%% ===== 
%% ===== 
%% ===== \begin{flushleft}
%% ===== When one or more interrupt cells in a system controller is set, the system controller
%% ===== \end{flushleft}
%% ===== 
%% ===== 
%% ===== \begin{flushleft}
%% ===== activates the interrupt present (XIP) line to all system controller ports having an assigned interrupt
%% ===== \end{flushleft}
%% ===== 
%% ===== 
%% ===== 
%% ===== 
%% ===== 
%% ===== \begin{flushleft}
%% ===== \newpage
%% ===== mask in which one or more of the interrupt cells that are set is unmasked. Interrupt masks should
%% ===== \end{flushleft}
%% ===== 
%% ===== 
%% ===== \begin{flushleft}
%% ===== be assigned only to processors. Each interrupt cell has associated with it a unique interrupt trap
%% ===== \end{flushleft}
%% ===== 
%% ===== 
%% ===== \begin{flushleft}
%% ===== pair located at an absolute main memory address equal to twice the cell number.
%% ===== \end{flushleft}
%% ===== 
%% ===== 
%% ===== 
%% ===== 
%% ===== 
%% ===== \begin{flushleft}

\subsubsection{Interrupt Sampling}

%% ===== \end{flushleft}
%% ===== 
%% ===== 
%% ===== \begin{flushleft}
%% ===== The processor always fetches instructions in pairs. At an appropriate point (as early as
%% ===== \end{flushleft}
%% ===== 
%% ===== 
%% ===== \begin{flushleft}
%% ===== possible) in the execution of a pair of instructions, the next sequential instruction pair is fetched
%% ===== \end{flushleft}
%% ===== 
%% ===== 
%% ===== \begin{flushleft}
%% ===== and held in a special instruction buffer register. The exact point depends on instruction sequence
%% ===== \end{flushleft}
%% ===== 
%% ===== 
%% ===== \begin{flushleft}
%% ===== and other conditions
%% ===== \end{flushleft}
%% ===== 
%% ===== 
%% ===== \begin{flushleft}
%% ===== If the interrupt inhibit bit (bit 28) is not set in the current instruction word at the point of
%% ===== \end{flushleft}
%% ===== 
%% ===== 
%% ===== \begin{flushleft}
%% ===== next sequential instruction pair virtual address formation, the processor samples the group 7
%% ===== \end{flushleft}
%% ===== 
%% ===== 
%% ===== \begin{flushleft}
%% ===== faults. If any of the group 7 faults is found an internal flag is set reflecting the presence of the
%% ===== \end{flushleft}
%% ===== 
%% ===== 
%% ===== \begin{flushleft}
%% ===== fault. The processor next samples the interrupt present lines from all eight memory interface ports
%% ===== \end{flushleft}
%% ===== 
%% ===== 
%% ===== \begin{flushleft}
%% ===== and loads a register with bits corresponding to the states of the lines. If any bit in the register is
%% ===== \end{flushleft}
%% ===== 
%% ===== 
%% ===== \begin{flushleft}
%% ===== set ON an internal flag is set to reflect the presence of the bit(s) in the register.
%% ===== \end{flushleft}
%% ===== 
%% ===== 
%% ===== \begin{flushleft}
%% ===== If the instruction pair virtual address being formed is the result of a transfer of control
%% ===== \end{flushleft}
%% ===== 
%% ===== 
%% ===== \begin{flushleft}
%% ===== condition or if the current instruction is Execute (xec), Execute Double (xed), Repeat (rpt), Repeat
%% ===== \end{flushleft}
%% ===== 
%% ===== 
%% ===== \begin{flushleft}
%% ===== Double (rpd), or Repeat Link (rpl), the group 7 faults and interrupt present lines are not sampled.
%% ===== \end{flushleft}
%% ===== 
%% ===== 
%% ===== \begin{flushleft}
%% ===== At an appropriate point in the execution of the current instruction pair, the processor
%% ===== \end{flushleft}
%% ===== 
%% ===== 
%% ===== \begin{flushleft}
%% ===== fetches the next instruction pair. At this point, it first tests the internal flags for group 7 faults and
%% ===== \end{flushleft}
%% ===== 
%% ===== 
%% ===== \begin{flushleft}
%% ===== interrupts. If either flag is set it does not fetch the next instruction pair.
%% ===== \end{flushleft}
%% ===== 
%% ===== 
%% ===== \begin{flushleft}
%% ===== At the
%% ===== \end{flushleft}
%% ===== 
%% ===== 
%% ===== \begin{flushleft}
%% ===== internal flags.
%% ===== \end{flushleft}
%% ===== 
%% ===== 
%% ===== \begin{flushleft}
%% ===== flag for group
%% ===== \end{flushleft}
%% ===== 
%% ===== 
%% ===== \begin{flushleft}
%% ===== fault present.
%% ===== \end{flushleft}
%% ===== 
%% ===== 
%% ===== 
%% ===== 
%% ===== 
%% ===== \begin{flushleft}
%% ===== completion of the current instruction pair the processor once again checks the
%% ===== \end{flushleft}
%% ===== 
%% ===== 
%% ===== \begin{flushleft}
%% ===== If neither flag is set, execution of the next instruction pair proceeds. If the internal
%% ===== \end{flushleft}
%% ===== 
%% ===== 
%% ===== \begin{flushleft}
%% ===== 7 faults is set, the processor enters a FAULT CYCLE for the highest priority group 7
%% ===== \end{flushleft}
%% ===== 
%% ===== 
%% ===== \begin{flushleft}
%% ===== If the internal flag for interrupts is set, the processor enters an INTERRUPT CYCLE.
%% ===== \end{flushleft}
%% ===== 
%% ===== 
%% ===== 
%% ===== 
%% ===== 
%% ===== \begin{flushleft}

\subsubsection{Interrupt Cycle Sequence}

%% ===== \end{flushleft}
%% ===== 
%% ===== 
%% ===== \begin{flushleft}
%% ===== In the INTERRUPT CYCLE, the processor safe-stores the Control Unit Data (see Section 3)
%% ===== \end{flushleft}
%% ===== 
%% ===== 
%% ===== \begin{flushleft}
%% ===== into program-invisible holding registers in preparation for a Store Control Unit (scu) instruction,
%% ===== \end{flushleft}
%% ===== 
%% ===== 
%% ===== \begin{flushleft}
%% ===== enters temporary absolute mode, and forces the current ring of execution C(PPR.PRR) to 0. It then
%% ===== \end{flushleft}
%% ===== 
%% ===== 
%% ===== \begin{flushleft}
%% ===== issues an XEC system controller command to the system controller on the highest priority port for
%% ===== \end{flushleft}
%% ===== 
%% ===== 
%% ===== \begin{flushleft}
%% ===== which there is a bit set in the interrupt present register.
%% ===== \end{flushleft}
%% ===== 
%% ===== 
%% ===== \begin{flushleft}
%% ===== The selected system controller responds by clearing its highest priority interrupt cell and
%% ===== \end{flushleft}
%% ===== 
%% ===== 
%% ===== \begin{flushleft}
%% ===== returning the interrupt trap pair address for that cell to the processor.
%% ===== \end{flushleft}
%% ===== 
%% ===== 
%% ===== \begin{flushleft}
%% ===== If there is no interrupt cell set in the selected system controller (implying that all have been
%% ===== \end{flushleft}
%% ===== 
%% ===== 
%% ===== \begin{flushleft}
%% ===== cleared in response to XEC system controller commands from other processors), the system
%% ===== \end{flushleft}
%% ===== 
%% ===== 
%% ===== \begin{flushleft}
%% ===== controller returns the address value 1, which is not a valid interrupt trap pair address. The
%% ===== \end{flushleft}
%% ===== 
%% ===== 
%% ===== \begin{flushleft}
%% ===== processor senses this value, aborts the INTERRUPT CYCLE, and returns to normal sequential
%% ===== \end{flushleft}
%% ===== 
%% ===== 
%% ===== \begin{flushleft}
%% ===== instruction processing.
%% ===== \end{flushleft}
%% ===== 
%% ===== 
%% ===== \begin{flushleft}
%% ===== The interrupt trap pair address returned and the operation code for the Execute Double
%% ===== \end{flushleft}
%% ===== 
%% ===== 
%% ===== \begin{flushleft}
%% ===== (xed) instruction are forced into the instruction register and executed as an instruction. Note that
%% ===== \end{flushleft}
%% ===== 
%% ===== 
%% ===== \begin{flushleft}
%% ===== the execution of the instruction is not done in a normal EXECUTE CYCLE but in the INTERRUPT
%% ===== \end{flushleft}
%% ===== 
%% ===== 
%% ===== \begin{flushleft}
%% ===== CYCLE with the processor in temporary absolute mode.
%% ===== \end{flushleft}
%% ===== 
%% ===== 
%% ===== \begin{flushleft}
%% ===== If the attempt to fetch and execute the instruction pair at the interrupt trap pair results in a
%% ===== \end{flushleft}
%% ===== 
%% ===== 
%% ===== \begin{flushleft}
%% ===== fault, the INTERRUPT CYCLE is aborted and a FAULT CYCLE for the trouble fault (fault number
%% ===== \end{flushleft}
%% ===== 
%% ===== 
%% ===== \begin{flushleft}
%% ===== 31) is initiated. In the FAULT CYCLE for a trouble fault, the processor does not safe-store the
%% ===== \end{flushleft}
%% ===== 
%% ===== 
%% ===== \begin{flushleft}
%% ===== Control Unit Data. Therefore, it may be possible to recover the conditions for the interrupt (except
%% ===== \end{flushleft}
%% ===== 
%% ===== 
%% ===== \begin{flushleft}
%% ===== the interrupt number) by use of the Store Control Unit (scu) instruction. The interrupt number
%% ===== \end{flushleft}
%% ===== 
%% ===== 
%% ===== 
%% ===== 
%% ===== 
%% ===== \begin{flushleft}
%% ===== \newpage
%% ===== may usually be recovered by analysis of the computed address for the interrupt trap pair stored in
%% ===== \end{flushleft}
%% ===== 
%% ===== 
%% ===== \begin{flushleft}
%% ===== the control unit history registers.
%% ===== \end{flushleft}
%% ===== 
%% ===== 
%% ===== \begin{flushleft}
%% ===== If either of the two instructions in the interrupt trap pair results in a transfer of control to a
%% ===== \end{flushleft}
%% ===== 
%% ===== 
%% ===== \begin{flushleft}
%% ===== computed address generated in absolute mode, the absolute mode indicator is set ON for the
%% ===== \end{flushleft}
%% ===== 
%% ===== 
%% ===== \begin{flushleft}
%% ===== transfer and remains ON thereafter until changed by program action.
%% ===== \end{flushleft}
%% ===== 
%% ===== 
%% ===== \begin{flushleft}
%% ===== If either of the two instructions in the interrupt trap pair results in a transfer of control to a
%% ===== \end{flushleft}
%% ===== 
%% ===== 
%% ===== \begin{flushleft}
%% ===== computed address generated in append mode (through the use of bit 29 of the instruction word or
%% ===== \end{flushleft}
%% ===== 
%% ===== 
%% ===== \begin{flushleft}
%% ===== by use of the itp or its modifiers), the transfer is made in the append mode and and the processor
%% ===== \end{flushleft}
%% ===== 
%% ===== 
%% ===== \begin{flushleft}
%% ===== remains in append mode thereafter.
%% ===== \end{flushleft}
%% ===== 
%% ===== 
%% ===== \begin{flushleft}
%% ===== If no transfer of control takes place, the processor returns to the mode in effect at the time
%% ===== \end{flushleft}
%% ===== 
%% ===== 
%% ===== \begin{flushleft}
%% ===== of the interrupt and resumes normal sequential execution with the instruction following the
%% ===== \end{flushleft}
%% ===== 
%% ===== 
%% ===== \begin{flushleft}
%% ===== interrupted instruction (C(PPR.IC) + 1). Note that the current ring of execution C(PPR.PRR) was
%% ===== \end{flushleft}
%% ===== 
%% ===== 
%% ===== \begin{flushleft}
%% ===== forced to 0 during the INTERRUPT CYCLE and that normal sequential execution will resume in
%% ===== \end{flushleft}
%% ===== 
%% ===== 
%% ===== \begin{flushleft}
%% ===== ring 0.
%% ===== \end{flushleft}
%% ===== 
%% ===== 
%% ===== \begin{flushleft}
%% ===== Due to the time required for many of the EIS data movement instructions, additional group
%% ===== \end{flushleft}
%% ===== 
%% ===== 
%% ===== \begin{flushleft}
%% ===== 7 fault and interrupt sampling is done during these instructions. After the initial load of the
%% ===== \end{flushleft}
%% ===== 
%% ===== 
%% ===== \begin{flushleft}
%% ===== decimal unit input data buffer, group 7 faults and interrupts are sampled for each input operand
%% ===== \end{flushleft}
%% ===== 
%% ===== 
%% ===== \begin{flushleft}
%% ===== virtual address formation. The instruction in execution is interrupted before the operand is
%% ===== \end{flushleft}
%% ===== 
%% ===== 
%% ===== \begin{flushleft}
%% ===== fetched and flags are set into Control Unit Data and Decimal Unit Data to allow the restart of the
%% ===== \end{flushleft}
%% ===== 
%% ===== 
%% ===== \begin{flushleft}
%% ===== instruction.
%% ===== \end{flushleft}
%% ===== 
%% ===== 
%% ===== \begin{flushleft}
%% ===== NOTE:
%% ===== \end{flushleft}
%% ===== 
%% ===== 
%% ===== 
%% ===== 
%% ===== 
%% ===== \begin{flushleft}
%% ===== The execution of a Store Pointers and Lengths (spl) instruction is required before an
%% ===== \end{flushleft}
%% ===== 
%% ===== 
%% ===== \begin{flushleft}
%% ===== interrupted EIS instruction may be restarted. Therefore, a fault or interrupt handling
%% ===== \end{flushleft}
%% ===== 
%% ===== 
%% ===== \begin{flushleft}
%% ===== routine must execute this instruction even though it does not use the decimal unit for its
%% ===== \end{flushleft}
%% ===== 
%% ===== 
%% ===== \begin{flushleft}
%% ===== processing.
%% ===== \end{flushleft}
%% ===== 
%% ===== 
%% ===== 
%% ===== 
%% ===== 
%% ===== \begin{flushleft}
%% ===== Many of the interrupts are deliberately or inadvertently caused by the software and do not
%% ===== \end{flushleft}
%% ===== 
%% ===== 
%% ===== \begin{flushleft}
%% ===== necessarily involve error conditions. The operating supervisor determines the proper action for
%% ===== \end{flushleft}
%% ===== 
%% ===== 
%% ===== \begin{flushleft}
%% ===== each interrupt by analyzing the detailed state of the processor at the time of the interrupt. In
%% ===== \end{flushleft}
%% ===== 
%% ===== 
%% ===== \begin{flushleft}
%% ===== order to accomplish this analysis, it is necessary that the first instruction in each of the interrupt
%% ===== \end{flushleft}
%% ===== 
%% ===== 
%% ===== \begin{flushleft}
%% ===== trap pairs be the Store Control Unit (scu) instruction and the second be a transfer to an interrupt
%% ===== \end{flushleft}
%% ===== 
%% ===== 
%% ===== \begin{flushleft}
%% ===== analysis routine. If an interrupt is to be intentionally ignored, the trap pair for that interrupt
%% ===== \end{flushleft}
%% ===== 
%% ===== 
%% ===== \begin{flushleft}
%% ===== should contain an scu/rcu pair referencing a unique Y-block8. By using this pair to ignore an
%% ===== \end{flushleft}
%% ===== 
%% ===== 
%% ===== \begin{flushleft}
%% ===== interrupt, the state of the processor for the ignored interrupt may be recovered if the ignored
%% ===== \end{flushleft}
%% ===== 
%% ===== 
%% ===== \begin{flushleft}
%% ===== interrupt causes a trouble fault. The use of the scu/rcu pair also ensures that execution is
%% ===== \end{flushleft}
%% ===== 
%% ===== 
%% ===== \begin{flushleft}
%% ===== resumed in the original ring of execution.
%% ===== \end{flushleft}
%% ===== 
%% ===== 
%% ===== 
%% ===== 
%% ===== 
%% ===== \begin{flushleft}
%% ===== \newpage
