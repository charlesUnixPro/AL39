
\section{FAULTS AND INTERRUPTS}

Faults and interrupts both result in an interruption of normal sequential
processing, but there is a difference in how they originate. Generally, faults
are caused by events or conditions that are internal to the processor and
interrupts are caused by events or conditions that are external to the
processor. Faults and interrupts enable the processor to respond promptly when
conditions occur that require system attention.


A unique word-pair is dedicated for the instructions to service each fault and
interrupt condition. The instruction pair associated with a fault or interrupt
is called the trap pair for that fault or interrupt. The set of all interrupt
trap pairs is called the interrupt vector and is located at absolute main
memory address 0. The set of all fault trap pairs is called the fault vector
and is located at a 0 modulo 32 absolute main memory address whose high-order
bits are given by the setting of the FAULT BASE switches on the processor
configuration panel. The fault vector is constrained to lie within the lowest
4096 words of main memory.


\subsection{FAULT CYCLE SEQUENCE}

Following the detection of a fault condition, the control unit determines the
proper time to initiate the fault sequence according to the fault group (Fault
groups are discussed later in this section). At that time, the control unit
interrupts normal sequential processing with an ABORT CYCLE.  The ABORT CYCLE
brings all overlapped and asynchronous functions within the processor to an
orderly halt. At the end of the ABORT CYCLE, the control unit initiates a FAULT
CYCLE.


In the FAULT CYCLE, the processor safe-stores the Control Unit Data (see
Section 3) into program-invisible holding registers in preparation for a Store
Control Unit (scu) instruction, then enters temporary absolute mode, forces the
current ring of execution C(PPR.PRR) to 0, and generates a computed address for
the fault trap pair by concatenating the setting of the FAULT BASE switches on
the processor configuration panel with twice the fault number (see Table 7-1).
This computed address and the operation code for the Execute Double (xed)
instruction are forced into the instruction register and executed as an
instruction. Note that the execution of the instruction is not done in a normal
EXECUTE CYCLE but in the FAULT CYCLE with the processor in temporary absolute
mode.

If the attempt to fetch and execute the instruction pair at the fault trap pair
results in another fault, the current FAULT CYCLE is aborted and a new FAULT
CYCLE for the trouble fault (fault number 31) is initiated. In the FAULT CYCLE
for a trouble fault, the processor does not safe-store the Control Unit Data.
Therefore, it may be possible to recover the conditions for the original fault
(except the fault number) by use of the Store Control Unit (scu) instruction.
The fault number may usually be recovered by analysis of the computed address
for the original fault trap pair stored in the control unit history registers.


If either of the two instructions in the fault trap pair results in a transfer
of control to a computed address generated in absolute mode, the absolute mode
indicator is set ON for the transfer and remains ON thereafter until changed by
program action.


If either of the two instructions in the fault trap pair results in a transfer
of control to a computed address generated in append mode (through the use of
bit 29 of the instruction word or by use of the its or itp modifiers), the
transfer is made in the append mode and the processor remains in append mode
thereafter.


If no transfer of control takes place, the processor returns to the mode in
effect at the time of the fault and resumes normal sequential execution with
the instruction following the faulting instruction (C(PPR.IC) + 1). Note that
the current ring of execution C(PPR.PRR) was forced to 0 during the FAULT CYCLE
and that normal sequential execution will resume in ring 0.

Many of the fault conditions are deliberately or inadvertently caused by the
software and do not necessarily involve error conditions. The operating
supervisor determines the proper action for each fault condition by analyzing
the detailed state of the processor at the time of the fault. In order to
accomplish this analysis, it is necessary that the first instruction in each of
the fault trap pairs be the Store Control Unit (scu) instruction and the second
be a transfer to a fault analysis routine. If a fault condition is to be
intentionally ignored, the fault trap pair for that condition should contain an
scu/rcu pair referencing a unique Y-block8. By using this pair to ignore a
fault, the state of the processor for the ignored fault condition may be
recovered if the ignored fault causes a trouble fault. The use of the scu/rcu
pair also ensures that execution is resumed in the original ring of execution.

Table 7-1. List of Faults

Decimal fault number Octal (1) fault address Fault mnemonic Fault name Command Priority Group

0 0 sdf Shutdown 27 7

1 2 str Store 10 4

2 4 mme Master mode entry 1 11 5

3 6 f1 Fault tag 1 17 5

4 10 tro Timer runout 26 7 

5 12 cmd 9 4 

6 14 drl Derail 15 5 

7 16 luf Lockup 5 4 

8 20 con Connect 25 7 

9 22 par Parity 8 4 

10 24 ipr Illegal procedure 16 5 

11 26 onc Operation not complete 4 2 

12 30 suf Startup 1 1 

13 32 ofl Overflow 7 3 

14 34 div Divide check 6 3 

15 36 exf Execute 2 1 

16 40 df0 Directed fault 0 20 6 

17 42 df1 Directed fault 1 21 6 

18 44 df2 Directed fault 2 22 6 

19 46 df3 Directed fault 3 23 6 

20 50 acv Access violation 24 6 

21 52 mme2 Master mode entry 2 12 5 

22 54 mme3 Master mode entry 3 13 5 

23 56 mme4 Master mode entry 4 14 5 

24 60 f2 Fault tag 2 18 5 

25 62 f3 Fault tag 3 19 5 

26 64 Unassigned 

27 66 Unassigned 

28 70 Unassigned 

29 72 Unassigned

30 74 Unassigned 

31 76 trb Trouble 3 2

(1)The octal fault address value is the value concatenated with the FAULT BASE
switch setting in forming the computed address for the fault trap pair.

\subsection{FAULT PRIORITY}

The processor has provision for 32 faults of which 27 are implemented. The
faults are classified into seven fault priority groups that roughly correspond
to the severity of the faults.  Fault priority groups are defined so that fault
recognition precedence may be established when two or more faults exist
concurrently. Overlapped and asynchronous functions in the processor allow the
simultaneous occurrence of faults. Group 1 has the highest priority and group 7
has the lowest. In groups 1 through 6, only one fault within each group is
allowed to be active at any one time. The first fault within a group occurring
through the normal program sequence is the one serviced.


Group 7 faults are saved by the hardware for eventual recognition. In the case
of simultaneous faults within group 7, shutdown has the highest priority with
timer runout next and connect the lowest.

There is a single exception to the handling of faults in priority group order.
If an operand fetch generates a parity fault and the use of the operand in
{``}closing out'' instruction execution generates an overflow fault or a divide
check fault, these faults are considered simultaneous but the parity fault
takes precedence.

\subsection{FAULT RECOGNITION}

For the discussion following, the term {``}function'' is defined as a major
processor functional cycle. Examples are: APPEND CYCLE, CA CYCLE, INSTRUCTION
FETCH CYCLE, OPERAND STORE CYCLE, DIVIDE EXECUTION CYCLE. Some of these cycles
are discussed in various sections of this manual.

Faults in groups 1 and 2 cause the processor to abort all functions immediately by entering a FAULT CYCLE.

Faults in group 3 cause the processor to {``}close out'' current functions
without taking any irrevocable action (such as setting PTW.U in an APPEND CYCLE
or modifying an indirect word in a CA CYCLE), then to discard any pending
functions (such as an APPEND CYCLE needed during a CA CYCLE), and to enter a
FAULT CYCLE.

Faults in group 4 cause the processor to suspend overlapped operation, to
complete current and pending functions for the current instruction, and then to
enter a FAULT CYCLE.  

Faults in groups 5 or 6 are normally detected during virtual address formation
and instruction decode. These faults cause the processor to suspend overlapped
operation, to complete the current and pending instructions, and to enter a
FAULT CYCLE. If a fault in a higher priority group is generated by the
execution of the current or pending instructions, that higher priority fault
will take precedence and the group 5 or 6 fault will be lost. If a group 5 or 6
fault is detected during execution of the current instruction (e.g., an access
violation, out of segment bounds, fault during certain interruptible EIS
instructions), the instruction is considered {``}complete'' upon detection of
the fault.


Faults in group 7 are held and processed (with interrupts) at the completion of
the current instruction pair. Group 7 faults are inhibitable by setting bit 28
of the instruction word.  

Faults in groups 3 through 6 must wait for the system controller to acknowledge the last access request before entering the FAULT CYCLE.  

\subsection{FAULT DESCRIPTIONS}

\subsubsection{Group 1 Faults}

Startup 

DC POWER has been turned on. When the POWER ON button is pressed, the processor
is first initialized and then the startup fault is generated.

Execute

1. The EXECUTE pushbutton on the processor maintenance panel has been pressed.

2. An external gate signal has been substituted for the EXECUTE pushbutton.

The selection between the above conditions is made by settings of various
switches on the processor maintenance panel.

\subsubsection{Group 2 Faults}

Operation Not Complete

Any of the following will cause an operation not complete fault:

1. The processor has addressed a system controller to which it is not attached,
that is, there is no main memory interface port having its ADDRESS ASSIGNMENT
switches set to a value including the main memory address desired.

2. The addressed system controller has failed to acknowledge the processor.

3. The processor has not generated a main memory access request or a direct
operand within 1 to 2 milliseconds and is not executing the Delay Until
Interrupt Signal (dis) instruction.

4. A main memory interface port received a data strobe without a preceding
acknowledgment from the system controller that it had received the access
request.

5. A main memory interface port received a data strobe before the data
previously sent to it was unloaded.

Trouble

The trouble fault is defined as the occurrence of a fault during the fetch or
execution of a fault trap pair or interrupt trap pair. Such faults may be
hardware generated (for example, operation not complete or parity), or
operating system generated (e.g., the page containing a trap pair instruction
operand is missing).  

\subsubsection{Group 3 Faults}

Overflow

An arithmetic overflow, exponent overflow, exponent underflow, or EIS
truncation fault has been generated. The generation of this fault is inhibited
when the overflow mask indicator is ON. Resetting of the overflow mask
indicator to OFF does not generate a fault from previously set indicators. The
overflow mask state does not affect the setting, testing or storing of
indicators. The determination of the specific overflow condition is by
indicator testing by the operating supervisor.

Divide Check

A divide check fault occurs when the actual division cannot be carried out for
one of the reasons specified with individual divide instructions.  

\subsubsection{Group 4 Faults}

Store

The processor attempted to select a disabled port, an out-of-bounds address was
generated in the BAR mode or absolute mode, or an attempt was made to access a
store unit that was not ready.

Command

1. The processor attempted to load or read the interrupt mask register in a
system controller in which it did not have an interrupt mask assigned.

2. The processor issued an XEC system controller command to a system controller
in which it did not have an interrupt mask assigned.  

3. The processor issued a connect to a system controller port that is masked
OFF.  

4. The selected system controller is in TEST mode and a condition determined by
certain system controller maintenance panel switches has been trapped.

5. An attempt was made to load a pointer register with packed pointer data in
which the BITNO field value was greater than or equal to 60(8).

Lockup

The program is in a code sequence which has inhibited sampling for interrupts
(whether present or not) and group 7 faults for longer than the prescribed
time. In absolute mode or privileged mode the lockup time is 32 milliseconds.
In normal mode or BAR mode the lockup time is specified by the setting for the
lockup time in the cache mode register. The lockup time is program settable to
2, 4, 8, or 16 milliseconds.  

While in absolute mode or privileged mode the lockup fault is signalled at the
end of the time limit set in the lockup timer but is not recognized until the
32 millisecond limit. If the processor returns to normal mode or BAR mode after
the fault has been signalled but before the 32 millisecond limit, the fault is
recognized before any instruction in the new mode is executed.

Parity

1. The selected system controller has returned an illegal action signal with an
illegal action code for one of the various main memory parity error conditions.

2. A cache memory data or directory parity error has occurred either for read,
write, or block load. Cache status bits for the condition have been set in the
cache mode register.

3. The processor has detected a parity error in the system controller interface
port while either generating outgoing parity or verifying incoming parity.  

\subsubsection{Group 5 Faults}

Master Mode Entries 1-4

The corresponding Master Mode Entry instruction has been decoded.

Fault Tags 1-3

The corresponding indirect then tally variation has been detected during
virtual address formation.  

Derail

The Derail instruction has been decoded.

Illegal Procedure

1. An illegal operation code has been decoded or an illegal instruction
sequence has been encountered.


2. An illegal modifier or modifier sequence has been encountered during virtual 
address formation.


3. An illegal address has been given in an instruction for which the ADDRESS
field is used for register selection.


4. An attempt was made to execute a privileged instruction in normal mode or
BAR mode.  

5. An illegal digit was encountered in a decimal numeric operand.

6. An illegal specification was found in an EIS operand descriptor.


The conditions for the fault will be set in the fault register, word 1 of the
Control Unit Data, or in both.

\subsubsection{Group 6 Faults}

Directed Faults 0-3


A faulted segment descriptor word (SDW) or page table word (PTW) with the
corresponding directed fault number has been fetched by the appending unit.  

Access Violation

The appending unit has detected one of the several access violations below.
Word 1 of the Control Unit Data contains status bits for the condition.

1. Not in read bracket (ACV3=ORB)

2. Not in write bracket (ACV5=OWB)

3. Not in execute bracket (ACV1=OEB)

4. No read permission (ACV4=R-OFF)

5. No write permission (ACV6=W-OFF)

6. No execute permission (ACV2=E-OFF)

7. Invalid ring crossing (ACV12=CRT)

8. Call limiter fault (ACV7=NO GA)

9. Outward call (ACV9=OCALL)

10.Bad outward call (ACV10=BOC)

11.Inward return (ACV11=INRET)

12.Ring alarm (ACV13=RALR)

13.Associative memory error

14.Out of segment bounds (ACV15=OOSB)

15.Illegal ring order (ACV0=IRO)

16.Out of call brackets (ACV8=OCB)

\subsubsection{Group 7 Faults}

Shutdown

An external power shutdown condition has been detected. DC POWER shutdown will 
occur in approximately one millisecond.

Timer Runout

The timer register has decremented to or through the value zero. If the
processor is in privileged mode or absolute mode, recognition of this fault is
delayed until a return to normal mode or BAR mode. Counting in the timer
register continues.  

Connect

A connect signal (\$CON strobe) has been received from a system controller.
This event is to be distinguished from a Connect Input/Output Channel (cioc)
instruction encountered in the program sequence.

(See the discussion of the floating faults in Section 3).

\subsection{INTERRUPTS AND EXTERNAL FAULTS}

Each system controller contains 32 interrupt cells that are used for
communication among the active system modules (processors, I/O multiplexers,
etc.). The interrupt cells are organized in a numbered priority chain. Any
active system module connected to a system controller port may request the
setting of an interrupt cell with the SXC system controller command.  

When one or more interrupt cells in a system controller is set, the system
controller activates the interrupt present (XIP) line to all system controller
ports having an assigned interrupt mask in which one or more of the interrupt
cells that are set is unmasked. Interrupt masks should be assigned only to
processors. Each interrupt cell has associated with it a unique interrupt trap
pair located at an absolute main memory address equal to twice the cell number.  
\subsubsection{Interrupt Sampling}

The processor always fetches instructions in pairs. At an appropriate point (as
early as possible) in the execution of a pair of instructions, the next
sequential instruction pair is fetched and held in a special instruction buffer
register. The exact point depends on instruction sequence and other conditions

If the interrupt inhibit bit (bit 28) is not set in the current instruction
word at the point of next sequential instruction pair virtual address
formation, the processor samples the group 7 faults. If any of the group 7
faults is found an internal flag is set reflecting the presence of the fault.
The processor next samples the interrupt present lines from all eight memory
interface ports and loads a register with bits corresponding to the states of
the lines. If any bit in the register is set ON an internal flag is set to
reflect the presence of the bit(s) in the register.

If the instruction pair virtual address being formed is the result of a
transfer of control condition or if the current instruction is Execute (xec),
Execute Double (xed), Repeat (rpt), Repeat Double (rpd), or Repeat Link (rpl),
the group 7 faults and interrupt present lines are not sampled.  

At an appropriate point in the execution of the current instruction pair, the
processor fetches the next instruction pair. At this point, it first tests the
internal flags for group 7 faults and interrupts. If either flag is set it does
not fetch the next instruction pair.  

At the completion of the current instruction pair the processor once again
checks the internal flags.  If neither flag is set, execution of the next
instruction pair proceeds. If the internal flag for group 7 faults is set, the
processor enters a FAULT CYCLE for the highest priority group 7 fault present.
If the internal flag for interrupts is set, the processor enters an INTERRUPT
CYCLE.  

\subsubsection{Interrupt Cycle Sequence}

In the INTERRUPT CYCLE, the processor safe-stores the Control Unit Data (see
Section 3) into program-invisible holding registers in preparation for a Store
Control Unit (scu) instruction, enters temporary absolute mode, and forces the
current ring of execution C(PPR.PRR) to 0. It then issues an XEC system
controller command to the system controller on the highest priority port for
which there is a bit set in the interrupt present register.


The selected system controller responds by clearing its highest priority
interrupt cell and returning the interrupt trap pair address for that cell to
the processor.

If there is no interrupt cell set in the selected system controller (implying
that all have been cleared in response to XEC system controller commands from
other processors), the system controller returns the address value 1, which is
not a valid interrupt trap pair address. The processor senses this value,
aborts the INTERRUPT CYCLE, and returns to normal sequential instruction
processing.


The interrupt trap pair address returned and the operation code for the Execute
Double (xed) instruction are forced into the instruction register and executed
as an instruction. Note that the execution of the instruction is not done in a
normal EXECUTE CYCLE but in the INTERRUPT CYCLE with the processor in temporary
absolute mode.


If the attempt to fetch and execute the instruction pair at the interrupt trap
pair results in a fault, the INTERRUPT CYCLE is aborted and a FAULT CYCLE for
the trouble fault (fault number 31) is initiated. In the FAULT CYCLE for a
trouble fault, the processor does not safe-store the Control Unit Data.
Therefore, it may be possible to recover the conditions for the interrupt
(except the interrupt number) by use of the Store Control Unit (scu)
instruction. The interrupt number may usually be recovered by analysis of the
computed address for the interrupt trap pair stored in the control unit history
registers.


If either of the two instructions in the interrupt trap pair results in a
transfer of control to a computed address generated in absolute mode, the
absolute mode indicator is set ON for the transfer and remains ON thereafter
until changed by program action.


If either of the two instructions in the interrupt trap pair results in a
transfer of control to a computed address generated in append mode (through the
use of bit 29 of the instruction word or by use of the itp or its modifiers),
the transfer is made in the append mode and and the processor remains in append
mode thereafter.

If no transfer of control takes place, the processor returns to the mode in
effect at the time of the interrupt and resumes normal sequential execution
with the instruction following the interrupted instruction (C(PPR.IC) + 1).
Note that the current ring of execution C(PPR.PRR) was forced to 0 during the
INTERRUPT CYCLE and that normal sequential execution will resume in ring 0.

Due to the time required for many of the EIS data movement instructions,
additional group 7 fault and interrupt sampling is done during these
instructions. After the initial load of the decimal unit input data buffer,
group 7 faults and interrupts are sampled for each input operand virtual
address formation. The instruction in execution is interrupted before the
operand is fetched and flags are set into Control Unit Data and Decimal Unit
Data to allow the restart of the instruction.


NOTE: The execution of a Store Pointers and Lengths (spl) instruction is
required before an interrupted EIS instruction may be restarted. Therefore, a
fault or interrupt handling routine must execute this instruction even though
it does not use the decimal unit for its processing.


Many of the interrupts are deliberately or inadvertently caused by the software
and do not necessarily involve error conditions. The operating supervisor
determines the proper action for each interrupt by analyzing the detailed state
of the processor at the time of the interrupt. In order to accomplish this
analysis, it is necessary that the first instruction in each of the interrupt
trap pairs be the Store Control Unit (scu) instruction and the second be a
transfer to an interrupt analysis routine. If an interrupt is to be
intentionally ignored, the trap pair for that interrupt should contain an
scu/rcu pair referencing a unique Y-block8. By using this pair to ignore an
interrupt, the state of the processor for the ignored interrupt may be
recovered if the ignored interrupt causes a trouble fault. The use of the
scu/rcu pair also ensures that execution is resumed in the original ring of
execution.





