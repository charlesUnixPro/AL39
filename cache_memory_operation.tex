
\section{DPS/L68 CACHE MEMORY OPERATION}


The Multics processor may be fitted with an optional cache memory. The
operation of this cache memory is described in this section.


\subsection{PHILOSOPHY OF CACHE MEMORY}

The cache memory is a high speed buffer memory located within the processor
that is intended to hold operands and/or instructions in expectation of their
immediate use. This concept is different from that of holding a single operand
(such as the divisor for a divide instruction) in the processor during
execution of a single instruction. A cache memory depends on the locality of
reference principle. Locality of reference involves the calculation of the
probability, for any value of d, that the next instruction or operand reference
after a reference to the instruction or operand at location A is to location
A+d.

The calculation of probabilities for a set of values of d requires the
statistical analysis of large volumes of real and simulated instruction
sequences and data organizations. If it can be shown that the average expected
data/instruction access time reduction (over the range 1 to d) is statistically
significant in comparison to the fixed main memory access time, then the
implementation of a cache memory with block size d will contribute a
significant improvement in performance.

The results of such studies for the Multics processor with a cache memory as
described below (with d!=!4) show a hit probability ranging between 80 and 95
percent (depending on instruction mix and data organization) and a performance
improvement ranging up to 30 percent.  


\subsection{CACHE MEMORY ORGANIZATION} 

The cache memory is implemented as 2048 36-bit words of high-speed register
storage with associated control and content directory circuitry within the
processor. It is fully integrated with the normal data path circuitry and is
virtually invisible to all programming sequences. Parity is generated, stored,
and/or checked on each data reference. The total storage is divided into 512
blocks of 4 words each and the blocks are organized into 128 columns of four
levels each.

\subsubsection{Cache Memory / Main Memory Mapping}

Main memory is mapped into the cache memory as described below and shown in
Figure 9-1.

Main memory is divided into N blocks of 4 words each arranged in ascending
order and numbered with the value of Y15,21 of the first word of the block.

All main memory blocks with numbers n modulo 128 are grouped associatively with
cache memory column n.

Each cache memory column may hold any four blocks of the associated set of main
memory blocks.

Each cache memory column has associated with it a four entry directory (one
entry for each level) and a 2-bit round robin counter. Parity is generated,
stored, and checked on each directory entry.

A cache directory entry consists of a 15-bit ADDRESS register, a pre-set, 2-bit
level number value and a level full flag bit.

When a main memory block is loaded into a cache memory block at some level in
the associated column, the directory ADDRESS register for that column and level
is loaded with Y0,14. (Level selection is discussed in {``}Cache Memory
Control'' later in this section.)

Figure 9-1. Main Memory/Cache Memory Mapping

\subsubsection{Cache Memory Addressing}

For a read operation, the 24-bit absolute main memory address prepared by the
appending unit is presented simultaneously to the cache control and to the main
memory port selection circuitry. While port selection is being accomplished,
the cache memory is accessed as follows.  

Yl5,21 are used to select a cache memory column.  

Y0,14 are matched associatively against the four directory ADDRESS registers
for the selected column.

If a match occurs for a level whose full flag is ON, a hit is signaled, the
main memory reference cycle is canceled, and the level number value is read
out.

The level number value and Y22,23 are used to select the level and word in the
selected column and the cache memory data is read out into the data circuitry.

If no hit is signaled, the main memory reference cycle proceeds and a cache
memory block load cycle is initiated (see {``}Cache Memory Control'' below).

For a write operation, the 24-bit absolute main memory address prepared by the
appending unit is presented simultaneously to the cache control and to the main
memory port selection circuitry. While port selection is being accomplished,
the cache memory is accessed as follows.  

Y15,21 are used to select a cache memory column.  

Y0,14 are matched associatively against the four directory ADDRESS registers
for the selected column.

If a match occurs for a level whose full flag is ON, a hit is signaled and the
level number value is read out.  

The level number value and Y22,23 are used to select the level and word in the
selected column, a cache memory write cycle is enabled, and the data is written
to the main memory and the cache memory simultaneously.

If no hit is signaled, the main memory reference cycle proceeds with no further
cache memory action.  

\subsection{CACHE MEMORY CONTROL}

\subsubsection{Enabling and Disabling Cache Memory}

The cache memory is controlled by the state of several bits in the cache mode
register (see Section 3). The cache mode register may be loaded with the Load
Central Processor Register (lcpr) instruction. The cache memory control bits
are as follows:

bit Value Action


54 0 The lower half of the cache memory (levels 0 and 1) is disabled.

1 The lower half of the cache memory is active and enabled as per the state of
bits 56-57.

55 0 The upper half of the cache memory (levels 2 and 3) is disabled.

1 The upper half of the cache memory is active and enabled as per the state of
bits 56-57.

56 0 The cache memory (if active) is not used for operands and indirect words.

1 The cache memory (if active) is used for operands and indirect words.

57 0 The cache memory (if active) is not used for instructions.

1 The cache memory (if active) is used for instructions.

58 0 The cache-to-register mode is not in effect (see {``}Dumping the Cache
Memory'' later in this section).

1 The cache-to-register mode is in effect.


NOTE: The cache memory option furnishes a switch panel maintenance aid that
attaches to the free edge of the cache memory control logic board. The switch
panel provides six switches for manual control of the cache memory:


Four of the switches inhibit the control functions of bits 54-57 of the cache
mode register and have the effect of forcing the corresponding function to be
disabled.


The fifth switch inhibits the store-aside feature wherein the processor is
permitted to proceed immediately after the cache memory write cycle on write
operations without waiting for a data acknowledgment from main memory. (There
is no software control corresponding to this switch).


The sixth switch forces the enabled condition on all cache memory controls
(except cache-to-register mode) without regard to the corresponding cache mode
register control bit.

There is no switch corresponding to the cache-to-register control bit.

While these switches are intended primarily for maintenance sessions, they have been found useful in testing the cache memory during normal operation and in permitting operation of the processor with the cache memory in degraded or partially disabled mode.  

\subsubsection{Cache Memory Control in Segment Descriptor Words}

Certain data have characteristics such that they should never be loaded into
the cache memory. Primary examples of such data are hardware mailboxes for the
I/O multiplexer, bulk store controller, etc., status return words, and various
dynamic operating system data base segments. In general, any data that is
modified by an agency external to a processor with the intent to convey
information to that processor should never be loaded into cache memory.  

Bit 57 of the segment descriptor word is used to reflect this property of
{``}encacheability'' for each segment. (See Section 5 for a discussion of the
segment descriptor word.) If the bit is set ON, data from the segment may be
loaded into the cache memory; if the bit is OFF, they may not.  The operating
system may set bit 57 ON or OFF as appropriate for the use of the segment.  


\subsubsection{Loading the Cache Memory}

The cache memory is loaded with data implicitly whenever a cache memory block
load is required. (See the discussion of read operations in {``}Cache Memory
Addressing'' earlier in this section.) There is no explicit method or
instruction to load data into the cache memory.  

When a cache memory block load is required, the level is selected from the
value of the round robin counter for the selected column, and the cache memory
write function is enabled.  (The round robin counter contains the number of the
least recently loaded level.) When the data arrives from main memory, it is
written into the cache memory and entered into the data circuitry.  The
processor proceeds with the execution of the instruction requiring the operand
if appropriate.  

When the cache memory write is complete, further virtual address formation is
inhibited, Y22 is inverted, and a second main memory access for the other half
of the block is made. When the second half data arrives from main memory, it is
written into the cache memory, Y0,14 are loaded into the directory ADDRESS
register, the level full flag is set ON, the round robin counter is advanced by
1, and virtual address formation is permitted to proceed.  

If all four level full flags for a column are set ON, a column full flag is
also set ON and remains ON until one or more levels in the column are cleared.

\subsubsection{Clearing the Cache Memory}

Cache memory can be cleared in two ways; general clear and selective clear. The
clearing action is the same in both cases, namely, the full flags of the
selected column(s) and/or level(s) are set OFF.

\subsubsubsection{General Clear}

The entire cache memory is cleared by setting all column and level full flags
to OFF in the following situations:

Upper or lower cache memory or both becoming enabled by appropriate bits in the
operand of the Load Central Processor Register (lcpr) instruction or by action
of the cache memory control logic board free edge switches.  

Execution of a Clear Associative Memory Segments (cams) instruction with bit 15
of the address field set ON.

\subsubsubsection{Selective Clear}

The cache memory is cleared selectively as follows:

If a read-and-clear operation (ldac, sznc, etc.) results in a hit on the cache
memory, that cache memory block hit is cleared.

Execution of a Clear Associative Memory Pages (camp) instruction with address
bit 15 set ON causes Y13,14 to be matched against all cache directory ADDRESS
registers.  All cache memory blocks hit are cleared.

\subsubsection{Dumping the Cache Memory}

When the cache-to-register mode flag (bit 59 of the cache mode register) is set
ON, the processor is forced to fetch the operands of all double-precision
operations unit load operations from the cache memory. Y0,12 are ignored,
Y15,21 select a column, and Y13,14 select a level. All other operations (e.g.,
instruction fetches, single-precision operands, etc.) are treated normally.

Note that the phrase {``}treated normally'' as used here includes the case
where the cache memory is enabled. If the cache memory is enabled, the
{``}other'' operations causes normal block loads and cache memory writes thus
destroying the original contents of the cache memory. The cache memory should
be disabled before dumping is attempted.

An indexed program loop involving the ldaq and staq instructions with the
cache-toregister mode bit set ON serves to dump any or all of the cache memory.  
The occurrence of a fault or interrupt sets the cache-to-register mode bit to
OFF.  




