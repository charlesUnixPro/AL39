
\section{DPS/L68 CACHE MEMORY OPERATION}

%% ===== \end{flushleft}
%% ===== 
%% ===== 
%% ===== \begin{flushleft}
%% ===== The Multics processor may be fitted with an optional cache memory. The operation of this
%% ===== \end{flushleft}
%% ===== 
%% ===== 
%% ===== \begin{flushleft}
%% ===== cache memory is described in this section.
%% ===== \end{flushleft}
%% ===== 
%% ===== 
%% ===== 
%% ===== 
%% ===== 
%% ===== \begin{flushleft}

\subsection{PHILOSOPHY OF CACHE MEMORY}

%% ===== \end{flushleft}
%% ===== 
%% ===== 
%% ===== \begin{flushleft}
%% ===== The cache memory is a high speed buffer memory located within the processor that is
%% ===== \end{flushleft}
%% ===== 
%% ===== 
%% ===== \begin{flushleft}
%% ===== intended to hold operands and/or instructions in expectation of their immediate use. This concept
%% ===== \end{flushleft}
%% ===== 
%% ===== 
%% ===== \begin{flushleft}
%% ===== is different from that of holding a single operand (such as the divisor for a divide instruction) in the
%% ===== \end{flushleft}
%% ===== 
%% ===== 
%% ===== \begin{flushleft}
%% ===== processor during execution of a single instruction. A cache memory depends on the locality of
%% ===== \end{flushleft}
%% ===== 
%% ===== 
%% ===== \begin{flushleft}
%% ===== reference principle. Locality of reference involves the calculation of the probability, for any value
%% ===== \end{flushleft}
%% ===== 
%% ===== 
%% ===== \begin{flushleft}
%% ===== of d, that the next instruction or operand reference after a reference to the instruction or operand
%% ===== \end{flushleft}
%% ===== 
%% ===== 
%% ===== \begin{flushleft}
%% ===== at location A is to location A+d.
%% ===== \end{flushleft}
%% ===== 
%% ===== 
%% ===== \begin{flushleft}
%% ===== The calculation of probabilities for a set of values of d requires the statistical analysis of
%% ===== \end{flushleft}
%% ===== 
%% ===== 
%% ===== \begin{flushleft}
%% ===== large volumes of real and simulated instruction sequences and data organizations. If it can be
%% ===== \end{flushleft}
%% ===== 
%% ===== 
%% ===== \begin{flushleft}
%% ===== shown that the average expected data/instruction access time reduction (over the range 1 to d) is
%% ===== \end{flushleft}
%% ===== 
%% ===== 
%% ===== \begin{flushleft}
%% ===== statistically significant in comparison to the fixed main memory access time, then the
%% ===== \end{flushleft}
%% ===== 
%% ===== 
%% ===== \begin{flushleft}
%% ===== implementation of a cache memory with block size d will contribute a significant improvement in
%% ===== \end{flushleft}
%% ===== 
%% ===== 
%% ===== \begin{flushleft}
%% ===== performance.
%% ===== \end{flushleft}
%% ===== 
%% ===== 
%% ===== \begin{flushleft}
%% ===== The results of such studies for the Multics processor with a cache memory as described
%% ===== \end{flushleft}
%% ===== 
%% ===== 
%% ===== \begin{flushleft}
%% ===== below (with d!=!4) show a hit probability ranging between 80 and 95 percent (depending on
%% ===== \end{flushleft}
%% ===== 
%% ===== 
%% ===== \begin{flushleft}
%% ===== instruction mix and data organization) and a performance improvement ranging up to 30 percent.
%% ===== \end{flushleft}
%% ===== 
%% ===== 
%% ===== 
%% ===== 
%% ===== 
%% ===== \begin{flushleft}

\subsection{CACHE MEMORY ORGANIZATION}

%% ===== \end{flushleft}
%% ===== 
%% ===== 
%% ===== \begin{flushleft}
%% ===== The cache memory is implemented as 2048 36-bit words of high-speed register storage with
%% ===== \end{flushleft}
%% ===== 
%% ===== 
%% ===== \begin{flushleft}
%% ===== associated control and content directory circuitry within the processor. It is fully integrated with
%% ===== \end{flushleft}
%% ===== 
%% ===== 
%% ===== \begin{flushleft}
%% ===== the normal data path circuitry and is virtually invisible to all programming sequences. Parity is
%% ===== \end{flushleft}
%% ===== 
%% ===== 
%% ===== \begin{flushleft}
%% ===== generated, stored, and/or checked on each data reference. The total storage is divided into 512
%% ===== \end{flushleft}
%% ===== 
%% ===== 
%% ===== \begin{flushleft}
%% ===== blocks of 4 words each and the blocks are organized into 128 columns of four levels each.
%% ===== \end{flushleft}
%% ===== 
%% ===== 
%% ===== 
%% ===== 
%% ===== 
%% ===== \begin{flushleft}

\subsubsection{Cache Memory / Main Memory Mapping}

%% ===== \end{flushleft}
%% ===== 
%% ===== 
%% ===== \begin{flushleft}
%% ===== Main memory is mapped into the cache memory as described below and shown in Figure
%% ===== \end{flushleft}
%% ===== 
%% ===== 
%% ===== 9-1.
%% ===== 
%% ===== 
%% ===== \begin{flushleft}
%% ===== Main memory is divided into N blocks of 4 words each arranged in ascending order
%% ===== \end{flushleft}
%% ===== 
%% ===== 
%% ===== \begin{flushleft}
%% ===== and numbered with the value of Y15,21 of the first word of the block.
%% ===== \end{flushleft}
%% ===== 
%% ===== 
%% ===== \begin{flushleft}
%% ===== All main memory blocks with numbers n modulo 128 are grouped associatively with
%% ===== \end{flushleft}
%% ===== 
%% ===== 
%% ===== \begin{flushleft}
%% ===== cache memory column n.
%% ===== \end{flushleft}
%% ===== 
%% ===== 
%% ===== \begin{flushleft}
%% ===== Each cache memory column may hold any four blocks of the associated set of main
%% ===== \end{flushleft}
%% ===== 
%% ===== 
%% ===== \begin{flushleft}
%% ===== memory blocks.
%% ===== \end{flushleft}
%% ===== 
%% ===== 
%% ===== \begin{flushleft}
%% ===== Each cache memory column has associated with it a four entry directory (one entry for
%% ===== \end{flushleft}
%% ===== 
%% ===== 
%% ===== \begin{flushleft}
%% ===== each level) and a 2-bit round robin counter. Parity is generated, stored, and checked
%% ===== \end{flushleft}
%% ===== 
%% ===== 
%% ===== \begin{flushleft}
%% ===== on each directory entry.
%% ===== \end{flushleft}
%% ===== 
%% ===== 
%% ===== \begin{flushleft}
%% ===== A cache directory entry consists of a 15-bit ADDRESS register, a pre-set, 2-bit level
%% ===== \end{flushleft}
%% ===== 
%% ===== 
%% ===== \begin{flushleft}
%% ===== number value and a level full flag bit.
%% ===== \end{flushleft}
%% ===== 
%% ===== 
%% ===== 
%% ===== 
%% ===== 
%% ===== \begin{flushleft}
%% ===== \newpage
%% ===== When a main memory block is loaded into a cache memory block at some level in the
%% ===== \end{flushleft}
%% ===== 
%% ===== 
%% ===== \begin{flushleft}
%% ===== associated column, the directory ADDRESS register for that column and level is
%% ===== \end{flushleft}
%% ===== 
%% ===== 
%% ===== \begin{flushleft}
%% ===== loaded with Y0,14. (Level selection is discussed in {``}Cache Memory Control'' later in
%% ===== \end{flushleft}
%% ===== 
%% ===== 
%% ===== \begin{flushleft}
%% ===== this section.)
%% ===== \end{flushleft}
%% ===== 
%% ===== 
%% ===== 
%% ===== 
%% ===== 
%% ===== \begin{flushleft}
%% ===== Main
%% ===== \end{flushleft}
%% ===== 
%% ===== 
%% ===== \begin{flushleft}
%% ===== Memory
%% ===== \end{flushleft}
%% ===== 
%% ===== 
%% ===== 
%% ===== 
%% ===== 
%% ===== \begin{flushleft}
%% ===== Block
%% ===== \end{flushleft}
%% ===== 
%% ===== 
%% ===== 0
%% ===== 
%% ===== 
%% ===== \begin{flushleft}
%% ===== Words
%% ===== \end{flushleft}
%% ===== 
%% ===== 
%% ===== 0,3
%% ===== 
%% ===== 
%% ===== 
%% ===== 
%% ===== 
%% ===== \begin{flushleft}
%% ===== Block
%% ===== \end{flushleft}
%% ===== 
%% ===== 
%% ===== 1
%% ===== 
%% ===== 
%% ===== \begin{flushleft}
%% ===== Words
%% ===== \end{flushleft}
%% ===== 
%% ===== 
%% ===== 4,7
%% ===== 
%% ===== 
%% ===== 
%% ===== 
%% ===== 
%% ===== \begin{flushleft}
%% ===== Block
%% ===== \end{flushleft}
%% ===== 
%% ===== 
%% ===== 2
%% ===== 
%% ===== 
%% ===== \begin{flushleft}
%% ===== Words
%% ===== \end{flushleft}
%% ===== 
%% ===== 
%% ===== 8,11
%% ===== 
%% ===== 
%% ===== 
%% ===== 
%% ===== 
%% ===== \begin{flushleft}
%% ===== Block
%% ===== \end{flushleft}
%% ===== 
%% ===== 
%% ===== 128
%% ===== 
%% ===== 
%% ===== \begin{flushleft}
%% ===== Words
%% ===== \end{flushleft}
%% ===== 
%% ===== 
%% ===== 512,515
%% ===== 
%% ===== 
%% ===== 
%% ===== 
%% ===== 
%% ===== \begin{flushleft}
%% ===== Block
%% ===== \end{flushleft}
%% ===== 
%% ===== 
%% ===== 129
%% ===== 
%% ===== 
%% ===== \begin{flushleft}
%% ===== Words
%% ===== \end{flushleft}
%% ===== 
%% ===== 
%% ===== 516,519
%% ===== 
%% ===== 
%% ===== 
%% ===== 
%% ===== 
%% ===== ...
%% ===== 
%% ===== 
%% ===== 
%% ===== 
%% ===== 
%% ===== \begin{flushleft}
%% ===== Block
%% ===== \end{flushleft}
%% ===== 
%% ===== 
%% ===== \begin{flushleft}
%% ===== N-128
%% ===== \end{flushleft}
%% ===== 
%% ===== 
%% ===== \begin{flushleft}
%% ===== Words
%% ===== \end{flushleft}
%% ===== 
%% ===== 
%% ===== -512,-509
%% ===== 
%% ===== 
%% ===== 
%% ===== 
%% ===== 
%% ===== ...
%% ===== 
%% ===== 
%% ===== 
%% ===== 
%% ===== 
%% ===== \begin{flushleft}
%% ===== Block
%% ===== \end{flushleft}
%% ===== 
%% ===== 
%% ===== \begin{flushleft}
%% ===== N-127
%% ===== \end{flushleft}
%% ===== 
%% ===== 
%% ===== \begin{flushleft}
%% ===== Words
%% ===== \end{flushleft}
%% ===== 
%% ===== 
%% ===== -508,-505
%% ===== 
%% ===== 
%% ===== 
%% ===== 
%% ===== 
%% ===== ...
%% ===== 
%% ===== 
%% ===== 
%% ===== 
%% ===== 
%% ===== \begin{flushleft}
%% ===== Block
%% ===== \end{flushleft}
%% ===== 
%% ===== 
%% ===== 126
%% ===== 
%% ===== 
%% ===== \begin{flushleft}
%% ===== Words
%% ===== \end{flushleft}
%% ===== 
%% ===== 
%% ===== 504,507
%% ===== 
%% ===== 
%% ===== 
%% ===== 
%% ===== 
%% ===== \begin{flushleft}
%% ===== Block
%% ===== \end{flushleft}
%% ===== 
%% ===== 
%% ===== 127
%% ===== 
%% ===== 
%% ===== \begin{flushleft}
%% ===== Words
%% ===== \end{flushleft}
%% ===== 
%% ===== 
%% ===== 508,511
%% ===== 
%% ===== 
%% ===== 
%% ===== 
%% ===== 
%% ===== \begin{flushleft}
%% ===== Block
%% ===== \end{flushleft}
%% ===== 
%% ===== 
%% ===== 130
%% ===== 
%% ===== 
%% ===== \begin{flushleft}
%% ===== Words
%% ===== \end{flushleft}
%% ===== 
%% ===== 
%% ===== 520,523
%% ===== 
%% ===== 
%% ===== 
%% ===== 
%% ===== 
%% ===== ...
%% ===== 
%% ===== 
%% ===== 
%% ===== 
%% ===== 
%% ===== \begin{flushleft}
%% ===== Block
%% ===== \end{flushleft}
%% ===== 
%% ===== 
%% ===== 254
%% ===== 
%% ===== 
%% ===== \begin{flushleft}
%% ===== Words
%% ===== \end{flushleft}
%% ===== 
%% ===== 
%% ===== 1016,1019
%% ===== 
%% ===== 
%% ===== 
%% ===== 
%% ===== 
%% ===== \begin{flushleft}
%% ===== Block
%% ===== \end{flushleft}
%% ===== 
%% ===== 
%% ===== 255
%% ===== 
%% ===== 
%% ===== \begin{flushleft}
%% ===== Words
%% ===== \end{flushleft}
%% ===== 
%% ===== 
%% ===== 1020,1023
%% ===== 
%% ===== 
%% ===== 
%% ===== 
%% ===== 
%% ===== ...
%% ===== 
%% ===== 
%% ===== 
%% ===== 
%% ===== 
%% ===== ...
%% ===== 
%% ===== 
%% ===== 
%% ===== 
%% ===== 
%% ===== ...
%% ===== 
%% ===== 
%% ===== 
%% ===== 
%% ===== 
%% ===== \begin{flushleft}
%% ===== Block
%% ===== \end{flushleft}
%% ===== 
%% ===== 
%% ===== \begin{flushleft}
%% ===== N-126
%% ===== \end{flushleft}
%% ===== 
%% ===== 
%% ===== \begin{flushleft}
%% ===== Words
%% ===== \end{flushleft}
%% ===== 
%% ===== 
%% ===== -504,-501
%% ===== 
%% ===== 
%% ===== 
%% ===== 
%% ===== 
%% ===== ...
%% ===== 
%% ===== 
%% ===== 
%% ===== 
%% ===== 
%% ===== ...
%% ===== 
%% ===== 
%% ===== 
%% ===== 
%% ===== 
%% ===== \begin{flushleft}
%% ===== Block
%% ===== \end{flushleft}
%% ===== 
%% ===== 
%% ===== \begin{flushleft}
%% ===== N-2
%% ===== \end{flushleft}
%% ===== 
%% ===== 
%% ===== \begin{flushleft}
%% ===== Words
%% ===== \end{flushleft}
%% ===== 
%% ===== 
%% ===== -8,-5
%% ===== 
%% ===== 
%% ===== 
%% ===== 
%% ===== 
%% ===== \begin{flushleft}
%% ===== Block
%% ===== \end{flushleft}
%% ===== 
%% ===== 
%% ===== \begin{flushleft}
%% ===== N-1
%% ===== \end{flushleft}
%% ===== 
%% ===== 
%% ===== \begin{flushleft}
%% ===== Words
%% ===== \end{flushleft}
%% ===== 
%% ===== 
%% ===== -4,-1
%% ===== 
%% ===== 
%% ===== 
%% ===== 
%% ===== 
%% ===== \begin{flushleft}
%% ===== Column
%% ===== \end{flushleft}
%% ===== 
%% ===== 
%% ===== 127
%% ===== 
%% ===== 
%% ===== \begin{flushleft}
%% ===== Level
%% ===== \end{flushleft}
%% ===== 
%% ===== 
%% ===== 0
%% ===== 
%% ===== 
%% ===== 
%% ===== 
%% ===== 
%% ===== ...
%% ===== 
%% ===== 
%% ===== 
%% ===== 
%% ===== 
%% ===== \begin{flushleft}
%% ===== Cache
%% ===== \end{flushleft}
%% ===== 
%% ===== 
%% ===== \begin{flushleft}
%% ===== Memory
%% ===== \end{flushleft}
%% ===== 
%% ===== 
%% ===== 
%% ===== 
%% ===== 
%% ===== \begin{flushleft}
%% ===== Column
%% ===== \end{flushleft}
%% ===== 
%% ===== 
%% ===== 0
%% ===== 
%% ===== 
%% ===== \begin{flushleft}
%% ===== Level
%% ===== \end{flushleft}
%% ===== 
%% ===== 
%% ===== 0
%% ===== 
%% ===== 
%% ===== 
%% ===== 
%% ===== 
%% ===== \begin{flushleft}
%% ===== Column
%% ===== \end{flushleft}
%% ===== 
%% ===== 
%% ===== 1
%% ===== 
%% ===== 
%% ===== \begin{flushleft}
%% ===== Level
%% ===== \end{flushleft}
%% ===== 
%% ===== 
%% ===== 0
%% ===== 
%% ===== 
%% ===== 
%% ===== 
%% ===== 
%% ===== \begin{flushleft}
%% ===== Column
%% ===== \end{flushleft}
%% ===== 
%% ===== 
%% ===== 2
%% ===== 
%% ===== 
%% ===== \begin{flushleft}
%% ===== Level
%% ===== \end{flushleft}
%% ===== 
%% ===== 
%% ===== 0
%% ===== 
%% ===== 
%% ===== 
%% ===== 
%% ===== 
%% ===== ...
%% ===== 
%% ===== 
%% ===== 
%% ===== 
%% ===== 
%% ===== \begin{flushleft}
%% ===== Column
%% ===== \end{flushleft}
%% ===== 
%% ===== 
%% ===== 126
%% ===== 
%% ===== 
%% ===== \begin{flushleft}
%% ===== Level
%% ===== \end{flushleft}
%% ===== 
%% ===== 
%% ===== 0
%% ===== 
%% ===== 
%% ===== 
%% ===== 
%% ===== 
%% ===== \begin{flushleft}
%% ===== Column
%% ===== \end{flushleft}
%% ===== 
%% ===== 
%% ===== 0
%% ===== 
%% ===== 
%% ===== \begin{flushleft}
%% ===== Level
%% ===== \end{flushleft}
%% ===== 
%% ===== 
%% ===== 1
%% ===== 
%% ===== 
%% ===== 
%% ===== 
%% ===== 
%% ===== \begin{flushleft}
%% ===== Column
%% ===== \end{flushleft}
%% ===== 
%% ===== 
%% ===== 1
%% ===== 
%% ===== 
%% ===== \begin{flushleft}
%% ===== Level
%% ===== \end{flushleft}
%% ===== 
%% ===== 
%% ===== 1
%% ===== 
%% ===== 
%% ===== 
%% ===== 
%% ===== 
%% ===== \begin{flushleft}
%% ===== Column
%% ===== \end{flushleft}
%% ===== 
%% ===== 
%% ===== 2
%% ===== 
%% ===== 
%% ===== \begin{flushleft}
%% ===== Level
%% ===== \end{flushleft}
%% ===== 
%% ===== 
%% ===== 1
%% ===== 
%% ===== 
%% ===== 
%% ===== 
%% ===== 
%% ===== ...
%% ===== 
%% ===== 
%% ===== 
%% ===== 
%% ===== 
%% ===== \begin{flushleft}
%% ===== Column
%% ===== \end{flushleft}
%% ===== 
%% ===== 
%% ===== 126
%% ===== 
%% ===== 
%% ===== \begin{flushleft}
%% ===== Level
%% ===== \end{flushleft}
%% ===== 
%% ===== 
%% ===== 1
%% ===== 
%% ===== 
%% ===== 
%% ===== 
%% ===== 
%% ===== \begin{flushleft}
%% ===== Column
%% ===== \end{flushleft}
%% ===== 
%% ===== 
%% ===== 127
%% ===== 
%% ===== 
%% ===== \begin{flushleft}
%% ===== Level
%% ===== \end{flushleft}
%% ===== 
%% ===== 
%% ===== 1
%% ===== 
%% ===== 
%% ===== 
%% ===== 
%% ===== 
%% ===== \begin{flushleft}
%% ===== Column
%% ===== \end{flushleft}
%% ===== 
%% ===== 
%% ===== 0
%% ===== 
%% ===== 
%% ===== \begin{flushleft}
%% ===== Level
%% ===== \end{flushleft}
%% ===== 
%% ===== 
%% ===== 2
%% ===== 
%% ===== 
%% ===== 
%% ===== 
%% ===== 
%% ===== \begin{flushleft}
%% ===== Column
%% ===== \end{flushleft}
%% ===== 
%% ===== 
%% ===== 1
%% ===== 
%% ===== 
%% ===== \begin{flushleft}
%% ===== Level
%% ===== \end{flushleft}
%% ===== 
%% ===== 
%% ===== 2
%% ===== 
%% ===== 
%% ===== 
%% ===== 
%% ===== 
%% ===== \begin{flushleft}
%% ===== Column
%% ===== \end{flushleft}
%% ===== 
%% ===== 
%% ===== 2
%% ===== 
%% ===== 
%% ===== \begin{flushleft}
%% ===== Level
%% ===== \end{flushleft}
%% ===== 
%% ===== 
%% ===== 2
%% ===== 
%% ===== 
%% ===== 
%% ===== 
%% ===== 
%% ===== ...
%% ===== 
%% ===== 
%% ===== 
%% ===== 
%% ===== 
%% ===== \begin{flushleft}
%% ===== Column
%% ===== \end{flushleft}
%% ===== 
%% ===== 
%% ===== 126
%% ===== 
%% ===== 
%% ===== \begin{flushleft}
%% ===== Level
%% ===== \end{flushleft}
%% ===== 
%% ===== 
%% ===== 2
%% ===== 
%% ===== 
%% ===== 
%% ===== 
%% ===== 
%% ===== \begin{flushleft}
%% ===== Column
%% ===== \end{flushleft}
%% ===== 
%% ===== 
%% ===== 127
%% ===== 
%% ===== 
%% ===== \begin{flushleft}
%% ===== Level
%% ===== \end{flushleft}
%% ===== 
%% ===== 
%% ===== 2
%% ===== 
%% ===== 
%% ===== 
%% ===== 
%% ===== 
%% ===== \begin{flushleft}
%% ===== Column
%% ===== \end{flushleft}
%% ===== 
%% ===== 
%% ===== 0
%% ===== 
%% ===== 
%% ===== \begin{flushleft}
%% ===== Level
%% ===== \end{flushleft}
%% ===== 
%% ===== 
%% ===== 3
%% ===== 
%% ===== 
%% ===== 
%% ===== 
%% ===== 
%% ===== \begin{flushleft}
%% ===== Column
%% ===== \end{flushleft}
%% ===== 
%% ===== 
%% ===== 1
%% ===== 
%% ===== 
%% ===== \begin{flushleft}
%% ===== Level
%% ===== \end{flushleft}
%% ===== 
%% ===== 
%% ===== 3
%% ===== 
%% ===== 
%% ===== 
%% ===== 
%% ===== 
%% ===== \begin{flushleft}
%% ===== Column
%% ===== \end{flushleft}
%% ===== 
%% ===== 
%% ===== 2
%% ===== 
%% ===== 
%% ===== \begin{flushleft}
%% ===== Level
%% ===== \end{flushleft}
%% ===== 
%% ===== 
%% ===== 3
%% ===== 
%% ===== 
%% ===== 
%% ===== 
%% ===== 
%% ===== ...
%% ===== 
%% ===== 
%% ===== 
%% ===== 
%% ===== 
%% ===== \begin{flushleft}
%% ===== Column
%% ===== \end{flushleft}
%% ===== 
%% ===== 
%% ===== 126
%% ===== 
%% ===== 
%% ===== \begin{flushleft}
%% ===== Level
%% ===== \end{flushleft}
%% ===== 
%% ===== 
%% ===== 3
%% ===== 
%% ===== 
%% ===== 
%% ===== 
%% ===== 
%% ===== \begin{flushleft}
%% ===== Column
%% ===== \end{flushleft}
%% ===== 
%% ===== 
%% ===== 127
%% ===== 
%% ===== 
%% ===== \begin{flushleft}
%% ===== Level
%% ===== \end{flushleft}
%% ===== 
%% ===== 
%% ===== 3
%% ===== 
%% ===== 
%% ===== 
%% ===== 
%% ===== 
%% ===== \begin{flushleft}
%% ===== Figure 9-1. Main Memory/Cache Memory Mapping
%% ===== \end{flushleft}
%% ===== 
%% ===== 
%% ===== 
%% ===== 
%% ===== 
%% ===== \begin{flushleft}

\subsubsection{Cache Memory Addressing}

%% ===== \end{flushleft}
%% ===== 
%% ===== 
%% ===== \begin{flushleft}
%% ===== For a read operation, the 24-bit absolute main memory address prepared by the appending
%% ===== \end{flushleft}
%% ===== 
%% ===== 
%% ===== \begin{flushleft}
%% ===== unit is presented simultaneously to the cache control and to the main memory port selection
%% ===== \end{flushleft}
%% ===== 
%% ===== 
%% ===== \begin{flushleft}
%% ===== circuitry. While port selection is being accomplished, the cache memory is accessed as follows.
%% ===== \end{flushleft}
%% ===== 
%% ===== 
%% ===== \begin{flushleft}
%% ===== Yl5,21 are used to select a cache memory column.
%% ===== \end{flushleft}
%% ===== 
%% ===== 
%% ===== 
%% ===== 
%% ===== 
%% ===== \begin{flushleft}
%% ===== \newpage
%% ===== Y0,14 are matched associatively against the four directory ADDRESS registers for the
%% ===== \end{flushleft}
%% ===== 
%% ===== 
%% ===== \begin{flushleft}
%% ===== selected column.
%% ===== \end{flushleft}
%% ===== 
%% ===== 
%% ===== \begin{flushleft}
%% ===== If a match occurs for a level whose full flag is ON, a hit is signaled, the main memory
%% ===== \end{flushleft}
%% ===== 
%% ===== 
%% ===== \begin{flushleft}
%% ===== reference cycle is canceled, and the level number value is read out.
%% ===== \end{flushleft}
%% ===== 
%% ===== 
%% ===== \begin{flushleft}
%% ===== The level number value and Y22,23 are used to select the level and word in the selected
%% ===== \end{flushleft}
%% ===== 
%% ===== 
%% ===== \begin{flushleft}
%% ===== column and the cache memory data is read out into the data circuitry.
%% ===== \end{flushleft}
%% ===== 
%% ===== 
%% ===== \begin{flushleft}
%% ===== If no hit is signaled, the main memory reference cycle proceeds and a cache memory
%% ===== \end{flushleft}
%% ===== 
%% ===== 
%% ===== \begin{flushleft}
%% ===== block load cycle is initiated (see {``}Cache Memory Control'' below).
%% ===== \end{flushleft}
%% ===== 
%% ===== 
%% ===== \begin{flushleft}
%% ===== For a write operation, the 24-bit absolute main memory address prepared by the appending
%% ===== \end{flushleft}
%% ===== 
%% ===== 
%% ===== \begin{flushleft}
%% ===== unit is presented simultaneously to the cache control and to the main memory port selection
%% ===== \end{flushleft}
%% ===== 
%% ===== 
%% ===== \begin{flushleft}
%% ===== circuitry. While port selection is being accomplished, the cache memory is accessed as follows.
%% ===== \end{flushleft}
%% ===== 
%% ===== 
%% ===== \begin{flushleft}
%% ===== Y15,21 are used to select a cache memory column.
%% ===== \end{flushleft}
%% ===== 
%% ===== 
%% ===== \begin{flushleft}
%% ===== Y0,14 are matched associatively against the four directory ADDRESS registers for the
%% ===== \end{flushleft}
%% ===== 
%% ===== 
%% ===== \begin{flushleft}
%% ===== selected column.
%% ===== \end{flushleft}
%% ===== 
%% ===== 
%% ===== \begin{flushleft}
%% ===== If a match occurs for a level whose full flag is ON, a hit is signaled and the level
%% ===== \end{flushleft}
%% ===== 
%% ===== 
%% ===== \begin{flushleft}
%% ===== number value is read out.
%% ===== \end{flushleft}
%% ===== 
%% ===== 
%% ===== \begin{flushleft}
%% ===== The level number value and Y22,23 are used to select the level and word in the selected
%% ===== \end{flushleft}
%% ===== 
%% ===== 
%% ===== \begin{flushleft}
%% ===== column, a cache memory write cycle is enabled, and the data is written to the main
%% ===== \end{flushleft}
%% ===== 
%% ===== 
%% ===== \begin{flushleft}
%% ===== memory and the cache memory simultaneously.
%% ===== \end{flushleft}
%% ===== 
%% ===== 
%% ===== \begin{flushleft}
%% ===== If no hit is signaled, the main memory reference cycle proceeds with no further cache
%% ===== \end{flushleft}
%% ===== 
%% ===== 
%% ===== \begin{flushleft}
%% ===== memory action.
%% ===== \end{flushleft}
%% ===== 
%% ===== 
%% ===== 
%% ===== 
%% ===== 
%% ===== \begin{flushleft}
%% ===== \newpage

\subsection{CACHE MEMORY CONTROL}

%% ===== \end{flushleft}
%% ===== 
%% ===== 
%% ===== \begin{flushleft}

\subsubsection{Enabling and Disabling Cache Memory}

%% ===== \end{flushleft}
%% ===== 
%% ===== 
%% ===== \begin{flushleft}
%% ===== The cache memory is controlled by the state of several bits in the cache mode register (see
%% ===== \end{flushleft}
%% ===== 
%% ===== 
%% ===== \begin{flushleft}
%% ===== Section 3). The cache mode register may be loaded with the Load Central Processor Register
%% ===== \end{flushleft}
%% ===== 
%% ===== 
%% ===== \begin{flushleft}
%% ===== (lcpr) instruction. The cache memory control bits are as follows:
%% ===== \end{flushleft}
%% ===== 
%% ===== 
%% ===== 
%% ===== 
%% ===== 
%% ===== \begin{flushleft}
%% ===== bit
%% ===== \end{flushleft}
%% ===== 
%% ===== 
%% ===== 54
%% ===== 
%% ===== 
%% ===== 
%% ===== 
%% ===== 
%% ===== 55
%% ===== 
%% ===== 
%% ===== 
%% ===== 
%% ===== 
%% ===== 56
%% ===== 
%% ===== 
%% ===== 57
%% ===== 
%% ===== 
%% ===== 59
%% ===== 
%% ===== 
%% ===== 
%% ===== 
%% ===== 
%% ===== \begin{flushleft}
%% ===== NOTE:
%% ===== \end{flushleft}
%% ===== 
%% ===== 
%% ===== 
%% ===== 
%% ===== 
%% ===== \begin{flushleft}
%% ===== Value Action
%% ===== \end{flushleft}
%% ===== 
%% ===== 
%% ===== 0
%% ===== 
%% ===== 
%% ===== 
%% ===== 
%% ===== 
%% ===== \begin{flushleft}
%% ===== The lower half of the cache memory (levels 0 and 1) is disabled.
%% ===== \end{flushleft}
%% ===== 
%% ===== 
%% ===== 
%% ===== 
%% ===== 
%% ===== 1
%% ===== 
%% ===== 
%% ===== 
%% ===== 
%% ===== 
%% ===== \begin{flushleft}
%% ===== The lower half of the cache memory is active and enabled as per the state of bits
%% ===== \end{flushleft}
%% ===== 
%% ===== 
%% ===== 56-57.
%% ===== 
%% ===== 
%% ===== 
%% ===== 
%% ===== 
%% ===== 0
%% ===== 
%% ===== 
%% ===== 
%% ===== 
%% ===== 
%% ===== \begin{flushleft}
%% ===== The upper half of the cache memory (levels 2 and 3) is disabled.
%% ===== \end{flushleft}
%% ===== 
%% ===== 
%% ===== 
%% ===== 
%% ===== 
%% ===== 1
%% ===== 
%% ===== 
%% ===== 
%% ===== 
%% ===== 
%% ===== \begin{flushleft}
%% ===== The upper half of the cache memory is active and enabled as per the state of bits
%% ===== \end{flushleft}
%% ===== 
%% ===== 
%% ===== 56-57.
%% ===== 
%% ===== 
%% ===== 
%% ===== 
%% ===== 
%% ===== 0
%% ===== 
%% ===== 
%% ===== 
%% ===== 
%% ===== 
%% ===== \begin{flushleft}
%% ===== The cache memory (if active) is not used for operands and indirect words.
%% ===== \end{flushleft}
%% ===== 
%% ===== 
%% ===== 
%% ===== 
%% ===== 
%% ===== 1
%% ===== 
%% ===== 
%% ===== 
%% ===== 
%% ===== 
%% ===== \begin{flushleft}
%% ===== The cache memory (if active) is used for operands and indirect words.
%% ===== \end{flushleft}
%% ===== 
%% ===== 
%% ===== 
%% ===== 
%% ===== 
%% ===== 0
%% ===== 
%% ===== 
%% ===== 
%% ===== 
%% ===== 
%% ===== \begin{flushleft}
%% ===== The cache memory (if active) is not used for instructions.
%% ===== \end{flushleft}
%% ===== 
%% ===== 
%% ===== 
%% ===== 
%% ===== 
%% ===== 1
%% ===== 
%% ===== 
%% ===== 
%% ===== 
%% ===== 
%% ===== \begin{flushleft}
%% ===== The cache memory (if active) is used for instructions.
%% ===== \end{flushleft}
%% ===== 
%% ===== 
%% ===== 
%% ===== 
%% ===== 
%% ===== 0
%% ===== 
%% ===== 
%% ===== 
%% ===== 
%% ===== 
%% ===== \begin{flushleft}
%% ===== The cache-to-register mode is not in effect (see {``}Dumping the Cache Memory''
%% ===== \end{flushleft}
%% ===== 
%% ===== 
%% ===== \begin{flushleft}
%% ===== later in this section).
%% ===== \end{flushleft}
%% ===== 
%% ===== 
%% ===== 
%% ===== 
%% ===== 
%% ===== 1
%% ===== 
%% ===== 
%% ===== 
%% ===== 
%% ===== 
%% ===== \begin{flushleft}
%% ===== The cache-to-register mode is in effect.
%% ===== \end{flushleft}
%% ===== 
%% ===== 
%% ===== 
%% ===== 
%% ===== 
%% ===== \begin{flushleft}
%% ===== The cache memory option furnishes a switch panel maintenance aid that attaches to the
%% ===== \end{flushleft}
%% ===== 
%% ===== 
%% ===== \begin{flushleft}
%% ===== free edge of the cache memory control logic board. The switch panel provides six
%% ===== \end{flushleft}
%% ===== 
%% ===== 
%% ===== \begin{flushleft}
%% ===== switches for manual control of the cache memory:
%% ===== \end{flushleft}
%% ===== 
%% ===== 
%% ===== \begin{flushleft}
%% ===== Four of the switches inhibit the control functions of bits 54-57 of the cache mode
%% ===== \end{flushleft}
%% ===== 
%% ===== 
%% ===== \begin{flushleft}
%% ===== register and have the effect of forcing the corresponding function to be disabled.
%% ===== \end{flushleft}
%% ===== 
%% ===== 
%% ===== \begin{flushleft}
%% ===== The fifth switch inhibits the store-aside feature wherein the processor is permitted to
%% ===== \end{flushleft}
%% ===== 
%% ===== 
%% ===== \begin{flushleft}
%% ===== proceed immediately after the cache memory write cycle on write operations without
%% ===== \end{flushleft}
%% ===== 
%% ===== 
%% ===== \begin{flushleft}
%% ===== waiting for a data acknowledgment from main memory. (There is no software control
%% ===== \end{flushleft}
%% ===== 
%% ===== 
%% ===== \begin{flushleft}
%% ===== corresponding to this switch).
%% ===== \end{flushleft}
%% ===== 
%% ===== 
%% ===== \begin{flushleft}
%% ===== The sixth switch forces the enabled condition on all cache memory controls (except
%% ===== \end{flushleft}
%% ===== 
%% ===== 
%% ===== \begin{flushleft}
%% ===== cache-to-register mode) without regard to the corresponding cache mode register
%% ===== \end{flushleft}
%% ===== 
%% ===== 
%% ===== \begin{flushleft}
%% ===== control bit.
%% ===== \end{flushleft}
%% ===== 
%% ===== 
%% ===== \begin{flushleft}
%% ===== There is no switch corresponding to the cache-to-register control bit.
%% ===== \end{flushleft}
%% ===== 
%% ===== 
%% ===== \begin{flushleft}
%% ===== While these switches are intended primarily for maintenance sessions, they have been
%% ===== \end{flushleft}
%% ===== 
%% ===== 
%% ===== \begin{flushleft}
%% ===== found useful in testing the cache memory during normal operation and in permitting
%% ===== \end{flushleft}
%% ===== 
%% ===== 
%% ===== \begin{flushleft}
%% ===== operation of the processor with the cache memory in degraded or partially disabled
%% ===== \end{flushleft}
%% ===== 
%% ===== 
%% ===== \begin{flushleft}
%% ===== mode.
%% ===== \end{flushleft}
%% ===== 
%% ===== 
%% ===== 
%% ===== 
%% ===== 
%% ===== \begin{flushleft}

\subsubsection{Cache Memory Control in Segment Descriptor Words}

%% ===== \end{flushleft}
%% ===== 
%% ===== 
%% ===== \begin{flushleft}
%% ===== Certain data have characteristics such that they should never be loaded into the cache
%% ===== \end{flushleft}
%% ===== 
%% ===== 
%% ===== \begin{flushleft}
%% ===== memory. Primary examples of such data are hardware mailboxes for the I/O multiplexer, bulk
%% ===== \end{flushleft}
%% ===== 
%% ===== 
%% ===== \begin{flushleft}
%% ===== store controller, etc., status return words, and various dynamic operating system data base
%% ===== \end{flushleft}
%% ===== 
%% ===== 
%% ===== \begin{flushleft}
%% ===== segments. In general, any data that is modified by an agency external to a processor with the
%% ===== \end{flushleft}
%% ===== 
%% ===== 
%% ===== \begin{flushleft}
%% ===== intent to convey information to that processor should never be loaded into cache memory.
%% ===== \end{flushleft}
%% ===== 
%% ===== 
%% ===== 
%% ===== 
%% ===== 
%% ===== \begin{flushleft}
%% ===== \newpage
%% ===== Bit 57 of the segment descriptor word is used to reflect this property of {``}encacheability'' for
%% ===== \end{flushleft}
%% ===== 
%% ===== 
%% ===== \begin{flushleft}
%% ===== each segment. (See Section 5 for a discussion of the segment descriptor word.) If the bit is set
%% ===== \end{flushleft}
%% ===== 
%% ===== 
%% ===== \begin{flushleft}
%% ===== ON, data from the segment may be loaded into the cache memory; if the bit is OFF, they may not.
%% ===== \end{flushleft}
%% ===== 
%% ===== 
%% ===== \begin{flushleft}
%% ===== The operating system may set bit 57 ON or OFF as appropriate for the use of the segment.
%% ===== \end{flushleft}
%% ===== 
%% ===== 
%% ===== 
%% ===== 
%% ===== 
%% ===== \begin{flushleft}

\subsubsection{Loading the Cache Memory}

%% ===== \end{flushleft}
%% ===== 
%% ===== 
%% ===== \begin{flushleft}
%% ===== The cache memory is loaded with data implicitly whenever a cache memory block load is
%% ===== \end{flushleft}
%% ===== 
%% ===== 
%% ===== \begin{flushleft}
%% ===== required. (See the discussion of read operations in {``}Cache Memory Addressing'' earlier in this
%% ===== \end{flushleft}
%% ===== 
%% ===== 
%% ===== \begin{flushleft}
%% ===== section.) There is no explicit method or instruction to load data into the cache memory.
%% ===== \end{flushleft}
%% ===== 
%% ===== 
%% ===== \begin{flushleft}
%% ===== When a cache memory block load is required, the level is selected from the value of the
%% ===== \end{flushleft}
%% ===== 
%% ===== 
%% ===== \begin{flushleft}
%% ===== round robin counter for the selected column, and the cache memory write function is enabled.
%% ===== \end{flushleft}
%% ===== 
%% ===== 
%% ===== \begin{flushleft}
%% ===== (The round robin counter contains the number of the least recently loaded level.) When the data
%% ===== \end{flushleft}
%% ===== 
%% ===== 
%% ===== \begin{flushleft}
%% ===== arrives from main memory, it is written into the cache memory and entered into the data circuitry.
%% ===== \end{flushleft}
%% ===== 
%% ===== 
%% ===== \begin{flushleft}
%% ===== The processor proceeds with the execution of the instruction requiring the operand if appropriate.
%% ===== \end{flushleft}
%% ===== 
%% ===== 
%% ===== \begin{flushleft}
%% ===== When the cache memory write is complete, further virtual address formation is inhibited,
%% ===== \end{flushleft}
%% ===== 
%% ===== 
%% ===== \begin{flushleft}
%% ===== Y22 is inverted, and a second main memory access for the other half of the block is made. When
%% ===== \end{flushleft}
%% ===== 
%% ===== 
%% ===== \begin{flushleft}
%% ===== the second half data arrives from main memory, it is written into the cache memory, Y0,14 are
%% ===== \end{flushleft}
%% ===== 
%% ===== 
%% ===== \begin{flushleft}
%% ===== loaded into the directory ADDRESS register, the level full flag is set ON, the round robin counter is
%% ===== \end{flushleft}
%% ===== 
%% ===== 
%% ===== \begin{flushleft}
%% ===== advanced by 1, and virtual address formation is permitted to proceed.
%% ===== \end{flushleft}
%% ===== 
%% ===== 
%% ===== \begin{flushleft}
%% ===== If all four level full flags for a column are set ON, a column full flag is also set ON and
%% ===== \end{flushleft}
%% ===== 
%% ===== 
%% ===== \begin{flushleft}
%% ===== remains ON until one or more levels in the column are cleared.
%% ===== \end{flushleft}
%% ===== 
%% ===== 
%% ===== 
%% ===== 
%% ===== 
%% ===== \begin{flushleft}

\subsubsection{Clearing the Cache Memory}

%% ===== \end{flushleft}
%% ===== 
%% ===== 
%% ===== \begin{flushleft}
%% ===== Cache memory can be cleared in two ways; general clear and selective clear. The clearing
%% ===== \end{flushleft}
%% ===== 
%% ===== 
%% ===== \begin{flushleft}
%% ===== action is the same in both cases, namely, the full flags of the selected column(s) and/or level(s) are
%% ===== \end{flushleft}
%% ===== 
%% ===== 
%% ===== \begin{flushleft}
%% ===== set OFF.
%% ===== \end{flushleft}
%% ===== 
%% ===== 
%% ===== 
%% ===== 
%% ===== 
%% ===== \begin{flushleft}

\subsubsubsection{General Clear}

%% ===== \end{flushleft}
%% ===== 
%% ===== 
%% ===== \begin{flushleft}
%% ===== The entire cache memory is cleared by setting all column and level full flags to OFF in the
%% ===== \end{flushleft}
%% ===== 
%% ===== 
%% ===== \begin{flushleft}
%% ===== following situations:
%% ===== \end{flushleft}
%% ===== 
%% ===== 
%% ===== \begin{flushleft}
%% ===== Upper or lower cache memory or both becoming enabled by appropriate bits in the
%% ===== \end{flushleft}
%% ===== 
%% ===== 
%% ===== \begin{flushleft}
%% ===== operand of the Load Central Processor Register (lcpr) instruction or by action of the
%% ===== \end{flushleft}
%% ===== 
%% ===== 
%% ===== \begin{flushleft}
%% ===== cache memory control logic board free edge switches.
%% ===== \end{flushleft}
%% ===== 
%% ===== 
%% ===== \begin{flushleft}
%% ===== Execution of a Clear Associative Memory Segments (cams) instruction with bit 15 of
%% ===== \end{flushleft}
%% ===== 
%% ===== 
%% ===== \begin{flushleft}
%% ===== the address field set ON.
%% ===== \end{flushleft}
%% ===== 
%% ===== 
%% ===== 
%% ===== 
%% ===== 
%% ===== \begin{flushleft}

\subsubsubsection{Selective Clear}

%% ===== \end{flushleft}
%% ===== 
%% ===== 
%% ===== \begin{flushleft}
%% ===== The cache memory is cleared selectively as follows:
%% ===== \end{flushleft}
%% ===== 
%% ===== 
%% ===== \begin{flushleft}
%% ===== If a read-and-clear operation (ldac, sznc, etc.) results in a hit on the cache memory,
%% ===== \end{flushleft}
%% ===== 
%% ===== 
%% ===== \begin{flushleft}
%% ===== that cache memory block hit is cleared.
%% ===== \end{flushleft}
%% ===== 
%% ===== 
%% ===== \begin{flushleft}
%% ===== Execution of a Clear Associative Memory Pages (camp) instruction with address bit 15
%% ===== \end{flushleft}
%% ===== 
%% ===== 
%% ===== \begin{flushleft}
%% ===== set ON causes Y13,14 to be matched against all cache directory ADDRESS registers.
%% ===== \end{flushleft}
%% ===== 
%% ===== 
%% ===== \begin{flushleft}
%% ===== All cache memory blocks hit are cleared.
%% ===== \end{flushleft}
%% ===== 
%% ===== 
%% ===== 
%% ===== 
%% ===== 
%% ===== \begin{flushleft}
%% ===== \newpage

\subsubsection{Dumping the Cache Memory}

%% ===== \end{flushleft}
%% ===== 
%% ===== 
%% ===== \begin{flushleft}
%% ===== When the cache-to-register mode flag (bit 59 of the cache mode register) is set ON, the
%% ===== \end{flushleft}
%% ===== 
%% ===== 
%% ===== \begin{flushleft}
%% ===== processor is forced to fetch the operands of all double-precision operations unit load operations
%% ===== \end{flushleft}
%% ===== 
%% ===== 
%% ===== \begin{flushleft}
%% ===== from the cache memory. Y0,12 are ignored, Y15,21 select a column, and Y13,14 select a level. All
%% ===== \end{flushleft}
%% ===== 
%% ===== 
%% ===== \begin{flushleft}
%% ===== other operations (e.g., instruction fetches, single-precision operands, etc.) are treated normally.
%% ===== \end{flushleft}
%% ===== 
%% ===== 
%% ===== \begin{flushleft}
%% ===== Note that the phrase {``}treated normally'' as used here includes the case where the cache
%% ===== \end{flushleft}
%% ===== 
%% ===== 
%% ===== \begin{flushleft}
%% ===== memory is enabled. If the cache memory is enabled, the {``}other'' operations causes normal block
%% ===== \end{flushleft}
%% ===== 
%% ===== 
%% ===== \begin{flushleft}
%% ===== loads and cache memory writes thus destroying the original contents of the cache memory. The
%% ===== \end{flushleft}
%% ===== 
%% ===== 
%% ===== \begin{flushleft}
%% ===== cache memory should be disabled before dumping is attempted.
%% ===== \end{flushleft}
%% ===== 
%% ===== 
%% ===== \begin{flushleft}
%% ===== An indexed program loop involving the ldaq and staq instructions with the cache-toregister mode bit set ON serves to dump any or all of the cache memory.
%% ===== \end{flushleft}
%% ===== 
%% ===== 
%% ===== \begin{flushleft}
%% ===== The occurrence of a fault or interrupt sets the cache-to-register mode bit to OFF.
%% ===== \end{flushleft}
%% ===== 
%% ===== 
%% ===== 
%% ===== 
%% ===== 
