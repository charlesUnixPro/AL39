
\section{INTRODUCTION}

The processor described in this reference manual is a hardware module designed
for use with Multics. The many distinctive features and functions of Multics
are enhanced by the powerful hardware features of the processor. The addressing
features, in particular, are designed to permit the Multics software to compute
relative and absolute addresses, locate data and programs in the Multics
virtual memory, and retrieve such data and programs as necessary.

%%\addcontentsline{toc}{subsection}{MULTICS PROCESSOR FEATURES}
\subsection{MULTICS PROCESSOR FEATURES}


The Multics processor contains the following general features:

\begin{enumerate}

\item
Storage protection to place access restrictions on specified segments.

\item
Capability to interrupt program execution in response to an external
signal (e.g., I/O termination) at the end of any even/odd instruction pair
(midinstruction interrupts are permitted for some instructions), to save
processor status, and to restore the status at a later time without loss of
continuity of the program.

\item
Capability to fetch instruction pairs and to buffer two instructions (up
to four instructions, depending on certain main memory overlap conditions)
including the one currently in execution.


\item
Overlapping instruction execution, address preparation, and instruction fetch.
While an instruction is being executed, address preparation for the next
operand (or even the operand following it) or the next instruction pair is
taking place. The operations unit can be executing instruction N, instruction
N+1 can be buffered in the operations unit (with its operand buffered in a main
memory port), and the control unit can be executing instructions N+2 or N+3 (if
such execution does not involve the main memory port or registers of
instructions N or N+1) or preparing the address to fetch instructions N+4 and
N+5. This includes the capability to detect store instructions that alter the
contents of buffered instructions and the ability to delay preprocessing of an
address using register modification if the instruction currently in execution
changes the register to be used in that modification.  



\item
Interlacing capability to direct main memory accesses to interlaced system
controller modules.


\item
Intermediate storage of address and control information in high-speed registers
addressable by content (associative memory).


\item
Intermediate storage of base address and control information in pointer
registers that are loaded by the executing program.


\item
Absolute address computation at execution time.

\item
Ability to hold recently referenced operands and instructions in a high-speed
lookaside memory (cache option).


\end{enumerate}




\subsubsection{Segmentation and Paging}
A segment is a collection of data or instructions that is assigned a symbolic
name and addressed symbolically by the user. Paging is controlled by the system
software; the user need not be aware of the existence of pages. User-visible
address preparation is concerned with the calculation of a virtual memory
address; the processor hardware completes address preparation by translating
the final virtual memory address into an absolute main memory address. The user
may view each of his segments as residing in an independent main memory unit.
Each segment has its own origin that can be addressed as location zero. The
size of each segment varies without affecting the addressing of the other
segments.  Each segment can be addressed like a conventional main memory image
starting at location zero. Maximum segment size is 262,144 words.

When viewed from the processor, main memory consists of blocks or page frames,
each of which has a length of {``}page-size'' words. The page size used by
Multics is 1024 words. Each frame begins at an absolute address which is zero
modulo the page size. Any page of a segment can be placed in any available main
memory frame. These pages may be addressed as if they were contiguous, even
though they may be in widely scattered absolute locations. Only currently
referenced pages need be in main memory. A segment need not be paged, in which
case the complete segment is located in contiguous words of main memory. In
Multics, all user segments are paged. See Section \ref{s5} for additional discussion.

\subsubsection{Address Modification and Address Appending}

Before each main memory access, two major phases of address preparation take place:

\begin{enumerate}

\item
Address modification by register or indirect word content, if specified by the
instruction word or indirect word.

\item
Address appending, in which a virtual memory address is translated into an
absolute address to access main memory.

\end{enumerate}

Although the above two types of modification are combined in most operations,
they are described separately in Sections \ref{s5} and \ref{s6}. The address modification
procedure can go on indefinitely, with one type of modification leading to
repetitions of the same type or to other types of modification prior to a main
memory access for an operand.  

\subsubsection{Faults and Interrupts}

The processor detects certain illegal instruction usages, faulty communication
with the main memory, programmed faults, certain external events, and
arithmetic faults. Many of the processor fault conditions are deliberately or
inadvertently caused by the software and do not necessarily involve error
conditions. The processor communicates with the other system modules (I/O
multiplexers, bulk store controllers, and other processors) by setting and
answering external interrupts. When a fault or interrupt is recognized, a
{``}trap'' results. The trap causes the forced execution of a pair of
instructions in a main memory location, unique to the fault or interrupt, known
as the fault or interrupt vector. The first of the forced instructions may
cause safe storage of the processor status. The second instruction in a fault
vector should be some form of transfer, or the faulting program will be resumed
at the point of interruption. Faults and interrupts are described in Section 7.


Interrupts and certain low-priority faults are recognized only at specific
times during the execution of an instruction pair. If, at these times, bit 28
in the instruction word is set ON, the trap is inhibited and program execution
continues. The interrupt or fault signal is saved for future recognition and is
reset only when the trap occurs.


\subsection{PROCESSOR MODES OF OPERATION}


There are three modes of main memory addressing (absolute mode, append mode,
and BAR mode), and two modes of instruction execution (normal mode and
privileged mode).

\subsubsection{Instruction Execution Modes}


\subsubsubsection{Normal Mode} 


Most instructions can be executed in the normal mode. Certain instructions, classed as
privileged, cannot be executed in normal mode. These are identified in the individual instruction
descriptions. An attempt to execute privileged instructions while in the normal mode results in an
illegal procedure fault. The processor executes instructions in normal mode only if it is forming
addresses in append mode and the segment descriptor word (SDW) for the executing segment
specifies a nonprivileged procedure.


\subsubsubsection{Privileged Mode}

In privileged mode, all instructions can be executed. The processor executes instructions in
privileged mode when forming addresses in absolute mode or when forming addresses in append
mode and the segment descriptor word (SDW) for the segment in execution specifies a privileged
procedure and the execution ring is equal to zero. See Sections 5 and 7 for additional discussion.


\subsubsection{Addressing Modes}

\subsubsubsection{Absolute Mode}

In absolute mode, the final computed address is treated as the absolute main
memory address unless the appending hardware mechanism is invoked for a
particular main memory reference. During instruction fetches, the procedure
pointer register is ignored. The processor enters absolute mode when it is
initialized or immediately after a fault or interrupt. It remains in absolute
mode until it executes a transfer instruction whose operand is obtained via
explicit use of the appending hardware mechanism.

The appending hardware mechanism may be invoked for an instruction by setting
bit 29 of the instruction word ON to cause a reference to a properly loaded
pointer register or by the use of indirect-to-segment (its) or
indirect-to-pointer (itp) modification in an indirect word.  

\subsubsubsection{Append Mode}


The append mode is the most commonly used main memory addressing mode. In
append mode the final computed address is either combined with the procedure
pointer register, or it is combined with one of the eight pointer registers. If
bit 29 of the instruction word contains a 0, then the procedure pointer
register is selected; otherwise, the pointer register given by bits 0-2 of the
instruction word is selected.  

\subsubsubsection{BAR Mode}

In BAR mode, the base address register (BAR) is used. The BAR contains an
address bound and a base address. All computed addresses are relocated by
adding the base address. The relocated address is combined with the procedure
pointer register to form the virtual memory address. A program is kept within
certain limits by subtracting the unrelocated computed address from the address
bound. If the result is zero or negative, the relocated address is out of
range, and a store fault occurs.  

\subsection{PROCESSOR UNIT FUNCTIONS}


Major functions of each principal logic element are listed below and are
described in subsequent sections of this manual.


\subsubsection{Appending Unit}

\begin{description}

\item
Controls data input/output to main memory

\item
Performs main memory selection and interlace

\item
Does address appending

\item
Controls fault recognition

\item
Interfaces with cache

\end{description}


\subsubsection{Associative Memory Assembly}


This assembly consists of sixteen 51-bit page table word associative memory
(PTWAM) registers and sixteen 108-bit segment descriptor word associative
memory (SDWAM) registers.  These registers are used to hold pointers to most
recently used segments (SDWs) and pages (PTWs). This unit reduces the need for
possible multiple main memory accesses before obtaining an absolute main memory
address of an operand or instruction.  



\subsubsection{Control Unit}

\begin{description}

\item
Performs address modification

\item
Controls mode of operation (privileged, normal, etc.)

\item
Performs interrupt recognition

\item
Decodes instruction words and indirect words

\item
Performs timer register loading and decrementing

\end{description}

\subsubsection{Operation Unit}

\begin{description}

\item
Does fixed- and floating-binary arithmetic

\item
Does shifting and Boolean operations

\end{description}

\subsubsection{Decimal Unit}

\begin{description}

\item
Does decimal arithmetic

\item
Does character-string and bit-string operations

\end{description}

